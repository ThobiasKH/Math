\documentclass[11pt]{article}
\usepackage[a4paper,margin=1in]{geometry}
\usepackage{fourier} % Fourier font
\usepackage{xcolor}
\usepackage{tikz}
\usepackage[most]{tcolorbox}
\usepackage{amsthm, amsmath, amssymb}
\usepackage{enumitem}
\usepackage{hyperref}
\usepackage[nameinlink,noabbrev]{cleveref}
\usepackage{titling} 

% Dark mode colors
\definecolor{bgcolor}{HTML}{FFFFFF}
\definecolor{textcolor}{HTML}{000000}
\definecolor{defcolor}{HTML}{E86873}
\definecolor{thmcolor}{HTML}{0A9396}
\definecolor{lemcolor}{HTML}{94D2BD}
\definecolor{corcolor}{HTML}{9B4AF7}
\definecolor{probcolor}{HTML}{EE9B00}
\definecolor{excolor}{HTML}{21E933}

% Background and text color
\pagecolor{bgcolor}
\color{textcolor}

% No paragraph indentation
\setlength{\parindent}{0pt}
\setlength{\parskip}{0.7em}

% Theorem box styles
\tcbset{
  enhanced,
  colback=bgcolor,
  colframe=thmcolor,
  coltext=white,
  coltitle=white,
  fonttitle=\bfseries,
  boxrule=0.7pt,
  left=1em,
  right=1em,
  top=0.7em,
  bottom=0.7em,
  before skip=10pt,
  after skip=10pt,
}

% Theorem environments with colored boxes
\newtcbtheorem[number within=section]{thm}{Theorem}{
  colframe=thmcolor, colback=thmcolor!15!bgcolor
}{thm} % The 'thm' here is the *prefix* for the label

\newtcbtheorem[number within=section]{defn}{Definition}{
  colframe=defcolor, colback=defcolor!15!bgcolor
}{def} % The 'def' here is the *prefix* for the label

\newtcbtheorem[number within=section]{lem}{Lemma}{
  colframe=lemcolor, colback=lemcolor!15!bgcolor
}{lem}

\newtcbtheorem[number within=section]{cor}{Corollary}{
  colframe=corcolor, colback=corcolor!15!bgcolor
}{cor}

\newtcbtheorem[number within=section]{prob}{Problem}{
  colframe=probcolor, colback=probcolor!15!bgcolor
}{prob}

\newtcbtheorem[number within=section]{ex}{Example}{
  colframe=excolor, colback=excolor!15!bgcolor
}{ex}

% Proof environment 
\renewenvironment{proof}[1][\proofname]{%
  \par\pushQED{\qed}\normalfont\topsep6pt \trivlist
  \item[\hskip\labelsep\itshape #1.]\ignorespaces
}{%
  \popQED\endtrivlist\addvspace{6pt}
}

% Cleveref name formats for tcolorbox environments
\crefname{thm}{theorem}{theorems}
\Crefname{thm}{Theorem}{Theorems}

\crefname{def}{definition}{definitions}
\Crefname{def}{Definition}{Definitions}

\crefname{lem}{lemma}{lemmas}
\Crefname{lem}{Lemma}{Lemmas}

\crefname{cor}{corollary}{corollaries}
\Crefname{cor}{Corollary}{Corollaries}

\crefname{prob}{problem}{problems}
\Crefname{prob}{Problem}{Problems}

\crefname{ex}{example}{examples}
\Crefname{ex}{Example}{Examples}


\title{\huge{Oblig 1}}
\author{\LARGE{Thobias Høivik}}
\date{}

\begin{document}
\maketitle

\newpage
\begin{prob*}[Problem 1] 
    La $\mathbb C^* = \mathbb C \setminus \{0\}$, hvor 
    gruppeoperasjonen er vanlig multiplikasjon med 
    komplekse tall. La 
    \[
        A = \left\{z \in \mathbb C^* \bigm| |z| = 1\right\}
    \]

    a) Er $A$ lukket under multiplikasjon?  

    b) Er $A$ en undergruppe av $\mathbb C^*$?
\end{prob*}
\begin{proof}[Løsning og bevis av (a)]
    $A$ er lukket under multiplikasjon. 

    Merk at for $z \in A$ kan vi skrive $z$ som 
    \[
        z = e^{i\theta}, \: \theta \in \mathbb R/2\pi\mathbb Z
    \]
    Vi vet at $\mathbb R/2\pi\mathbb Z$ er lukket under 
    addisjon (faktisk en gruppe), og isomorf med 
    $A$ via $\phi:\mathbb R/2\pi\mathbb Z \to A$
    \[
        \phi(\theta) = e^{i\theta}
    \]
    så $A$ er lukket. 

    Alternativt: 
    \begin{align*}
        z,w \in A \Rightarrow |z| = |w| = 1 \\ 
        |z\cdot w| = |z| \cdot |w| = 1 \\ 
        \Rightarrow z\cdot w \in A
    \end{align*}
\end{proof}
\medskip
\begin{proof}[Løsning og bevis av (b)]
    Vi hevder at $A \leq \mathbb C^*$. 

    Med tanke på at $A \subseteq \mathbb C^*$, og at $A$ er 
    en gruppe (isomorf med $\mathbb R/2\pi \mathbb Z$, som vist 
    i (a)), kan vi konkludere at $A \leq \mathbb C^*$. 

    Alternativt kan vi gjøre følgende:

    $A \subseteq \mathbb C^*$, per definisjon, og 
    \[
        1 = e^{0i} \in A \neq \emptyset
    \]
    Da gjennstår det å vise at for vilkårlig valgte 
    $a,b \in A$ så har vi at 
    \[
        ab^{-1} \in A
    \]
    For $a,b \in A$ finnes det $\theta_a, \theta_b \in 
    \mathbb R/2\pi\mathbb Z$ slik at 
    \[
        a = e^{i\theta_a}, \: b = e^{i\theta_b} 
    \]
    Vi ser at 
    \begin{align*}
        ab^{-1} &= e^{i\theta_a} (e^{i\theta_b})^{-1} \\ 
                &= e^{i\theta_a} e^{i(-\theta_b)} \\ 
                &= e^{i(\theta_a - \theta_b)} \in A
    \end{align*}
    og konkluderer dermed at $A \leq \mathbb C^*$.
\end{proof}
\newpage 
\begin{prob*}[Problem 2]
    La $G$ være en endelig gruppe med identitetselement $e$. 
    Anta at for alle $g \in G$ gjelder 
    \[
        g^2 = e.
    \]

    a) La $a,b \in G$. Vis at 
    \[
        aba^{-1}b^{-1} = e
    \]

    b) Vis at $G$ er abelsk. 

    c) Vis at $G$ er enten triviell eller isomorf med gruppen 
    \[
        \prod_{n=1}^n \mathbb Z_2 = \overbrace{\mathbb Z_2
        \times \cdots \times \mathbb Z_2}^n
    \]
    for en eller annen $n \geq 1$.
\end{prob*}
For simplisitet viser vi (b) først. 

\begin{proof}[Bevis av (b)]
    $G = \{e,g_1,\ldots,g_n\}$ med $g^2 = e, \: \forall g \in G$.
    
    Vi har da at $g = g^{-1}$. Ta vilkårlige $x,y \in G$. 
    Da får vi 
    \begin{align*}
        (xy)^{-1} &= y^{-1}x^{-1} \\ 
        (xy)^{-1} &= yx & (g = g^{-1}) \\ 
        xy = (xy)^{-1} &= yx
    \end{align*}
    $x,y \in G$ var vilkårlig valgt så vi konkluderer at 
    alle $x,y \in G$ kommuterer (sikkert ikke riktig Norsk begrep) 
    med hverrandre. Altså $G$ er abelsk. 
\end{proof}
\medskip
\begin{proof}[Bevis av (a)]
    Nå følger $aba^{-1}b^{-1} = e$ direkte siden $G$ er abelsk. 

    Viss vi skulle vist det uten å vite at $G$ er abelsk 
    kan vi igjen observere at $g = g^{-1}$ og da 
    \begin{align*}
        e &= abb^{-1}a^{-1} \\ 
          &= abba & \text{siden} \ g = g^{-1} \\ 
          &= aba^{-1}b^{-1} & \text{siden} \ (ba)^{-1} = a^{-1}b^{-1}
    \end{align*}
\end{proof}
\medskip
\begin{proof}[Bevis av (c)]
    $G$ er endelig (og dermed endelig generert) og abelsk, 
    så 
    \[
        G \cong \left(\prod_{i=1}^n \mathbb Z_{p_i^{(e_i)}}\right) 
        \times \mathbb Z \times \cdots \times \mathbb Z
    \]
    Ettersom $|G| < \infty$ må $G$ være isomorf til produktet 
    av grupper av formen $\mathbb Z_{p_i^{(e_i)}}$, altså 
    \[
        G \cong \prod_{i=1}^n \mathbb Z_{p_i^{(e_i)}}
    \]
    Betegn 
    \[
        \Gamma := \prod_{i=1}^n \mathbb Z_{p_i^{(e_i)}}
    \]
    med antagelsen at $G \cong \Gamma$. 
    Viss vi antar, for en kontradiksjon, at dette produktet 
    inneholder $\mathbb Z_{p_k^{e_k}}$ med $p_k^{e_k} > 2$ ser 
    vi at det finnes et element $\gamma = (0,\ldots,1,\ldots,0) 
    \in \Gamma$ (0 i alle innslag uten om $k$) med 
    \[
        |\gamma| = p_k^{e_k} > 2 
    \]
    Med antagelsen at $\Gamma \cong G$ får vi da 
    at et element med orden $>2$ må bli sent til et element av 
    orden $2$, som ikke er mulig. Videre kan vi se 
    at fra Lagrange's teorem må 
    \[
        |g| = 2 \Bigm| |G|
    \]
    så $|G| = 2^m, \: m \in \mathbb N$ (siden en odd prim divisor for 
    $|G|$ hadde motsigt $|g| \leq 2$ pga. Cauchy's). 
    Så for $|G| = 2^m$ blir 
    \[
        G \cong \prod_{i=1}^{n:=m} \mathbb Z_2
    \]

    Dersom $G \not\cong \Gamma$ så er den eneste andre endelige 
    genererte abelske gruppen med $g^2 = e, \: \forall g \in G$ 
    den trivielle gruppen $\{e\}$.
\end{proof}

\newpage 
\begin{prob*}[Problem 3]
    La 
    \[
        \sigma = 
        \begin{pmatrix}
            1 & 2 & 3 & 4 & 5 \\ 
            2 & 5 & 4 & 3 & 1
        \end{pmatrix}
        \in S_5
    \]

    a) Skriv $\sigma$ som et produkt av disjunkte sykler. 

    b) Ligger $\sigma$ i $A_5$? 

    c) Hva er ordenen til $\sigma$? 

    d) For den sykliske gruppen $\langle \sigma \rangle$ og 
    elementet $(1,4) \in S_5$, vis at venstre restklasse 
    $(1,4)\sigma$ er ikke lik høyre restklasse $\sigma(1,4)$.
\end{prob*}
\begin{proof}[Løsning av (a)]
    $\sigma$ sender $1 \to 2 \to 5 \to 1$ og 
    $3 \to 4 \to 3$ så 
    \[
        \sigma = (1 \ 2 \ 5)(3 \ 4)
    \]
\end{proof}
\medskip 
\begin{proof}[Løsning av b)]
    Nei, 
    \[
        \sigma = (1 \ 5)(1 \ 2)(3 \ 4)
    \]
    Altså, $\sigma$ er ikke komposisjonen av et partal 
    transposisjoner. 
\end{proof}
\medskip
\begin{proof}[Løsning av (c)]
    Ordenen av en komposisjon av disjunkte sykler $\sigma_1, \sigma_2$
    er $\text{lcm}(|\sigma_1|, |\sigma_2|)$.  

    \[
        \sigma = (1 \ 2 \ 5)(3 \ 4)
    \]
    og siden ordenen av en $3$-sykel er $3$ og en $2$-sykel $2$ 
    så konkluderer vi at 
    \[
        |\sigma| = \text{lcm}(2,3) = 6
    \]
\end{proof}
\begin{proof}[Bevis av (d)]
    Anta at de er like. 
    Husk at $gH = Hg \Leftrightarrow gHg^{-1} = H$, så 
    viss 
    \[
        (1 \ 4)\sigma(1 \ 4)^{-1} = \sigma^r
    \]
    for en $r \in \mathbb Z$ ville vi hatt 
    at venstre restklasse er lik høyre. 
    For simplisitet bruker vi at 
    \[
        \tau(a_1 \ \cdots \ a_n) \tau^{-1} = (\tau(a_1) \ 
        \cdots \ \tau(a_n))
    \]
    så 
    \begin{align*}
        (1 \ 4) \sigma (1 \ 4) &= (1 \ 4)(1 \ 2 \ 5)(3 \ 4)(1 \ 4) \\ 
                               &= (4 \ 2 \ 5)(3 \ 1) 
    \end{align*}
    Merk her at $3 \leftrightarrow 1$, men ingen potens av 
    $\sigma$ vil transponere $3$ og $1$. Dermed er antagelsen at 
    $(1 \ 4)\langle \sigma \rangle = \langle \sigma \rangle (1 \ 4)$
    feil. 
\end{proof}
\newpage 
\begin{prob*}[Problem 4]
    La $GL_2(\mathbb R)$ være gruppen av reelle, inverterbare 
    $(2\times 2)$-matriser, dvs. 
    \[
        GL_2(\mathbb R) = \left\{ 
            \begin{pmatrix}
                a & b \\ 
                c & d
            \end{pmatrix} 
            \Biggm| a,b,c,d \in \mathbb R, ad - bc \neq 0
        \right\}, 
    \]  
    med gruppeoperasjonen matrisemultiplikasjon. La 
    \[
        B = \left\{ 
            \begin{pmatrix}
                a & b \\ 
                0 & d
            \end{pmatrix} 
            \Biggm| 
            a,b,d \in \mathbb R, ad \neq 0
        \right\}
        \subseteq GL_2(\mathbb R).
    \] 

    a) Vis at $B \leq GL_2(\mathbb R)$. 

    La nå $M_1, M_2 \in GL_2(\mathbb R)$ være gitt ved 
    \[
        M_1 = 
        \begin{pmatrix}
            a_1 & b_1 \\ 
            c_1 & d_1
        \end{pmatrix}, \: 
        M_2 = 
        \begin{pmatrix}
            a_2 & b_2 \\ 
            c_2 & d_2
        \end{pmatrix}.
    \]
    
    b) Vis at hvis $M_1B = M_2B$, så finnes det en $a \in \mathbb R 
    \setminus \{0\}$ slik at 
    \[
        \begin{pmatrix}
            a_1 \\ 
            c_1
        \end{pmatrix} 
        = 
        a \begin{pmatrix}
            a_2 \\ 
            c_2
        \end{pmatrix}
    \]

    c) Vis at det finnes $a \in \mathbb R \setminus \{0\}$ slik at 
    \[
        \begin{pmatrix}
            a_1 \\ 
            c_1
        \end{pmatrix} 
        = 
        a \begin{pmatrix}
            a_2 \\ 
            c_2
        \end{pmatrix}
    \]
    så er $M_1B = M_2B$. 
\end{prob*}
\begin{proof}[Bevis av (a)]
    Merk at identitetsmatrisen er i $B$; 
    \[
        e = 
        \begin{pmatrix}
            1 & 0 \\ 
            0 & 1
        \end{pmatrix} 
        \in B
    \]
    så $B \neq \emptyset$. 
    Videre la $x,y \in B$. 
    \begin{align*}
        xy^{-1} &= 
        \begin{pmatrix}
            a_x & b_x \\ 
            0 & d_x
        \end{pmatrix}
        \frac{1}{a_yd_y}
        \begin{pmatrix}
            d_y & -b_y \\ 
            0 & a_y
        \end{pmatrix}
        \\ 
                &=
        \begin{pmatrix}
            a_x & b_x \\ 
            0 & d_x
        \end{pmatrix}
        \begin{pmatrix}
            \frac{1}{a_y} & -\frac{b_y}{a_yd_y} \\ 
            0 & \frac{1}{d_y}
        \end{pmatrix}
        \\ 
                &= 
        \begin{pmatrix}
            \frac{a_x}{a_y} &
            -\frac{a_x b_y}{a_y d_y}
            +
            \frac{b_x}{d_y} \\[6pt]
            0 & \frac{d_x}{d_y}
        \end{pmatrix}.
    \end{align*}
\end{proof}
Vi ser at $(a_x / a_y) \cdot (d_x / d_y) \neq 0$, så 
$xy^{-1} \in B$ og vi konkluderer at 
$B \leq GL_2(\mathbb R)$. (Kanskje one-step test ikke var det 
enkleste her)

\newpage 
\begin{proof}[Bevis av (b)]
    Vi antar at $M_1B = M_2B$ så 
    \[
        M_2^{-1}M_1 \in B
    \]
    Viss vi gjør utregninge får vi da 
    \begin{align*}
        M_2^{-1}M_1 &=
        \frac{1}{a_2d_2 - b_2c_2}
        \begin{pmatrix}
            d_2 & -b_2 \\ 
            -c_2 & a_2
        \end{pmatrix}
        \begin{pmatrix}
            a_1 & b_1 \\ 
            c_1 & d_1
        \end{pmatrix}
        \\ 
                    &= 
        \frac{1}{a_2d_2 - b_2c_2} 
        \begin{pmatrix} 
            a_1 d_2 - b_2 c_1 & b_1 d_2 - b_2 d_1 \\ 
            a_2 c_1 - a_1 c_2 & a_2 d_1 - b_1 c_2 
        \end{pmatrix} \in B
    \end{align*}
    så $a_2c_1 - a_1c_2 = 0 \Rightarrow a_1c_2 = a_2c_1$. Med andre 
    ord er determinanten av 
    \[
        \det\begin{pmatrix}
            a_1 & a_2 \\ 
            c_1 & c_2 
        \end{pmatrix} = 0
    \]
    så kolonnevektorene $(a_1, c_1), (a_2, c_2)$ er lineært avhengige.
    Altså 
    \[
        \begin{pmatrix}
            a_1 \\ c_1 
        \end{pmatrix}
        = 
        a 
        \begin{pmatrix}
            a_2 \\ c_2 
        \end{pmatrix}
        , \: a \in \mathbb R \setminus \{0\}
    \]
    som ønsket. 
\end{proof}
\begin{proof}[Bevis av (c)]
    Vi går fram med samme argument som i (b). 

    Viss vi antar at de første kolonnevektorene av $M_1$
    er lineært uavhengige kan vi jobbe baklengs for å finne 
    at 
    \[
        M_2^{-1}M_1 = \frac{1}{a_2d_2 - b_2c_2} 
        \begin{pmatrix}
            \alpha & \beta \\
            \det (\text{col}_1(M_1) \ \text{col}_1(M_2)) & 
            \delta
        \end{pmatrix}
    \]
    Så $M_2^{-1}M_1$ er en upper-triangular, $2\times 2$ 
    inverterbar, reell, matrise. 
    Altså er $M_2^{-1}M_1 \in B$ så vi konkluderer at 
    $M_2 B = M_1 B$. 
\end{proof}

\end{document}
