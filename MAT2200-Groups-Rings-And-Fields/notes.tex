\documentclass[11pt]{article}
\usepackage[a4paper,margin=1in]{geometry}
\usepackage{fourier} % Fourier font
\usepackage{xcolor}
\usepackage{tikz}
\usepackage[most]{tcolorbox}
\usepackage{amsthm, amsmath, amssymb}
\usepackage{enumitem}
\usepackage{hyperref}
\usepackage[nameinlink,noabbrev]{cleveref}
\usepackage{titling} 

% Dark mode colors
\definecolor{bgcolor}{HTML}{FFFFFF}
\definecolor{textcolor}{HTML}{000000}
\definecolor{defcolor}{HTML}{E86873}
\definecolor{thmcolor}{HTML}{0A9396}
\definecolor{lemcolor}{HTML}{94D2BD}
\definecolor{corcolor}{HTML}{9B4AF7}
\definecolor{probcolor}{HTML}{EE9B00}
\definecolor{excolor}{HTML}{21E933}

% Background and text color
\pagecolor{bgcolor}
\color{textcolor}

% No paragraph indentation
\setlength{\parindent}{0pt}
\setlength{\parskip}{0.7em}

% Theorem box styles
\tcbset{
  enhanced,
  colback=bgcolor,
  colframe=thmcolor,
  coltext=white,
  coltitle=white,
  fonttitle=\bfseries,
  boxrule=0.7pt,
  left=1em,
  right=1em,
  top=0.7em,
  bottom=0.7em,
  before skip=10pt,
  after skip=10pt,
}

% Theorem environments with colored boxes
\newtcbtheorem[number within=section]{thm}{Theorem}{
  colframe=thmcolor, colback=thmcolor!15!bgcolor
}{thm} % The 'thm' here is the *prefix* for the label

\newtcbtheorem[number within=section]{defn}{Definition}{
  colframe=defcolor, colback=defcolor!15!bgcolor
}{def} % The 'def' here is the *prefix* for the label

\newtcbtheorem[number within=section]{lem}{Lemma}{
  colframe=lemcolor, colback=lemcolor!15!bgcolor
}{lem}

\newtcbtheorem[number within=section]{cor}{Corollary}{
  colframe=corcolor, colback=corcolor!15!bgcolor
}{cor}

\newtcbtheorem[number within=section]{prob}{Problem}{
  colframe=probcolor, colback=probcolor!15!bgcolor
}{prob}

\newtcbtheorem[number within=section]{ex}{Example}{
  colframe=excolor, colback=excolor!15!bgcolor
}{ex}

% Proof environment 
\renewenvironment{proof}[1][\proofname]{%
  \par\pushQED{\qed}\normalfont\topsep6pt \trivlist
  \item[\hskip\labelsep\itshape #1.]\ignorespaces
}{%
  \popQED\endtrivlist\addvspace{6pt}
}

% Cleveref name formats for tcolorbox environments
\crefname{thm}{theorem}{theorems}
\Crefname{thm}{Theorem}{Theorems}

\crefname{def}{definition}{definitions}
\Crefname{def}{Definition}{Definitions}

\crefname{lem}{lemma}{lemmas}
\Crefname{lem}{Lemma}{Lemmas}

\crefname{cor}{corollary}{corollaries}
\Crefname{cor}{Corollary}{Corollaries}

\crefname{prob}{problem}{problems}
\Crefname{prob}{Problem}{Problems}

\crefname{ex}{example}{examples}
\Crefname{ex}{Example}{Examples}

\usepackage{tikz-cd}

\title{\huge{Groups, Rings and Fields}}
\author{\LARGE{Thobias Høivik}}
\date{\Large{Spring 2026}}

\begin{document}
\maketitle

\newpage
\tableofcontents

\newpage
\section{Groups}

\begin{defn}[Group]
\label{defn:group}
A group is a set $S$ together with a binary operation 
$\circ$ such that the following properties hold: 

\begin{enumerate}
    \item Identity: There exists an element $e \in S$, satisfying
        $$ 
            e \circ a = a \circ e = a 
        $$ 
        for every $a \in S$.

    \item Inverses: For every $a \in S$ there exists $b \in S$ such 
        that 
        $$ 
            a \circ b = b \circ a = e, \text{ the identity element}
        $$ 
        We usually denote this element as $a^{-1}$ or $-a$, depending
        on context.

    \item Associativity: For any $a,b,c \in S$, we require 
        $$ 
            a \circ (b \circ c) = (a \circ b) \circ c
        $$ 
\end{enumerate}

A group is then the tupple $(S,\circ)$. We will often just write 
the set to refer to the group, e.g. refering to the group 
$(\mathbb Z, +)$ as just $\mathbb Z$.
    
\end{defn}

\subsection{Basic Examples}

\begin{ex}[Integers under addition]
The set of integers $\mathbb{Z}$ with the operation $+$ forms a group:
\begin{itemize}
    \item Identity: $0$ since $0 + n = n + 0 = n$ for all $n \in \mathbb{Z}$.
    \item Inverses: For $n \in \mathbb{Z}$, the inverse is $-n$.
    \item Associativity: Addition is associative.
\end{itemize}
Hence $(\mathbb{Z}, +)$ is a group.
\end{ex}

\begin{ex}[Non-example: Natural numbers under addition]
The set $\mathbb{N}$ under $+$ is \emph{not} a group since
there is no inverse for $n > 0$.
\end{ex}

\subsection{Basic Properties}

\begin{thm}[Uniqueness of identity]
\label{thm:identity-unique}
The identity element in a group is unique.
\end{thm}

\begin{proof}
Suppose $e$ and $e'$ are both identities. Then
$$
e = e \circ e' = e',
$$
so the identity is unique.
\end{proof}

\begin{thm}[Uniqueness of inverses]
\label{thm:inverse-unique}
Each element in a group has a unique inverse.
\end{thm}

\begin{proof}
Suppose $b$ and $c$ are inverses of $a$. Then
$$
b = b \circ e = b \circ (a \circ c) = (b \circ a) \circ c = e \circ c = c.
$$
\end{proof}

\subsection{Example Problem}

\begin{prob}
Determine whether the set 
$$
G = \{1, -1, i, -i\} \subset \mathbb{C}
$$ 
with multiplication is a group.
\end{prob}

\begin{proof}[Solution]
We check the group axioms:
\begin{enumerate}
    \item \textbf{Closure:} Multiplying any two elements of $G$ yields another element in $G$. True.
    \item \textbf{Identity:} The element $1$ acts as identity. True.
    \item \textbf{Inverses:} Each element has an inverse in $G$: $1^{-1}=1$, $(-1)^{-1}=-1$, $i^{-1}=-i$, $(-i)^{-1}=i$. True.
    \item \textbf{Associativity:} Multiplication of complex numbers is associative. True.
\end{enumerate}
Hence $(G, \cdot)$ is a group, in fact it is an abelian group 
which we shall describe below in \cref{defn:abelian_group}. 
\end{proof}

\newpage
\subsection{Abelian Groups}

\begin{defn}[Abelian Group]
\label{defn:abelian_group}
A group $(G, \circ)$ is \emph{abelian} (or commutative) if
$$
a \circ b = b \circ a \quad \forall a,b \in G.
$$
\end{defn}

\begin{ex}
The group $(\mathbb{Z}, +)$ is abelian because $m+n = n+m$.
\end{ex}

\newpage 
\section{Important Groups}
First let's define some often used notation. 

For $g \in G$, define: 
\begin{itemize}
    \item $g^n = gg\ldots g(n > 0)$ 
    \item $g^0 = e$
    \item $g^{-n} = (g^{-1})^n$
\end{itemize} 
and recognize the following identities (which are provable 
by induction): 
$$ 
    g^m g^n = g^{m+n}, \quad (g^m)^n = g^{mn}
$$ 

\begin{defn}[Roots of Unity]
    \label{defn:roots_of_unity}
    $$ 
        U_n = \{e^{2\pi ik/n} : 0 \leq k < n\}
    $$ 
    are called the n-th roots of unity.
\end{defn}

\begin{defn}
    An subset $\{g_1,\ldots,g_k\} \subseteq G$ is a generating set 
    if every $g \in G$ can be expressed as 
    $$ 
        g_1^{n_1} \cdots g_k^{n_k}
    $$ 

    If a singleton $g'$ alone generates a group then it is called 
    a generator of $G$ and $G$ is said to be cyclic. 
\end{defn}

The $n$-th roots of unity are finite, with exactly $n$ elements. 
They are cyclic, with generator $e^{2i\pi/n}$.

\begin{defn}[Symmetric Group]
    \label{defn:symmetric_group}
    The symmetric group $S_n$ is the set of all permutations of 
    $n$ elements under composition. 
    In other words it is the set of all $\sigma : [n] \to [n]$ with 
    composition as its operation. 
\end{defn}

The symmetric group is non-abelian for $n \geq 3$. The identity 
corresponds to the identity permutation (doing nothing) and inverses 
are undoing a permutation. 
We have the following notation for permutations in the symmetric Group
(assume $n = 3$ for this example): 
$$ 
    \begin{pmatrix}
        1 & 2 & 3 \\ 
        2 & 3 & 1
    \end{pmatrix}
$$ 
corresponds to $1 \mapsto 2, 2 \mapsto 3, 3 \mapsto 1$. 

Another way to write the same permutation is: 
$$ 
    \left(1 \: 2 \: 3\right)
$$ 

\newpage 
\section{Subgroups}
\begin{defn}[Subgroup]
    \label{defn:subgroup}
    Let $G$ be a group under $\star$.
    A subgroup $H \leq G$ is a subset $H \subseteq G$ that is itself 
    a group under $\star$.
\end{defn}

Having to mechanically check if $H \subseteq G$ satisfies all the 
required axioms can be a bit tedious so next we introduce a 
powerful theorem which let's us easily determine whether some 
subset is a subgroup or not. 

\begin{thm}[Subgroup Test]
    \label{thm:subgroup_test}
    A nonempty subset $H \subseteq G$ is a subgroup if and only 
    if 
    $$ 
        \forall x,y \in H, \; xy^{-1} \in H.
    $$ 
\end{thm}
\begin{proof}[Proof Idea]
    If $H$ is a subgroup the required conditions follows. 

    Conversely: 
    \begin{itemize}
        \item Nonempty so some $h \in H$.
        \item $hh^{-1} = e \in H$. 
        \item Closure under $xy^{-1}$ will give inverses and 
            closure. 
    \end{itemize}
\end{proof}

We get some nice corollaries from this, as well as a nice and tidy 
way to prove whether or not a subset is a subgroup. 

\begin{cor}
    The intersection of any collection of subgroups is a subgroup. 

    Given any subset $S \subseteq G$, there is a smallest subgroup 
    containing it. 
\end{cor}

\begin{defn}[Cyclic Subgroups]
    Let $G$ be a group. 

    For any $g \in G$, define: 
    $$ 
        \langle g \rangle  = \{g^n : n \in \mathbb Z\}.
    $$ 

    $\langle g \rangle$ is called the cyclic subgroup generated 
    by $g$. 
\end{defn}

It is very straightforward to show that the cyclic subgroup generated 
by some $g \in G$ is indeed a subgroup with the use of 
\cref{thm:subgroup_test}.

On page \pageref{defn:roots_of_unity} we say that a group is cyclic 
if it has a signle element which generates it, but now we can simply
say that a group $G$ is cyclic if $G = \langle g \rangle$ for some 
$g \in G$.

\begin{thm}
    Suppose a group $G$ is cyclic, i.e. $G = \langle g \rangle$ 
    for some $g \in G$. 

    Then it is isomorphic to one of the following: 
    \begin{itemize}
        \item $(\mathbb Z, +)$, or 
        \item $(\mathbb Z/n\mathbb Z, +)$ for some $n \geq 1$. 
    \end{itemize}
\end{thm}

\subsection{Structure of Cyclic Subgroups}
\begin{defn}[Order]
    The order of an element $g \in G$ is 
    $$ 
        \text{ord}(g) = |\langle g \rangle|
    $$ 
\end{defn}

Furthermore, if $\text{ord}(g) = n < \infty$, then $g^n = e$ and 
$n$ is minimal, otherwise we say $g$ has infinite order.  

If $\text{ord}(g) = n$, then: 
$$ 
    \langle g \rangle = \{e, g, g^2, \ldots, g^{n-1}\}
$$ 

Moreover: 
\begin{itemize}
    \item $g^k = e \Leftrightarrow n \mid k$
    \item $\langle g^k \rangle = \langle g \rangle \Leftrightarrow 
        \gcd(k,n) = 1$
\end{itemize}

\begin{thm}
    Every subgroup of a cyclic group is cyclic. 

    More precisely: 
    \begin{itemize}
        \item Subgroups of $\mathbb Z$ are exactly $n\mathbb Z$.
        \item Subgroups of $\mathbb Z/n\mathbb Z$ correspond 
            to the divisors of $n$.
    \end{itemize}
\end{thm}

With all of this machinery we can approach group theory with geometric
and algebraic intuition. Cyclic groups are like repeated motion, 
finite cyclic groups corresponding to rotations by rational angles 
and infinite ones corresponding to translations. 

\newpage 
\section{Cosets and Lagrange's Theorem}
\begin{defn}[Left and Right Cosets]
    Let $H \leq G$, and let $g \in G$. 

    The left- and right cosets are: 
    $$ 
        gH = \{gh : h \in H\} \quad Hg = \{hg : h \in H\}
    $$ 
    respectively.
\end{defn}

A key intuition is that a coset is like a copy of $H$ shifted 
by $g$. 
We may also recognize that the cosets are the same size as $H$ by 
identifying $h \mapsto gh$ to be a bijection and so forth. 

The next natural question is when are two cosets equal? 
Like with showing when a subset is a subgroup we have a little trick 
for this problem. 

\begin{thm}
    Let $G$ be a group. 

    For $g_1, g_2 \in G$, the following are equivalent: 
    \begin{align*}
        g_1 H &= g_2 H \\ 
              &\Updownarrow \\ 
        g_2^{-1}g_1 &\in H \\ 
                    &\Updownarrow \\ 
        g_1 &\in g_2 H 
    \end{align*}
\end{thm}

\begin{thm}
    \label{thm:coset_partition}
    The set of all left cosets of $H \leq G$ forms a 
    partition of $G$. 

    That is, 
    \begin{itemize}
        \item Every element of $G$ is in exactly on coset. 
        \item Two cosets are either the same or disjoint. 
    \end{itemize}
\end{thm}

\begin{defn}
    Let $G$ be a group with $H \leq G$. 

    The index of $H$ in $G$, written $[G:H]$, is the number of 
    left cosets of $H$. 
\end{defn}

If $G$ is finite we get $|G| = [G:H] \cdot |H|$.

\newpage 
\begin{thm}[Lagrange's Theorem]
    \label{thm:lagrange} 
    Let $G$ be a group with $H \leq G$. 

    Then: 
    $$ 
        |H| \bigm\vert |G|
    $$ 
\end{thm}
\begin{proof}
    Let $G$ be a group with $H \leq G$. 

    Recall that by \cref{thm:coset_partition} we have that the 
    set of left cosets of $H$, $g_1 H, g_2 H, \ldots, g_k H$ 
    (this set is finite since $G$ is finite) are 
    pairwise disjoint, satisfying 
    $$ 
        G = \bigcup_{i=1}^k g_i H
    $$ 
    meaning 
    $$ 
        |G| = \displaystyle\sum_{i=1}^{k}|g_i H|
    $$ 

    Recall also that $|H| = |g_i H|, \, \forall i \in [k]$. 
    Thus 
    $$ 
        |G| = \displaystyle\sum_{i=1}^{k}|H| = k|H|
    $$ 

    In other words $|G|$ is $|H|$ times some integer $k$, therefore 
    $|H|$ divides $|G|$. 
\end{proof}


\Cref{thm:lagrange} has the following immediate consequences: 
\begin{itemize}
    \item The order of any element divides $|G|$. 
    \item $g^{|G|} = e$ for every $g \in G$. 
\end{itemize}

Beware that it does not follow from Lagrange's Theorem that there 
exists a subgroup with the order of a divisor of $|G|$ for every 
divisor of $G$. The alternating group of order $12$, $A_{4}$, 
has no subgroup of order $6$ for example.

\begin{prop}
    A group of order $p$ where $p$ is prime, is cyclic. 
\end{prop}
\begin{proof}[Proof Sketch]
    Let $G$ be a group with $|G| = p$ (prime). 

    Let $g \in G \setminus \{1\}$. Consider it's generated subgroup 
    $\langle g \rangle$. By Lagrange's Theorem the order of this 
    subgroup divides $p$ so $|\langle g \rangle|$ is $1$ or $p$, but 
    it can't be $1$ as $g \neq 1$ so it's order is $p$, i.e. 
    $g$ generates the entirety of $G$. 
\end{proof}

\subsection{Problems relating to subgroups}
\begin{prob}
    Let $G$ be a group and $g \in G$ with  $\text{ord}(g) = 12$. 

    \begin{itemize}
        \item List all distinct subgroups of $\langle g \rangle$.
        \item For which integers $k$ does $g^k$ generate $\langle g \rangle$?
    \end{itemize}
\end{prob}
\begin{proof}[Solution]
    Recalling that $\langle g \rangle = \{g^k : k \in \mathbb Z\}$, 
    we want to find all the distinct subgroups of $\langle g \rangle$. 
    In other words, which subsets of $\langle g \rangle$ are also 
    groups with respect to the operation of $G$? 

    $\text{ord}(g) = |\langle g \rangle| = 12$ hence $\langle g \rangle = \{g^0, g^1, \ldots, g^{11}\}$.
    We know that there are two trivial subgroups of $\langle g \rangle$, namely $\{g^0\}$ and $\langle g \rangle$. 
    Otherwise, we know that every cyclic subgroub of a cyclic group (which $\langle g\rangle$ is) is itself 
    cyclic, so we can find the subgroups by looking at the generated groups of each element 
    of $\langle g \rangle$. Otherwise, we could use the fact that 
    $\langle g\rangle $ is isomorphic to $\mathbb Z/12\mathbb Z$ together with realizing that the number 
    of subgroups is then the same as the number of positive divisors of $12$. 

    Using the first approach we get: 
    \begin{align*}
        \langle g^2 \rangle &= \{g^n : n = 2k, \: k \in \mathbb Z\} \\ 
                            &= \{e, g^2, \ldots, g^10\} = \langle g^10 \rangle \\ 
        \langle g^3 \rangle &= \{e, g^3, g^6, g^9\} = \langle g^9 \rangle \\ 
        \langle g^4 \rangle &= \{e, g^4, g^8\} = \langle g^8 \rangle \\ 
        \langle g^6 \rangle &= \{e, g^6\} \\ 
    \end{align*}
    
    Those integers which are coprime to $12$, like $1,5,7,11$, will generate $\langle g \rangle$. 
    Thus we have $6$ distinct subgroups, exactly the same as the number of 
    positive divisors of $12$, as expected. 
\end{proof}

\begin{prob}
    Let $G$ be any group $H = \langle g \rangle$ where $\text{ord}(g) < \infty$. 

    Prove that if $g^m \in H$, then $\langle g^m \rangle \leq H$. 
\end{prob}
\begin{proof}
    Assume, for contradiction, that $g^m \in H$, but $\langle g^m \rangle$ 
    is not a subgroup of $H$. 

    Let $s = m \mod \text{ord}(g)$. 
    Then $\langle g^m \rangle = \{g^{sk} : k \in \mathbb Z\} \neq \emptyset$.  
    It is clear that this is a subset of $\langle g \rangle = H$, since 
    $\langle g \rangle$ is closed under the group operation and 
    $g^m \in H$. 
    Then the only way our assumption is true is if 
    there exists some $x,y \in \langle g^m \rangle$ such that 
    $xy^{-1} \not\in \langle g^m \rangle$. 

    Every element $x,y \in \langle g^m \rangle$ is of the form 
    $x = g^{sn}, \, y = g^{st}$ for some $n,l \in \mathbb N$. 
    By assumption, 
    $xy^{-1} = g^{s(n-t)} \not\in \langle g^m \rangle$ 
    which would necessarily mean that $n-t \not \in \mathbb Z$, 
    impossible. 
\end{proof}
The above argument can be made more rigorous 
by considering the isomorphism between $\langle g^m \rangle$ and 
$\mathbb Z/s\mathbb Z$, or by proving the claim 
directly (I am an idiot, and this was a bad approach). 

\newpage 
\section{Normal subgroups}
\begin{defn}[Normal subgroup]
    \label{defn:normal_subgroup}
    A subgroup $H \leq G$ is normal, written $H \trianglelefteq G$, if:
    $$ 
        gH = Hg, \: \forall g \in G
    $$ 
    with the equivalent characterization:
    $$ 
        H \trianglelefteq G \Leftrightarrow gHg^{-1} = H, \: \forall g \in G
    $$ 
\end{defn}

A few things to tick off immediately; 
\begin{enumerate}
    \item Every subgroup of an abelian group is normal. 
    \item \{e\} and $G$ are always normal. 
    \item The alternating group $A_n \trianglelefteq S_n$. 
    \item Subgroups of index $2$ are always normal. 
\end{enumerate}

If you consider the map (conjugation)
$$ 
    x \mapsto gxg^{-1},
$$ 
which constitutes an automorphism of $G$ we can say 
the following. 

Normal subgroups are precisely those subgroups which are 
fixed under every conjugation. 

\begin{defn}[Quotient Group]
    \label{defn:quotient_group}
    If $H \trianglelefteq G$, define: 
    $$ 
        G/H := \{gH : g \in G\}
    $$ 
    with an operation we'll call multiplication by: 
    $$ 
        (gH)(kH) = (gk)H
    $$ 
\end{defn}

\begin{thm}
    The quotient group with multiplication, as defined above, 
    is a well-defined group. 
\end{thm}

\newpage 
\begin{defn}
    We define the \textbf{canonical projection} as 
    $$ 
        \pi : G \to G/H, \: \pi(g) = gH.
    $$ 
\end{defn}
Notably, the projection $\pi$ constitutes 
a homomorphism with kernel $\ker \pi = H$.
Notice that this claim of the kernel being $H$ itself 
should be immediately obvious as $hH$ for some 
$h \in H$ is itself $H$ as it's closed. 

\newpage 
\section{Exercises for Week 1}
\begin{prob}[29, p.48]
    Show that if $G$ is a finite group with identity $e$ and 
    an even number of elements, then there is 
    $a \neq e$ in the group such that $a * a = e$.
\end{prob}
\begin{proof}
    We are being asked to show that in a finite group with even 
    order, there exists some element other than the identity which 
    is it's own inverse. 

    Recall uniqueness of inverses and that for $g \in G$, 
    $(g^{-1})^{-1} = g$. Thus for every element there exists one 
    and only one element which is it's inverse. $e^{-1} = e$ is 
    covered. In $G \setminus \{e\}$ there are an odd number of 
    elements. Each of these have exactly one unique inverse meaning 
    that there is one element "left" over which then necessarily 
    has itself as it's inverse. 
\end{proof}

\begin{prob}[33, p.49]
    Let $G$ be an abelian group and let $c^n = c * c * \cdots * c$ 
    for $n$ factors c, where $c \in G$ and $n \in \mathbb Z^+$. 
    Give a proof by mathematical induction that $(a*b)^n = a^n * b^n$ 
    for all $a,b \in G$. 
\end{prob}
\begin{proof}
    Let $a,b \in G$. The base case where $n = 1$ is trivial so 
    we proceed to our assumption that for any $k \in \mathbb Z^+$ we 
    have $(a*b)^k = a^k * b^k$. 

    Let $n = k + 1$. 
    \begin{align*}
        (a*b)^n &= (a*b)^{k+1} \\ 
                &= (a*b)^k * (a*b) \\  
                &= a^k * b^k * (a*b) & \text{by hypothesis} \\ 
                &= a^k * (b^k * a) * b & \text{by associativity} \\ 
                &= a^k * (a * b^k) * b & \text{by commutativity} \\ 
                &= (a^k * a) * (b^k * b) \\ 
                &= (a^{k+1}) * (b^{k+1}) \\ 
                &= a^n * b^n
    \end{align*}
     
    By the principle of mathematical induction we have that 
    $$  
        (a*b)^n = a^n * b^n, \: \forall n \in \mathbb Z^+
    $$ 
\end{proof}

\newpage 
\begin{prob}[35, p. 49]
    Show that if $(a*b)^2 = a^2 * b^2$ for $a,b$ in a group, then 
    $a*b = b*a$.  
\end{prob}
\begin{proof}
    \begin{align*}
        (a*b)^2 &= (a*b)*(a*b) \\ 
        a^2 * b^2 &= (a*a) * (b*b) \\ 
        \Rightarrow (a*b) * (a*b) &= (a*a) * (b*b) \\ 
                    a * (b*a) * b &= a*(a*b)*b
    \end{align*}
    
    By one application each of left- and right cancellation we have 
    $$ 
        a*b = b*a
    $$ 
\end{proof}

\begin{prob}[41, p. 49]
    Let $G$ be a group and fix $g \in G$. Show that the map 
    $i_g$, s.t. $i_g(x) = gxg^{-1}$ for $x \in G$. Show that 
    $i_g$ is an automorphism on $G$.
\end{prob}
\begin{proof}
    Let $i_g : G \to G$ be defined as above. We need to show that 
    it satisfies all the properties of an isomorphism. 

    It is clear, through left- and right cancellation, that 
    $i_g$ is injective. 

    Notice also that for any $x \in G$ we have the pre-image 
    $g^{-1}xg$ since $i_g(g^{-1}xg) = gg^{-1}xgg^{-1} = x$, hence 
    $i_g$ is surjective. 

    Lastly we verify the homomorphism property: 
    \begin{align*}
        i_g(xy) &= gxyg^{-1} \\ 
                &= gxeyg^{-1} \\ 
                &= gxg^{-1}gyg^{-1} \\ 
                &= i_g(x) i_g(y)
    \end{align*}
    completing the proof. 
\end{proof}

\newpage 
\section{Problems for week of Jan 26}
\begin{prob}[Problem 44 Fraleigh's p.67]
    Let $G = \langle a \rangle$ be a cyclic group isomorphic 
    to a group $G'$. 
    If $\phi : G \to G'$ is an isomorphism, show that 
    for every $x \in G$, $\phi(x)$ is uniquely determined by $\phi(a)$.
    That is, if $\phi$ and $\psi$ constitutes isomorphisms from 
    $G \to G'$ with $\phi(a) = \psi(a)$, then $\phi(x) = \psi(x)$ 
    for every $x \in G$.
\end{prob}
\begin{proof}
    Let $G$ be generated by $a$, isomorphic to $G'$ via 
    $\phi$ and $\psi$, satisfying  
    \[
        \phi(a) = \psi(a)
    \]
    Let $x \in G$. Then $x = a^n$ for some $n \in \mathbb Z$, as 
    $G = \langle a \rangle$. 
    Observe the following:
    \begin{align*}
        \phi(x) &= \phi(a^n) \\ 
                &= \phi(a)^n \\ 
                &= \psi(a)^n \\ 
                &= \psi(a^n) = \psi(x)
    \end{align*}
    This completes the proof.
\end{proof}

\begin{prob}[Problem 6 Fraleigh's p.83]
    Compute $|\sigma|$ for 
    \[
        \sigma = 
        \begin{pmatrix}
            1 & 2 & 3 & 4 & 5 & 6 \\ 
            3 & 1 & 4 & 5 & 6 & 2
        \end{pmatrix}
    \]
\end{prob}
\begin{proof}[Solution]
    Observe that $\sigma$ is the permutation which sends 
    \begin{align*}
        1 &\to 3 \\ 
        2 &\to 1 \\ 
        & \ \vdots \\ 
        6 &\to 2
    \end{align*}
    so $\sigma^2$ is 
    \begin{align*}
        1 &\to 4 \\ 
        2 &\to 3 \\ 
          & \ \vdots \\ 
        6 &\to 1
    \end{align*}
    We want to find $n \in \mathbb N$ such that 
    $\sigma^n = $ the identity permutation. 
    It's easy to see that performing the permutation $6$ times 
    gives $1 \to 1$ which we would require. We also see that 
    $2 \to 2$, $3 \to 3$ and so on. 

    We could also write the permutation in compact form and see 
    that $\sigma$ is a cycle of length $6$, so the order is $6$. 
\end{proof}

\newpage 
\begin{prob}[Problem 7 Fraleigh's p.83]
    Compute $|\tau|$ for 
    \[
        \tau = 
        \begin{pmatrix}
            1 & 2 & 3 & 4 & 5 & 6 \\ 
            2 & 4 & 1 & 3 & 6 & 5 
        \end{pmatrix}
    \]
\end{prob}
\begin{proof}[Solution]
    Note that $\tau$ in compact notation is 
    \[
        \tau = (1 \ 2 \ 4 \ 3)(5 \ 6)
    \]
    so $\tau$ is the composition of a $2$- and $4$-cycle. 
    \[
        |\tau| = \text{lcm}(4, 2) = 4
    \]
\end{proof}
\begin{prob}[Problem 49 Fraleigh's p.86]
    If $A$ is a set, then a subgroup $H \leq S_A$ is transitive 
    on $A$ if for each $a,b \in A$ there exists $\sigma \in H$ 
    such that $\sigma(a)=b$. Show that if $A$ is a nonempty finite 
    set, then there exists a finite cylic subgroup $H$ of $S_A$ 
    with $|H| = |A|$ that is transitive on $A$.
\end{prob}
\begin{proof}
    Let $A = \{a_1, a_2, \ldots, a_n\}$ with $S_A$ being the 
    permutations on $A$. Let $\sigma \in S_A$ be the permutation which 
    sends $a_i \to a_{i+1}$ (take $i$ modulo $n$). It is clear that 
    $\sigma^k$ is the permutation which sends $a_i$ to $a_{i+k}$ and 
    furthermore $|\sigma| = n = |A|$. It is not hard to see 
    that $\langle\sigma\rangle$ forms a subgroup 
    where each permutation shifts 
    all elements of $A$. 
    
    It is clear that for any $a_i, a_j \in A$ we can find 
    a permutation which sends one to the other. Suppose, without 
    loss of generality, $i < j$. Then 
    $\sigma^{j-i}(a_i) = a_j$, and the inverse of said permutation 
    sends $a_j$ to $a_i$.

    A bit inelegantly written, but I digress. 
\end{proof}
\begin{prob}[Problem 1 Fraleigh's p.94]
    Find the orbit of 
    \[
        \sigma = 
        \begin{pmatrix}
            1 & 2 & 3 & 4 & 5 & 6 \\ 
            5 & 1 & 3 & 6 & 2 & 4
        \end{pmatrix}
    \]
\end{prob}
\begin{proof}[Solution]
    We are looking at $\sigma \in S_6$ acting on $[6]$ (I assume)
    so we want to find 
    \[
        \left\{ 
            \text{Orb}_\sigma(n) = \{\sigma^k(x) : 
            k \in \mathbb Z\} 
            \mid n = 1, 2, \ldots, 6
        \right\}
    \]

    Let us rewrite $\sigma$ more compactly. 
    \[
        1 \to 5 \to 2 \to 1 
    \]
    \[
        3 \to 3 
    \]
    \[
        4 \to 6 \to 4
    \]
    so 
    \[
        \sigma = (1 \ 5 \ 2)(4 \ 6)
    \]
    
    Thus we have the following orbits: 
    \begin{align*}
        \text{Orb}_\sigma(1) = \text{Orb}_\sigma(2) = 
        \text{Orb}_\sigma(5) = \{1,2,5\} \\ 
        \text{Orb}_\sigma(4) = \text{Orb}_\sigma(6) = \{4,6\} \\
        \text{Orb}_\sigma(3) = \{3\} \\ 
    \end{align*}
\end{proof}

\begin{prob}[Problem 7 Fraleigh's p.94]
    Compute: 
    \[
        (1 \ 4 \ 5)(7 \ 8)(2 \ 5 \ 7)
    \]
\end{prob}
\begin{proof}[Solution]
    \begin{align*}
        (1 \ 4 \ 5)(7 \ 8)(2 \ 5 \ 7)
        &= (1 \ 4 \ 5) \circ ((7 \ 8)(2 \ 5 \ 7)) \\ 
        &= (1 \ 4 \ 5) (2 \ 5 \ 8 \ 7) \\ 
        &= (2 \ 1 \ 4 \ 5 \ 8 \ 7)
    \end{align*}
\end{proof}
\begin{prob}[Problem 29 Fraleigh's p.96]
    Show that for every subgroup $H$ of $S_n$ for $n \geq 2$, 
    either all permutations in $H$ are even or exactly half are. 
\end{prob}
\begin{proof}
    Recall that an even permutation is a permutation which can be 
    expressed as an even number of transpositions (swaps of 
    $2$ elements). 

    Let $\text{sgn}(\sigma) = 1$ if $\sigma$ is even, 
    $\text{sgn}(\sigma) = -1$ otherwise for all $\sigma \in S_n$.  
    $\text{sgn}:S_n \to \{1,-1\}$ is a homomorphism.  
    Restrict this map to $H$. Then, the image: 
    \[
        \text{sgn}[H] = 
        \begin{cases}
            \{+1\} \\ 
            \{+1,-1\}
        \end{cases}
    \]
    In the first case it is clear that every element of $H$ must 
    be even. 

    In the other case where $\text{sgn}$ is surjective, the 
    kernel 
    \[
        \ker(\text{sgn}[H]) = H \cap A_n,
    \]
    the set of even permutations in $H$. 
    By the first isomorphism theorem, 
    \[
        H/(H\cap A_n) \cong \{+1, -1\}
    \]
    so 
    \[
        [H:H\cap A_n] = 2
    \]
    hence 
    \[
        |H \cap A_n| = \frac{1}{2} H
    \]
    i.e. half of $H$ is even. 
\end{proof}

\newpage 
\section{External- and Internal Direct Products}
\begin{defn}
    Let $H,K \leq G$, 
    \begin{align*}
        G' = H \times K = \{(h,k) \bigm| h \in H, \ k \in K\} \\ 
        G = H \oplus K = \{hk \bigm| h \in H, \ k \in K\}
    \end{align*}
    The internal direct product $H \oplus K$ presupposes that 
    $H, K \unlhd G$, and $H \cap K = \{e\}$.
\end{defn}
We really want to be able to say that $H \times K \cong H \oplus K$.

\begin{ex}
    Consider the dihedral group $D_3 = \{1,x,x^2, y, xy, x^y\}$. 
    Let $H = \langle x \rangle$ and $K = \langle y \rangle$.
    It's easy to see that 
    $H \cap K = \{e\} = \{1\}$, and $H \unlhd D_3$. 
    However $xK = \{x, xy\}$, but $Kx = \{x,yx\} = \{x, x^2y\}$. 
    Hence 
    \[
        D_3 \neq H \oplus K
    \]
\end{ex}

\begin{lem}
    $H,K \unlhd G$ and $H \cap K = \{e\}$. 
    \begin{enumerate}
        \item if $ab = a'b'$ where $a,a' \in H$ and $b,b' \in K$ 
            then $a = a'$ and $b = b'$.
        \item if $a \in H$ and $b \in K $ then $ab = ba$.
    \end{enumerate}
\end{lem}

\begin{thm}
    Assuming $H,K$ normal subgroups of $G$ with trivial intersection, 
    then 
    \[
        H \times K \cong H \oplus K
    \]
\end{thm}
\begin{proof}
Define a map
\[
\varphi : H \times K \to HK, \qquad \varphi(h,k) = hk.
\]

This map is well-defined since \(h \in H\), \(k \in K\) implies \(hk \in HK\).

Define a second map
\[
\psi : HK \to H \times K
\]
by sending \(g \in HK\) to the unique pair \((h,k)\) such that \(g = hk\).
Such a decomposition exists since \(HK = G\), and is unique because
\(H \cap K = \{e\}\).

We now verify that these maps are inverse to one another. For all
\((h,k) \in H \times K\),
\[
(\psi \circ \varphi)(h,k) = \psi(hk) = (h,k),
\]
and for all \(g \in HK\),
\[
(\varphi \circ \psi)(g) = g.
\]
Thus \(\varphi\) is bijective with inverse \(\psi\).

Finally, \(\varphi\) is a homomorphism, since for \((h_1,k_1),(h_2,k_2) \in H \times K\),
\begin{align*}
    \varphi((h_1,k_1)(h_2,k_2))
    &= \varphi(h_1h_2,k_1k_2) \\
    &= h_1h_2k_1k_2 \\
    &= h_1k_1h_2k_2 \\
    &= \varphi(h_1,k_1)\varphi(h_2,k_2),
\end{align*}
where we use that elements of \(H\) and \(K\) commute.

Therefore \(\varphi\) is an isomorphism, and \(HK \cong H \times K\).
\end{proof}

\begin{thm}
    If $|x|, |y| < \infty$ and $(x,y) \in G \times H$ then 
    $|(x,y)| = \text{lcm}(|x|,|y|)$.
\end{thm}

\newpage 
\section{Problems from week of 8 Feb}
\begin{prob}[Problem 47 Fraleigh's p.113]
    Let $G$ be an abelian group. Let $H \subseteq G$ consisting of the 
    identity and all elements of $H$ with order $2$. Show that 
    $H \leq G$.
\end{prob}
\begin{proof}
    Let $H = \{e\} \cup \{g \in G \bigm| |g| = 2\} \subseteq G$. 
    Let $x,y \in H$, and for the sake of simplicity assume they 
    are nontrivial, i.e. $x\neq e \neq y$. 
    To show that $H$ constitutes a subgroup of $G$ 
    it suffices to show that $xy^{-1} \in H$ (we already 
    know it is nonempty).

    \begin{align*}
        (xy^{-1})^2 &= xy^{-1}xy^{-1} \\ 
                    &= xxy^{-1}y^{-1} &(G \ \text{abelian}) \\ 
                    &= x^2(y^2)^{-1} \\ 
                    &= e(e) \\ 
                    &= e
    \end{align*}
    Hence $|xy^{-1}| = 2$, so $xy^{-1} \in H$.
    Thus $H$ is a subgroup of $G$.
\end{proof}

\begin{prob}[Problem 52 Fraleigh's p.113]
    Show that a finie abelian group is not cyclic if and only 
    if it contains a subgroup isomorphic to $\mathbb Z_p 
    \times \mathbb Z_p$ for some prime $p$.
\end{prob}
\begin{proof}
    Let $G$ be a finite abelian group. 

    We begin by proving the right-to-left implication. 
    Assume then that there is some $H \leq G$ with 
    $H \cong \mathbb Z_p \times \mathbb Z_p$. 
    Suppose, for a contradiction, 
    that $G = \langle g \rangle$ (cyclic) for some $g \in G$.
    Let $\phi : H \to \mathbb Z_p \times \mathbb Z_p$ be an 
    isomorphism. 

    Recall that as $G$ is cyclic, generated by $g$, we have that  
    there is some $k \in \mathbb Z$, for every $h \in H$, 
    such that $h = k\cdot g$. As $G$ is finite, $k \cdot g = 
    (|G| - k) \cdot g$, so we can safely assume $k \geq 0$ without 
    any loss of generality. 

    Since $\phi$ is bijective we know that, for every  
    $y \in \mathbb Z_p \times \mathbb Z_p$, there is some 
    $h \in H$ such that 
    \[
        \phi(h) = y
    \]
    Using the homomorphism property, and the 
    fact that $h = kg, \ k \in \mathbb N$, we get that: 
    \begin{align*}
        y &= \phi(h) \\ 
          &= \phi(k\cdot g) \\ 
          &= \phi\left(\displaystyle\sum_{i=1}^{k}g\right) \\ 
          &= \displaystyle\sum_{i=1}^{k} \phi(g) \\ 
          &= k \cdot \phi(g)
    \end{align*}
    As $y$ was chosen arbitrarily we now see that 
    $\mathbb Z_p \times \mathbb Z_p = \langle \phi(g) \rangle$, i.e. 
    cyclic. Being cyclic, together with the fact that 
    $|\mathbb Z_p \times \mathbb Z_p| = p^2$, it is the 
    case that there must be some element with order 
    $p^2$, but no such element exists so $\mathbb Z_p \times 
    \mathbb Z_p$ is not cyclic. We have arrived at a contradiction 
    so our assumption that $G$ is cyclic must be false. 

    Now we prove the other direction. 
    We now assume that we have a finite abelian group $G$ which 
    is not cyclic. As $G$ is finite we know that $G$ is finitely 
    generated. Thus by the fundamental theorem of finitely generated
    abelian groups we know that $G$ is isomorphic to a direct product 
    of cyclic groups of the form 
    \[
        \mathbb Z_{(p_1)^{\gamma_1}}
        \times 
        \mathbb Z_{(p_2)^{\gamma_2}}
        \times \cdots \times 
        \mathbb Z_{(p_n)^{\gamma_n}} 
        \times 
        \mathbb Z 
        \times \mathbb Z
        \times \cdots \times 
        \mathbb Z
    \]
    As $G$ is finite, the direct product does not include 
    $\mathbb Z$. Furthermore as $G$ is not cyclic we know that 
    \[
        G \not\cong \mathbb Z_{p^\gamma}
    \]
    Hence $G$ must be the direct product of at least $2$ 
    $\mathbb Z_{(p_i)^{\gamma_i}}$. As $\gamma_i \in \mathbb N 
    \setminus \{0\}$ and distinct from $\gamma_j$ for $j \neq i$ 
    there must be some $\mathbb Z_{(p_i)^{\gamma_l}}$ with 
    $\gamma_l \geq 2$. Notice that this group necessarily 
    has 
    \[
        H = \{0\} \times \cdots \times \{0\} \times 
        \mathbb Z_{p_i} \times \{0\} \times \cdots 
        \times \{0\} \times \mathbb Z_{p_i} \times \{0\} \times 
        \cdots \times \{0\}
    \]
    as a subgroup, which itself is isomorphic to 
    $\mathbb Z_{p_i} \times \mathbb Z_{p_i}$. 
    Then $\phi^{-1}[H]$ is a subgroup of $G$, which is 
    isomorphic to $\mathbb Z_{p_i} \times \mathbb Z_{p_i}$.
\end{proof}

\begin{prob}[Problem 46 Fraleigh's p.135]
    Let a group $G$ be generated by $\{a_i \bigm| i \in I\}$, 
    where $I$ is some indexing set and $a_i \in G$ for all $i \in I$.
    Let $\phi,\mu$ be two homomorphisms from $G$ into a group $G'$, 
    such that $\phi(a_i) = \mu(a_i)$ for every $i \in I$. 
    Prove that $\phi = \mu$.
\end{prob}
\begin{proof}
    Let $G = \{a_i \bigm| i\in I\}$ with homomorphisms 
    $\phi, \mu : G \to G'$ which coincide on $a_i$ for all $i \in I$.
    Let $x \in G$. 
    \begin{align*}
        \phi(x) &= \phi\left(\prod_{i \in I} a_i^{k_i}\right), \: 
        k_i \in \mathbb Z \\ 
                &= \prod_{i\in I} \phi(a_i^{k_i}) \\ 
                &= \prod_{i\in I} 
                \underbrace{\phi(a_i)\cdots 
                \phi(a_i)}_{k_i \; \text{times}} \\ 
                &= \prod_{i\in I} \underbrace{\mu(a_i)\cdots 
                \mu(a_i)}_{k_i \; \text{times}} \\ 
                &= \prod_{i\in I} \mu(a_i^{k_i}) \\ 
                &= \mu\left(\prod_{i \in I}a_i^{k_i}\right) 
                = \mu(x)
    \end{align*}
    $x$ was chosen arbitrarily hence $\phi$ and $\mu$ coincide 
    on all $x \in G$. In other words, $\phi = \mu$.
\end{proof}

\begin{prob}[52 Fraleigh's p.135]
    Let $\phi : G \to G'$ be a homomorphism with kernel $H$ and 
    let $a \in G$. Prove the set equality $\{x \in G \bigm| 
    \phi(x) = \phi(a)\} = Ha$.
\end{prob}
\begin{proof}
    We begin by proving that $Ha \subseteq \{x \in G \bigm| 
    \phi(x) = \phi(a)\}$.

    Suppose $ha \in Ha$. 
    Then $h \in H$ so 
    \[
    \phi(ha) = \underbrace{\phi(h)}_{\in H} \phi(a) 
        = e' \phi(a) = \phi(a)
    \]
    Thus $ha \in \{x \in G \bigm| \phi(x) = \phi(a)\}$. As $x$ 
    was arbitrarily chosen, we conclude that 
    \[ 
        Ha\subseteq \{x \in G \bigm| \phi(x) 
        = \phi(a)\}
    \]
    Let $x \in \{x \in G \bigm| \phi(x) = \phi(a)\}$. 
    Then 
    \begin{align*}
        \phi(x) &= \phi(a) \\ 
        \phi(x) \phi(a)^{-1} &= e' \\ 
        \phi(x a^{-1}) &= e'
    \end{align*}
    hence $xa^{-1} \in H$ which implies $x \in Ha$.
    Once again $x$ was arbitrarily chosen so 
    \[
        \{x \in G \bigm| \phi(x) = \phi(a)\} \subseteq Ha
    \]
    
    Summing up we conclude that 
    \[
        Ha = \{x \in G \bigm| \phi(x) = \phi(a)\}
    \]
\end{proof}

\newpage
\section{Problems week of 16 Feb, Quotient grps, simple grps}
\begin{prob}[Problem 3 Fraleigh's p.142]
    Find the order of the given factor group: 

    \[
        (\mathbb Z_4 \times \mathbb Z_2) / \langle (2,1)\rangle
    \]
\end{prob}
\begin{proof}[Solution]
We first compute the subgroup generated by $(2,1)$:
\[
\langle (2,1) \rangle = \{(0,0),(2,1)\},
\]
since $(2,1)+(2,1)=(0,0)$.

Because $\mathbb Z_4 \times \mathbb Z_2$ is abelian, every subgroup is normal,
so the quotient group is well-defined.

The elements of the quotient
\[
(\mathbb Z_4 \times \mathbb Z_2)/\langle(2,1)\rangle
\]
are the cosets
\[
(x,y)+\langle(2,1)\rangle
= \{(x,y),(x+2,y+1)\},
\quad (x,y)\in \mathbb Z_4\times\mathbb Z_2.
\]

Each coset has exactly $2$ elements. Since
\[
|\mathbb Z_4 \times \mathbb Z_2|=8,
\]
the number of distinct cosets is
\[
\frac{8}{2}=4.
\]

Hence the quotient group has order $4$.
\end{proof}
\begin{prob}[Problem 34 Fraleigh's p.143]
    Show that if a finite group $G$ has exactly one subgroup $H$ 
    of a given order, then $H$ is a normal subgroup of $G$.
\end{prob}
\begin{proof}
    Suppose $G$ is a finite group with only one subgroup $H$ of some 
    order $k \in \mathbb N$. 
    Recall that for the conjugacy of $H$ is still a subgroup, 
    furthermore
    \[
        |gHg^{-1}| = |H| = k, \: g \in G
    \]
    However $H$ is the only subgroup with order $k$ so 
    \[
        gHg^{-1} = H, \: \forall g \in G
    \]
    which implies that $H$ is normal.

\end{proof}
\begin{prob}[Problem 35 Fraleigh's p.143]
    Show that if $H$ and $N$ are subgroups of a group $G$, and 
    $N$ is normal in $G$, then $H \cap N$ is normal in $H$. 
\end{prob}
\begin{proof}
    Let $G$ be a group with $H \leq G$ and $N \trianglelefteq G$.
    We take it for granted that $H \cap N$ is indeed a group 
    (this is not hard to show). 

    Fix some arbitrary $h \in H$.
    We must show that $h(H\cap N)h^{-1} \subseteq H \cap N$. 
    
    Let $x \in H \cap N$. As $x, h \in H$ we ceirtanly have 
    $hxh^{-1} \in H$. As $x \in N$, $N$ normal, we have that 
    for any $h \in H \subseteq G$, $hxh^{-1} \in N$. 

    Hence 
    \[
        hxh^{-1} \in H \cap N
    \]
\end{proof}

\begin{prob}[Problem 3 Fraleigh's p.151]
    Classify the groups according to the fundamental theorem of 
    finitely generated abelian groups.
    \[
        (\mathbb Z_2 \times \mathbb Z_4)/\langle(1,2)\rangle
    \]
    \[
        (\mathbb Z_4 \times \mathbb Z_8)/\langle(1,2)\rangle
    \]
    \[ 
        (\mathbb Z_4 \times \mathbb Z_4 \times \mathbb Z_8)/
        \langle (1,2,4) \rangle
    \]
\end{prob}
\begin{proof}[Solution]
    We begin with 
    $(\mathbb Z_2 \times \mathbb Z_4)/\langle(1,2)\rangle$. 

    \begin{align*}
        \langle (1,2) \rangle &= \{(0,0), (1,2)\} \\ 
        (0,0) \{(0,0), (1,2)\} &= \{(0,0), (1,2)\} = A \\ 
        (1,0) \{(0,0), (1,2)\} &= \{(1,0), (0,2)\} = B \\ 
        (0,1) \{(0,0), (1,2)\} &= \{(0,1), (1,3)\} = C \\ 
        (1,1) \{(0,0), (1,2)\} &= \{(1,1), (0,3)\} = D \\ 
        (0,2) \{(0,0), (1,2)\} &= \{(1,0), (0,2)\} = B \\ 
        (1,2) \{(0,0), (1,2)\} &= \{(1,2), (0,0)\} = A \\ 
        (0,3) \{(0,0), (1,2)\} &= \{(0,3), (1,1)\} = D \\ 
        (1,3) \{(0,0), (1,2)\} &= \{(1,3), (0,1)\} = C \\ 
    \end{align*}
    We get an abelian group with $4$ distinct elements.
    Crucially, we can see that the coset of $(0,1)$ has order 
    $4$ and is thus a generator. 
    We conclude that $(\mathbb Z_2 \times \mathbb Z_4)/
    \langle (1,2)\rangle \cong \mathbb Z_4$.

    Now we look at $(\mathbb Z_4 \times \mathbb Z_8)/
    \langle(1,2)\rangle$.

    Notice that 
    \[
        (\mathbb Z_4 \times \mathbb Z_8) = \langle a,b 
        \bigm| 4a = 0, \: 8b = 0, \: ab = ba\rangle
    \]
    so quotienting with $\langle(1,2)\rangle$ only imposes 
    the extra relation that 
    \[
        a + 2b = 0 \Leftrightarrow a = -2b
    \]
    then $\text{ord}(1,2) = \text{lcm}(4,4) = 4$ so 
    $\langle (1,2)\rangle$ has $4$ elements.  
    $|\mathbb Z_4 \times \mathbb Z_8| = 32$ so the quotient group has 
    \[
        \frac{32}{4} = 8
    \]
    elements. 
    Up to isomorphism there exist $3$ abelian groups of order 
    $8$.
    The relation imposed by $\langle (1,2)\rangle$ changes nothing 
    so we require $8b = 0$. Clearly 
    \[
        (\mathbb Z_4 \times \mathbb Z_8)/\langle (1,2)\rangle 
        \cong \mathbb Z_8
    \]

    Lastly we look at 
    \[
        (\mathbb Z_4 \times \mathbb Z_4 \times \mathbb Z_8)/
        \langle (1,2,4) \rangle
    \]
    Notice that 
    \[
        \mathbb Z_4 \times \mathbb Z_4 \times \mathbb Z_8 
        = \{a,b,c \bigm| 4a = 0, \: 4b = 0, \: 8c = 0, \; 
        ab = ba, \; ac = ca, \; bc = ca\}
    \]
    with 
    \[
        \langle (1,2,4) \rangle
    \]
    imposing $a + 2b + 4c = 0$ which then means that 
    $a = -2b -4c$. The order of $\mathbb Z_4 \times 
    \mathbb Z_4 \times \mathbb Z_8$ is $128$. The order
    of the subgroup is $\text{lcm}(4, 2, 2)$ which is $4$. 
    Thus the order of the quotient group is 
    \[
        \frac{128}{4} = 32 = 2^5
    \]
    There are precisely $7$ abelian groups unique up to isomorphism.
    $b$ has order $4$ and $c$ has $8$ so we conclude that 
    the quotient group is isomorphic to 
    \[
        \mathbb Z_4 \times \mathbb Z_8 
    \]
\end{proof}
\begin{prob}[Problem 34 Fraleigh's p.153]
    Let $G$ be finite, containing a nontrivial subgroup of index $2$
    in $G$. Show that $G$ is not simple. 
\end{prob}
\begin{proof}
    There exists a subgroup $H \leq G$ with 
    $2$ left cosets. Since $H$ is always a left coset of $H$ there 
    must be one more left coset $gH$ for any $g \not\in H$. 
    Since the number of right cosets is the same we 
    have right cosets $H$ and $Hg$. There are only $2$ cosets 
    so the non-$H$ cosets must be the same 
    \[
        gH = Hg
    \]
    Thus left and right cosets coincide. Therefore 
    \[
        H \trianglelefteq G
    \]
    $H$ is nontrivial, and crucially $H \neq G$ (since the index 
    is $2$). Hence $G$ has a nontrivial proper subgroup. 
    In other words, $G$ is not simple. 
\end{proof}


\end{document}
