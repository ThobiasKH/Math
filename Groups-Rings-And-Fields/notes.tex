\documentclass[11pt]{article}
\usepackage[a4paper,margin=1in]{geometry}
\usepackage{fourier} % Fourier font
\usepackage{xcolor}
\usepackage{tikz}
\usepackage[most]{tcolorbox}
\usepackage{amsthm, amsmath, amssymb}
\usepackage{enumitem}
\usepackage{hyperref}
\usepackage[nameinlink,noabbrev]{cleveref}
\usepackage{titling} 

% Dark mode colors
\definecolor{bgcolor}{HTML}{FFFFFF}
\definecolor{textcolor}{HTML}{000000}
\definecolor{defcolor}{HTML}{E86873}
\definecolor{thmcolor}{HTML}{0A9396}
\definecolor{lemcolor}{HTML}{94D2BD}
\definecolor{corcolor}{HTML}{9B4AF7}
\definecolor{probcolor}{HTML}{EE9B00}
\definecolor{excolor}{HTML}{21E933}

% Background and text color
\pagecolor{bgcolor}
\color{textcolor}

% No paragraph indentation
\setlength{\parindent}{0pt}
\setlength{\parskip}{0.7em}

% Theorem box styles
\tcbset{
  enhanced,
  colback=bgcolor,
  colframe=thmcolor,
  coltext=white,
  coltitle=white,
  fonttitle=\bfseries,
  boxrule=0.7pt,
  left=1em,
  right=1em,
  top=0.7em,
  bottom=0.7em,
  before skip=10pt,
  after skip=10pt,
}

% Theorem environments with colored boxes
\newtcbtheorem[number within=section]{thm}{Theorem}{
  colframe=thmcolor, colback=thmcolor!15!bgcolor
}{thm} % The 'thm' here is the *prefix* for the label

\newtcbtheorem[number within=section]{defn}{Definition}{
  colframe=defcolor, colback=defcolor!15!bgcolor
}{def} % The 'def' here is the *prefix* for the label

\newtcbtheorem[number within=section]{lem}{Lemma}{
  colframe=lemcolor, colback=lemcolor!15!bgcolor
}{lem}

\newtcbtheorem[number within=section]{cor}{Corollary}{
  colframe=corcolor, colback=corcolor!15!bgcolor
}{cor}

\newtcbtheorem[number within=section]{prob}{Problem}{
  colframe=probcolor, colback=probcolor!15!bgcolor
}{prob}

\newtcbtheorem[number within=section]{ex}{Example}{
  colframe=excolor, colback=excolor!15!bgcolor
}{ex}

% Proof environment 
\renewenvironment{proof}[1][\proofname]{%
  \par\pushQED{\qed}\normalfont\topsep6pt \trivlist
  \item[\hskip\labelsep\itshape #1.]\ignorespaces
}{%
  \popQED\endtrivlist\addvspace{6pt}
}

% Cleveref name formats for tcolorbox environments
\crefname{thm}{theorem}{theorems}
\Crefname{thm}{Theorem}{Theorems}

\crefname{def}{definition}{definitions}
\Crefname{def}{Definition}{Definitions}

\crefname{lem}{lemma}{lemmas}
\Crefname{lem}{Lemma}{Lemmas}

\crefname{cor}{corollary}{corollaries}
\Crefname{cor}{Corollary}{Corollaries}

\crefname{prob}{problem}{problems}
\Crefname{prob}{Problem}{Problems}

\crefname{ex}{example}{examples}
\Crefname{ex}{Example}{Examples}


\title{\huge{Groups, Rings and Fields}}
\author{\LARGE{Thobias Høivik}}
\date{\Large{Spring 2026}}

\begin{document}
\maketitle

\newpage
\tableofcontents

\newpage
\section{Groups}

\begin{defn}[Group]
\label{defn:group}
A group is a set $S$ together with a binary operation 
$\circ$ such that the following properties hold: 

\begin{enumerate}
    \item Identity: There exists an element $e \in S$, satisfying
        $$ 
            e \circ a = a \circ e = a 
        $$ 
        for every $a \in S$.

    \item Inverses: For every $a \in S$ there exists $b \in S$ such 
        that 
        $$ 
            a \circ b = b \circ a = e, \text{ the identity element}
        $$ 
        We usually denote this element as $a^{-1}$ or $-a$, depending
        on context.

    \item Associativity: For any $a,b,c \in S$, we require 
        $$ 
            a \circ (b \circ c) = (a \circ b) \circ c
        $$ 
\end{enumerate}

A group is then the tupple $(S,\circ)$. We will often just write 
the set to refer to the group, e.g. refering to the group 
$(\mathbb Z, +)$ as just $\mathbb Z$.
    
\end{defn}

\subsection{Basic Examples}

\begin{ex}[Integers under addition]
The set of integers $\mathbb{Z}$ with the operation $+$ forms a group:
\begin{itemize}
    \item Identity: $0$ since $0 + n = n + 0 = n$ for all $n \in \mathbb{Z}$.
    \item Inverses: For $n \in \mathbb{Z}$, the inverse is $-n$.
    \item Associativity: Addition is associative.
\end{itemize}
Hence $(\mathbb{Z}, +)$ is a group.
\end{ex}

\begin{ex}[Non-example: Natural numbers under addition]
The set $\mathbb{N}$ under $+$ is \emph{not} a group since
there is no inverse for $n > 0$.
\end{ex}

\subsection{Basic Properties}

\begin{thm}[Uniqueness of identity]
\label{thm:identity-unique}
The identity element in a group is unique.
\end{thm}

\begin{proof}
Suppose $e$ and $e'$ are both identities. Then
$$
e = e \circ e' = e',
$$
so the identity is unique.
\end{proof}

\begin{thm}[Uniqueness of inverses]
\label{thm:inverse-unique}
Each element in a group has a unique inverse.
\end{thm}

\begin{proof}
Suppose $b$ and $c$ are inverses of $a$. Then
$$
b = b \circ e = b \circ (a \circ c) = (b \circ a) \circ c = e \circ c = c.
$$
\end{proof}

\subsection{Example Problem}

\begin{prob}
Determine whether the set 
$$
G = \{1, -1, i, -i\} \subset \mathbb{C}
$$ 
with multiplication is a group.
\end{prob}

\begin{proof}[Solution]
We check the group axioms:
\begin{enumerate}
    \item \textbf{Closure:} Multiplying any two elements of $G$ yields another element in $G$. True.
    \item \textbf{Identity:} The element $1$ acts as identity. True.
    \item \textbf{Inverses:} Each element has an inverse in $G$: $1^{-1}=1$, $(-1)^{-1}=-1$, $i^{-1}=-i$, $(-i)^{-1}=i$. True.
    \item \textbf{Associativity:} Multiplication of complex numbers is associative. True.
\end{enumerate}
Hence $(G, \cdot)$ is a group, in fact it is an abelian group 
which we shall describe below in \cref{defn:abelian_group}. 
\end{proof}

\newpage
\subsection{Abelian Groups}

\begin{defn}[Abelian Group]
\label{defn:abelian_group}
A group $(G, \circ)$ is \emph{abelian} (or commutative) if
$$
a \circ b = b \circ a \quad \forall a,b \in G.
$$
\end{defn}

\begin{ex}
The group $(\mathbb{Z}, +)$ is abelian because $m+n = n+m$.
\end{ex}

\newpage 
\section{Important Groups}
First let's define some often used notation. 

For $g \in G$, define: 
\begin{itemize}
    \item $g^n = gg\ldots g(n > 0)$ 
    \item $g^0 = e$
    \item $g^{-n} = (g^{-1})^n$
\end{itemize} 
and recognize the following identities (which are provable 
by induction): 
$$ 
    g^m g^n = g^{m+n}, \quad (g^m)^n = g^{mn}
$$ 

\begin{defn}[Roots of Unity]
    \label{defn:roots_of_unity}
    $$ 
        U_n = \{e^{2\pi ik/n} : 0 \leq k < n\}
    $$ 
    are called the n-th roots of unity.
\end{defn}

\begin{defn}
    An subset $\{g_1,\ldots,g_k\} \subseteq G$ is a generating set 
    if every $g \in G$ can be expressed as 
    $$ 
        g_1^{n_1} \cdots g_k^{n_k}
    $$ 

    If a singleton $g'$ alone generates a group then it is called 
    a generator of $G$ and $G$ is said to be cyclic. 
\end{defn}

The $n$-th roots of unity are finite, with exactly $n$ elements. 
They are cyclic, with generator $e^{2i\pi/n}$.

\begin{defn}[Symmetric Group]
    \label{defn:symmetric_group}
    The symmetric group $S_n$ is the set of all permutations of 
    $n$ elements under composition. 
    In other words it is the set of all $\sigma : [n] \to [n]$ with 
    composition as its operation. 
\end{defn}

The symmetric group is non-abelian for $n \geq 3$. The identity 
corresponds to the identity permutation (doing nothing) and inverses 
are undoing a permutation. 
We have the following notation for permutations in the symmetric Group
(assume $n = 3$ for this example): 
$$ 
    \begin{pmatrix}
        1 & 2 & 3 \\ 
        2 & 3 & 1
    \end{pmatrix}
$$ 
corresponds to $1 \mapsto 2, 2 \mapsto 3, 3 \mapsto 1$. 

Another way to write the same permutation is: 
$$ 
    \left(1 \: 2 \: 3\right)
$$ 

\newpage 
\section{Subgroups}
\begin{defn}[Subgroup]
    \label{defn:subgroup}
    Let $G$ be a group under $\star$.
    A subgroup $H \leq G$ is a subset $H \subseteq G$ that is itself 
    a group under $\star$.
\end{defn}

Having to mechanically check if $H \subseteq G$ satisfies all the 
required axioms can be a bit tedious so next we introduce a 
powerful theorem which let's us easily determine whether some 
subset is a subgroup or not. 

\begin{thm}[Subgroup Test]
    \label{thm:subgroup_test}
    A nonempty subset $H \subseteq G$ is a subgroup if and only 
    if 
    $$ 
        \forall x,y \in H, \; xy^{-1} \in H.
    $$ 
\end{thm}
\begin{proof}[Proof Idea]
    If $H$ is a subgroup the required conditions follows. 

    Conversely: 
    \begin{itemize}
        \item Nonempty so some $h \in H$.
        \item $hh^{-1} = e \in H$. 
        \item Closure under $xy^{-1}$ will give inverses and 
            closure. 
    \end{itemize}
\end{proof}

We get some nice corollaries from this, as well as a nice and tidy 
way to prove whether or not a subset is a subgroup. 

\begin{cor}
    The intersection of any collection of subgroups is a subgroup. 

    Given any subset $S \subseteq G$, there is a smallest subgroup 
    containing it. 
\end{cor}

\begin{defn}[Cyclic Subgroups]
    Let $G$ be a group. 

    For any $g \in G$, define: 
    $$ 
        \langle g \rangle  = \{g^n : n \in \mathbb Z\}.
    $$ 

    $\langle g \rangle$ is called the cyclic subgroup generated 
    by $g$. 
\end{defn}

It is very straightforward to show that the cyclic subgroup generated 
by some $g \in G$ is indeed a subgroup with the use of 
\cref{thm:subgroup_test}.

On page \pageref{defn:roots_of_unity} we say that a group is cyclic 
if it has a signle element which generates it, but now we can simply
say that a group $G$ is cyclic if $G = \langle g \rangle$ for some 
$g \in G$.

\begin{thm}
    Suppose a group $G$ is cyclic, i.e. $G = \langle g \rangle$ 
    for some $g \in G$. 

    Then it is isomorphic to one of the following: 
    \begin{itemize}
        \item $(\mathbb Z, +)$, or 
        \item $(\mathbb Z/n\mathbb Z, +)$ for some $n \geq 1$. 
    \end{itemize}
\end{thm}

\subsection{Structure of Cyclic Subgroups}
\begin{defn}[Order]
    The order of an element $g \in G$ is 
    $$ 
        \text{ord}(g) = |\langle g \rangle|
    $$ 
\end{defn}

Furthermore, if $\text{ord}(g) = n < \infty$, then $g^n = e$ and 
$n$ is minimal, otherwise we say $g$ has infinite order.  

If $\text{ord}(g) = n$, then: 
$$ 
    \langle g \rangle = \{e, g, g^2, \ldots, g^{n-1}\}
$$ 

Moreover: 
\begin{itemize}
    \item $g^k = e \Leftrightarrow n \mid k$
    \item $\langle g^k \rangle = \langle g \rangle \Leftrightarrow 
        \gcd(k,n) = 1$
\end{itemize}

\begin{thm}
    Every subgroup of a cyclic group is cyclic. 

    More precisely: 
    \begin{itemize}
        \item Subgroups of $\mathbb Z$ are exactly $n\mathbb Z$.
        \item Subgroups of $\mathbb Z/n\mathbb Z$ correspond 
            to the divisors of $n$.
    \end{itemize}
\end{thm}

With all of this machinery we can approach group theory with geometric
and algebraic intuition. Cyclic groups are like repeated motion, 
finite cyclic groups corresponding to rotations by rational angles 
and infinite ones corresponding to translations. 

\newpage 
\section{Cosets and Lagrange's Theorem}
\begin{defn}[Left and Right Cosets]
    Let $H \leq G$, and let $g \in G$. 

    The left- and right cosets are: 
    $$ 
        gH = \{gh : h \in H\} \quad Hg = \{hg : h \in H\}
    $$ 
    respectively.
\end{defn}

A key intuition is that a coset is like a copy of $H$ shifted 
by $g$. 
We may also recognize that the cosets are the same size as $H$ by 
identifying $h \mapsto gh$ to be a bijection and so forth. 

The next natural question is when are two cosets equal? 
Like with showing when a subset is a subgroup we have a little trick 
for this problem. 

\begin{thm}
    Let $G$ be a group. 

    For $g_1, g_2 \in G$, the following are equivalent: 
    \begin{align*}
        g_1 H &= g_2 H \\ 
              &\Updownarrow \\ 
        g_2^{-1}g_1 &\in H \\ 
                    &\Updownarrow \\ 
        g_1 &\in g_2 H 
    \end{align*}
\end{thm}

\begin{thm}
    \label{thm:coset_partition}
    The set of all left cosets of $H \leq G$ forms a 
    partition of $G$. 

    That is, 
    \begin{itemize}
        \item Every element of $G$ is in exactly on coset. 
        \item Two cosets are either the same or disjoint. 
    \end{itemize}
\end{thm}

\begin{defn}
    Let $G$ be a group with $H \leq G$. 

    The index of $H$ in $G$, written $[G:H]$, is the number of 
    left cosets of $H$. 
\end{defn}

If $G$ is finite we get $|G| = [G:H] \cdot |H|$.

\newpage 
\begin{thm}[Lagrange's Theorem]
    \label{thm:lagrange} 
    Let $G$ be a group with $H \leq G$. 

    Then: 
    $$ 
        |H| \bigm\vert |G|
    $$ 
\end{thm}
\begin{proof}
    Let $G$ be a group with $H \leq G$. 

    Recall that by \cref{thm:coset_partition} we have that the 
    set of left cosets of $H$, $g_1 H, g_2 H, \ldots, g_k H$ 
    (this set is finite since $G$ is finite) are 
    pairwise disjoint, satisfying 
    $$ 
        G = \bigcup_{i=1}^k g_i H
    $$ 
    meaning 
    $$ 
        |G| = \displaystyle\sum_{i=1}^{k}|g_i H|
    $$ 

    Recall also that $|H| = |g_i H|, \, \forall i \in [k]$. 
    Thus 
    $$ 
        |G| = \displaystyle\sum_{i=1}^{k}|H| = k|H|
    $$ 

    In other words $|G|$ is $|H|$ times some integer $k$, therefore 
    $|H|$ divides $|G|$. 
\end{proof}


\Cref{thm:lagrange} has the following immediate consequences: 
\begin{itemize}
    \item The order of any element divides $|G|$. 
    \item $g^{|G|} = e$ for every $g \in G$. 
\end{itemize}

Beware that it does not follow from Lagrange's Theorem that there 
exists a subgroup with the order of a divisor of $|G|$ for every 
divisor of $G$. The alternating group of order $12$, $A_{4}$, 
has no subgroup of order $6$ for example.

\begin{prop}
    A group of order $p$ where $p$ is prime, is cyclic. 
\end{prop}
\begin{proof}[Proof Sketch]
    Let $G$ be a group with $|G| = p$ (prime). 

    Let $g \in G \setminus \{1\}$. Consider it's generated subgroup 
    $\langle g \rangle$. By Lagrange's Theorem the order of this 
    subgroup divides $p$ so $|\langle g \rangle|$ is $1$ or $p$, but 
    it can't be $1$ as $g \neq 1$ so it's order is $p$, i.e. 
    $g$ generates the entirety of $G$. 
\end{proof}

\subsection{Problems relating to subgroups}
\begin{prob}
    Let $G$ be a group and $g \in G$ with  $\text{ord}(g) = 12$. 

    \begin{itemize}
        \item List all distinct subgroups of $\langle g \rangle$.
        \item For which integers $k$ does $g^k$ generate $\langle g \rangle$?
    \end{itemize}
\end{prob}
\begin{proof}[Solution]
    Recalling that $\langle g \rangle = \{g^k : k \in \mathbb Z\}$, 
    we want to find all the distinct subgroups of $\langle g \rangle$. 
    In other words, which subsets of $\langle g \rangle$ are also 
    groups with respect to the operation of $G$? 

    $\text{ord}(g) = |\langle g \rangle| = 12$ hence $\langle g \rangle = \{g^0, g^1, \ldots, g^{11}\}$.
    We know that there are two trivial subgroups of $\langle g \rangle$, namely $\{g^0\}$ and $\langle g \rangle$. 
    Otherwise, we know that every cyclic subgroub of a cyclic group (which $\langle g\rangle$ is) is itself 
    cyclic, so we can find the subgroups by looking at the generated groups of each element 
    of $\langle g \rangle$. Otherwise, we could use the fact that 
    $\langle g\rangle $ is isomorphic to $\mathbb Z/12\mathbb Z$ together with realizing that the number 
    of subgroups is then the same as the number of positive divisors of $12$. 

    Using the first approach we get: 
    \begin{align*}
        \langle g^2 \rangle &= \{g^n : n = 2k, \: k \in \mathbb Z\} \\ 
                            &= \{e, g^2, \ldots, g^10\} = \langle g^10 \rangle \\ 
        \langle g^3 \rangle &= \{e, g^3, g^6, g^9\} = \langle g^9 \rangle \\ 
        \langle g^4 \rangle &= \{e, g^4, g^8\} = \langle g^8 \rangle \\ 
        \langle g^6 \rangle &= \{e, g^6\} \\ 
    \end{align*}
    
    Those integers which are coprime to $12$, like $1,5,7,11$, will generate $\langle g \rangle$. 
    Thus we have $6$ distinct subgroups, exactly the same as the number of 
    positive divisors of $12$, as expected. 
\end{proof}

\begin{prob}
    Let $G$ be any group $H = \langle g \rangle$ where $\text{ord}(g) < \infty$. 

    Prove that if $g^m \in H$, then $\langle g^m \rangle \leq H$. 
\end{prob}
\begin{proof}
    Assume, for contradiction, that $g^m \in H$, but $\langle g^m \rangle$ 
    is not a subgroup of $H$. 

    Let $s = m \mod \text{ord}(g)$. 
    Then $\langle g^m \rangle = \{g^{sk} : k \in \mathbb Z\} \neq \emptyset$.  
    It is clear that this is a subset of $\langle g \rangle = H$, since 
    $\langle g \rangle$ is closed under the group operation and 
    $g^m \in H$. 
    Then the only way our assumption is true is if 
    there exists some $x,y \in \langle g^m \rangle$ such that 
    $xy^{-1} \not\in \langle g^m \rangle$. 

    Every element $x,y \in \langle g^m \rangle$ is of the form 
    $x = g^{sn}, \, y = g^{st}$ for some $n,l \in \mathbb N$. 
    By assumption, 
    $xy^{-1} = g^{s(n-t)} \not\in \langle g^m \rangle$ 
    which would necessarily mean that $n-t \not \in \mathbb Z$, 
    impossible. 
\end{proof}
The above argument can be made more rigorous 
by considering the isomorphism between $\langle g^m \rangle$ and 
$\mathbb Z/s\mathbb Z$, or by proving the claim 
directly (I am an idiot, and this was a bad approach). 

\newpage 
\section{Normal subgroups}
\begin{defn}[Normal subgroup]
    \label{defn:normal_subgroup}
    A subgroup $H \leq G$ is normal, written $H \trianglelefteq G$, if:
    $$ 
        gH = Hg, \: \forall g \in G
    $$ 
    with the equivalent characterization:
    $$ 
        H \trianglelefteq G \Leftrightarrow gHg^{-1} = H, \: \forall g \in G
    $$ 
\end{defn}

A few things to tick off immediately; 
\begin{enumerate}
    \item Every subgroup of an abelian group is normal. 
    \item \{e\} and $G$ are always normal. 
    \item The alternating group $A_n \trianglelefteq S_n$. 
    \item Subgroups of index $2$ are always normal. 
\end{enumerate}

If you consider the map (conjugation)
$$ 
    x \mapsto gxg^{-1},
$$ 
which constitutes an automorphism of $G$ we can say 
the following. 

Normal subgroups are precisely those subgroups which are 
fixed under every conjugation. 

\begin{defn}[Quotient Group]
    \label{defn:quotient_group}
    If $H \trianglelefteq G$, define: 
    $$ 
        G/H := \{gH : g \in G\}
    $$ 
    with an operation we'll call multiplication by: 
    $$ 
        (gH)(kH) = (gk)H
    $$ 
\end{defn}

\begin{thm}
    The quotient group with multiplication, as defined above, 
    is a well-defined group. 
\end{thm}

\newpage 
\begin{defn}
    We define the \textbf{canonical projection} as 
    $$ 
        \pi : G \to G/H, \: \pi(g) = gH.
    $$ 
\end{defn}
Notably, the projection $\pi$ constitutes 
a homomorphism with kernel $\ker \pi = H$.
Notice that this claim of the kernel being $H$ itself 
should be immediately obvious as $hH$ for some 
$h \in H$ is itself $H$ as it's closed. 
\end{document}
