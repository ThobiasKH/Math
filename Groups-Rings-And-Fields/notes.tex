\documentclass[11pt]{article}
\usepackage[a4paper,margin=1in]{geometry}
\usepackage{fourier} % Fourier font
\usepackage{xcolor}
\usepackage{tikz}
\usepackage[most]{tcolorbox}
\usepackage{amsthm, amsmath, amssymb}
\usepackage{enumitem}
\usepackage{hyperref}
\usepackage[nameinlink,noabbrev]{cleveref}
\usepackage{titling} 

% Dark mode colors
\definecolor{bgcolor}{HTML}{FFFFFF}
\definecolor{textcolor}{HTML}{000000}
\definecolor{defcolor}{HTML}{E86873}
\definecolor{thmcolor}{HTML}{0A9396}
\definecolor{lemcolor}{HTML}{94D2BD}
\definecolor{corcolor}{HTML}{9B4AF7}
\definecolor{probcolor}{HTML}{EE9B00}
\definecolor{excolor}{HTML}{21E933}

% Background and text color
\pagecolor{bgcolor}
\color{textcolor}

% No paragraph indentation
\setlength{\parindent}{0pt}
\setlength{\parskip}{0.7em}

% Theorem box styles
\tcbset{
  enhanced,
  colback=bgcolor,
  colframe=thmcolor,
  coltext=white,
  coltitle=white,
  fonttitle=\bfseries,
  boxrule=0.7pt,
  left=1em,
  right=1em,
  top=0.7em,
  bottom=0.7em,
  before skip=10pt,
  after skip=10pt,
}

% Theorem environments with colored boxes
\newtcbtheorem[number within=section]{thm}{Theorem}{
  colframe=thmcolor, colback=thmcolor!15!bgcolor
}{thm} % The 'thm' here is the *prefix* for the label

\newtcbtheorem[number within=section]{defn}{Definition}{
  colframe=defcolor, colback=defcolor!15!bgcolor
}{def} % The 'def' here is the *prefix* for the label

\newtcbtheorem[number within=section]{lem}{Lemma}{
  colframe=lemcolor, colback=lemcolor!15!bgcolor
}{lem}

\newtcbtheorem[number within=section]{cor}{Corollary}{
  colframe=corcolor, colback=corcolor!15!bgcolor
}{cor}

\newtcbtheorem[number within=section]{prob}{Problem}{
  colframe=probcolor, colback=probcolor!15!bgcolor
}{prob}

\newtcbtheorem[number within=section]{ex}{Example}{
  colframe=excolor, colback=excolor!15!bgcolor
}{ex}

% Proof environment 
\renewenvironment{proof}[1][\proofname]{%
  \par\pushQED{\qed}\normalfont\topsep6pt \trivlist
  \item[\hskip\labelsep\itshape #1.]\ignorespaces
}{%
  \popQED\endtrivlist\addvspace{6pt}
}

% Cleveref name formats for tcolorbox environments
\crefname{thm}{theorem}{theorems}
\Crefname{thm}{Theorem}{Theorems}

\crefname{def}{definition}{definitions}
\Crefname{def}{Definition}{Definitions}

\crefname{lem}{lemma}{lemmas}
\Crefname{lem}{Lemma}{Lemmas}

\crefname{cor}{corollary}{corollaries}
\Crefname{cor}{Corollary}{Corollaries}

\crefname{prob}{problem}{problems}
\Crefname{prob}{Problem}{Problems}

\crefname{ex}{example}{examples}
\Crefname{ex}{Example}{Examples}


\title{\huge{Groups, Rings and Fields}}
\author{\LARGE{Thobias Høivik}}
\date{\Large{Spring 2026}}

\begin{document}
\maketitle

\newpage
\tableofcontents

\newpage
\section{Groups}

\begin{defn}[Group]
\label{defn:group}
A group is a set $S$ together with a binary operation 
$\circ$ such that the following properties hold: 

\begin{enumerate}
    \item Identity: There exists an element $e \in S$, satisfying
        $$ 
            e \circ a = a \circ e = a 
        $$ 
        for every $a \in S$.

    \item Inverses: For every $a \in S$ there exists $b \in S$ such 
        that 
        $$ 
            a \circ b = b \circ a = e, \text{ the identity element}
        $$ 
        We usually denote this element as $a^{-1}$ or $-a$, depending
        on context.

    \item Associativity: For any $a,b,c \in S$, we require 
        $$ 
            a \circ (b \circ c) = (a \circ b) \circ c
        $$ 
\end{enumerate}

A group is then the tupple $(S,\circ)$. We will often just write 
the set to refer to the group, e.g. refering to the group 
$(\mathbb Z, +)$ as just $\mathbb Z$.
    
\end{defn}

\subsection{Basic Examples}

\begin{ex}[Integers under addition]
The set of integers $\mathbb{Z}$ with the operation $+$ forms a group:
\begin{itemize}
    \item Identity: $0$ since $0 + n = n + 0 = n$ for all $n \in \mathbb{Z}$.
    \item Inverses: For $n \in \mathbb{Z}$, the inverse is $-n$.
    \item Associativity: Addition is associative.
\end{itemize}
Hence $(\mathbb{Z}, +)$ is a group.
\end{ex}

\begin{ex}[Non-example: Natural numbers under addition]
The set $\mathbb{N}$ under $+$ is \emph{not} a group since
there is no inverse for $n > 0$.
\end{ex}

\subsection{Basic Properties}

\begin{thm}[Uniqueness of identity]
\label{thm:identity-unique}
The identity element in a group is unique.
\end{thm}

\begin{proof}
Suppose $e$ and $e'$ are both identities. Then
$$
e = e \circ e' = e',
$$
so the identity is unique.
\end{proof}

\begin{thm}[Uniqueness of inverses]
\label{thm:inverse-unique}
Each element in a group has a unique inverse.
\end{thm}

\begin{proof}
Suppose $b$ and $c$ are inverses of $a$. Then
$$
b = b \circ e = b \circ (a \circ c) = (b \circ a) \circ c = e \circ c = c.
$$
\end{proof}

\subsection{Example Problem}

\begin{prob}
Determine whether the set 
$$
G = \{1, -1, i, -i\} \subset \mathbb{C}
$$ 
with multiplication is a group.
\end{prob}

\begin{proof}[Solution]
We check the group axioms:
\begin{enumerate}
    \item \textbf{Closure:} Multiplying any two elements of $G$ yields another element in $G$. True.
    \item \textbf{Identity:} The element $1$ acts as identity. True.
    \item \textbf{Inverses:} Each element has an inverse in $G$: $1^{-1}=1$, $(-1)^{-1}=-1$, $i^{-1}=-i$, $(-i)^{-1}=i$. True.
    \item \textbf{Associativity:} Multiplication of complex numbers is associative. True.
\end{enumerate}
Hence $(G, \cdot)$ is a group, in fact it is an abelian group 
which we shall describe below in \cref{defn:abelian_group}. 
\end{proof}

\newpage
\subsection{Abelian Groups}

\begin{defn}[Abelian Group]
\label{defn:abelian_group}
A group $(G, \circ)$ is \emph{abelian} (or commutative) if
$$
a \circ b = b \circ a \quad \forall a,b \in G.
$$
\end{defn}

\begin{ex}
The group $(\mathbb{Z}, +)$ is abelian because $m+n = n+m$.
\end{ex}



\end{document}
