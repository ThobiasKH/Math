\documentclass[12pt]{article}
\usepackage{amsmath, amssymb, amsfonts, geometry}

% Page Setup
\geometry{top=1in, bottom=1in, left=1in, right=1in}

\title{Lecture Notes: Abstract Algebra (Course By: Alvaro Lozano-Robledo)}
\author{Thobias K. Høivik}
\date{\today}

\begin{document}

\maketitle
\section*{Comparing Groups}
Our long term goal here is to classify/characterize all groups, a tremendous task. 
We might start by doing this for finit groups or, if we let \(n \ge 1\), characterize 
all groups G of order n. 

\subsection*{Example} 
Consider the group \(<Z/4Z, +>\), and it's cayley table
\[
\begin{array}{c|cccc}
+ & 0 & 1 & 2 & 3 \\
\hline
0 & 0 & 1 & 2 & 3 \\
1 & 1 & 2 & 3 & 0 \\
2 & 2 & 3 & 0 & 1 \\
3 & 3 & 0 & 1 & 2 \\
\end{array}
\]
vs. \(<G, \star>\) with 
\[
\begin{array}{c|cccc}
\star & a & b & c & d \\
\hline
a & c & d & b & a \\
b & d & c & a & b \\
c & b & a & d & c \\
d & a & b & c & d \\
\end{array}
\]

\noindent 
The question then becomes: is \(G \cong Z/4Z\)? 

\noindent 
\textbf{Answer:} Yes it is true if you rearrange 
\[
\begin{cases} 
a = 1 \\
b = 3 \\ 
c = 2 \\ 
d = 0
\end{cases}
\]
so the table may be written as 
\[
\begin{array}{c|cccc}
\star & d & a & c & b \\
\hline
d &  &  &  &  \\
a &  &  &  &  \\
c &  &  &  &  \\
b &  &  &  &  \\
\end{array}
\]
to make it match \(Z/4Z\).

\subsection*{Example} 
What if we want to find finite groups \(G, |G| = 8\)? 
\begin{enumerate}
    \item \(Z/8Z = \{0,1,2,3,4,5,6,7 \text{ mod } 8\}\) 
    \item \(Z/4Z \times Z/2Z = \{(a \text{ mod } 4, b \text{ mod } 2)\}\)
    \item \(Z/2Z \times Z/2Z \times Z/2Z =
    \{(a \text{ mod } 2),(b \text{ mod } 2),(c \text{ mod } 2)\},\)
    \item \(D_4 = \text{Sym}(\square) = \{id, r, r^2, r^3, s, rs, r^2s, r^3s\}\)
    \item \(Q_8 = \{\pm 1, \pm i, \pm j, \pm k\}\)
\end{enumerate}

\noindent 
\textbf{Q1:} Are these different, 
and \textbf{Q2:} are these all the unique groups of order 8?
The answer to both of these as it turns out is yes; 
There are only 5 unique structure on sets with 8 elements which 
satisfy the group axioms. Any "new" structure of size 8 with the group axioms fulfilled
would be one of these 5 groups in disguise. 

\noindent 
Now, how would we go about showing that these are indeed different? 
Well, the first three groups are abelian, while the last two are non-abelian. 
This means that the first three are distinct from the last two. 
\(\#1 = Z/8Z\) is cyclic, but \#2 and \#3 are not cyclic, 
making \#1 distinct from the other abelian groups. 
We can continue comparing the qualities of groups to see if they are indeed distinct. 
For instance for the \#4 and \#5 we can look at the order of the elements 
and observe that \(Q_8\) has 1 element with order 2 while \(D_4\) has multiple.  

\end{document}
