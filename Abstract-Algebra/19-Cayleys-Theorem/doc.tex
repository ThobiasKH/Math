\documentclass[12pt]{article}
\usepackage{amsmath, amssymb, amsthm, amsfonts, geometry}
\newtheorem{theorem}{Theorem}
\newtheorem{lemma}{Lemma}
\newtheorem{proposition}{Proposition}
\newtheorem{corollary}{Corollary}
\newtheorem{definition}{Definition}

\geometry{top=1in, bottom=1in, left=1in, right=1in}

\title{Lecture Notes: Abstract Algebra | Cayley's Theorem (Course By: Alvaro Lozano-Robledo)}
\author{Thobias K. Høivik}
\date{\today}

\begin{document}

\maketitle
\begin{theorem}[Cayley's Theorem]\label{thm:cayley}
    Every finite group is isomorphic to a subgroup of a permutation group.
\end{theorem}

\subsection*{Example}
\(\mathbb Z/_3\mathbb Z\)
\[
\begin{array}{c|ccc}
+ & 0 & 1 & 2 \\
\hline
0 & 0 & 1 & 2 \\
1 & 1 & 2 & 0 \\
2 & 2 & 0 & 1 \\
\end{array}
\]
Notice how each row is a permutation of \(\{0,1,2\}\), 
namely the permutations:
\begin{gather*}
    \binom{0 \quad 1 \quad 2}{0 \quad 1 \quad 2} \\
    \binom{0 \quad 1 \quad 2}{1 \quad 2 \quad 0} \\
    \binom{0 \quad 1 \quad 2}{2 \quad 0 \quad 1} \\
\end{gather*}
or alternatively,

\begin{gather*}
    (0) \\ 
    (1,2,3) \\ 
    (1,3,2)
\end{gather*}
written in single-line notation.
Consider then the isomorphism 
\[
    \phi : \mathbb Z /_3 \mathbb Z 
    \to \{(1),(123),(132)\} = <(123)>\subseteq S_3
\]
\begin{gather*}
    0 \to (1) \\
    1 \to (123) \\ 
    2 \to (132) \\ 
    \phi (n \text{ mod } 3) = (123)^n \\ 
    \phi(n + m) = (123)^{n+m} = (123)^n (123)^m = \phi(n) \phi(m) \\ 
    \mathbb Z /_3 \mathbb Z \cong <(123)>
\end{gather*}

\begin{lemma} \label{lem:lambda}
    Let G be a finite group and \(g \in G\). 
    Let \(\lambda_g: G \to G, \quad \lambda_g(a)=a\star g\).
    Then \(\lambda_g\) is a bijection.
    
\end{lemma}
\begin{proof}
    Since G is closed \(\lambda_g\) is well defined: 
    \[ 
        a\in G \land g \in G \Rightarrow a\star g \in G \land g\star a \in G
    \]
    Suppose we have  
    \[
        \lambda_g(a) = \lambda_g(b), \quad a,b \in G
    \]
    \begin{gather*}
        g\star a = g \star b \\ 
        a = b \\ \because \forall g \in G : \exists g^{-1} \in G 
    \end{gather*} 
    hence \(\lambda_g\) is injective. Now to show surjectivity:
    \begin{gather*}
        \lambda_g(a) = b \\ 
        g \star a = b \\ 
        a = g^{-1}\star b \in G
    \end{gather*}
    obviously, lol (tired).
\end{proof}

\subsection*{Proving the Theorem}
Now to prove Cayley's Theorem \ref{thm:cayley}.
\begin{proof}[Proof of Cayley's Theorem]
    Let G be a finite group, G = \(\{g_1, g_2, \dots, g_n\}\). 
    For \(g \in G\), let \(\lambda_g:G \to G, \quad \lambda_g(a) = g\star a\), 
    \(\lambda_g\) is a bijection as shown in \ref{lem:lambda}, making 
    \(\lambda \in Sym(G)\) a permutation of G. 
    \(Sym(G) = Sym(\{g_1, g_2, \dots, g_n\}) = Sym(\{1,2,\dots, n\}) = S_n\).
    Let \(\overline{G} = \{\lambda_g : g\in  G\} \subseteq S_n\). 

    \noindent
    \textbf{Claim: } \(\overline{G} = \{\lambda_g : g\in G\}\) is a group. 
    \(<\overline{G}, \circ>\) is closed:
    \begin{gather*}
        \lambda_g \circ \lambda_{g^\star}(a) = \lambda_g(\lambda_{g^\star}(a)) = gg'a \\
        = \lambda{gg^\star} \in \overline{G} \quad \because gg^\star \in G
    \end{gather*}
    \(<\overline{G},\circ>\) is associative
    \begin{gather*}
        (\lambda_{x} \circ \lambda_{y}) \circ \lambda_{z} \\ 
        = \lambda_{xyz} \\ 
        = \lambda_{x} \circ (\lambda_{y} \circ \lambda_{z}) \\ 
    \end{gather*}
    \(<\overline{G},\circ>\) has identity  
    \[ 
        \lambda_e(a) = e \star a = a 
    \]
    \(<\overline{G}, \circ>\) has inverse 
    \[ 
        \lambda_g \circ \lambda_{g^{-1}} = \lambda_e
    \]
    Moreover \(G \cong \overline{G}\). 
    Consider \(\phi: G \to \overline{G}\) 
    \[ 
        \phi(g) = \lambda_g
    \]
    Injective: 
    \[ 
        \lambda_a(e) = \lambda_b(e) \Leftrightarrow ae = be \Leftrightarrow a = b
    \]
    Surjective: 
    \[ 
        \lambda_a \in \overline{G}, \phi(a) = \lambda_a
    \]
    by definition.
    Structure: 
    \[ 
        \phi(gh) = \lambda_{gh} = \lambda_g \circ \lambda_h = \phi(g)\phi(h)
    \]
    thus we have an isomorphism.

\end{proof}

\end{document}
