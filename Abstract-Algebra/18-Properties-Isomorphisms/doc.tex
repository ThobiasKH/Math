\documentclass[12pt]{article}
\usepackage{amsmath, amssymb, amsthm, amsfonts, geometry}
\newtheorem{theorem}{Theorem}
\newtheorem{lemma}{Lemma}
\newtheorem{proposition}{Proposition}
\newtheorem{corollary}{Corollary}
\newtheorem{definition}{Definition}

% Page Setup
\geometry{top=1in, bottom=1in, left=1in, right=1in}

\title{Lecture Notes: Abstract Algebra | Properties of Isomorphisms (Course By: Alvaro Lozano-Robledo)}
\author{Thobias K. Høivik}
\date{\today}

\begin{document}

\maketitle
\section*{Theorems}
Let \(\phi: G \to H\) be a group isomorphism.

\noindent 
Then: 
\begin{theorem}[]
    \(\phi^{-1}: H \to G\) is also an isomorphism.

    \begin{proof}
        Since \(\phi: G \to H\) is an isomorphism \(\Rightarrow \phi\) is a bijection 
        and therefore there exists an inverse map \(\phi^{-1} : H \to G\) 
        and since \(\phi\) is a bijection we know \(\phi^{-1}\) is a bijection.
        Observe
        \begin{gather*}
            h,k \in H \Rightarrow \exists a,b, \in G : 
            \phi(a) = h \land \phi(b) = k \\ 
            \therefore \phi^{-1}(hk) = \phi^{-1}(\phi(a)\phi(b)) \\ 
            = \phi^{-1}(\phi(ab)) = ab = \phi^{-1}(h) \phi^{-1}(k)
        \end{gather*}
        thus \(\phi^{-1}\) is an isomorphism.
    \end{proof}
\end{theorem} 
\begin{theorem}

    \(|G| = |H|\), the cardinalities of G and H are the same.

    \begin{proof}
        \(\phi\) is an isomorphism \(\Rightarrow \phi\) is a bijection 
        and bijections between sets are how we define two sets to 
        have the same cardinality.
    \end{proof}

\end{theorem}

\begin{theorem}
    If G is abelian, then H is abelian. 

    \begin{proof}
        Take \(h,k \in H\) arbitrary elements, then
        \begin{gather*}
            \exists a,b \in G : \phi(a) = h \land \phi(b) = k\\ 
            \phi(ab) = \phi(ba) \\ 
            \phi(a)\phi(b) = \phi(b)\phi(a) \\ 
            hk = kh
        \end{gather*}
        Since we took h and k to be arbitrary elements of H, 
        we have shown all elements of H commute, hence H is abelian. 
    \end{proof}
\end{theorem}

\begin{theorem} 
    If G is cyclic, then H is cyclic. 

    \begin{proof} 
        Let's assume \(G = <a>\) and take \(h \in H\) arbitrary. 
        \begin{gather*}
            \exists b \in G : h = \phi(b)\\ 
            b \in G = <a> \Rightarrow b = a^n, n \in \mathbb Z^+ \\ 
            \phi(b) = \phi(a^n) = \phi(a)^n = \phi(a)\phi(a)\dots \phi(a) \\ 
            \therefore h = \phi(a)^n 
        \end{gather*}
        Since we took h to be arbitrary \(h = \phi(a)^n \forall h \in H\), 
        thus \(H = <\phi(a)>\). 
        
    \end{proof}
    
\end{theorem}

\begin{theorem}
    If G has a subgroup of order n, then H has a subgroup of order n.

    \begin{proof}
        Let \( J \subseteq G \) be a subgroup with \( |J| = n \), 
        and consider its image under \( \phi \):
        \[
            \phi[J] = \{ \phi(a) \mid a \in J \} \subseteq H.
        \]
        We verify that \( \phi[J] \) is a subgroup of \( H \):
        \textbf{Closure:}  
        If \( h, k \in \phi[J] \), then there exist \( a, b \in J \) 
        such that \( h = \phi(a) \) and \( k = \phi(b) \). 
        Since \( J \) is a subgroup, \( ab^{-1} \in J \). Applying \( \phi \), we get
        \[
            hk^{-1} = \phi(a) \phi(b)^{-1} = \phi(ab^{-1}) \in \phi[J].
        \]
        So, \( \phi[J] \) is closed under the group operation.
        \textbf{Identity:}  
        Since \( J \) is a subgroup, it contains the 
        identity element \( e_G \). Applying \( \phi \), we obtain
        \[
            \phi(e_G) = e_H \in \phi[J].
        \]
        Thus, \( \phi[J] \) contains the identity element of \( H \).
        \textbf{Inverses:}  
        If \( h \in \phi[J] \), then \( h = \phi(a) \) for 
        some \( a \in J \). Since \( J \) is a subgroup, \( a^{-1} \in J \), so
        \[
            \phi(a^{-1}) = \phi(a)^{-1} = h^{-1} \in \phi[J].
        \]
        Hence, \( \phi[J] \) contains inverses.
        Thus, \( \phi[J] \) is a subgroup of \( H \).  
        Now, to show \( |\phi[J]| = |J| = n \), we need \( \phi \) 
        to be injective on \( J \), meaning \( \ker \phi \cap J = \{ e_G \} \). 
        This holds if \( \phi \) is injective or if \( J \) 
        is mapped bijectively onto \( \phi[J] \). 
        In that case, \( |\phi[J]| = |J| = n \), as required.  
        Thus, if \( \phi \) is injective on \( J \), then \( H \) has a 
        subgroup of order \( n \), completing the proof. 
    \end{proof}
\end{theorem}

\begin{theorem}
    All cyclic groups of infinite order are isomorphic to the integers under addition. 
    Which means that there is only one infinite cyclic group structure.

    \begin{proof}
        Let G be a group, \(|G| = \infty\), and \(G =<a>\). 
        Let \(\phi: \mathbb Z \to G, \phi(n) = a^n\). Then: 
        \[ 
            g \in G \Rightarrow g = a^m, m\in \mathbb Z 
        \]
        \[ 
            \Rightarrow \phi(m)=g 
        \]
        making \(\phi\) surjective and 
        \[ 
            \phi(n) = \phi(m) \Rightarrow a^n = a^m  
        \]
        \[ 
            a^{n-m} = e \Rightarrow n-m = 0 (\because |G| = \infty)\Rightarrow n = m
        \]
        thus \(\phi\) is injective.
        Lastly 
        \[ 
            \phi(n+m) = a^{n+m} = a^n a^m = \phi(n)\phi(m)
        \]

    \end{proof}

\end{theorem}

\end{document}
