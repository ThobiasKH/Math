\documentclass[12pt]{article}
\usepackage{amsmath, amssymb, amsthm, amsfonts, geometry}
\newtheorem{theorem}{Theorem}
\newtheorem{lemma}{Lemma}
\newtheorem{proposition}{Proposition}
\newtheorem{corollary}{Corollary}
\newtheorem{definition}{Definition}

% Page Setup
\geometry{top=1in, bottom=1in, left=1in, right=1in}

\title{Lecture Notes: Quotient Groups \& Normal Subgroups (Course By: Alvaro Lozano-Robledo)} 
\author{Thobias K. Høivik}
\date{\today}

\begin{document}

\maketitle
\section*{Normal Subgroups}
\begin{definition}
    Let \( G \) be a group and \( N \) be a subgroup of \( G \). 
    We say that \( N \) is a \textbf{normal subgroup} of \( G \), 
    written as \( N \triangleleft G \), if it satisfies the condition  
    \[
        gN = Ng \quad \text{for all } g \in G.
    \]
    Equivalently, \( N \) is normal in \( G \) if for all 
    \( g \in G \) and \( n \in N \), we have  
    \[
        gng^{-1} \in N.
    \]
    This condition ensures that the left and right cosets of \( N \) in \( G \)
    are the same, which allows the construction of the quotient group \( G/N \).
\end{definition}

\begin{proposition}    
    Let G be an abelian group, N any subgroup of G. Then \(N \triangleleft G\).
\end{proposition}
\begin{proof}
    Let G be an abelian group and N a subgroup of G. 
    The left cosets of N are then \(gN = \{gn : n\in N\}\) for some \(g \in G\). 
    Since G is abelian and \(n \in G \land g \in G\) we get
    \[
        gn = ng
    \]
    for any arbitrary \(g \in G\). Thus 
    \[ 
        gN = \{gn : n \in N\} = \{ng : n \in N\} = Ng, \quad \forall g \in G
    \]
\end{proof}


\break
\section*{Quotient Group}
\begin{theorem}
    Let \(G / N = \{gN : g \in G\}\) respecting \(gN \star g'N = (gg')N\) is 
    a group with \([G:N]\) elements, where \([G:N]\) is the index of N in G. 
    \([G:N]\) is the \(n\) in \(G = \bigsqcup^n_{i = 1} g_iN\).
\end{theorem}
\begin{proof}
    Assume \(N \triangleleft G\).
    \(\# G / N = \#\{gN : g \in G\} = \)     
    the number of distinct cosets of \(N = [G:N]\). 
    \(\left( G / N, \star \right)\) is a group: 

    \noindent 
    \textbf{\(\star\) well-defined: }
    \[ 
        aN = a'N
    \]
    \[ 
        bN = b'N 
    \]
    \[ 
        \Rightarrow a' = a \cdot n_1 \land b' = b \cdot n_2, n_1,n_2 \in N  
    \]
    \[ 
        aNbN = a'Nb'N = a'b'N
    \]
    \[ 
        aNbN = abN
    \]
    \[ 
        a'b'N = a \cdot n_1 b \cdot n_2N = abN, \because n_1,n_2 \in N
    \]
    \textbf{\(\star\) associative: }
    \[ 
        aN \star (bN \star cN) = aN \star (bcN) = abcN 
    \]
    \[ 
        = (ab)cN = (abN) \star cN = (aN \star bN) \star cN
    \]
    \textbf{Identity:}
    \("e" = eN = N\)
    \[ 
        aN \star eN = (ae)N = aN
    \]
    \[ 
        eN \star aN = (ea)N = aN
    \]
    \textbf{Inverses:}
    \[ 
        (gN)^{-1} := g^{-1}N
    \]
    \[
        aN \star a^{-1}N = (aa^{-1})N = eN = N
    \]
    \[ 
        a^{-1}N \star aN = (a^{-1}a)N = eN = N
    \]
    Thus \(\left( G / N, \star\right)\) is a group.
    
\end{proof}

\begin{corollary}
    If G is an abelian group and \(N \subseteq G\) is any subgroup. 
    Then, \(G / N\) is an abelian group.
\end{corollary}
\begin{proof}
    Since G is abelian \(\Rightarrow N \triangleleft G\) is a normal subgroup. 
    By theorem 1 \(G / N\) is a group.
    Now since G is abelian we have 
    \[ 
        \forall aN,bN \in G/N : aN \star bN = (ab)N = (baN) = bN \star aN
    \]
    Thus the quotient group of an abelian group G with any subgroup N is an 
    abelian group.
\end{proof}



\begin{definition}
    Let \( G \) be a group and \( N \) a normal subgroup of \( G \) 
    (i.e., \( N \triangleleft G \)). 
    The \textbf{quotient group} or \textbf{factor group} \( G/N \) 
    is the set of left cosets of \( N \) in \( G \), defined as
    \[
        G/N = \{ gN \mid g \in G \}.
    \]
    The group operation on \( G/N \) is given by
    \[
        (gN)(hN) = (gh)N \quad \text{for all } g,h \in G.
    \]
    This operation is well-defined, and \( G/N \) forms a group under this operation.
\end{definition}

\end{document}
