\documentclass[12pt]{article}
\usepackage{amsmath, amssymb, amsthm, amsfonts, geometry}
\newtheorem{theorem}{Theorem}
\newtheorem{lemma}{Lemma}
\newtheorem{proposition}{Proposition}
\newtheorem{corollary}{Corollary}
\newtheorem{definition}{Definition}

% Page Setup
\geometry{top=1in, bottom=1in, left=1in, right=1in}

\title{Some Practice Problems from Fraleigh}
\author{Thobias K. Høivik}
\date{\today}

\begin{document}
\maketitle
\noindent
These are some practice problems, primarily to work on proof writing.

\section*{Problem 44 | Page 67}
Let \( G \) be a cyclic group with generator \( a \), and let \( G' \)
be a group isomorphic to \( G \). If \( \varphi : G \to G' \) is an isomorphism, 
show that, for every \( x \in G \), \( \varphi(x) \) is completely 
determined by the value \( \varphi(a) \). 
That is, if \( \varphi : G \to G' \) and \( \psi : G \to G' \) 
are two isomorphisms such that \( \varphi(a) = \psi(a) \), 
then \( \varphi(x) = \psi(x) \) for all \( x \in G \).

\begin{proof}
    Let \(G = <a> = \{e, a, a^2, \dots\}\) 
    be a cyclic group generated by \(a\) isomorphic to \(G'\) 
    via the arbitrary isomorphisms: 
    \[ 
        \varphi: G \to G' \land \psi: G \to G'
    \]
    where  
    \[
        \varphi(a) = \psi(a)
    \]
    Given that G is cyclic with a as generator we deduce
    \[
        \forall x \in G \quad \exists n \in \mathbb Z : x = a^n 
    \] 
    \[ 
        \therefore \varphi(x) = \varphi(a^n) = \varphi(aa\dots a) 
    \]
    which, by the structure preserving property of group isomorphisms yields 
    \[ 
        \varphi(aa\dots a) = \varphi(a^n) = \varphi(a)^n 
        = \varphi(a) \varphi(a) \dots \varphi(a)
    \]
    which means that \(\varphi(x)\) can be written in terms of \(\varphi(a)\) alone. 
    Furthermore, recall
    \[ 
        \varphi(a) = \psi(a)
    \]
    meaning 
    \begin{gather*}
        \varphi(a)\varphi(a) = \psi(a)\psi(a) \\ 
        \varphi(a)\varphi(a)\dots\varphi(a) = \psi(a)\psi(a)\dots\psi(a) \\
        \varphi(a)^n = \psi(a)^n\\
        \varphi(a^n) = \psi(a^n) \\ 
        \varphi(x) = \psi(x)
    \end{gather*}

    
\end{proof}

\section*{Problem 45 | Page 67}
Let \( a \) and \( b \) be elements of a group \( G \). 
Show that if \( ab \) has finite order \( n \), then \( ba \) also has order \( n \).
\begin{proof}
    Take a and b to be arbitrary elements of G, with \(|ab| = n \neq \infty\), then
    \begin{gather*}
        |ab| = n \Rightarrow (ab)^n = e \\ 
        (ab)^n = e \\ 
        (ab)(ab)\dots(ab) = e \\
    \end{gather*}
    by associativity of group operations: 
    \begin{gather*}
        a(ba)(ba)\dots(ba)b = e \\ 
        a(ba)^{n-1}b = e \\
        (ba)^{n-1}b = a^{-1} \\ 
        (ba)^{n-1} = a^{-1}b^{-1} \\ 
        (ba)^{n-1}b = a^{-1}b^{-1}b \\ 
        (ba)^{n-1}b = a^{-1} \\ 
        (ba)^{n-1}ba = a^{-1}a \\ 
        (ba)^{n-1}(ba) = e \\ 
        (ba)^n = e 
    \end{gather*}
    thus \(|ba| = n\).
    
    
\end{proof}

\end{document}
