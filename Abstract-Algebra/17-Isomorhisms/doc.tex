\documentclass[12pt]{article}
\usepackage{amsmath, amssymb, amsthm, amsfonts, geometry}

\newtheorem{theorem}{Theorem}
\newtheorem{lemma}{Lemma}
\newtheorem{proposition}{Proposition}
\newtheorem{corollary}{Corollary}
\newtheorem{definition}{Definition}
% Page Setup
\geometry{top=1in, bottom=1in, left=1in, right=1in}

\title{Lecture Notes: Abstract Algebra - Isomorphisms (Course By: Alvaro Lozano-Robledo)}
\author{Thobias K. Høivik}
\date{\today}

\begin{document}

\maketitle

\begin{definition} 
    Two groups \((G,\star_G)\) and \((H, \star_H)\) are said to be 
    \textbf{isomorphic} \(\Leftrightarrow G \cong H\) 
    if there exists a bijective function 
    \[
        \phi: G \to H
    \]
    that preserves the group operation, i.e., \(\forall a, b \in G \),
    \[ 
        \phi (a \star_G b) = \phi (a) \star_H \phi (b).
    \]
    A function 
    \[
        \psi: G \to H
    \]
    that preserves the group operation,
    but is not necessarily bijective is called a \textbf{homomorphism}.
    In other words an \textbf{isomorphism} is a bijective \textbf{homomorphism}.
\end{definition}

\subsection*{Example} 
Show \((\mathbb Z/_4 \mathbb Z, +) \cong (\mu_4, \times) \) 
where \(\mu_4 = \{1, -1, i, -i\}\) 
are the 4-th roots of unity in \(\mathbb C^x\). 
\begin{proof}
    Let \(\phi : \mathbb Z/_4 \mathbb Z \rightarrow \mu_4\), 
    \(\phi (n) = i^n\). 
    \begin{gather*}
        \phi(1) = i         \\
        \phi(2) = i^2 = -1  \\
        \phi(3) = i^3 = -i  \\
        \phi(0) = i ^ 0 = 1 
    \end{gather*}
    thus it is injective and surjective \(\Leftrightarrow\) it is bijective, 
    and 
    \begin{gather*}
        \phi (m+n) = i^{m+n} = i^m \times i^n 
    \end{gather*}
    so \(\phi\) is a structure preserving bijection, making it an isomorphism. 
\end{proof}

\subsection*{Another Example}
Let  
\[
F = C^\infty(\mathbb{R}) = \{ f: \mathbb{R} \to \mathbb{R} \mid f^{(n)} \text{ exists for all } n \in \mathbb{Z}_{\geq 0} \}
\]
be the space of infinitely differentiable functions. Define addition as \((f+g)(x) = f(x) + g(x)\) for all \( f, g \in F \). 
Show that \( F \cong F \) via the mapping  
\[
\phi : F \to F, \quad \phi(f)(x) = \int_0^x f(t) dt.
\]
\begin{proof}
    We set out to prove that \(\phi\) is a bijective homomorphism. 
    Let \(f,g \in F\),

    \noindent 
    \textbf{1: Homomorphism} 
    
    \begin{gather*}
        \phi (f + g)(x) = \displaystyle\int_{0}^{x} f(t) + g(t) dt \\ 
        = \displaystyle\int_{0}^{x} f(t)dt + \displaystyle\int_{0}^{x} g(t)dt
        = \phi(f)(x) + \phi(g)(x)
    \end{gather*}

    \noindent 
    \textbf{2: Injectivity} 

    \begin{gather*}
        \phi(f)(x) = \phi(g)(x) \\ 
        \displaystyle\int_{0}^{x} f(t)dt = \displaystyle\int_{0}^{x} g(t)dt \\
        \frac{d}{dx} \displaystyle\int_{0}^{x} f(t) dt 
        = \frac{d}{dx} \displaystyle\int_{0}^{x} g(t) dt \\ 
        \text{By the fundemental theorem of calculus: } \\ 
        f(x) = g(x)
    \end{gather*}

    \noindent 
    \textbf{3: Surjectivity} 
    \begin{gather*}
        \phi(f)(x) = g(x) \\ 
        \displaystyle\int_{0}^{x} f(t) dt = g(x) \\
        \text{By the fundemental theorem of calculus: } \\ 
        \frac{d}{dx} \displaystyle\int_{0}^{x} f(t)dt = f(x) \\
        \Rightarrow f(x) = \frac{d}{dx} g(x) \quad \\ 
        \forall g \in F : g' \in F, \quad \therefore f \in F
    \end{gather*}
    Thus we have shown \(\phi\) to be a bijective homomorphism, completing the proof.
    
\end{proof}

\begin{lemma}
    Let \(\phi : G\rightarrow H\) be an isomorphism of groups. 
    Then \(\phi(e_H) = e_H\).
\end{lemma}
\begin{proof}
    Let \(\phi (e_G) = h \in H\). Also 
    \[
        \phi(e_G) = \phi(e_G \star_G e_G) 
        = \phi(e_G) \star_H \phi(e_G)
        = h \star_H h 
    \]
    So \(h = h \star_H h\) and so 
    \[
        h \star_H h^-1 = h^-1 \star_H (h \star_H h) = h
    \]
    Hence \(h = e_H\).
    
\end{proof}

\subsection*{Example} 
Prove that \(\mathbb Z/_4 \mathbb Z \ncong 
\mathbb Z/_2 \mathbb Z \times \mathbb Z/_2 \mathbb Z\). 
\begin{proof}
    Suppose there is an isomorphism  
    \(
        \phi: \mathbb Z/_4 \mathbb Z \rightarrow
        \mathbb Z/_2 \mathbb Z \times \mathbb Z/_2 \mathbb Z
    \). 
    Then
    \[
        \exists g \in \mathbb Z/_2\mathbb Z \times \mathbb Z/_2\mathbb Z\text{ s.t } 
        \phi (g) = 1 \text{ mod } 4
    \] 
    If so, 
    \[ 
        \phi(g+g) = \phi(g)+\phi(g) = 1 + 1 = 2 \text{ mod } 4,
    \]
    but 
    \[ 
        g+g = 2g = (a,b)+(a,b) = (2a,2b) \equiv (0,0) \text{ mod } 2 
    \]
    \[ 
        \phi(2g) = \phi(0) = 0 \not\equiv 2 \text{ mod } 4
    \] 
    Thus we get 
    \[ 
        \phi(g+g) = 2 \text{ mod } 4 \land \phi(g+g) = 0 \Rightarrow \bot,
    \]
    a contradiction. So an isomorphism cannot exist between these groups, 
    hence they are not isomorphic.
\end{proof}




\end{document}
