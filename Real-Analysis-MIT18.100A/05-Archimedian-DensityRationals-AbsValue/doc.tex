\documentclass[12pt]{article}
\usepackage{amsmath, amssymb, amsthm, amsfonts, geometry}
\newtheorem{theorem}{Theorem}
\newtheorem{lemma}{Lemma}
\newtheorem{proposition}{Proposition}
\newtheorem{corollary}{Corollary}
\newtheorem{definition}{Definition}

% Page Setup
\geometry{top=1in, bottom=1in, left=1in, right=1in}

\title{Lecture Notes: Real Analysis | The Archimedian Property, Density of the Rationals, and Absolute Value (Course: MIT 18.100A)}
\author{Thobias K. Høivik}
\date{\today}

\begin{document}

\maketitle
\noindent 
\(\mathbb R\) is the unique ordered field with the least upper bound property 
containing \(\mathbb Q\).

\noindent 
\textbf{Simple fact: }
If \(x,y \in \mathbb R \land x < y \Rightarrow \exists r \in \mathbb R\) such that 
\(x < r < y\), for instance \(r = \frac{x+y}{2}\).
Then, we ask the question: \(x,y \in \mathbb R \land x < y\), then, does there 
exist \(r \in \mathbb Q\) such that \(x< r < y\)?

\noindent 
\begin{theorem}[The Archimedian Property]
    If \(x,y \in \mathbb R\) and \(x > 0\) then \(\exists n \in \mathbb N\) such 
    that \(nx > y \). 
\end{theorem}

\begin{theorem}[Density of the Rationals (the answer)]
    If \(x,y \in \mathbb R\) and \(x < y\) then \(\exists r \in \mathbb Q\)
    such that \(x < r < y\)
\end{theorem}
\begin{proof}[Proof (Archimedian Property)]
We prove this by contradiction.

\begin{enumerate}
    \item \textbf{Assume the negation}:  
    Suppose the claim is false. Then there exists \( x, y \in \mathbb{R} \) with \( x > 0 \) such that for every \( n \in \mathbb{N} \),
    \[
    nx \leq y.
    \]
    This means that the set
    \[
    S = \{ nx \mid n \in \mathbb{N} \}
    \]
    is bounded above by \( y \).

    \item \textbf{Apply the Least Upper Bound (LUB) property}:  
    Since \( \mathbb{R} \) has the \textit{Least Upper Bound Property}, the set \( S \) has a least upper bound, say \( b_0 \), i.e.,
    \[
    b_0 = \sup S.
    \]
    By the definition of supremum, for every \( \varepsilon > 0 \), there exists some \( n \in \mathbb{N} \) such that
    \[
    b_0 - \varepsilon < nx \leq b_0.
    \]

    \item \textbf{Derive a contradiction}:  
    Consider \( b_0 - x \). Since \( b_0 \) is the least upper bound, it must be that
    \[
    b_0 - x < nx
    \]
    for some \( n \in \mathbb{N} \), implying
    \[
    (n+1)x = nx + x > b_0.
    \]
    But this contradicts the assumption that \( b_0 \) is an upper bound for \( S \), because \( (n+1)x \in S \) and should not exceed \( b_0 \).

\end{enumerate}

\noindent
Thus, the assumption was false, proving that for any \( x > 0 \) and any \( y \in \mathbb{R} \), there exists \( n \in \mathbb{N} \) such that  
\[
nx > y.
\]
\end{proof}

\begin{proof}[Proof (Density of the Rationals)]
    Suppose \(x,y \in \mathbb R, x < y\). Then, there are three cases: 
    \begin{enumerate}
        \item \(x < 0 < y\) 
        \item \(0 \leq x < y\) 
        \item \(x < y \leq 0\)
    \end{enumerate}
    \textbf{Case 1:}
    Choose \(r = 0 \in \mathbb Q\)

    \noindent 
    \textbf{Case 2:}
    Suppose \(0 \leq x < y\). By the archimedian property \(\exists n\in \mathbb N\) 
    such that \(n(y-x) > 1\). Then, by the archimedian property again 
    \(\exists l \in \mathbb N\) such that \(l > nx\). 
    
    \noindent
    Thus, \(S = \{k \in \mathbb N : k > nx\} \neq \emptyset\).
    The well-ordering property of \(\mathbb N\), \(S\) has least element \(m\).
    Since \(m \in S \Rightarrow nx < m\). Since \(m\) is the least element of 
    \(S\), \(m-1 \not\in S \Rightarrow m-1 \leq nx \Rightarrow m \leq nx + 1\).
    Then, \(nx < m \leq nx + 1 < ny \Rightarrow x < \frac{m}{n} < y\), 
    so \(r = \frac{m}{n} \in \mathbb Q\).

    \noindent 
    \textbf{Case 3: }
    Suppose \(x < y \leq 0\). Then, \(0 \leq -y < x\). 
    By case 2, \(\exists r' \in \mathbb Q\) such that \(-y < r' < -x\). 
    Then, multiplying by \(-1\) gives \(x < -r' < y\) so \(r = r'\).
\end{proof}

\begin{theorem} 
    Assume \(S \subset \mathbb R\) is nonempty and bounded above, i.e \(\exists \sup S\).
    Then \(x = \sup S \Leftrightarrow \) 
    \begin{enumerate}
        \item x is an upper bound of S
        \item \(\forall \mathcal E > 0, \exists y \in S 
            \text{ s.t } x - \mathcal E < y \leq x\)
    \end{enumerate}
\end{theorem}

\begin{proof}
    Suppose \(S \subset \mathbb R\) nonempty and bounded above.  
    First we prove the left-to-right implication:
    Assume x is the supremum of S. Then x is an upper bound of S.  
    Let \(\mathcal E > 0\). Then, 
    \(x - \mathcal E \in \mathbb R\). By the density of \(\mathbb Q\) in \(\mathbb R\), 
    \(\exists y \in \mathbb Q\) such that \(x- \mathcal E < y < x \Rightarrow 
    x - \mathcal E < y \leq x\). 

    \noindent 
    Now for the right-to-left implication: 

    \noindent 
    Suppose x is an upper bound of S and that \(\forall \mathcal E > 0, 
    \exists y \in S \text{ s.t } x - \mathcal E < y \leq x\), but x is not the supremum 
    of S. Then \(\exists z \in \mathbb R \text{ s.t } x-\mathcal E < y \leq z < x\).
    Since \(z < x \Rightarrow x - z > 0\). Choose \(\mathcal E = x-z\). 
    Then, \(x-(x-z) < y \leq z < x \Rightarrow z < y \leq z < x\), which contradicts 
    our assumption that \(z < x\) so our assumption must be false and properties 
    1 and 2 must imply that x is the supremum of S.

    \noindent 
    Thus, we have shown that x is the supremum of S if and only if 
    x is an upper bound of S and for any value arbitrarily close to x there exists 
    some value y in between them.
    
\end{proof}

\noindent 
\textbf{Remark:}
There is an analogous statement for the infimum of a set S, which works in the same way.

\break
\begin{theorem}
    \(\sup\{1-\frac{1}{n} : n \in \mathbb N\} = 1\)
\end{theorem}

\begin{proof}
    Note \(1 - \frac{1}{n} < 1, \forall n \in \mathbb N\) so 1 is an upper bound. 
    Let \(\mathcal E > 0\). Then by the archimedian principle \(\exists n \in \mathbb N\) 
    such that \(\frac{1}{\mathcal E} < n\). 
    Then, \(1 - \mathcal E < 1 - \frac{1}{n} < 1\). 
    Thus \(\sup\{1-\frac{1}{n} : n \in \mathbb N\} = 1\). 
\end{proof}

\begin{definition}
    For \(x \in \mathbb R, A \subset \mathbb R\) we define 
    \(x + A = \{x + a : a \in A\}\), 
    \(x \cdot A = \{x \cdot a : a \in A\}\).
\end{definition}

\begin{theorem}
    If \(x \in \mathbb R\) and A is bounded above then 
    \(x + A\) is bounded above and \(\sup(x+A) = x+\sup A\).
\end{theorem}

\begin{proof}
    Suppose \(x\in \mathbb R\) and A is bounded above with \(\sup A = b_0\).
    Then \(\forall a \in A: a \leq b\), where \(b\) is any arbitrary upper bound of A. 
    \(b_0 \leq b \Rightarrow a \leq b_0 \Rightarrow x + a \leq x + b_0\). 
    So, \(x + b_0\) is an upper bound, but we can more concretely show that it 
    is in fact the l.u.b. Suppose \(x + b_0\) is not in fact the least upper bound. 
    Then, \(\exists c < x +b_0\) a smaller upper bound. However, subtracting x 
    \(c - x < b_0\) which means that \(b_0\) is not the least upper bound of A anymore 
    with this assumption, and thus it must be false and \(x + b_0\) is in fact the l.u.b.
    Thus \(\sup(x+A) = x+\sup A\).
\end{proof}
\noindent 
Note that in the proof of \textbf{Theorem 5} we could also (and it may have been 
more rigorous to) use the epsilon definition to prove we had indeed found the l.u.b. 

\begin{theorem}
    If \(x \in \mathbb R\) and A is bounded above then 
    \(x \cdot A\) is bounded above and \(\sup(x\cdot A) = x \cdot \sup A\).
\end{theorem}
\noindent
This is proved in a similar fashion as in \textbf{Theorem 5}, one just needs to 
find some epsilon that will make it magically work.

\begin{theorem}
    If \(A,B \subset \mathbb R\) with A bounded above and B bounded below and 
    \(\forall x \in A, \forall y \in B\) if \(x \leq y\) then \(\sup A \leq \inf B\).
\end{theorem}
\begin{proof}
    Let \(y \in B\). Then \(\forall x \in A : x \leq y \Rightarrow y\) is an upper bound 
    of A and therefore \(\sup A \leq y\). Thus, \(\sup A\) is a lower bound of B 
    \(\Rightarrow \sup A \leq \inf B\).
    
\end{proof}

\begin{definition}[Absolute Value]
    If \(x \in \mathbb R\)
    \[ 
        |x| = 
        \begin{cases}
            x \text{ if } x\geq 0 \\ 
            -x \text{ if } x < 0
        \end{cases}
    \]
\end{definition}


\end{document}
