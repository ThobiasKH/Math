\documentclass[12pt]{article}
\usepackage{amsmath, amssymb, amsthm, amsfonts, geometry}
\newtheorem{theorem}{Theorem}
\newtheorem{lemma}{Lemma}
\newtheorem{proposition}{Proposition}
\newtheorem{corollary}{Corollary}
\newtheorem{definition}{Definition}

% Page Setup
\geometry{top=1in, bottom=1in, left=1in, right=1in}

\title{Lecture Notes: Real Analysis | Set Theory Revision and Least Upper Bound Property (Course: MIT 18.100A)}
\author{Thobias K. Høivik}
\date{\today}

\begin{document}

\maketitle
\section*{Cardinalities}
\begin{theorem}[Cantor's Theorem]
    If A is a set, then \(|A| < |P(A)|\).
\end{theorem}
\begin{proof}
    Let A be a set. Define \(f: A \to P(A)\) by \(f(x) = \{x\}\). 
    We can see that \(f\) is injective:
    \begin{gather*}
        f(x) = y(x) \\ 
        \{x\} = \{y\} \\ 
        x = y
    \end{gather*}
    thus \(f\) is injective which implies 
    \[ \boxed{|A| \leq |P(A)|}\] 
    Now we shall show that \(|A| \neq |P(A)|\). 
    By contradiction suppose \(|A| = |P(A)|\). 
    Then \(\exists\) bijective function \(g: A \to P(A)\). 
    Define a subset \(B \subset A\) by 
    \(B = \{x\in A : x \not\in g(x)\}\). 
    Then by the definition of the powerset we have 
    \(B \in P(A)\). Since \(g\) is surjective,  
    \(\exists b \in A\) such that \(g(b)=B\). 
    
    \noindent 
    Case 1: 
    \begin{gather*}
        b\in g(b) = B \Rightarrow b \in B \\ 
        \Rightarrow b \not\in g(b)
    \end{gather*}
    
    \noindent 
    Case 2: 
    \begin{gather*}
        b \not \in g(b) \Rightarrow b \in B \\ 
        b \in g(b)
    \end{gather*}

    \noindent
    Thus we have arrived at a contradiction 
    \(b \in g(b) \Leftrightarrow b \not \in g(b)\) 
    by our assumption and so it cannot be true. 
    Thus 
    \[ 
        \boxed{|A| \neq |P(A)|}
    \]
    \begin{gather*}
        |A| \leq |P(A)| \land |A| \neq |P(A)| \\
        \Rightarrow \boxed{|A| < |P(A)|}
    \end{gather*}
    
\end{proof}

\section*{The Reals and Order Relations}
We will now restate the definition of the real numbers, slightly 
differently from before, as a theorem rather than a definition.
\begin{theorem}[Real Numbers]
    There exists a unique ordered field
    containing \(\mathbb Q\) with the least upper bound property, 
    which we denote by \(\mathbb R\).
\end{theorem}

\noindent 
We shall not go through the process of proving this theorem at this 
time, but rather explore the properties of \(\mathbb R\).

\subsection*{Ordered Sets}
\begin{definition}[Ordered Set]
    An ordered set is a set S with a relation \(<\) 
    such that 
    \begin{enumerate}
        \item \(\forall x,y \in S : x=y \lor x < y \lor y < x\)
        \item If \(x < y\) and \(y < z\) then \(x < z\), i.e 
            the relation is transitive.
    \end{enumerate}
\end{definition}

\noindent 
\textbf{Examples:}

\noindent 
For \(m,n \in \mathbb Z\), \(m < n\) if \(m-n \in \mathbb Z\).

\noindent 
For \(q,r \in \mathbb Q\), \(q < r\) if 
\(\exists m,n \in \mathbb N \text{ s.t } r-q = \frac{m}{n}\).

\noindent 
\textbf{Non-Example:}

\noindent 
Let S be a subset \(S = P(\mathbb N)\). 
We define the relation \(R\), where \((A,B) \in R\) 
if \(A \subset B\). Clearly this is transitive, but
property one is not satisfied because, for example
\(\{0\} \neq \{1\}\), and neither \((\{0\}, \{1\}) \not\in R\) nor
\((\{1\}, \{0\}) \not\in R\).

\section*{Bounds}
Recall the concepts of bounds, supremum and infimum. 

\noindent 
\textbf{Examples:}

\noindent
Let \(S = \mathbb Z, E = \{-1,0,2\} \subset S\). 
Examples of upper- and lower bounds would be \(2,3,4,\dots\), 
and \(\dots, -2,-1\), respectively.  
The supremum \(sup(E)\) would be 2 and the infimum \(inf(E)\) would be -1.

\noindent 
Let \(S = \mathbb Q\) and \(E = \{q \in \mathbb Q : 0 \leq q \leq 1\}\). 
Upper bounds: \(1, \frac{3}{2}, \frac{5}{3}, \dots\), and lower bounds: 
\(0, -1, -\frac{2}{3}, \dots\). Supremum \(sup(E) = 1\) and infimum \(inf(E) = 0\).

\subsection*{Least Upper Bound Property}
\begin{definition}
    An ordered set S has the least upper bound property if every nonempty 
    \(E \subset S\) which is bounded above has a supremum in S. 
\end{definition} 

\noindent 
A trivial example of this is \(S = \{0\}\) where every nonzero subset is the set itself 
and so every supremum lies in S.
 
\noindent 
\textbf{Example:} 

\noindent 
\(S = \{-1,-2,-3,\dots\}\) with the regular ordering relation found in the integers. 
If \(E \subset S\), \(E\) nonempty then \(-E = \{-x : x \in E\} \subset \mathbb N\). 
By the well-ordering property of the natural numbers, there exists \(m \in -E\) 
such that \(\forall x \in E : m \leq -x\) \(\Rightarrow -m \in E\) and 
\(\forall x \in E : x\leq -m \Rightarrow -m\) is the supremum of \(E\).

\noindent 
\begin{proposition}
    \(\mathbb Q\) does not have the least upper bound property.
\end{proposition} 
\begin{proof}
    Consider the set 
    \[
        E = \{ q \in \mathbb{Q} \mid q > 0 \text{ and } q^2 < 2 \}.
    \]
    The set \(E\) is nonempty since, for example, \( q = 1 \) 
    belongs to \( E \). Moreover, \( E \) is bounded above in \( \mathbb{Q} \) 
    by any rational number greater than \( \sqrt{2} \), such as \( r = 2 \).

    \noindent 
    We claim that \( E \) has no least upper bound in \( \mathbb{Q} \). 
    First, observe that for all \( q \in E \), we have \( q < \sqrt{2} \), 
    so any least upper bound of \( E \) in \( \mathbb{R} \) must be 
    \( \sup(E) = \sqrt{2} \). However, since 
    \( \sqrt{2} \not\in \mathbb{Q} \), \( \sup(E) \) cannot be a rational number.

    \noindent 
    Suppose for contradiction that \( E \) has a least upper bound \( s \in \mathbb{Q} \).
    Then \( s \geq q \) for all \( q \in E \), 
    and for any \( \mathcal E > 0 \), 
    there exists some \( q \in E \) with \( s - \epsilon < q \). 
    If \( s = \sup(E) \), then we must have \( s^2 = 2 \). 
    However, no rational number satisfies this equation, contradicting our 
    assumption that \( s \) is rational. Thus, \( E \) has no least upper bound 
    in \( \mathbb{Q} \), proving that \( \mathbb{Q} \) does not satisfy the
    least upper bound property.
\end{proof}

\end{document}
