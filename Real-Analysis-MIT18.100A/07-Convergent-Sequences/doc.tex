\documentclass[12pt]{article}
\usepackage{amsmath, amssymb, amsthm, amsfonts, geometry}
\newtheorem{theorem}{Theorem}
\newtheorem{lemma}{Lemma}
\newtheorem{proposition}{Proposition}
\newtheorem{corollary}{Corollary}
\newtheorem{definition}{Definition}

% Page Setup
\geometry{top=1in, bottom=1in, left=1in, right=1in}

\title{Lecture Notes: Real Analysis | Convergent Sequences (Course: MIT 18.100A)}
\author{Thobias K. Høivik}
\date{\today}

\begin{document}

\maketitle
\begin{definition}
    \(\{x_n\}\) converges to \(x\) if \(\forall \epsilon > 0, \exists M\in \mathbb N\)
    such that \(\forall n \geq M : |x_n - x| < \epsilon\).
\end{definition}
\noindent
\textbf{Notation:} \(\displaystyle\lim_{x \to \infty} x_n\) or \(x_n \to x\).

\noindent 
To find the definition of a non-convergent sequence we can negate the definition: 
\begin{definition}
    A sequence \(\{x_n\}\) does not converge to x if 
    \(\exists \epsilon > 0\) such that \(\forall M \in \mathbb N, \exists n \geq M\) 
    such that \(|x_n - x| \geq \epsilon\).
\end{definition}
\noindent 
\textbf{Examples:}

\noindent 
\(\displaystyle\lim_{x\to\infty} \frac{1}{n^2+30n+1} = 0\). 
\begin{proof}
    We aim to prove that \(\displaystyle\lim_{x\to\infty} \frac{1}{n^2 + 30n + 1} = 0\). 
    We do this by finding an expression for \(M\) in terms of \(\epsilon\) to satisfy 
    the definition of convergence. 
    First note that \[\frac{1}{n^2} > \frac{1}{n^2 + 30n + 1}, \forall n \in \mathbb N \setminus \{0\}\] 
    We can prove this claim via induction:
    
    \noindent 
    \textbf{Base case: n = 1} 
    \[ 
        \frac{1}{1} > \frac{1}{1+30+1}
    \]
    Suppose, for our inductive hypothesis, that  
    \[
        \frac{1}{n^2} > \frac{1}{n^2 + 30n + 1} 
    \]
    Then, for \(k = n+1\) 
    \[ 
        \text{LHS: }\frac{1}{k^2} = \frac{1}{n^2 + 2n + 1} 
    \]
    \[ 
        \text{RHS: }\frac{1}{k^2 + 30k + 1} = \frac{1}{n^2 + 2n + 1 + 30n + 31}
        = \frac{1}{n^2 + 32n + 32}
    \]
    Suppose, for a contradiction, 
    \begin{gather*}
        \frac{1}{n^2+2n+1} < \frac{1}{n^2 + 32n + 32} \\ 
        \text{multiply by } -1 \\ 
        n^2 + 2n + 1 > n^2 + 32n + 32 \\
        0 > 32n + 31 
    \end{gather*}
    which is obviously not true since \(n \geq 1\). 
    Thus our assumption was false and therefore, 
    by the principles of mathematical induction, 
    \[ 
        \frac{1}{n^2} > \frac{1}{n^2 +30n +1}, \forall n \in \mathbb N\setminus\{0\}
    \]
    
    \noindent 
    Now to prove that the sequence converges let \(\epsilon > 0\) and choose 
    \(M > \frac{1}{\sqrt{\epsilon}}\) via Archimedes' principle. 
    Then, using Archimedes' principle again, 
    we can choose any natural number \(n > M\) which will have the following property: 
    \[ 
        n > M \Rightarrow n > \frac{1}{\sqrt{\epsilon}}
    \]
    \begin{gather*}
        \Rightarrow n^2 > \frac{1}{\epsilon} \\ 
        \Rightarrow \epsilon n^2 > 1 \\ 
        \Rightarrow \epsilon > \frac{1}{n^2} > \frac{1}{n^2 + 30n + 1} 
        = |\frac{1}{n^2 + 30n + 1}| \\ 
        \Rightarrow \frac{1}{n^2 + 30n + 1} < \epsilon
    \end{gather*}
    In other words we have shown that 
    \[ 
        \forall \epsilon > 0,  \exists M \in \mathbb N \text{ s.t } \forall n \geq M : 
        |\frac{1}{n^2+30n+1}| < \epsilon
    \]
    Thus \((n^2 + 30n + 1)^{-1}\) converges to 0.  
\end{proof}

\begin{theorem}
    If \(\{x_n\}\) is convergent then 
    \(\{x_n\}\) is bounded.
    
\end{theorem}
\begin{proof}
     Suppose \(x_n \to x\). Then \(\exists M \in \mathbb N\) such that 
     \(\forall n \geq M, |x_n - x| < 1\). Which implies \(\forall n \geq M\):
     \[ 
        |x_n| = |x_n - x + x| \leq |x_n - x| + |x| = 1 + |x| 
     \]
     Define \(B = |x_1| + |x_2| + \dots + |x_{M-1}| + 1 + |x|\) to be 
     the bound of \(\{x_n\}\). 
     Then \(\forall n \in \mathbb N\), 
     \[ 
        |x_n| \leq B
     \]
     Thus \(\{x_n\}\) is bounded by \(B = 1 + |x| + \displaystyle\sum_{k=1}^{M-1} |x_k|\).
\end{proof}

\begin{definition}
    A sequence \(\{x_n\}\) is \textbf{monotone increasing} if \(\forall n \in \mathbb N: 
    x_n \leq x_{n+1}\).
    I.e \(x_1 \leq x_2 \leq x_3 \leq \dots\) 

    \noindent
    A sequence \(\{x_n\}\) is \textbf{monotone decreasing} if \(\forall n \in 
    \mathbb M: x_n \geq x_{n+1}\).

    \noindent 
    If a sequence satisfies one of these conditions we say it is \textbf{monotone} 
    or \textbf{monotonic}.
    
\end{definition}
\begin{theorem}
    A monotonic sequence \(\{x_n\}\) is convergent \textbf{if and only if} it is bounded.
\end{theorem}
\begin{proof}
    Let \(\{x_n\}\) be a convergent sequence, converging to \(x\). 
    We already know that any convergent series is bounded by 
    \[ 
        B = x_1 + x_2 + \dots x_{M-1} + 1 + |x| = 1 + |x| + 
        \displaystyle\sum_{k=1}^{M-1}|x_k| 
    \]
    so it must hold for convergent monotonic sequences also. 

    \noindent 
    Now to prove that any bounded monotonic sequence, there are two cases to consider.

    \noindent 
    \textbf{Case 1: Monotone Increasing} 

    \noindent 
    \(\{x_n\} \leq \{x_{n+1}\}\) and \(\exists B \geq 0 : |x_n| \leq B\).
    Let \(x = \sup\{x_n : n \in \mathbb N\}\) which must exist since the set 
    of terms in the sequence is non-empty and bounded by \(B\).
    Let \(\epsilon > 0\). Since \(x = \sup\{x_n\}\) there exists a natural number 
    \(M\) such that \(x - \epsilon < x_M \leq x\).
    Since the sequence is increasing we know that 
    \(\forall n \geq M : x_n \geq x_M > x- \epsilon\) and \(x_n \leq x\). 
    Which implies 
    \[ 
        |x_n - x| < \epsilon
    \]

    \noindent 
    \textbf{Case 2: Monotone Decreasing}

    \noindent 
    By using a symmetric argument and picking \(x = \inf\{x_n\}\) we reach 
    the same conclusion.
    
    
\end{proof}
\begin{theorem}

    \noindent
    \begin{enumerate}
        \item  If \(C \in (0,1)\) then \(\displaystyle\lim_{x\to\infty} C^n = 0\). 
        \item If \(C > 1\) then \(\{C^n\}\) is unbounded. 
    \end{enumerate}
    
    
\end{theorem}
\begin{proof}[Proof of \#2]
    We need to show that \(\forall B \geq 0, \exists n \in \mathbb N : c^n > B\).
    Let \(n \in \mathbb N\) such that \[n > \frac{B}{C-1}\]
    \begin{gather*}
        C^n = (1 + (C-1))^n \geq 1 + n(C-1) \geq n(C-1) > \frac{B}{C-1}C-1 = B\\ 
        C^n > B
    \end{gather*}
\end{proof}
\begin{proof}[Proof of \#1]
    Clearly if \(C \in (0,1)\) then \(\{C^n\}\) is bounded by 1. 
    If we can then show that \(\{C^n\}\) is monotone decreasing, then we know 
    it must be convergent to \(L\).
    We prove the claim via induction. 

    \noindent 
    We start with our base case
    \[ 
        0 < c^2 < c
    \]
    Suppose \(0 < C^{m+1} < C^m\). 
    Then 
    \[
        0 < C^{m+2} < C^{m+1} < C^m
    \]
    So \(\{C^n\}\) is monotone decreasing and bounded which implies it converges to \(L\).
    Now we'll show \(L=0\).
    Let \(\epsilon > 0\). Then \(\exists M \in \mathbb M\) such that \(\forall n \geq M\)
    \(|C^n - L| < (1 - c) \epsilon/2\). 
    Then  
    \[ 
        (1-C)|L| = |L - CL| \leq |L - C^{M+1} + C^{M+1} - cL| \leq |L - C^{M+1}| + 
        C|C^M + 1|
    \]
    \[ 
        < \frac{\epsilon}{2}(1-C) + C\frac{\epsilon}{2}(1-C) < \epsilon(1-C)
    \]
    Thus 
    \[ 
        |L| < \epsilon \Rightarrow |L| = 0 \Rightarrow L = 0
    \]
    
\end{proof}

\section*{Subsequences}
\begin{definition}
    Let \(\{x_n\}\) be a sequence, and let \(\{n_k\}\) be a sequence of natural numbers 
    such that \(n_1 < n_2 < n_3 < \dots\) (strictly decreasing).
    The sequence \(\{x_{n_k}\}\) is called a subsequence of \(\{x_n\}\).
        
    
\end{definition}

\noindent 
\textbf{Example:}

\noindent 
\(1,2,3,4,5,6,\dots\) is a sequence with 
\(1,3,5,7,9,\dots\) being a subsequence of the odd entries.
\(1,1,1,1,\dots\) would be a \textbf{non-example}, since \(n_k\) 
is not strictly increasing.

\begin{theorem}
    If \(\{x_n\}\) converges to \(x\) and \(\{x_{n_k}\}\) is a subsequence 
    then 
    \[
        \displaystyle\lim_{k\to\infty} x_{n_k} = x
    \]
        
    
\end{theorem}
\begin{proof}
    Since \(1 \leq n_1 < n_2 < n_3 < \dots \Rightarrow \forall 
    k \in \mathbb N : n_k \geq k\). 
    Let \(\epsilon > 0\). Since \(x_n \to x\), \(\exists M_0 \in \mathbb N\) such that 
    \(\forall n > M_0, |x_n - x| < \epsilon\).
    Choose \(M = M_0\). 
    If \(k \geq M \Rightarrow n_k \geq M = M_0 \Rightarrow |x_{n_k} - x| < \epsilon\)
        
    
\end{proof}








\end{document}
