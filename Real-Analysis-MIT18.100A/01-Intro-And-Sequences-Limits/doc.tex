\documentclass[12pt]{article}
\usepackage{amsmath, amssymb, amsthm, amsfonts, geometry}
\newtheorem{theorem}{Theorem}
\newtheorem{lemma}{Lemma}
\newtheorem{proposition}{Proposition}
\newtheorem{corollary}{Corollary}
\newtheorem{definition}{Definition}

% Page Setup
\geometry{top=1in, bottom=1in, left=1in, right=1in}

\title{Lecture Notes: Real Analysis | First Lecture (Course By: The Bright Side of Mathematics) } 
\author{Thobias K. Høivik}
\date{\today}

\begin{document}

\maketitle
\section*{Introduction to Real Analysis}
\begin{definition}[Axioms of The Reals]

    A non-empty set \(\mathbb R\) together with operations 
    \(+,\times\) and ordering \(\leq\) is called 
    the real numbers if it satisfies: 
    \begin{itemize}
        \item (A) \((\mathbb R, +)\) is an abelian group with additive identity 0.
        \item (M) \((\mathbb R, \cdot)\) is an abelian group with multiplicative identity 1.
        \item (D) Distributive law: \(x \cdot (y+z) = x\cdot y + x\cdot z\).
        \item (O) \(\leq\) is a total order, compatible with \(+\) and \(\cdot\),
            Archimedian property .
        \item (C) Every Cauchy sequence is a convergent sequence.
\end{itemize}
\end{definition}

\noindent 
Notice that properties A, M and D makes \(\mathbb R\) a field.

\begin{definition}[The Absolute Value Function]
    Let \(x \in \mathbb R\). Then the absolute value of x is: 
    \[ 
        |x| = 
        \begin{cases}
            x  & x\geq 0 \\
            -x & x < 0
        \end{cases}
    \] 
\end{definition}

\section*{Sequences and Limits} 
\begin{definition}[Sequences]
    A sequence of real numbers is a map 
    \(a_n: \mathbb N \to \mathbb R\) 
    or \(a_n: \mathbb N_0 \to \mathbb R\) if you have \(0 \in \mathbb N\). 
    (Truly, a number theorist's worst nightmare)

    \noindent 
    We will more often use the notations \((a_1, a_2, a_3, \dots)\) or 
    \((a_n)_{n\in \mathbb N}\) or \((a_n)_{n=1}^\infty\) or \((a_n)\).
\end{definition}

\noindent
\textbf{Examples:}
\begin{enumerate}
    \item 
        \[
            (a_n)_{n \in \mathbb N} = ((-1)^n)_{n\in \mathbb N}
            = (-1, 1, -1, 1, \dots)
        \]

    \item  
        \[
            (a_n)_{n\in \mathbb N} = \left(\frac{1}{n}\right)_{n\in\mathbb N} 
            = (1, \frac{1}{2}, \frac{1}{3}, \dots)
        \] 

    \item 
        \[ 
            (a_n)_{n \in \mathbb N} = (2^n)_{n\in \mathbb N}
            = (2, 4, 8, 16, 32, \dots)
        \]
\end{enumerate}

\subsection*{Convergent Series}
\begin{definition}[Convergent Series]
    A sequence \((a_n)_{a\in \mathbb N}\) is called convergent to \(a\in \mathbb R\) if
    \begin{gather*}
        \mathcal E > 0, \quad \exists N \in \mathbb N, \quad
        \forall n \geq N : |a_n - a | < \mathcal E
    \end{gather*}
\end{definition}

\noindent 
\textbf{Example:}
\((a_n) = (\frac{1}{n})\) (change in notation) is convergent to \(0 \in \mathbb R\).

\begin{proof}
    Let \(\mathcal E > 0\). Choose \(N \in \mathbb N\) such that 
    \(N \cdot \mathcal E > 1\) or, in other words, let \(N > \frac{1}{\mathcal E}\) 
    which must exist because of the Archimedian property.
    Then for \(n \geq N\), we have:
    \[
        |a_n - 0| = |\frac{1}{n} - 0| = |\frac{1}{n}| \leq \frac{1}{N}
        < \mathcal E
    \]
\end{proof}

\end{document}
