\documentclass[11pt]{article}
\usepackage[a4paper,margin=1in]{geometry}
\usepackage{fourier} % Fourier font
\usepackage{xcolor}
\usepackage{tikz}
\usepackage[most]{tcolorbox}
\usepackage{amsthm, amsmath, amssymb}
\usepackage{enumitem}
\usepackage{hyperref}
\usepackage[nameinlink,noabbrev]{cleveref}
\usepackage{titling} 

% Dark mode colors
\definecolor{bgcolor}{HTML}{FFFFFF}
\definecolor{textcolor}{HTML}{000000}
\definecolor{defcolor}{HTML}{E86873}
\definecolor{thmcolor}{HTML}{0A9396}
\definecolor{lemcolor}{HTML}{94D2BD}
\definecolor{corcolor}{HTML}{9B4AF7}
\definecolor{probcolor}{HTML}{EE9B00}
\definecolor{excolor}{HTML}{21E933}

% Background and text color
\pagecolor{bgcolor}
\color{textcolor}

% No paragraph indentation
\setlength{\parindent}{0pt}
\setlength{\parskip}{0.7em}

% Theorem box styles
\tcbset{
  enhanced,
  colback=bgcolor,
  colframe=thmcolor,
  coltext=white,
  coltitle=white,
  fonttitle=\bfseries,
  boxrule=0.7pt,
  left=1em,
  right=1em,
  top=0.7em,
  bottom=0.7em,
  before skip=10pt,
  after skip=10pt,
}

% Theorem environments with colored boxes
\newtcbtheorem[number within=section]{thm}{Theorem}{
  colframe=thmcolor, colback=thmcolor!15!bgcolor
}{thm} % The 'thm' here is the *prefix* for the label

\newtcbtheorem[number within=section]{defn}{Definition}{
  colframe=defcolor, colback=defcolor!15!bgcolor
}{def} % The 'def' here is the *prefix* for the label

\newtcbtheorem[number within=section]{lem}{Lemma}{
  colframe=lemcolor, colback=lemcolor!15!bgcolor
}{lem}

\newtcbtheorem[number within=section]{cor}{Corollary}{
  colframe=corcolor, colback=corcolor!15!bgcolor
}{cor}

\newtcbtheorem[number within=section]{prob}{Problem}{
  colframe=probcolor, colback=probcolor!15!bgcolor
}{prob}

\newtcbtheorem[number within=section]{ex}{Example}{
  colframe=excolor, colback=excolor!15!bgcolor
}{ex}

% Proof environment 
\renewenvironment{proof}[1][\proofname]{%
  \par\pushQED{\qed}\normalfont\topsep6pt \trivlist
  \item[\hskip\labelsep\itshape #1.]\ignorespaces
}{%
  \popQED\endtrivlist\addvspace{6pt}
}

% Cleveref name formats for tcolorbox environments
\crefname{thm}{theorem}{theorems}
\Crefname{thm}{Theorem}{Theorems}

\crefname{def}{definition}{definitions}
\Crefname{def}{Definition}{Definitions}

\crefname{lem}{lemma}{lemmas}
\Crefname{lem}{Lemma}{Lemmas}

\crefname{cor}{corollary}{corollaries}
\Crefname{cor}{Corollary}{Corollaries}

\crefname{prob}{problem}{problems}
\Crefname{prob}{Problem}{Problems}

\crefname{ex}{example}{examples}
\Crefname{ex}{Example}{Examples}


\title{\huge{Mathematical Logic}}
\author{\LARGE{Thobias Høivik}}
\date{\Large{Fall 2025}}

\begin{document}
\maketitle

\newpage
\tableofcontents

\newpage
\section{Introduction}
These notes are for (introductory) mathematical logic.  
As of writing this introduction, I have not yet been accepted 
to take the mathematical logic course (University of Oslo). 
These notes will be based on the book used in the aforementioned 
course, as well as resources I can find online (e.g. publicly 
avaliable courses like UCLA's Math 220A). Therefore 
the structure of these notes might be a bit scattered and 
disorganized.

\subsection{High level overview / rambling}
First-order logic is the standard way to formalize mathematics. 
For instance Peano arithmetic formalizes number theory, 
Zermelo-Fraenkel set theory formalizes set theory. 
In symbolic logic, with formal languages, 
we don't consider interpretations of symbols.
Using structures we can have a notion of validity or satisfaction 
to logical expressions. With these, we have notions of syntax 
and semantics. 

\subsection{Gödel's Completeness Theorem} 
This is a fundamental result in mathematical logic which states 
(informally) that a formula with no free 
variables can be formally deduced 
from a given set of axioms if and only if it is valid in every 
structure satisfying these axioms. 

\newpage
\section{Languages and Structures}
\subsection{Languages}
What does \(\forall x (x > 0 \Rightarrow \exists y (y \cdot y = x))\) 
mean? While we might recognize that this is a statement 
that holds in an ordered field where every positive element is a square
, it is just a sequence of symbols. 
\begin{defn}{First order language}{}
     A (first-order) language is a set of symbols \(\mathfrak L\)
     composed of two disjoint subsets: 
     \begin{enumerate}
        \item The first part (common to all languages) consists 
            of "(" and ")" together with the following logical 
            symbols: the set of variables \(\mathcal V = 
            \{v_n : n \in \mathbb N\}\), 
            the equality symbol "\(=\)", 
            connectives "\(\lnot\)", "\(\land\)", 
            the existential quantifier "\(\exists\)"

        \item The second part, called the signature of 
            \(\mathfrak L\) denoted \(\sigma^\mathfrak L\), 
            consists of the non-logical symbols of \(\mathfrak L\).
            It consists of: 
            \begin{itemize}
                \item a set of constant symbols 
                    \(C^\mathfrak L\) 
                \item a sequence of sets 
                    \(\mathcal F_n^\mathfrak L, n 
                    \in \mathbb N_{\geq 1}\),
                    where 
                    elements of this set are called the n-ary function 
                    symbols 
                \item a sequence of sets \(\mathcal R_n^\mathfrak L, 
                    n \in \mathbb N_{\geq 1}\), 
                    where elements of this set 
                    are called n-ary relation symbols 
                    (or, n-ary predicates)
            \end{itemize}
    \end{enumerate}
    The language \(\mathfrak L\) is given by the disjoint union 
    of these sets. 
\end{defn}
Note that the existential quantifier and logical or may be swapped 
for their negations: \(\forall, \lor\) and nothing will change. 
The choice is a matter of taste. Also note that while we give 
practical names to the sets in \(\mathfrak L\), we should 
attempt not to imbue any sort of interpretation on languages.

Now, a few \emph{remarks}: 
\begin{enumerate}
    \item A language is always infinite. Furthermore, the logical 
        part is countable while the non-logical can have 
        arbitrary cardinality. 
    \item Abusing the notation: \(\mathfrak L\) and 
        \(\sigma^\mathfrak L\) may be identified. 
\end{enumerate}

\begin{ex}{}{}
    \(\mathfrak L_\emptyset\) the empty language 
    (still contains logical part). 

    \(\mathfrak L_{ring} = \{\underline 0, \underline 1, 
    +, -, \cdot\}\) the ring language. 

    \(\mathfrak L_{ord} = \{<\}\) the order language. 

    \(\mathfrak L_{o.ring} := \mathfrak L_{ring} \cup 
    \mathfrak L_{ord}\)
    the ordered ring language. 

    \(\mathfrak L_{set} = \{\in\}\) the language of set theory. 

    \(\mathfrak L_{grp} = \{\cdot, ^{-1}\}, 1\) the language 
    of groups. 

    \(\mathfrak L_{graph} = \{E\}\) the language of graphs.

    \(\mathfrak L_{ar} = \{\underline 0, S, +, \cdot, <\}\) 
    the language of arithmetic.

\end{ex}

\subsection{Structures}

\begin{defn}{Structures}{}
    Let \(\mathfrak L\) be a first-order language. 
    An \(\mathfrak L\)-structure, \(\mathfrak A\), constists 
    of a non-empty set A (called the base set or universe of 
    \(\mathfrak A\) together with an element \(c^\mathfrak A \in A\) 
    for each \(c \in c^\mathfrak L\), a function 
    \(f^\mathfrak A: A^n \to A\) for each \(f\in F_n^\mathfrak L\), 
    and a subset \(R^\mathfrak A \subseteq A^n\) for each 
    \(R \in R_n^\mathfrak L\). We 
    write \(\mathfrak A = \left(A, \left(Z^\mathfrak A\right)_{Z \in 
    \sigma^\mathfrak L}\right)\). 
    \(Z^\mathfrak A\) is called the interpretation of the symbol 
    \(Z \in \sigma^\mathfrak L\) in the structure \(\mathfrak A\)
\end{defn}
Note that there are different ways of formally defining this 
and the specifics of this can safely be taken for granted later 
on. 

\begin{ex}{}{}
    \begin{enumerate}
        \item \(\mathcal N = (\mathbb N, 0, 
            \underset{x \mapsto x + 1}{S}, 
            +, \cdot, <)\) is an \(\mathfrak L_{ar}\)-structure

        \item \(C = (\mathbb C, 0, 1, +, -, \cdot)\) is an 
            \(\mathfrak L_{ring}\)-structure 

        \item \(R = (\mathbb R, 0, 1, +, -, \cdot, <)\) the ordered 
            field of real numbers, 
            is an \(\mathfrak L_{o.ring}\)-structures
    \end{enumerate}
\end{ex}


\begin{defn}{isomorphism of structures}{}
    We say that two \(\mathfrak L\)-structures 
    \(\mathfrak A, \mathfrak B\) are isomorphic, \(\mathfrak A 
    \cong \mathfrak B\), if there exists an isomorphism 

    \(F:A \underset{\text{bijection}}{\to} B\) between the base 
    sets which commutes with the interpretations of the symbols 
    \(\sigma^\mathfrak L\), that is: 
    \begin{enumerate}
        \item \(F(c^\mathfrak A) = c^\mathfrak B\) for every 
            constant symbol \(c\in C^\mathfrak L\)

        \item \(F(f^\mathfrak A(a_1, \dots, a_n)) = 
            f^\mathfrak B(F(a_1), \dots, F(a_n))\) for every 
            function symbol \(f \in \mathcal F^\mathfrak L_n\), and 
            every tuple \((a_1, \dots, a_n) \in A^n\)

        \item \((a_1,\dots,a_n) \in R^\mathfrak A \Leftrightarrow 
            (F(a_1), \dots, F(a_n)) \in R^\mathfrak B\) for every 
            predicate \(R \in \mathcal R^\mathfrak L_n\) and every 
            tuple \((a_1, \dots, a_n) \in A^n\)
    \end{enumerate}
    
\end{defn}

\newpage
\section{Terms and Formulas}

\subsection{Terms}
\begin{defn}{}{}
    A word \(w\) over a set (alphabet) \(E\) is a finite string 
    \(w=a_0a_1\dots a_{k-1}\) with \(a_i \in E\) for every 
    \(i\). 
    We call \(k\) the length of \(w\), and we denote \(E^*\)
    the set of all words over \(E\). 
\end{defn}

\begin{defn}{}{}
    Let $\mathfrak L$ be a language. The set 
    \(\mathcal T^\mathfrak L\) of 
    $\mathfrak L$-terms is the smallest subset $D$ of $\mathfrak L^*$ 
    containing the variables and the constants of $\mathfrak L$, 
    such that if $f \in F_n^\mathfrak L$ and $t_1,\dots,t_n \in D$, 
    then $ft_1\dots t_n \in D$. 

    \[\Rightarrow \mathcal T^\mathfrak L = \bigcup_{n\in\mathbb N} 
    \mathcal T^\mathfrak L_n\] where \(\mathcal T^\mathfrak L_0 
    = C^\mathfrak L \cup \mathcal V^\mathfrak L\) and, inductively 
    \[\mathcal T^\mathfrak L_{n+1} = \mathcal T^{\mathfrak L}_n \cup 
    \{ft_1\dots t_k : k\in\mathbb N_{\geq 1}, f \in \mathcal
    F^\mathfrak L_k \text{ and } t_1,\dots,t_k 
    \in \mathcal T_n^\mathfrak L\}\].

\end{defn}

\emph{Proposition}:

Any term \(t \in \mathcal T^\mathfrak L\) satisfies one 
and only one of the following: 
\begin{enumerate}
    \item \(t\) is a variable 
    \item $t$ is a constant symbol 
    \item there exists a unique integer $n \geq 1$, a unique 
        n-ary function symbol \(f\in \mathcal F^\mathfrak L_n\) 
        and a unique sequence \((t_1, \dots, t_n)\) of terms such 
        that \(t = ft_1\dots t_n\)
\end{enumerate}
\begin{proof}{}{}
     
We proceed by structural induction on the term \( t \in \mathcal{T}^{\mathfrak{L}} \), as defined by the inductive construction of terms in a first-order language.

\textbf{Base cases.}

If \( t \) is a \emph{variable} (i.e., \( t \in \mathcal{V} \)), then it satisfies condition (1).

If \( t \) is a \emph{constant symbol} (i.e., a 0-ary function symbol), then it satisfies condition (2).

These two cases are mutually exclusive since the sets of variables and constant symbols are disjoint.

\textbf{Inductive step.}

Assume the statement holds for terms \( t_1, \dots, t_n \in \mathcal{T}^{\mathfrak{L}} \).

Let \( f \in \mathcal{F}^{\mathfrak{L}}_n \) be an \( n \)-ary function symbol for some \( n \geq 1 \), and define a new term 
\[
t = f t_1 \dots t_n.
\]

Then \( t \) is neither a variable nor a constant symbol. Furthermore, the structure of the term guarantees that there exists a \emph{unique} integer \( n \), a unique function symbol \( f \), and a unique sequence \( (t_1, \dots, t_n) \) of terms such that 
\[
t = f t_1 \dots t_n,
\]
due to the syntactic rules of term formation in first-order logic. Thus, condition (3) holds uniquely.

\textbf{Exclusivity.}

We now verify that the three cases are mutually exclusive:

\begin{itemize}
    \item A term cannot be both a variable and a constant symbol (by definition of the syntax).
    \item A term cannot be a variable or constant symbol and simultaneously of the form \( f t_1 \dots t_n \) for any \( n \geq 1 \).
    \item Finally, due to the unique structure of terms, a term of the form \( f t_1 \dots t_n \) cannot be written in any other way.
\end{itemize}

Therefore, every term falls under exactly one of the three cases, completing the proof.

\end{proof}

We introduce some \emph{notation} here, for practical purposes: 
we shall often write \(f(t_1,\dots,t_n)\) instead of \(ft_1\dots t_n\).
When \(f\) is binary we might write \(t_1 f t_2\) instead of 
\(ft_1t_2\). For example: \((x+y) \cdot z\) means \(\cdot + x y z\). 

\begin{defn}{}{}
    The height of a term \(t\), denoted \(ht(t)\), is defined as the 
    smallest natural number \(k\) such that 
    \(t \in \mathcal T^\mathfrak L_k\).
\end{defn}
From this definition and the unique reading 
property for terms, it follows that 
\(ht(f(t_1,\dots,t_n)) = 1 + \max\{ht(t_i) : 1 \leq i \leq n\}\). 


\subsection{Formulas}
\begin{defn}{Atomic formula}{}
    An atomic $\mathfrak L$-formula is one of the following
    \begin{itemize}
        \item a word of the form \(t_1 = t_2\), where \(t_1, t_2\) are 
            terms of the language 
        \item a word of the form \(Rt_1\dots t_n\), 
            where \(R\in \mathcal R^\mathfrak L_n\) and all 
            \(t_1,\dots,t_n\) are terms of the language 
    \end{itemize}
    Then the set \(Fml^\mathfrak L\) of $\mathfrak L$-formulas 
    is the smallest subset \(D\) of \(\mathfrak L^*\) that contains 
    all atomic $\mathfrak L$-formulas such that 
    if \(x \in \mathcal V^\mathfrak L\) and \(\phi, \psi \in D\), 
    then $\lnot \phi, (\phi \land \psi)$ and $\exists x \phi $ are 
    all in $D$.

    $$\Rightarrow Fml^\mathfrak L = \bigcup_{n\in \mathbb N} 
    Fml_n^\mathfrak L$$ where $Fml_0^\mathfrak L$ is the set of 
    atomic $\mathfrak L$-formulas, and inductively 
    $$Fml_{n+1}^\mathfrak L := Fml_{n}^\mathfrak L \cup
    \{\lnot \phi : \phi \in Fml_n^\mathfrak L\} \cup 
    \{(\phi \land \psi) : \phi, \psi \in Fml_n^\mathfrak L\} 
    \cup \{\exists x \phi : \phi \in Fml_n^\mathfrak L, x \in 
    \mathcal V^\mathfrak L\}$$
\end{defn}
Note that the inclusion of $Fml_n^\mathfrak L$ in the 
union in the inductive definition is technically redundant, but 
we defined it like this for practicallity in other definitions.

The same can be said for the definition of $\mathcal T^\mathfrak L_n$.

\newpage
\subsection{Exercises from 1.6.1}
\begin{prob}{Task 4}{}
    Suppose that $\mathfrak A$ and $\mathfrak B$ are two 
    $\mathfrak L$-structures. 
    We say that they are \emph{isomorphic}, 
    $\mathfrak A \cong \mathfrak B$, if there 
    exists a bijection \(i: A \to A\) such that 
    for each constant symbol \(c\) of $C^\mathfrak L$, 
    \(i(c^\mathfrak A) = c^\mathfrak B\), 
    for each n-ary function symbol \(f\) in $\mathcal F^\mathfrak L_n$,
    and for each $(a_1,\dots,a_n) \in A^n$, 
    $i(f^\mathfrak A(a_1,\dots,a_n)) = f^\mathfrak B(i(a_1),\dots 
    , i(a_n))$, and for each n-ary relation/predicate symbol 
    $R$ in $\mathcal R^\mathfrak L_n$, and for each 
    $(a_1,\dots,a_n) \in A^n$, we have $(a_1,\dots,a_n) 
    \in R^\mathfrak A \Leftrightarrow (i(a_1),\dots,i(a_n)) \in 
    R^\mathfrak B$. 


    \begin{enumerate}[label=\Alph*]
        \item Show that $\cong$ is an equivalence relation.
        \item Find two different structures for a particular 
            language and prove that they are not isomorphic. 
    \end{enumerate}

\end{prob}
\begin{proof}[Proof of first]
    Let $\mathfrak L$ be an arbitrary language, and 
    $\mathfrak A, \mathfrak B, 
    \mathfrak C$ be $\mathfrak L$-structures. 
    
    First we show that $\mathfrak A \cong \mathfrak A$, i.e. 
    $\cong$ is reflexive. Let $i:A\to A$ be the identity map 
    $i(a) = a$. It is trivial to show that $i$ is bijective,
    being its own inverse. Then we have 
    $i(c^\mathfrak A) = c^\mathfrak A$ satisfying the first property 
    of structure isomorphisms. Furthermore, if we take $f$ to be 
    an $n$-ary function symbol of the language, and 
    $(a_1,\dots,a_n)\in A^n$, we get that 
    $i(f^\mathfrak A(a_1,\dots,a_n)) = f^\mathfrak A(a_1,\dots,a_n)$. 
    Now recognize that $i(a_j) = a_j$ for all $j$, and we find that 
    $i(f^\mathfrak A(a_1,\dots,a_n)) = 
    f^\mathfrak A(i(a_1),\dots,i(a_n))$, showing that the second  
    property holds. 
    Lastly, with $R$ being some $n$-ary relation symbol in the language
    and $(a_1,\dots,a_n) \in A^n$, 
    we have, through the same identity argument, 
    \[
        (a_1, \dots, a_n) \in R^\mathfrak A 
        \Leftrightarrow 
        (i(a_1),\dots,i(a_n)) \in R^\mathfrak A.
    \]
    Thus $\mathfrak A \cong \mathfrak A$.

    Second we show that $\cong$ is symmetric.   
    Suppose $i: A \to B$ is an isomorphism, so $\mathfrak A \cong \mathfrak B$ via $i$. 
    Since $i$ is bijective, it has an inverse $j := i^{-1} : B \to A$ 
    which is also a bijection.

    We now show that $j$ is a structure isomorphism from $\mathfrak B$ to $\mathfrak A$.

    \textbf{Constants:} For any constant symbol $c$ in the language, we know 
    $i(c^\mathfrak A) = c^\mathfrak B$. Applying $j$ to both sides gives:
    \[
        j(c^\mathfrak B) = j(i(c^\mathfrak A)) = c^\mathfrak A.
    \]
    So the condition on constants is satisfied.

    \textbf{Functions:} Let $f$ be an $n$-ary function symbol and 
    $(b_1,\dots,b_n) \in B^n$. Since $j$ maps from $B$ to $A$, we define
    $a_k := j(b_k)$ for each $k$. Then $i(a_k) = b_k$ for all $k$, and:
    \begin{align*}
        j(f^\mathfrak B(b_1,\dots,b_n)) 
        &= j(f^\mathfrak B(i(a_1),\dots,i(a_n))) \\
        &= j(i(f^\mathfrak A(a_1,\dots,a_n))) 
        = f^\mathfrak A(a_1,\dots,a_n),
    \end{align*}
    which shows that:
    \[
        f^\mathfrak A(j(b_1),\dots,j(b_n)) 
        = j(f^\mathfrak B(b_1,\dots,b_n)),
    \]
    satisfying the function compatibility condition.

    \textbf{Relations:} Let $R$ be an $n$-ary relation symbol and 
    $(b_1,\dots,b_n) \in B^n$. Let $a_k := j(b_k)$ as before. 
    Then, since $i(a_k) = b_k$, we get:
    \[
        (b_1,\dots,b_n) \in R^\mathfrak B 
        \Leftrightarrow 
        (i(a_1),\dots,i(a_n)) \in R^\mathfrak B 
        \Leftrightarrow 
        (a_1,\dots,a_n) \in R^\mathfrak A
    \]
    by the isomorphism definition for $i$. Applying $j$ to the tuple 
    and reversing the direction of the implication gives the desired 
    equivalence. Hence $\mathfrak B \cong \mathfrak A$.

    Lastly, we show that $\cong$ is transitive. 
    Suppose $\mathfrak A \underset{i}{\cong} \mathfrak B$ and 
    $\mathfrak B \underset{j}{\cong} \mathfrak C$. 
    Define $k := j \circ i : A \to C$, which is bijective 
    since it is the composition of two bijections.

    We show that $k$ is a structure isomorphism from 
    $\mathfrak A$ to $\mathfrak C$.

    \textbf{Constants:} For any constant symbol $c$, we have:
    \[
        k(c^\mathfrak A) = j(i(c^\mathfrak A)) = j(c^\mathfrak B) = c^\mathfrak C.
    \]

    \textbf{Functions:} Let $f$ be an $n$-ary function symbol 
    and $(a_1,\dots,a_n) \in A^n$. Then:
    \begin{align*}
        k(f^\mathfrak A(a_1,\dots,a_n)) 
        &= j(i(f^\mathfrak A(a_1,\dots,a_n))) \\ 
        &= j(f^\mathfrak B(i(a_1),\dots,i(a_n)))\\
        &= f^\mathfrak C(j(i(a_1)), \dots, j(i(a_n)))\\
        &= f^\mathfrak C(k(a_1),\dots,k(a_n)).
    \end{align*}

    \textbf{Relations:} Let $R$ be an $n$-ary relation symbol 
    and $(a_1,\dots,a_n) \in A^n$. Then:
    \[
        (a_1,\dots,a_n) \in R^\mathfrak A 
        \Leftrightarrow 
        (i(a_1),\dots,i(a_n)) \in R^\mathfrak B 
        \Leftrightarrow 
        (j(i(a_1)),\dots,j(i(a_n))) \in R^\mathfrak C
        \Leftrightarrow 
        (k(a_1),\dots,k(a_n)) \in R^\mathfrak C.
    \]
    The reverse direction follows similarly by applying the inverses 
    of $j$ and $i$. Hence, $k$ preserves all relations.

    Thus, $\mathfrak A \cong \mathfrak C$, and we have shown that 
    $\cong$ is transitive.

    Therefore, $\cong$ is an equivalence relation 
    (reflexive, symmetric, transitive).
\end{proof}

\emph{Solution to the second (Informal)}. 
Let $\mathfrak L_{NT} = \{0, S, +, \cdot, E, <\}$
be the language of number theory. 

Let $\mathcal N, \mathfrak A$ be $\mathfrak L_{NT}$-structures. 
More specifically, let $\mathcal N$ be the usual structure 
which models number theory. In $\mathfrak A$, let 
$a +^\mathfrak A b = a$ for all $a,b \in A$. 
To show that two structures like these cannot be isomorphic we 
proceed by contradiction, supposing there does exist an isomorphism 
$i: \mathbb N \to A$. 

$i$ is some function that, among other properties, satisfies 
$i(n + m) = i(n) +^\mathfrak A i(m)$, much like 
a group homomorphism, fun fun. 
Now since $i(n) +^\mathfrak A i(m) = i(n)$ we have 
$$ 
    i(n+m) = i(n) 
$$ 
Choose $n = 1 = m$ and we get 
$$ 
    i(2) = i(1) 
$$
which contradicts injectivity of $i$. 

\newpage 
\section{Semantics of first-order formulas}
\subsection{Unique readability, free variables, notation}
\emph{Proposition. (Unique readability of formulas)} 
Any $\mathfrak L$-formula $\phi$ satisfies one 
and only one of the following: 
\begin{enumerate}
    \item $\phi$ is an atomic formula 
    \item $\phi$ is equal to $\lnot \psi$ for some unique 
        $\mathfrak L$-formula $\psi$
    \item $\phi$ is equal to $\psi \land \mathfrak X$ for some unique 
        $\mathfrak L$-formulas $\psi, \mathfrak X$
    \item $\phi$ is equal to $\exists x \psi$ for some unique 
        variable x and some unique $\mathfrak L$-formula $\psi$
\end{enumerate}

\begin{proof}
We proceed by structural induction on the formation of formulas in $\mathfrak{L}$.

Let $Fml_n^{\mathfrak{L}}$ denote the set of formulas of depth at most $n$:
\begin{itemize}
    \item Base case: $Fml_0^{\mathfrak{L}}$ consists of atomic formulas.
    \item Inductive step: $Fml_{n+1}^{\mathfrak{L}}$ consists of formulas obtained by applying negation, conjunction, or existential quantification to formulas in $Fml_n^{\mathfrak{L}}$.
\end{itemize}

\textbf{Base case $(n = 0)$:}  
Any formula $\phi \in Fml_0^{\mathfrak{L}}$ is, by definition, atomic.  
Hence, case (1) holds, and the rest do not apply. The decomposition is trivially unique.

\medskip

\textbf{Inductive hypothesis:}  
Assume that for all $\psi \in Fml_n^{\mathfrak{L}}$, $\psi$ satisfies exactly one of the four cases of the proposition, with the associated subformulas and variables being unique.

\medskip

\textbf{Inductive step:}  
Let $\phi \in Fml_{n+1}^{\mathfrak{L}} \setminus Fml_n^{\mathfrak{L}}$. Then $\phi$ is constructed as exactly one of the following:
\begin{itemize}
    \item $\phi = \lnot \psi$, where $\psi \in Fml_n^{\mathfrak{L}}$,
    \item $\phi = \psi \land \chi$, where $\psi, \chi \in Fml_n^{\mathfrak{L}}$,
    \item $\phi = \exists x\, \psi$, where $x$ is a variable and $\psi \in Fml_n^{\mathfrak{L}}$.
\end{itemize}

Each of these constructions matches exactly one of cases (2), (3), or (4). We now prove uniqueness for each case.

\medskip

\textit{Case 2:} Suppose $\phi = \lnot \psi_1 = \lnot \psi_2$.  
Then by the syntax of negation, it follows that $\psi_1 = \psi_2$. Hence, $\psi$ is uniquely determined.

\medskip

\textit{Case 3:} Suppose $\phi = \psi_1 \land \chi_1 = \psi_2 \land \chi_2$.  
Then the only way such a formula can be parsed as a conjunction is with:
\[
\psi_1 = \psi_2 \quad \text{and} \quad \chi_1 = \chi_2.
\]
Thus, both components are uniquely determined.

\medskip

\textit{Case 4:} Suppose $\phi = \exists x_1\, \psi_1 = \exists x_2\, \psi_2$.  
By the syntax of quantification, it must be that $x_1 = x_2$ and $\psi_1 = \psi_2$.  
Hence, both the variable and subformula are uniquely determined.

\medskip

\textbf{Mutual exclusivity:}  
Finally, we verify that no formula can belong to more than one of the cases (1)–(4).
\begin{itemize}
    \item Atomic formulas cannot start with $\lnot$, $\land$, or $\exists$.
    \item Formulas of the form $\lnot \psi$ cannot be written as conjunctions or quantifiers.
    \item Conjunctions and quantifiers have distinct syntactic structures and cannot be confused.
\end{itemize}

\medskip

Hence, each formula $\phi$ satisfies exactly one of the four cases, and the decomposition is unique in each.

\end{proof}

\begin{defn}{}{}
    Let $v_k \in \mathcal V^\mathfrak L$ be a variable.
    We define: a free occurence of $v_k$ in $\phi$: 
    \begin{itemize}
        \item If $\phi$ is atomic (i.e. $t_1 = t_2$ for some terms), 
            all occurences of $v_k$ in $\phi$ are free. 

        \item If $\phi$ is of the form $\lnot \psi$, then the 
            free occurences of $v_k$ in $\phi$ are those which are 
            free in $\psi$.

        \item If $\phi$ is of the form $(\psi \land \chi)$, 
            then the free occurences of $v_k$ in $\phi$ are those 
            that are free in $\psi$ and those that are free in 
            $\chi$

        \item If $\phi$ is of the form $\exists v_l \psi$ and 
            $l \neq k$, then the free occurences of $v_k$ in $\phi$
            are those which are free in $\psi$

        \item If $\phi$ is of the form $\exists v_k \psi$, then no 
            occurence of $v_k$ in $\phi$ is free
    \end{itemize}

    Occurences of $v_k$ in $\phi$ that are not free are called 
    bound. 

    The free variables of $\phi$ are those having at least one 
    reoccurence in $\phi$. Denote $Free(\phi)$ the set of free 
    variables of $\phi$. 

    A sentence is a formula with no free variables. 
\end{defn}

\begin{ex}{}{}
    If $\phi$ is the formula $(\exists v_0 v_0 < v_1 \land v_0 = v_1)$ 
    (note that there is no use of "(" or ")" in the formula itself). 

    Then: 
    \begin{itemize}
        \item The first two occurences of $v_0$ are bound 

        \item The third $v_0$ is free

        \item All occurences of $v_1$ are free
    \end{itemize}
\end{ex}

\emph{Notation.}
We will use the following abbreviations: 
\begin{itemize}
    \item $(\phi \lor \psi)$ for $\lnot(\lnot \phi \land \lnot \psi)$

    \item $(\phi \Rightarrow \psi)$ for $\lnot (\phi \land \lnot \psi)$

    \item $(\phi \Leftrightarrow \psi)$ for 
        $((\phi \Rightarrow \psi) \land (\psi \Rightarrow \phi))$

    \item $\forall x \phi$ for $\lnot \exists x \lnot \phi$
\end{itemize}
Note that these do not constitute new symbols in 
the meaning of formulas in the formation rules, this is 
simply for convenience. We, formally, stick to the minimum  
amount of symbols for the full power of first-order logic.

We will also write $\exists x_1, \dots, x_n$ instead of 
$\exists x_1,\dots,\exists x_n$ and do so for the universal 
quantifier, also. 
Furthermore we write $R(t_1,\dots,t_n)$ instead of $Rt_1,\dots,t_n$ 
as well as write $t_1 R t_2$ instead of $Rt_1t_2$ when $R$ is 
a binary relation. 
We will write $(\phi_0 \land \dots \land \phi_n)$ or 
sometimes $\bigwedge_{i=0}^n \phi_i$ instead of 
$(\dots(\phi_0 \land \phi_1)\land \phi_2) \land \dots \phi_n)$, 
as well as similar abbreviations for disjunction.
We will add parentheses, or omit them for readability; 
with the convention that symbols from $\{\exists, \lnot, \forall\}$
bind strongest, followed by $\land$, then $\lor$, then the symbols 
from $\{\Rightarrow, \Leftrightarrow\}$.

\begin{ex}{}{}
    $\forall x \phi \land \psi \Rightarrow \chi$ is read as 
    $((\forall x \phi \land \psi) \Rightarrow \chi)$, that is 
    $\lnot(\forall x \phi \land \psi) \land \lnot \chi)$, and so 
    finally we get 
    $$ 
        \lnot((\lnot \exists x \lnot \phi \land \psi) \land \lnot \chi)
    $$
\end{ex}

\subsection{Semantics}
\begin{defn}{}{}
    Let $\mathfrak A$ be an $\mathfrak L$-structure. 
    \begin{enumerate}
        \item An assignment (with value in $\mathfrak A$) is a 
            function $\alpha : \mathcal V^\mathfrak L \to A$ from 
            the set of variables to the base set or universe of 
            $\mathfrak A$.

        \item If $\alpha$ is an assignment and $t$ is an 
            $\mathfrak L$-term, we define $t^\mathfrak A[\alpha]$
            by induction on the height of t $ht(t)$: 
            \begin{itemize}
                \item $v_i^\mathfrak A[\alpha] := \alpha(v_i)$ 
                    (for $v_i \in \mathcal V^\mathfrak A$) and 
                    $c^\mathfrak A[\alpha] := c^\mathfrak A$ 
                    (for constant symbol $c \in C^\mathfrak L$)
                \item $f(t_1,\dots,t_n)^\mathfrak A[\alpha] := 
                    f^\mathfrak A(t^\mathfrak A_1[\alpha],\dots,
                    t^\mathfrak A_n[\alpha])$
            \end{itemize}
    \end{enumerate}
\end{defn}

\begin{lem}{}{}
    If two assignments $\alpha, \beta$ concide on all variables 
    occuring in $t$, then $t[\alpha] = t[\beta]$.
\end{lem}

\emph{Notation.} 
If $t$ is a term, we might denote it by $t(x_1,\dots,x_n)$ if the 
variables $x_i$ are all distinct and all variables having 
at least one occurence in $t$ all belong to the set 
$\{x_1, \dots, x_n\}$. If $t(x_1,\dots,x_n)$ and a tuple of elements 
$a_1,\dots,a_n \in A$, we define $t^\mathfrak A[a_1,\dots,a_n]$ 
by $t^\mathfrak A[\alpha]$ with $\alpha$ defined to be 
$\alpha(x_i) = a_i$ for all $i$. 

\begin{defn}{Satisfaction of a formula/"Tarski's definition of truth"}
    {}
    Let $\mathfrak A$ be an $\mathfrak L$-structure. By induction 
    on the height $ht(\phi)$ of a formula we define 
    $\mathfrak A \vDash \phi[\alpha]$ 

    (read as "$\phi$ is satisfied in $\mathfrak A$ in $\alpha$"): 
    \begin{itemize}
        \item $\mathfrak A \vDash t_1 = t_2$ if and only if 
            $t_1^\mathfrak A[\alpha] = t_2^\mathfrak A[\alpha]$ 
        
        \item $\mathfrak A \vDash Rt_1 \dots t_n[\alpha]$ if and 
            only if $(t_1[\alpha],\dots,t_n[\alpha] \in R^\mathfrak A$

        \item $\mathfrak A \vDash \lnot \psi[\alpha]$ if and only if 
            $\mathfrak \nvDash \psi[\alpha]$ 

        \item $\mathfrak A \vDash (\psi \land \chi)[\alpha]$ if 
            and only if $\mathfrak A \vDash \psi[\alpha]$ and 
            $\mathfrak A \vDash \chi[\alpha]$

        \item $\mathfrak A \vDash \exists x \psi[\alpha]$ if 
            and only if there exists $a \in A$ with 
            $\mathfrak A \vDash \psi[\alpha_{a/x}]$, 
            where $\alpha_{a/x}$ denotes the assignment 
            defined by $\alpha_{a/x}(x) = a$ and 
            $\alpha_{a/x}(y) = \alpha(y)$ for all 
            $y$ different from $x$
    \end{itemize}
\end{defn}

\newpage
\emph{Proposition.}
If two assignments $\alpha, \beta$ coincide on $Free(\phi)$, then 
one has:
$$
    \mathfrak A \vDash \phi[\alpha] 
    \text{ if and only if } 
    \mathfrak A \vDash \phi[\beta]
$$

\begin{proof}
    By induction on $ht(\phi)$. The case of atomic formulas follows 
    from the earlier lemma about terms. 

    \emph{Inductive step:}

    Consider the case $\phi$ is equal to $\exists x \psi$. 
    If we have that $\mathfrak A \vDash \phi[\alpha]$, there exists 
    $a \in A$ such that $\mathfrak A \vDash \psi[\alpha_{a/x}]$. 
    Any variable $y \neq x$ that is free in $\psi$ is also free in 
    $\phi$. Hence we have $\mathfrak A \vDash \psi[\beta_{a/x}] $.
    By the inductive hypothesis, we also have 
    $\mathfrak A \vDash \phi[\beta]$.
\end{proof}

\emph{Notation.}
A formula $\phi$ will be denoted by $\phi(x_1, \dots, x_n)$ if the 
variables $x_i$ are all distinct and free in $\phi$ and 
they belong to the set $\{x_1,\dots,x_n\}$. 
If a formula $\phi(x_1,\dots,x_n)$ and elements $a_1,\dots,a_n \in A$
are given, we define $\mathfrak A \vDash \phi[a_1,\dots,a_n]$ by 
$\mathfrak A \vDash \phi[\alpha]$ where $\alpha$ is an assignment 
such that $\alpha(x_i) = a_i$ 
(well defined by the previous proposition).
Thus the formula $\phi(x_1,\dots,x_n)$ defines an n-ary relation 
on the underlying base set $A$ given by 
$\{(a_1,\dots,a_n) \in A^n : \mathfrak A \vDash \phi[a_1,\dots,a_n]\}$.
In particular, when $\phi$ is a sentence, then the relation 
$\mathfrak A \vDash \phi$ is interpreted as "$\phi$ is satisfied 
or true in $\mathfrak A$" or "$\mathfrak A$ is a model of $\phi$".

\newpage
\section{Substitution}
\subsection{Substitution of terms into formulas}
Earlier we have discussed satisfaction of formulas 
in structures with assignment functions, 
$\mathfrak A \vDash \phi[\alpha]$. 
We may write $\phi(x_1,\dots,x_n)$ if $x_i$ are distinct, free 
variables in $x_1,\dots,x_n$. Given $a_1,\dots,a_n \in A$, 
$\mathfrak A \vDash \phi[a_1,\dots,a_n] 
:\Leftrightarrow \mathfrak A \vDash \phi[\alpha]$ for $\alpha$ 
some assignment where $\alpha(x_i) = a_i$. 
Then the formula $\phi(x_1,\dots,x_n)$ defines an n-ary 
relation on $A$, given by 
$\phi[\mathfrak A] := \{(a_1,\dots,a_n) \in A^n : \mathfrak A
\vDash \phi[a_1,\dots,a_n]\}$.

\begin{defn}{}{}
    Let $\mathfrak A$ be an $\mathfrak L$-structure, and 
    $D \subseteq A^n$.  

    \begin{enumerate}
        \item The set $D$ is $\emptyset$-definable (or, $0$-definable)
            if $D = \phi[\mathfrak A]$ for some 
            formula $\phi(x_1,\dots,x_n)$.

        \item Let $B \subseteq A$. Then $D$ is $B$-definable 
            (or, definable over $B$) if 
            there exists a formula $\phi(x_1,\dots,x_n,y_1,\dots,y_m)$
            and $\overline b\in B^m$ such that  
            $$D = 
            \phi[\mathfrak A, \overline b] := 
            \{\overline a \in A^n : 
            \mathfrak A \vDash \phi(a_1,\dots,a_n,b_1,\dots,b_m)\} 
            $$
    \end{enumerate}
\end{defn}
Intuitively, a subset $D \subseteq A^n$ of 
an $\mathfrak{L}$-structure $\mathfrak{A}$ is 
\emph{definable} if it can be described by a formula in the 
language $\mathfrak{L}$, possibly using parameters from the domain. 

\begin{itemize}
    \item If $D$ is \emph{$\emptyset$-definable} (or $0$-definable), 
        then there exists a formula 
    $\phi(x_1,\dots,x_n)$ with no parameters from $A$ such that 
    \[
        D = \{\overline{a} \in A^n : \mathfrak{A} \vDash 
        \phi(\overline{a})\}.
    \]
    In other words, $D$ can be described without naming any 
    specific elements of $A$ beyond those already available 
    through the symbols of $\mathfrak{L}$.

    \item If $D$ is \emph{$B$-definable} for some $B \subseteq A$, 
        then there exists a formula 
    $\phi(x_1,\dots,x_n,y_1,\dots,y_m)$ and a tuple $\overline{b} 
    \in B^m$ such that 
    \[
        D = \{\overline{a} \in A^n : \mathfrak{A} \vDash 
        \phi(\overline{a},\overline{b})\}.
    \]
    Here, the formula may refer to specific elements of $B$ 
    through the parameters $\overline{b}$.
\end{itemize}

\begin{ex}{}{}
    Let $\mathfrak{A} = (\mathbb{N}, 0, S, +, \times)$, 
    where $S$ is the successor function.

    \begin{enumerate}
        \item The set of even natural numbers 
            \[
                E = \{ n \in \mathbb{N} : \exists y \ (n = y + y) \}
            \]
            is $\emptyset$-definable, since it can be defined 
            by the formula 
            $\phi(x) := \exists y \ (x = y + y)$, which uses no
            parameters from $\mathbb{N}$.

    \item The singleton set $\{3\}$ is not $\emptyset$-definable 
        in this language (unless a constant symbol for $3$ is 
        present in $\mathfrak{L}$), but it is definable over 
        $B = \{3\}$ using the formula 
    $\psi(x,y) := (x = y)$ with $y$ interpreted as $3$.

    \item The set $\{ x \in \mathbb{N} : x > 3 \}$ is 
        definable over $B = \{3\}$ by the formula 
    $\theta(x,y) := x > y$, again with $y$ interpreted as $3$.
\end{enumerate}
\end{ex}

When it comes to substitution we may, naively, want to 
replace every free occurence of some variable $x$ by a term $s$. 
For example $\phi(x_0)$ of the form $\exists v_1 \lnot(v_1 = v_0)$.
Now, what if $s = v_1$. This is bad because we would get 
$$ 
    \exists v_1 \lnot(v_1 = v_1)
$$ 
which is satisfied in no structure.


\begin{defn}{}{}
    Let $x_0,\dots, x_r$ be distinct variables and $s_0,\dots, s_r$ 
    be terms. Then we define the simultaneous substitution of 
    the $x_i$ by the terms $s_i$ in the following manner: 
    \begin{enumerate}
        \item Let $t$ be a term. 
            Then $t_{s_0/x_0,\dots,s_r/x_r} = 
            t_{\overline s / \overline x}$ is the word 
            obtained by: 
            $$
                x_{\overline s/\overline x} = 
                        \begin{cases}
                            x & \text{if } x\neq x_0,\dots, x_r \\ 
                            s_i & \text{if } x = x_i
                        \end{cases}
            $$
            for $x\in \mathcal V^\mathfrak L$. 
            $$ 
                c_{\overline s / \overline x} = c
            $$ 
            for $c \in C^\mathfrak L$.
            $$ 
                \left[f t^1\dots t^n\right]_{\overline s/\overline x} = 
                f t^1_{\overline s/\overline x} \dots 
                t^n_{\overline s/\overline x}
            $$ 
            for $f \in \mathcal F^\mathfrak L$ and 
            $t^i$ a term.
             
            By induction on the height of terms, this is well-defined.

        \item By induction on the height of a formula: 

            $[t = t']_{\overline s / \overline x} = 
            t_{\overline s / \overline x} = t'_{\overline s/ \overline x}$.  

            $[R t^1\dots t^n]_{\overline s / \overline x} = 
            R t^1_{\overline s/ \overline x} \dots 
            t^n_{\overline s/\overline x}$

            $[\lnot \psi]_{\overline s / \overline x} = 
            \lnot [\psi]_{\overline s / \overline x}$

            $(\psi \land \chi)_{\overline s / \overline x} = 
            \psi_{\overline s / \overline x} \land  
            \chi_{\overline s / \overline x}$

            For quantifiers: let $x_{i_1},\dots,x_{i_k}$ 
            ($i_1 < \dots < i_k$) be those variables among $x_0,\dots,x_r$ 
            that are free in $\exists x \psi$ 
            (in particular $x \neq x_{i_1} \neq \dots \neq x_{i_k}$). 
            Then we define:
            $$
                (\exists x \, \psi)_{\overline s / \overline x} = 
                \exists x \, (\psi_{\overline s' / \overline x'})
            $$
            where $\overline x' = (x_{i_1},\dots,x_{i_k})$ 
            and $\overline s' = (s_{i_1},\dots,s_{i_k})$.
             
            If $x$ has some occurence in one of 
            $s_{i_1},\dots,s_{i_k}$, we define it to be 
            $\exists u \left[\psi_{\overline s / 
            \overline x, u / x}\right]$
            where $u$ is the first variable appearing in 
            $v_1,v_2,\dots$ which does not occur in 
            $\exists x \psi, s_{i_1},\dots,s_{i_k}$.
            
    \end{enumerate}
\end{defn}

\begin{defn}{}{}
    Let $x_0,\dots,x_r$ be distinct variables, $\alpha$ an assignment
    with values in $\mathfrak A$ and $a_0,\dots,a_r$ are elements 
    of $A$. We define the assignment 
    $\alpha_{a_0/x_0, \dots, a_r/x_r} = 
    \alpha_{\overline a / \overline x}$ 
    by $\alpha_{\overline a / \overline x}(x_i) = a_i$ 
    and $\alpha_{\overline a / \overline x}(y) = \alpha(y)$ for 
    $y \neq x_i$.
\end{defn}

\begin{lem}{}{}
    $Let x_0, \dots, x_r$ be distinct variables, 
    $s_0, \dots, x_r$ be terms, and $\alpha$ be an assignment 
    with values in $\mathfrak A$.

    \begin{enumerate}
        \item For every term $t$, 
            $t^\mathfrak A_{\overline s / \overline x}[\alpha] = 
            t^\mathfrak A[\alpha_{s_0^\mathfrak A[\alpha]/x_0, 
            \dots, s_r^\mathfrak A[\alpha]/x_r}]$

        \item For every formula $\phi$, 
            $\mathfrak A \vDash 
            \phi_{\overline s / \overline x}[\alpha]$
            if and only if 
            $\mathfrak A \vDash 
            \phi[\alpha_{s_0^\mathfrak A[\alpha]/x_r,\dots,
            s_r^\mathfrak A[\alpha]/x_r}]$
    \end{enumerate}


    
\end{lem}
\begin{proof}[Proof of (1)]
        We proceed by induction on the structure of the term $t$.

        \textbf{Base case.} Suppose $t$ is a variable $y$.
        \begin{itemize}
            \item If $y = x_i$ for some $i \leq r$, 
                then by definition of substitution 
                \[
                    t_{\overline s/\overline x} = s_i.
                \]
                Hence
                \[
                    t^\mathfrak A_{\overline s/\overline x}[\alpha] 
                    = s_i^\mathfrak A[\alpha].
                \]
                On the other hand,
                \[
                    t^\mathfrak A
                    \bigl[\alpha_{s_0^\mathfrak A[\alpha]/x_0,\dots,
                    s_r^\mathfrak A[\alpha]/x_r}\bigr]
                    = 
                    (\alpha_{s_0^\mathfrak A[\alpha]/x_0,\dots})(x_i)
                    = s_i^\mathfrak A[\alpha].
                \]
                Thus the equality holds.

            \item If $y \notin \{x_0,\dots,x_r\}$, then 
                $t_{\overline s/\overline x} = y$, so
                \[
                    t^\mathfrak A_{\overline s/\overline x}[\alpha] 
                    = \alpha(y).
                \]
                Since the modified assignment 
                $\alpha_{s_0^\mathfrak A[\alpha]/x_0,
                \dots,s_r^\mathfrak A[\alpha]/x_r}$
                agrees with $\alpha$ on $y$, we have
                \[
                    t^\mathfrak A\bigl[\alpha_{s_0^\mathfrak A
                    [\alpha]/x_0,\dots}\bigr] 
                    = \alpha(y).
                \]
        \end{itemize}

        \textbf{Inductive step.} Suppose $t = f(t_1,\dots,t_n)$ 
        with function symbol $f$.
        By definition of substitution, 
        \[
            t_{\overline s/\overline x} 
            = f\bigl((t_1)_{\overline s/\overline x},\dots,
            (t_n)_{\overline s/\overline x}\bigr).
        \]
        Hence
        \[
            t^\mathfrak A_{\overline s/\overline x}[\alpha]
            = f^\mathfrak A\bigl((t_1)^\mathfrak A_{\overline 
                s/\overline x}[\alpha],\dots,
            (t_n)^\mathfrak A_{\overline s/\overline x}[\alpha]\bigr).
        \]
        By the induction hypothesis, for each $j$,
        \[
            (t_j)^\mathfrak A_{\overline s/\overline x}[\alpha]
            = t_j^\mathfrak A\bigl[\alpha_{s_0^\mathfrak A[\alpha]
                    /x_0,\dots,
            s_r^\mathfrak A[\alpha]/x_r}\bigr].
        \]
        Thus
        \[
            t^\mathfrak A_{\overline s/\overline x}[\alpha]
            = f^\mathfrak A\bigl(
                t_1^\mathfrak A[\alpha_{...}],\dots,
                t_n^\mathfrak A[\alpha_{...}]
            \bigr)
            = t^\mathfrak A\bigl[\alpha_{
                    s_0^\mathfrak A[\alpha]/x_0,\dots,
            s_r^\mathfrak A[\alpha]/x_r}\bigr].
        \]

        This completes the induction and the proof.
    \end{proof}


    \begin{proof}[Proof of (2)]
        We proceed by induction on the height $ht(\phi)$ of 
        the formula $\phi$.

        \textbf{Base cases.}  

        \begin{itemize}
            \item If $\phi$ is an atomic formula of the form 
                $t_1 = t_2$, then
                \[
                    \mathfrak A \vDash 
                    (t_1 = t_2)_{\overline s/\overline x}[\alpha]
                    \iff
                    t^\mathfrak A_{1_{\overline s/\overline x}}
                    [\alpha]
                    = t^\mathfrak A_{2_{\overline s/\overline x}}[\alpha].
                \]
                By part (1), this is equivalent to
                \[
                    t_1^\mathfrak A\bigl[\alpha_{s_0^\mathfrak A
                    [\alpha]/x_0,\dots,s_r^\mathfrak A[\alpha]/x_r}
                \bigr]
                    = t_2^\mathfrak A\bigl[\alpha_{s_0^\mathfrak A
                    [\alpha]/x_0,\dots,s_r^\mathfrak A[\alpha]/x_r}
                \bigr],
                \]
                which is precisely
                \[
                    \mathfrak A \vDash (t_1 = t_2)
                    [\alpha_{s_0^\mathfrak A[\alpha]/x_0,\dots,
                    s_r^\mathfrak A[\alpha]/x_r}].
                \]

            \item If $\phi$ is atomic of the form $R(t_1,\dots,t_n)$, then
                \[
                    \mathfrak A \vDash (R(t_1,\dots,t_n))_{\overline 
                    s/\overline x}[\alpha]
                    \iff 
                    (t^\mathfrak A_{1_{\overline s/\overline x}}
                    [\alpha],\dots,
                    t^\mathfrak A_{n_{\overline s/\overline x}}
                    [\alpha]) \in R^\mathfrak A.
                \]
                By part (1), this is equivalent to
                \[
                    (t_1^\mathfrak A[\alpha_{...}],\dots,
                    t_n^\mathfrak A[\alpha_{...}]) \in R^\mathfrak A,
                \]
                i.e.
                \[
                    \mathfrak A \vDash R(t_1,\dots,t_n)
                    [\alpha_{s_0^\mathfrak A[\alpha]/x_0,\dots,
                    s_r^\mathfrak A[\alpha]/x_r}].
                \]
        \end{itemize}

        \textbf{Inductive step.} 
        Assume the claim holds for all formulas of smaller
        height than $\phi$.

        \begin{itemize}
            \item If $\phi = \lnot \psi$, then
                \[
                    \mathfrak A \vDash 
                    (\lnot \psi)_{\overline s/\overline x}[\alpha]
                    \iff
                    \mathfrak A \not\vDash 
                    \psi_{\overline s/\overline x}[\alpha].
                \]
                By the induction hypothesis,
                \[
                    \mathfrak A \not\vDash \psi[\alpha_{...}]
                    \iff
                    \mathfrak A \vDash (\lnot \psi)[\alpha_{...}].
                \]

            \item If $\phi = (\psi \land \theta)$, then
                \[
                    \mathfrak A \vDash 
                    (\psi \land \theta)_{\overline s/\overline x}
                    [\alpha]
                    \iff
                    \mathfrak A \vDash \psi_{\overline s/\overline x}
                    [\alpha]
                    \;\text{and}\;
                    \mathfrak A \vDash \theta_{\overline s/
                    \overline x}[\alpha].
                \]
                By induction hypothesis, this is equivalent to
                \[
                    \mathfrak A \vDash \psi[\alpha_{...}] \;
                    \text{and}\;
                    \mathfrak A \vDash \theta[\alpha_{...}],
                \]
                which is
                \[
                    \mathfrak A \vDash (\psi \land \theta)
                    [\alpha_{...}].
                \]
                The same reasoning works for $\lor, \to, 
                \leftrightarrow$.

            \item If $\phi = \forall y\,\psi$, 
                then by definition of substitution
                \[
                    (\forall y\,\psi)_{\overline s/\overline x}
                    = \forall y\,(\psi_{\overline s/\overline x}),
                \]
                assuming $y \notin \{x_0,\dots,x_r\}$ 
                (or else with variable renaming).
                Then
                \[
                    \mathfrak A \vDash (\forall y\,\psi)_{\overline s
                    /\overline x}[\alpha]
                    \iff
                    \text{for all } a \in A,\;
                    \mathfrak A \vDash \psi_{\overline s/\overline x}
                    [\alpha_{a/y}].
                \]
                By induction hypothesis, this is equivalent to
                \[
                    \text{for all } a \in A,\;
                    \mathfrak A \vDash \psi[\alpha_{a/y,\,
                        s_0^\mathfrak A[\alpha]/x_0,\dots,
                    s_r^\mathfrak A[\alpha]/x_r}],
                \]
                which is precisely
                \[
                    \mathfrak A \vDash \forall y\,
                    \psi[\alpha_{s_0^\mathfrak A[\alpha]/x_0,\dots,
                    s_r^\mathfrak A[\alpha]/x_r}].
                \]

                The case $\phi = \exists y\,\psi$ is analogous.
        \end{itemize}

        Thus by structural induction the equivalence 
        holds for every formula $\phi$.
    \end{proof}


\subsection{Exercises from 1.8.1}
\begin{prob}{Task 4}{}
    Show that $x$ is always substitutable for $x$ in $\phi$.
\end{prob}
\begin{proof}
    Let $\phi$ be some $\mathfrak L$-formula 
    and let $x \in \mathcal V^\mathfrak L$.
    Let $Fml_n^\mathfrak L, n\in\mathbb N$ denote 
    the $\mathfrak L$-formulas with height at most $n$.
    We proceed by structural induction on the height of the formula 
    $ht(\phi)$.

    For the base case suppose $ht(\phi) = 0$, 
    i.e. $\phi \in Fml_0^\mathfrak L$ is an atomic formula.
    Then all occurences of variables in $\phi$ are free 
    since there are no quantifiers, and so are substitutable. 

    Suppose, for our inductive hypothesis, that $x$ is substitutable
    for $x$, for any formula in $Fml_n^\mathfrak L$.

    Then, ignoring the trivial case where $\phi$ is atomic, 
    we have $3$ cases to check. 

    \begin{enumerate}
        \item $\phi :\equiv \lnot \psi$

        \item $\phi :\equiv (\psi \lor \gamma)$

        \item $\phi :\equiv (\forall y (\psi))$ 
    \end{enumerate}

    \emph{Case 1.} 

    Suppose $\phi$ is of the form $\lnot \psi$, 
    where $\psi \in Fml_n^\mathfrak L$, 
    and by our inductive 
    hypothesis $x$ is substituable for $x$ in 
    $\psi \in Fml_n^\mathfrak L$. 
    Then $x$ is substitutable for $x$ in $\phi$.

    \emph{Case 2.}

    Suppose $\phi$ is of the form $(\psi \lor \gamma)$, 
    where $\psi,\gamma \in Fml_n^\mathfrak L$, and 
    by our inductive hypothesis $x$ is substitutable for $x$ in 
    any formula of height $n$. Then $x$ is substitutable for $x$ 
    in both $\psi$ and $\gamma$. Thus $x$ is substitutable for $x$ 
    in $\phi$.

    \emph{Case 3.}

    Suppose $\phi$ is of the form $(\forall y (\psi))$, 
    where $\psi \in Fml_n^\mathfrak L$, and 
    by our inductive hypothesis $x$ is substitutable for $x$ 
    in $\psi$. Since $x$ is substitutable for $x$ in $\psi$ and 
    $y$ does not occur in $x$ we have that $x$ is substitutable for 
    $x$ in $\phi$.

    This completes the proof.
\end{proof}

\subsection{Exercises from 1.9.1} 
\begin{prob}{Task 3}{}
    Suppose that $\phi$ is an $\mathfrak L$-formula and $x$ is 
    a variable. Prove that $\phi$ is valid if and only if 
    $\left(\forall x\right)\left(\phi\right)$ is valid.
\end{prob}

\begin{proof}
    Let $\mathfrak{L}$ be a language, 
    let $\phi$ be an $\mathfrak{L}$-formula, and let 
    $x \in \mathcal{V}^{\mathfrak{L}}$ be a variable of the language.  
    We show that $\phi$ is valid if and only if 
    $\forall x\,\phi$ is valid.

    \emph{($\Rightarrow$) Suppose $\phi$ is valid.}  
    This means that for every $\mathfrak{L}$-structure 
    $\mathfrak{A}$ and every assignment $\alpha$, we have
    \[
        \mathfrak{A} \vDash \phi[\alpha].
    \]
    Let $\mathfrak{A}$ be any $\mathfrak{L}$-structure 
    and $\alpha$ any assignment.  
    To check $\mathfrak{A} \vDash \forall x\,\phi[\alpha]$, 
    we must verify that
    \[
        \mathfrak{A} \vDash \phi[\alpha[a/x]] \quad 
        \text{for every } a \in A.
    \]
    But since $\phi$ is valid, $\mathfrak{A} \vDash 
    \phi[\alpha[a/x]]$ holds for every $a \in A$.  
    Thus $\mathfrak{A} \vDash \forall x\,\phi[\alpha]$ 
    for all $\mathfrak{A}$ and $\alpha$, so $\forall x\,\phi$ is valid.

    \emph{($\Leftarrow$) Suppose $\forall x\,\phi$ is valid.}  
    This means that for every $\mathfrak{L}$-structure 
    $\mathfrak{A}$ and every assignment $\alpha$,
    \[
        \mathfrak{A} \vDash \forall x\,\phi[\alpha].
    \]
    By the semantics of $\forall$, this means that for 
    every $a \in A$,
    \[
        \mathfrak{A} \vDash \phi[\alpha[a/x]].
    \]
    In particular, if we take $a = \alpha(x)$, 
    then $\alpha[a/x]$ agrees with $\alpha$ on all variables, so
    \[
        \mathfrak{A} \vDash \phi[\alpha].
    \]
    Since this holds for all $\mathfrak{A}$ and $\alpha$, 
    $\phi$ is valid.
\end{proof}

\begin{prob}{Task 4}{}
    a) Assume that $\vDash (\phi \Rightarrow \psi)$. Show that 
    $\phi \vDash \psi$. 

    b) Suppose that $\phi$ is $x < y$ and $\psi$ is 
    $z < w$. Show that $\phi \vDash \psi$, 
    but $\nvDash (\phi \Rightarrow \psi)$.
\end{prob}
\begin{proof}[Proof of a]
     Let $\mathfrak L$ be a language $\phi, \psi$ 
     $\mathfrak L$-formulas and for any arbitrary 
     $\mathfrak L$-structure $\mathfrak A$ and assignment $\alpha$, 
     we are have 
     $\mathfrak A \vDash (\phi \Rightarrow \psi)$.
     Our goal is to show that for any $\mathfrak A$ with assignment 
     $\alpha$, if $\mathfrak A \vDash \phi$ then 
     $\mathfrak A \vDash \psi$.

     As mentioned we have 
     $\mathfrak A \vDash (\phi \Rightarrow \psi)[\alpha]$ 
     for arbitrary structure and assignment. Recall that 
     $\phi \Rightarrow \psi$ is short-hand for 
     $\lnot \phi \lor \psi$. 
     Suppose $\mathfrak A \vDash (\lnot \phi \lor \psi)[\alpha]$ and 
     $\mathfrak A \vDash \phi[\alpha]$, 
     where structure and assignment are arbitrary, as mentioned. 
     Then by the semantics of disjunction we have 
     $\mathfrak A \vDash \lnot\phi[\alpha]$ or 
     $\mathfrak A \vDash \psi[\alpha]$, 
     and $\mathfrak A \vDash \phi[\alpha]$. 
     In particular $\mathfrak A \vDash \lnot \phi[\alpha]$ means 
     that $\mathfrak A \nvDash \phi[\alpha]$, but this is not the 
     case since we have $\mathfrak A \vDash \phi[\alpha]$.
     Therefore $\mathfrak A \vDash \psi[\alpha]$ must hold. 
     Hence $\phi \vDash \psi$.
\end{proof}


\begin{proof}[Proof of b]
    Let $\phi$ be the formula $x < y$ and $\psi$ 
    be the formula $z < w$.

    \emph{Step 1: Show $\phi \vDash \psi$.}  
    Assume $\mathfrak{A}$ is a structure with domain $A$ and 
    interpretation $<^{\mathfrak{A}} \subseteq A \times A$ of the 
    relation symbol $<$. Suppose $\mathfrak{A} \vDash \phi$, 
    i.e.\ for every assignment $\alpha$ we have 
    \[
        (\alpha(x), \alpha(y)) \in <^{\mathfrak{A}}.
    \]
    Let $a,b \in A$ be arbitrary.  
    Define an assignment $\alpha$ with $\alpha(x) = a$ and 
    $\alpha(y) = b$ (the values of other variables are irrelevant).  
    By assumption, $(a,b) \in <^{\mathfrak{A}}$.  
    Since $a,b$ were arbitrary, this shows that
    \[
        <^{\mathfrak{A}} = A \times A.
    \]
    Now let $\beta$ be any assignment.  
    Then $(\beta(z), \beta(w)) \in A \times A = <^{\mathfrak{A}}$,  
    so $\mathfrak{A} \vDash \psi$.  
    Thus every structure that satisfies $\phi$ also satisfies $\psi$, 
    so $\phi \vDash \psi$.

    \emph{Step 2: Show $\nvDash (\phi \to \psi)$.}  
    Consider the structure $\mathfrak{A}$ with domain $A = \{0,1\}$ 
    and $<^{\mathfrak{A}}$ interpreted as the usual strict order 
    $\{(0,1)\}$.  
    Let $\alpha$ be an assignment with
    \[
        \alpha(x) = 0, \quad \alpha(y) = 1, \quad
        \alpha(z) = 1, \quad \alpha(w) = 0.
    \]
    Then $\mathfrak{A} \vDash \phi[\alpha]$ since $0 < 1$,  
    but $\mathfrak{A} \nvDash \psi[\alpha]$ since $1 \not< 0$.  
    Hence $\mathfrak{A} \nvDash (\phi \to \psi)[\alpha]$, so 
    $\phi \to \psi$ is not valid.

    Therefore, $\phi \vDash \psi$ but $\nvDash (\phi \to \psi)$.
\end{proof}

\newpage
\section{Elementary Classes and Equivalence}
\begin{defn}{}{}
    For a set of well-formed-formulas $\Gamma$, 
    let $Mod \Gamma$ (sometimes written $M(\Gamma)$) 
    be the class of all structures which 
    model $\Gamma$. 

    A class of structures $\mathcal K$ is an elementary class 
    if there exists a well-formed-formula $\phi$ such that 
    $\mathcal K = Mod \phi$.

    A class of structures $\mathcal K$ is a weakly elementary class 
    if there exists a set of well-formed-formulas $\Gamma$ 
    such that $\mathcal K = Mod \Gamma$.
\end{defn}

\begin{defn}{}{}
    Two structures $\mathfrak A, \mathfrak B$ are 
    elementary equivalent
    (written $\mathfrak A \equiv \mathfrak B$) if, for all sentences 
    $\phi$ 
    $$ 
        \mathfrak A \vDash \phi \Leftrightarrow \mathfrak B 
        \vDash \phi
    $$ 
\end{defn}

\begin{ex}{}{}
    \begin{itemize}
        \item $(\mathbb N, 0, +) \not\equiv (\mathbb Z, 0, +)$
        \item $(\mathbb Q, 0, +) \not\equiv (\mathbb Z, 0, +)$
        \item $(\mathbb Q, 0, +) \equiv (\mathbb R, 0, +)$
    \end{itemize}
    The first example can be shown by using the formula 
    for existence of inverses, which $\mathbb Z$ has, while 
    $\mathbb N$ does not.

    For the second we can use the fact that for all $x$, there 
    exists $\frac{x}{2}$.

    The last one seems counterintuitive, but without multiplication
    and a notion of subsets, the two structures are equivalent.
\end{ex}

\begin{defn}{}{}
    Let $Th(\mathfrak A)$ be the set of all sentences 
    true in $\mathfrak A$.

    Note: 
    $$ 
        \mathfrak A \equiv \mathfrak B  
        \Leftrightarrow 
        Th(\mathfrak A) = Th(\mathfrak B)
    $$ 
\end{defn}

\newpage 
\section{Deductions}
This concept is not explored in the 
supplementary lecture videos I use and I do not feel like 
copy and pasting theorems and definitions from the whole 
second chapter of the book, so this section will be 
mostly problems.

\subsection{Exercises from 2.2.1}
\begin{prob}{Task 1}{}
    Let the collection of non-logical axioms be
    $$ 
    \Sigma = \{(A(x) \land A(x)) \rightarrow B(x,y), 
    A(x), B(x,y) \rightarrow A(x)\} 
    $$ 
    and let the rule of inference be modus ponens.
    Determine whether the following are deductions or not.

    \begin{enumerate}
        \item $A(x)$ 

            $A(x) \land A(x)$

            $(A(x) \land A(x)) \rightarrow B(x,y)$

            $B(x,y)$

        \item $B(x,y) \rightarrow A(x)$

            $A(x)$

            $B(x,y)$

        \item $(A(x) \land A(x)) \rightarrow B(x,y)$

            $B(x,y) \rightarrow A(x)$

            $(A(x) \land A(x)) \rightarrow A(x)$
    \end{enumerate}
\end{prob}
\emph{Solution.}

The first is not valid since we cannot conclude $A(x) \land A(x)$  
from $A(x)$.

The second is not valid since $B(x,y)$ does not follow from any of 
the preceding. 

The third is valid since we can conclude $B(x,y)$ from 
the first line. Then we can conclude $A(x)$ from the second 
line. Then we can conclude $(A(x) \land A(x)) \rightarrow A(x)$.  
In other words we can get $A(x)$ from $(A(x)\land A(x))$.

\begin{prob}{Task 4}{}
    Let $\mathcal L$ be a language consisting of a single unary 
    predicate symbol $R$, and $B$ be the infinite set of axioms 
    \begin{align*}
        B = \{&R(x_1),
            \\
              &R(x_1) \rightarrow R(x_2), 
              \\
              &R(x_2) \rightarrow R(x_3),
              \\
              & \quad \quad \quad \vdots 
            \\ 
              &R(x_i) \rightarrow R(x_{i+1}),
              \\
              & \quad \quad \quad \vdots
        \\
    \}  
    \end{align*}
    Using modus ponens, prove by induction that 
    $B \vdash R(x_j)$, for each $j \geq 1$.
\end{prob}

\begin{proof}
    For our base case we use the axiom $R(x_i) \in B$, which 
    gives us $B \vdash R(x_i)$. For our inductive hypothesis, 
    assume $B \vdash R(x_i)$ for some $i \geq 1$. 
    Then since $R(x_i) \rightarrow R(x_{i+1})$ we have, 
    via modus ponens, $B \vdash R(x_{i+1})$. 
    Then since $B \vdash R(x_1)$ and 
    $(R(x_i), R(x_i) \rightarrow R(x_{i+1}))$ we get 
    $B \vdash R(x_2), B \vdash R(x_3), \dots, B \vdash R_{x_j}$ 
    for each $j \geq 1$.
\end{proof}

\subsection{Tasks from 2.5.1}
\begin{prob}{Task 4}{}
    Show that, if $x$ is not free in $\psi$, 
    $(\phi \rightarrow \psi) \vDash 
    \left[(\exists x\phi) \rightarrow \psi\right]$.
\end{prob}

\begin{proof}
    Let $\phi, \psi$ be $\mathfrak L$-formulas such that 
    $x$ is not free in $\psi$. 

    Assume, for any $\mathfrak L$-structure, 
    $\mathfrak A \vDash (\phi \rightarrow \psi)[s]$.

    Suppose, ignoring the trivial case, 
    $\mathfrak A \vDash (\exists x \phi)[s]$.
    Then there exists $a \in A$ such that 
    $$ 
        \mathfrak A \vDash \phi[s[x|a]]
    $$ 

    From $\mathfrak A \vDash \phi$, we know that 
    for every assignment $t$, if $\mathfrak A \vDash \phi$, 
    then $\mathfrak A \vDash \psi$. In particular, 
    this holds for $t = s[x|a]$. 
    Hence 
    $$ 
        \mathfrak A \vDash \psi[s[x|a]]
    $$ 
    since $\psi$ is invariant under changes to the value 
    of $x$.

    This completes the proof.
\end{proof}

\newpage
\subsection{Tasks from 2.7.1}

\begin{prob}{Task 4}{}
    Suppose that $\eta$ is a sentence. 
    Prove that $\Sigma \vdash \eta$ if and only if 
    $\Sigma \cup (\lnot \eta) \vdash [(\forall x) x=x] \land 
    \lnot [(\forall x) x= x]$. Notice that 
    this exercise tells us that our deductive system allows us 
    to do proofs by contradiction.
\end{prob}

\begin{proof}
    Let $\eta$ be a sentence. 

    Suppose $\Sigma \vdash \eta$. 
    Then consider $\Sigma \cup (\lnot \eta)$. 
    Clearly $\Sigma \cup (\lnot \eta) \vdash \eta$, since 
    the smaller set of non-logical axioms $\Sigma \vdash \eta$. 
    Thus $\Sigma \cup (\lnot \eta)\vdash \eta \land (\lnot \eta)$. 
    Since $\eta$ has no free variables, 
    $\eta \land (\lnot \eta)$ is a contradiction. 
    Then, for any formula $\phi$, 
    $$ 
        \eta \land (\lnot \eta) \vDash \phi
    $$  
    more specifically, 
    $$ 
        \eta \land (\lnot \eta) \vDash 
        [(\forall x) x = x] \land \lnot[(\forall x) x = x]
    $$ 

    By the completeness theorem, we get 
    $$ 
        \eta \land (\lnot \eta) \vdash 
        [(\forall x) x = x] \land \lnot[(\forall x) x = x]
    $$ 

    Since, with our assumptions, 
    $\eta \land (\lnot \eta)$ is provable from 
    $\Sigma \cup (\lnot \eta)$, 
    $[(\forall x) x = x) \land \lnot [(\forall x) x = x]$ 
    is provable from the same set of non-logical axioms.

    Now, the other way, assume 
    $\Sigma \cup (\lnot \eta) \vdash [(\forall x) x = x) 
    \land \lnot [(\forall x) x = x]$. 
    Then, by the soundness theorem, 
    $$ 
        \Sigma \cup (\lnot \eta ) \vDash [(\forall x) x = x) \land 
        \lnot [(\forall x) x = x]
    $$ 
    which means that for arbitrary structure $\mathfrak A$, 
    $\mathfrak A \nvDash \Sigma \cup (\lnot \eta)$ or 
    $\mathfrak A \vDash [(\forall x) x = x) 
    \land \lnot [(\forall x) x = x]$.

    Since no structure can satisfy a contradiction it must be 
    the case that 
    $$ 
        \mathfrak A \nvDash \Sigma \cup (\lnot \eta) 
    $$ 

    So $\Sigma \cup (\lnot \eta)$ has no modles which means that 
    it is inconsistent. Then, any contradiction can be 
    proven from it. 
    For example 
    $$ 
        \Sigma \cup (\lnot \eta) \vdash (\eta \land \lnot \eta)
    $$ 
    which can be rewritten via the deduction theorem to 
    $$ 
        \Sigma \vdash (\lnot \eta) \rightarrow (\eta \land \lnot \eta)
    $$ 
    which can again be rewritten, using the laws of propositional 
    logic, to 
    $$ 
        \Sigma \vdash \eta \lor (\eta \land \lnot \eta)
    $$ 
    which simplifies to 
    $$ 
        \Sigma \vdash \eta
    $$ 

    Since, from any contradiction the implication to some formula 
    is valid, we can pick any contradiction 
    to be provable from an inconsistent set of non-logical axioms. 
    It does not necessarily have to be 
    $[(\forall x) x = x] \land \lnot [(\forall x) x = x]$. 
    This equips us to do proofs by contradiction since 
    if we prove a contradiction our assumptions must be 
    inconsistent, which alows us to prove the 
    contradiction which simplifies to the thing we want to prove.
\end{proof}

\newpage
\section{3rd chapter, completeness and compactness}
\section{Tasks from 3.2.1}
\begin{prob}{Task 1}{}
    Suppose that $\Sigma$ is inconsistent and $\phi$ is an 
    $\mathfrak L$-formula. 
    Prove that $\Sigma \vdash \phi$.
\end{prob}
\begin{proof}
    Let $\Sigma$ be an inconsistent set of non-logical axioms 
    and $\phi$ an $\mathfrak L$-formula.

    Since $\Sigma$ is inconsistent we have 
    $$ 
        \Sigma \vdash \bot
    $$ 

    From our logical axioms $\Lambda$, we have
    $$ 
        \vdash \bot \rightarrow \phi
    $$ 
    for any $\mathfrak L$-formula $\phi$.

    Then, since 
    $$ 
        \Sigma \vdash \bot
    $$ 
    and 
    $$ 
        \bot \rightarrow \phi
    $$ 
    we have 
    $$ 
        \Sigma \vdash \phi
    $$ 
    through modus ponens.
\end{proof}

\begin{prob}{Task 2}{}
    Assume that 
    $\Sigma_0 \subseteq \Sigma_1 \subseteq \Sigma_2 \cdots$
    are such that each $\Sigma_i$ is a consistent set of 
    sentences in a language $\mathfrak L$. Show that 
    $\bigcup \Sigma_i$ is consistent.
\end{prob}
\begin{proof}
    We prove the claim by contradiction. 

    Suppose $\Gamma := \bigcup \Sigma_i$ is inconsistent. 
    Then there exists a finite subset $\Delta \subseteq \Gamma$. 
    But $\Delta$ is finite, so each sentence in $\Delta$ lies 
    in some $\Sigma_i$, and because $\Sigma_i$ forms an increasing 
    chain there is an index $N$ such that 
    $\Delta \subseteq \Sigma_N$. Thus $\Sigma_N$ is inconsistent 
    since it contains an inconsistent subset, contradicting 
    the assumption that each $\Sigma_i$ is consistent. 
    Therefore $\Gamma$ must be consistent.
\end{proof}

\end{document}
