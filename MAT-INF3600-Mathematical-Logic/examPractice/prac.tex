\documentclass[11pt]{article}
\usepackage[a4paper,margin=1in]{geometry}
\usepackage{fourier} % Fourier font
\usepackage{xcolor}
\usepackage{tikz}
\usepackage[most]{tcolorbox}
\usepackage{amsthm, amsmath, amssymb}
\usepackage{enumitem}
\usepackage{hyperref}
\usepackage[nameinlink,noabbrev]{cleveref}
\usepackage{titling} 

% Dark mode colors
\definecolor{bgcolor}{HTML}{FFFFFF}
\definecolor{textcolor}{HTML}{000000}
\definecolor{defcolor}{HTML}{E86873}
\definecolor{thmcolor}{HTML}{0A9396}
\definecolor{lemcolor}{HTML}{94D2BD}
\definecolor{corcolor}{HTML}{9B4AF7}
\definecolor{probcolor}{HTML}{EE9B00}
\definecolor{excolor}{HTML}{21E933}

% Background and text color
\pagecolor{bgcolor}
\color{textcolor}

% No paragraph indentation
\setlength{\parindent}{0pt}
\setlength{\parskip}{0.7em}

% Theorem box styles
\tcbset{
  enhanced,
  colback=bgcolor,
  colframe=thmcolor,
  coltext=white,
  coltitle=white,
  fonttitle=\bfseries,
  boxrule=0.7pt,
  left=1em,
  right=1em,
  top=0.7em,
  bottom=0.7em,
  before skip=10pt,
  after skip=10pt,
}

% Theorem environments with colored boxes
\newtcbtheorem[number within=section]{thm}{Theorem}{
  colframe=thmcolor, colback=thmcolor!15!bgcolor
}{thm} % The 'thm' here is the *prefix* for the label

\newtcbtheorem[number within=section]{defn}{Definition}{
  colframe=defcolor, colback=defcolor!15!bgcolor
}{def} % The 'def' here is the *prefix* for the label

\newtcbtheorem[number within=section]{lem}{Lemma}{
  colframe=lemcolor, colback=lemcolor!15!bgcolor
}{lem}

\newtcbtheorem[number within=section]{cor}{Corollary}{
  colframe=corcolor, colback=corcolor!15!bgcolor
}{cor}

\newtcbtheorem[number within=section]{prob}{Problem}{
  colframe=probcolor, colback=probcolor!15!bgcolor
}{prob}

\newtcbtheorem[number within=section]{ex}{Example}{
  colframe=excolor, colback=excolor!15!bgcolor
}{ex}

% Proof environment 
\renewenvironment{proof}[1][\proofname]{%
  \par\pushQED{\qed}\normalfont\topsep6pt \trivlist
  \item[\hskip\labelsep\itshape #1.]\ignorespaces
}{%
  \popQED\endtrivlist\addvspace{6pt}
}

% Cleveref name formats for tcolorbox environments
\crefname{thm}{theorem}{theorems}
\Crefname{thm}{Theorem}{Theorems}

\crefname{def}{definition}{definitions}
\Crefname{def}{Definition}{Definitions}

\crefname{lem}{lemma}{lemmas}
\Crefname{lem}{Lemma}{Lemmas}

\crefname{cor}{corollary}{corollaries}
\Crefname{cor}{Corollary}{Corollaries}

\crefname{prob}{problem}{problems}
\Crefname{prob}{Problem}{Problems}

\crefname{ex}{example}{examples}
\Crefname{ex}{Example}{Examples}


\title{\huge{MAT-INF3600 Exam Practice}}
\author{\LARGE{Thobias Høivik}}
\date{}

\begin{document}
\maketitle

\newpage 
\tableofcontents

\newpage
\section{Problem set}
\begin{prob}
    Let $\mathfrak L$ be the language with a binary relation 
    symbol $R$. 
    Consider the sentence $\phi$: 
    $$ 
        \forall x \lnot R(x,x) \land \forall x,y,z [R(x,y) \land 
        R(y,z) \rightarrow R(x,z)]
    $$ 

    Prove or disprove: If a structure $\mathcal M$ satisfies 
    $\phi$ with a finite domain $M$, then there must exist 
    $a \in M$ such that $(a,b) \not\in R, \forall b\in M$.
\end{prob}
\begin{proof}
    The claim is true. 
    Assume $\mathcal M \vDash \phi$ with $|M| = |\{m_1,\dots,m_n\}| 
    < \infty$.
    Suppose, for a contradiction, that there 
    is some $m_j$ for every $m_i$ such that $(m_i,m_j) \in R$.

    Then we have 
    $(m_{i_1}, m_{i_2}) \in R$. $R$ is irreflexive so 
    $m_{i_2}$ must relate to some element other than 
    $m_{i_1}$ (as otherwise we would get $(m_{i_1}, m_{i_1}) \in R$ 
    through transitivity), thus 
    $(m_{i_2}, m_{i_3}) \in R$.
    Extrapolating we get $(m_{i_{n-1}}, m_{i_n}) \in R$, but now 
    $m_{i_n}$ must relate to some element, but this would lead to 
    $(m_{i_n}, m_{i_n}) \in R$ through transitivity so 
    $\mathcal M \not\vDash \phi$, a contradiction.
\end{proof}

\begin{prob}
    Provide a formal derivation of: 
    $$
        (\lnot P \rightarrow \lnot Q) \vdash (Q \rightarrow P)
    $$ 
\end{prob}
\begin{proof}
    The deduction, using L \& K's system, is trivial. 

    Let $\Sigma = \{\lnot P \rightarrow \lnot Q\}$.

    \begin{align*}
        &1. \lnot P \rightarrow \lnot Q & (\Sigma) \\ 
        &2. Q \rightarrow P & 1, (PC) \\ 
    \end{align*}

    We could also use the deduction theorem where we have 
    $\phi \vdash \psi$ if and only if $\vdash \phi \rightarrow \psi$ 
    for formula $\psi$ and sentence $\phi$.
\end{proof}

\newpage
\begin{prob}
    \label{prob:set1_prob3}
    Let $\Sigma$ be a set of first-order sentences. Suppose 
    that for every natural number $n$, there exists a model 
    $\mathcal M_n$ of $\Sigma$ such that $\mathcal M_n$ has 
    $n$ elements. Prove $\Sigma$ has an infinite model.  
\end{prob}

\begin{proof}
    Assume $\mathcal M_n \vDash \Sigma, |M| = n$ for every 
    $n \in \mathbb N$.

    Consider the expanded set $\Sigma' \supseteq \Sigma$ defined as 
    $$ 
        \Sigma' := \Sigma \cup 
        \{\exists x_1,\dots,\exists x_n 
        \bigwedge_{i < j} x_i \neq x_j 
        : n \in \mathbb N \}
    $$ 

    This set is essentially $\Sigma$ as well as 
    "there are at least $n$ elements" (for every $n$). 
    Consider the set
    $$
        \Delta \subset \Sigma' 
    $$
    
    such that $\phi_k \in \Delta$ where $\phi_k$ is the 
    statement "there are at least $k$ elements". 
    It is then straightforward that every finite 
    subset of $\Sigma'$ is satisfiable so $\Sigma'$ is as well, 
    and in particular, it's model is infinite. 
    $\Sigma' \supseteq \Sigma$ so this model also satisfies 
    $\Sigma$. 
\end{proof} 

\newpage 
\section{Problem set}
\begin{prob}
    Show that the class of all finite structures (in a language 
    with at least one relation) is not first-order axiomatizable.
\end{prob}
\begin{proof}[Sketch proof]
    Classifying all finite structures includes 
    structures that are arbitrarily large. 
    It is straightforward to show that if some set of sentences 
    has arbitrarily large finite models then it also has 
    an infinite model.  

    As seen in \cref{prob:set1_prob3}, defining "at least $n$ elements"
    for every natural $n$
    and having that sentence satisfied for every $n$ means 
    that it is satisfied by an infinite structure. 
\end{proof}



\end{document}
