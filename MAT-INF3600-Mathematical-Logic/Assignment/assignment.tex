\documentclass[11pt]{article}
\usepackage[a4paper,margin=1in]{geometry}
\usepackage{fourier} % Fourier font
\usepackage{xcolor}
\usepackage{tikz}
\usepackage[most]{tcolorbox}
\usepackage{amsthm, amsmath, amssymb}
\usepackage{enumitem}
\usepackage{hyperref}
\usepackage[nameinlink,noabbrev]{cleveref}
\usepackage{titling} 

% Dark mode colors
\definecolor{bgcolor}{HTML}{FFFFFF}
\definecolor{textcolor}{HTML}{000000}
\definecolor{defcolor}{HTML}{E86873}
\definecolor{thmcolor}{HTML}{0A9396}
\definecolor{lemcolor}{HTML}{94D2BD}
\definecolor{corcolor}{HTML}{9B4AF7}
\definecolor{probcolor}{HTML}{EE9B00}
\definecolor{excolor}{HTML}{21E933}

% Background and text color
\pagecolor{bgcolor}
\color{textcolor}

% No paragraph indentation
\setlength{\parindent}{0pt}
\setlength{\parskip}{0.7em}

% Theorem box styles
\tcbset{
  enhanced,
  colback=bgcolor,
  colframe=thmcolor,
  coltext=white,
  coltitle=white,
  fonttitle=\bfseries,
  boxrule=0.7pt,
  left=1em,
  right=1em,
  top=0.7em,
  bottom=0.7em,
  before skip=10pt,
  after skip=10pt,
}

% Theorem environments with colored boxes
\newtcbtheorem[number within=section]{thm}{Theorem}{
  colframe=thmcolor, colback=thmcolor!15!bgcolor
}{thm} % The 'thm' here is the *prefix* for the label

\newtcbtheorem[number within=section]{defn}{Definition}{
  colframe=defcolor, colback=defcolor!15!bgcolor
}{def} % The 'def' here is the *prefix* for the label

\newtcbtheorem[number within=section]{lem}{Lemma}{
  colframe=lemcolor, colback=lemcolor!15!bgcolor
}{lem}

\newtcbtheorem[number within=section]{cor}{Corollary}{
  colframe=corcolor, colback=corcolor!15!bgcolor
}{cor}

\newtcbtheorem[number within=section]{prob}{Problem}{
  colframe=probcolor, colback=probcolor!15!bgcolor
}{prob}

\newtcbtheorem[number within=section]{ex}{Example}{
  colframe=excolor, colback=excolor!15!bgcolor
}{ex}

% Proof environment 
\renewenvironment{proof}[1][\proofname]{%
  \par\pushQED{\qed}\normalfont\topsep6pt \trivlist
  \item[\hskip\labelsep\itshape #1.]\ignorespaces
}{%
  \popQED\endtrivlist\addvspace{6pt}
}

% Cleveref name formats for tcolorbox environments
\crefname{thm}{theorem}{theorems}
\Crefname{thm}{Theorem}{Theorems}

\crefname{def}{definition}{definitions}
\Crefname{def}{Definition}{Definitions}

\crefname{lem}{lemma}{lemmas}
\Crefname{lem}{Lemma}{Lemmas}

\crefname{cor}{corollary}{corollaries}
\Crefname{cor}{Corollary}{Corollaries}

\crefname{prob}{problem}{problems}
\Crefname{prob}{Problem}{Problems}

\crefname{ex}{example}{examples}
\Crefname{ex}{Example}{Examples}


\title{\huge{MAT-INF3600 Assignment}}
\author{\LARGE{Thobias Høivik}}
\date{}

\begin{document}
\maketitle

\newpage
\section*{Problem 1}

\begin{proof}[Solution for (a)]
    Let $A = \{0,1\}$, $c^\mathfrak A$ be any element in $A$ and 
    $R^\mathfrak A(x,y)$ if $x = y$. Let $f^\mathfrak A(x) = x$ 
    be the identity function on $A$.
    
    Then (i) is satisfied and (ii), $(\forall x)[R(x,f(x)]$ 
    is satisfied.
\end{proof}

\begin{proof}[Solution for (b)]
    Let $A = \{0,1\}$, $c^\mathfrak A$ be any element in $A$ and 
    $R^\mathfrak A(x,y)$ if $x < y$ with $f^\mathfrak A(x) = x$ 
    as before. Then, we get a $2$ element universe, 
    but $R(x,f(x))$ is not satisfied for any $x$ since 
    $0$ is not less than itself and $1$ is not less than itself.
\end{proof}

\begin{proof}[Solution for (c)]
    Now we must construct a $5$ element universe with 
    an injective function. 
    
    Let $A = \{1,\dots,5\}$ with $c^\mathfrak A = 1$ 
    (or any other choice), $R = \emptyset$ (again, arbitrary) 
    and $f(5) = 1$, $f(x) = x + 1$ otherwise.
\end{proof}

\begin{proof}[Solution and proof of (d)]
    Let $A = \mathbb N$ ($0$ included) with $c^\mathfrak A = 0$, 
    $R = \emptyset$ (choice is arbitrary) 
    and $f^\mathfrak A(n) = n + 1$ 
    be the successor function. 

    Then $\forall x[f(x) \neq c]$ since $0$ is not the successor 
    of any natural number. 
    Furthermore the second condition 
    of $f$ being injective is satisfied since 
    \begin{align*}
        f(x) &= f(y) \\ 
        \rightarrow x + 1 &= y + 1 \\ 
        \rightarrow x &= y
    \end{align*}

    Hence $\mathfrak A \vDash \Gamma$. 

    Now to prove that any model of $\Gamma$ has an infinite 
    universe. 
    
    Suppose we have some model of $\Gamma$ with a finite universe 
    $A = \{c^\mathfrak A, x_1, x_2, \dots, x_n\}$. 
    We require $f:A \to A \setminus \{c^\mathfrak A\}$ and for 
    it to be injective. Since $A$ is finite we have an injective map 
    from a set of size $n + 1$ to a set of size $n$ which is not 
    possible by the pigeonhole principle, thus we arrive at 
    a contradiction.
     
    To vizualize this more clearly we can attempt to construct 
    an injection $f:A \to A$. 
    \begin{align*}
        f^\mathfrak A(c^\mathfrak A) &= x_{i_1} \text{ where } 
        x_{i_1} \neq c^\mathfrak A \\ 
        f^\mathfrak A(x_1) &= x_{i_2} \text{ where } x_{i_2} 
        \neq x_{i_1}, \text{ and } 
        x_{i_2} \neq c \\ 
                           &\vdots \\ 
        f^\mathfrak A(x_{n-1}) &= x_{i_n} \text{ where } 
        x_{i_n} \neq x_{i_{n-1}} 
        ,\dots, x_{i_1}, \text{ and } x_{i_n} \neq c^\mathfrak A
    \end{align*}
    But now we arrive at $f^\mathfrak A(x_n)$ which 
    cannot go to $c^\mathfrak A$
    as that violates $f(x) \neq c$ and $f^\mathfrak A(x_n)$ 
    cannot go to any 
    $x_i$ as that would violate injectivity. So 
    we cannot construct a well-defined injection that satisfies 
    $f(x) \neq c$ for all $x$ given a finite universe.

    Hence any model of $\Gamma$ necessarily has an infinite universe.
\end{proof}

\newpage 
\section*{Problem 2}
\begin{proof}
    Let $n \geq 1$ and let $\theta_1, \dots, \theta_n$ be sentences. 
    Let $\Sigma$ be a set of formulas. We will prove, by induction, 
    that $\Sigma \cup \{\theta_1, \dots, \theta_n\} \vdash \phi$ 
    if and only if $\Sigma \vdash \theta_1 \land \dots \land \theta_n 
    \rightarrow \phi$.

    \emph{Base case $n = 1$}.

    $\Sigma \cup \theta \vdash \phi$ if any only if 
    $\Sigma \vdash \theta \rightarrow \phi$, by 
    \textbf{Theorem 2.7.4 (The Deduction Theorem)}. 

    Assume that for $n = k$. 
    Now look at $n = k + 1$,
    \begin{align*}
        \Sigma \cup \{\theta_1,\dots,\theta_k, \theta_{k + 1}\} 
        &\vdash \phi \\ 
        \Sigma \cup \{\theta_1,\dots,\theta_n\} 
        &\vdash \theta_{k+1} \rightarrow \phi 
        \text{ by the regular Deduction Theorem}
        \\ 
        \Sigma 
        &\vdash \left[\bigwedge_{i = 1}^k \theta_i\right] \rightarrow 
        (\theta_{k+1} \rightarrow \phi)
        \text{ by assumption} 
        \\ 
        \alpha \rightarrow (\beta \rightarrow \gamma) 
        & \text{ equivalent to }
        (\alpha \land \beta) \rightarrow \gamma
        \\ 
        \text{Hence } 
        \Sigma &\vdash \left[\bigwedge_{i=1}^{k+1=n}
        \theta_i \right] \rightarrow \phi
    \end{align*}

    Now to show the implication in the other direction: 
    \begin{align*}
        \Sigma &\vdash \left[\bigwedge_{i=1}^{k+1=n} \theta_i\right]
        \rightarrow \phi
        \\ 
        \Sigma &\vdash \left[\bigwedge_{i=1}^k \theta_i\right]
        \rightarrow (\theta_{k+1} \rightarrow \phi) \\ 
        \Sigma \cup \{\theta_1,\dots,\theta_k\} 
               &\vdash \theta_{k+1} \rightarrow \phi 
               \text{ by assumption } \\ 
        \Sigma \cup \{\theta_1,\dots,\theta_{k+1}\} 
               &\vdash \phi \text{ by the regular Deduction Theorem}
    \end{align*}

    This completes the proof.
\end{proof}

\newpage 
\section*{Problem 3}
\begin{proof}[Proof of (a)]
    \begin{align*}
        &1. \forall x [Rx \rightarrow Sx]  \\ 
        &2. \forall x [Rx \rightarrow Sx] \rightarrow 
        (Rt \rightarrow St) &(Q1) \\ 
        &3. (Rt \rightarrow St) \rightarrow (\lnot St \rightarrow 
        \lnot Rt) &(PC) \\ 
        &4. \forall x[Rx \rightarrow Sx] \rightarrow 
        (\lnot St \rightarrow \lnot Rt) &2,3 \quad (PC) \\
        &5. \forall x [Rx \rightarrow Sx] \rightarrow 
        \forall y [\lnot Sy \rightarrow \lnot Ry] &4, \quad (QR) \\
        &6. \forall y [\lnot Sy \rightarrow \lnot Ry] &1,5 \quad (PC) 
    \end{align*}
    Thus we have a deduction of 
    $\forall y[\lnot Sy \rightarrow \lnot Ry]$ from 
    $\forall x[Rx \rightarrow Sx]$.
\end{proof}

\begin{proof}[Proof of (b)]
    First we recognize that $\phi \not\vdash \psi$ if and only 
    if $\{\phi, \lnot \psi\}$ is satisfiable. 

    Let $\mathfrak A$ be a structure with universe 
    $A = \{a,b\}$, $R^\mathfrak A = \{a\}$, 
    $S^\mathfrak A = \{a,b\}$. 

    Then if $t = a$, $Ra$ is true and $Sa$ is true so 
    $Rx \rightarrow Sx$ is true. 
    If $t = b$, $Rb$ is false so the implication is true regardless. 
    Therefore $\mathfrak A \vDash \forall x[Rx \rightarrow Sx]$.

    Now to check the other formula. If $t = b$ we have 
    $Sb$, but we do not have $Rb$. Hence the implication does 
    not hold for all terms and 
    $\mathfrak A \not\vDash \forall y[Sy \rightarrow Ry]$.
\end{proof}

\newpage 
\section*{Problem 4}
\begin{proof}[Proof of (a)]
    We will appeal to Lemma 1 by showing that the negation 
    leads to the given contradiction. 
    \begin{align*}
        &1. \exists x (R(x)) \land \forall x (\lnot R(x) \land 
        \lnot S(x)) 
        \\ 
        &2. \exists x (R(x)) &1 \quad (PC) 
        \\ 
        &3. \forall x (\lnot R(x) \land \lnot S(x)) &1 \quad (PC)
        \\ 
        &4. \forall x (\lnot R(x) \land \lnot S(x))
        \rightarrow (\lnot R(x) \land \lnot S(x)) & (Q1)
        \\ 
        &5. \lnot R(x) \land \lnot S(x) & 3,4 \quad (PC)
        \\ 
        &6. \lnot R(x) & 5 \quad (PC)
        \\ 
        &7. \lnot R(x) \rightarrow (R(x) \rightarrow 
        \lnot (\forall x [x = x]) & (PC)
        \\
        &8. R(x) \rightarrow \lnot(\forall x [x = x]) &6,7 \quad (PC)
        \\ 
        &9. \exists x (R(x)) \rightarrow \lnot(\forall x [x = x])
        & 8 \quad (QR)
        \\ 
        &10. \lnot(\forall x [x = x]) &2,9 \quad (PC) 
        \\
        &11. \forall x (x = x) & (E1)
        \\ 
        &12. \bot &10,11 \quad (PC)
    \end{align*}

    Therefore 
    $$ 
        \vdash \exists x (R(x)) \rightarrow [\exists x(R(x) 
        \lor S(x))]
    $$ 
\end{proof}
\begin{proof}[Proof of (b)]
    Once again, we will appeal to Lemma 1 by showing that 
    $$ 
        \forall x (R(x) \land \exists x \lnot R(x)) 
        \vdash \bot
    $$ 
    \begin{align*}
        &1. \forall x (R(x) \land \exists x \lnot R(x))
        \\ 
        &2. \forall x (R(x) \land \exists x \lnot R(x))
        \rightarrow R(x) \land \exists x \lnot (R(x)) 
        & (Q1)
        \\ 
        &3. R(x) \land \exists x \lnot (R(x))
        & 1,2 \quad (PC)
        \\ 
        &4. R(x) 
        & 3 \quad (PC)
        \\ 
        &5. \exists \lnot (R(x)) 
        & 3 \quad (PC)
        \\ 
        &6. R(x) \rightarrow (\lnot R(x) \rightarrow 
        \lnot [\forall x (x=x)]) 
        & (PC)
        \\
        &7. \lnot R(x) \rightarrow \lnot [\forall x (x=x)]
        & 4,6 \quad (PC)
        \\ 
        &8. \exists x (\lnot R(x)) \rightarrow 
        \lnot [\forall x (x = x)] 
        & 7 \quad (QR)
        \\ 
        &9. \lnot [\forall x (x = x)]
        & 5,8 \quad (PC)
        \\ 
        &10. \forall x (x = x) 
        & (E1)
        \\ 
        &11. \bot 
        & 9,10 \quad (PC)
    \end{align*}
\end{proof}

\newpage 
\section*{Problem 5}
\begin{proof}
    We proceed by structural induction on $t$. 

    \textbf{Base case $(t = 0)$}. 

    Then there is $p = 0$, $T \vdash 0 = 0$.

    Assume that for any variable-free $\mathfrak L$-term $t$ 
    of height $n$, 
    we can find some prime term $p$ such that 
    $$ 
        T \vdash t = p
    $$ 

    \textbf{Inductive step}.

    Now we look at variable-free terms of height $n + 1$.

    \emph{Case 1 (f)}.
    Suppose $t = f(t_1, t_2)$ is variable-free. Then 
    $t_1,t_2$ are variable free $\mathfrak L$-terms. 
    By our hypothesis there exist prime terms $p_1,p_2$ such that 
    $$ 
        T \vdash p_1 = t_1, \quad T \vdash p_2 = t_2
    $$ 

    By $E2$ we have 
    $$ 
        p_1 = t_1 \land p_2 = t_2 \rightarrow 
        f(t_1,t_2) = f(p_1,p_2)
    $$ 

    By the inductive definition of prime terms, 
    $f(p_1, p_2)$ is a prime term and we have 
    $$ 
        T \vdash t = p
    $$ 
    where $t = f(t_1, t_2)$ and $p = f(p_1,p_2)$.

    \emph{Case 2 (v)}. 
    Suppose $t = v(t')$ is variable-free. Then $t'$ is 
    variable-free. By our hypothesis there exists a prime 
    term $p'$ such that 
    $$ 
        T \vdash p' = t'
    $$ 

    $p'$ is prime so it is the constant $0$ or of the form 
    $f(p'_1, p'_2)$ where $p'_1, p'_2$ are prime. 
    If $p'$ is $0$ we have (by $E2$) 
    $$ 
        v(t') = v(0) = 0
    $$ 
    which is a prime term. Otherwise if $p' = f(p'_1, p'_2)$ then 
    $$ 
        v(t') = v(f(p'_1, p'_2)) = p'_1
    $$
    by $T_3$ (and $E2$), which is prime.

    In other words there exists, in both cases, prime $p$ such that 
    $$ 
        T \vdash t = p
    $$ 

    \emph{Case 3 (h)}. 
    Suppose $t = h(t')$ is variable-free. Then $t'$ is 
    variable-free. By our hypothesis there exists a prime 
    term $p'$ such that 
    $$ 
        T \vdash p' = t'
    $$ 

    $p'$ is prime so it is the constant $0$ or of the form 
    $f(p'_1, p'_2)$ where $p'_1, p'_2$ are prime. 
    If $p'$ is $0$ we have (by $E2$) 
    $$ 
        h(t') = h(0) = 0
    $$ 
    which is a prime term. Otherwise if $p' = f(p'_1, p'_2)$ then 
    $$ 
        h(t') = h(f(p'_1, p'_2)) = p'_2
    $$
    by $T_4$ (and $E2$), which is prime.

    In other words there exists, in both cases, prime $p$ such that 
    $$ 
        T \vdash t = p
    $$ 
\end{proof}



\end{document}
