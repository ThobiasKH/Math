\documentclass[11pt]{article}
\usepackage[a4paper,margin=1in]{geometry}
\usepackage{fourier} % Fourier font
\usepackage{xcolor}
\usepackage{tikz}
\usepackage[most]{tcolorbox}
\usepackage{amsthm, amsmath, amssymb}
\usepackage{enumitem}
\usepackage{hyperref}
\usepackage[nameinlink,noabbrev]{cleveref}
\usepackage{titling} 

% Dark mode colors
\definecolor{bgcolor}{HTML}{FFFFFF}
\definecolor{textcolor}{HTML}{000000}
\definecolor{defcolor}{HTML}{E86873}
\definecolor{thmcolor}{HTML}{0A9396}
\definecolor{lemcolor}{HTML}{94D2BD}
\definecolor{corcolor}{HTML}{9B4AF7}
\definecolor{probcolor}{HTML}{EE9B00}
\definecolor{excolor}{HTML}{21E933}

% Background and text color
\pagecolor{bgcolor}
\color{textcolor}

% No paragraph indentation
\setlength{\parindent}{0pt}
\setlength{\parskip}{0.7em}

% Theorem box styles
\tcbset{
  enhanced,
  colback=bgcolor,
  colframe=thmcolor,
  coltext=white,
  coltitle=white,
  fonttitle=\bfseries,
  boxrule=0.7pt,
  left=1em,
  right=1em,
  top=0.7em,
  bottom=0.7em,
  before skip=10pt,
  after skip=10pt,
}

% Theorem environments with colored boxes
\newtcbtheorem[number within=section]{thm}{Theorem}{
  colframe=thmcolor, colback=thmcolor!15!bgcolor
}{thm} % The 'thm' here is the *prefix* for the label

\newtcbtheorem[number within=section]{defn}{Definition}{
  colframe=defcolor, colback=defcolor!15!bgcolor
}{def} % The 'def' here is the *prefix* for the label

\newtcbtheorem[number within=section]{lem}{Lemma}{
  colframe=lemcolor, colback=lemcolor!15!bgcolor
}{lem}

\newtcbtheorem[number within=section]{cor}{Corollary}{
  colframe=corcolor, colback=corcolor!15!bgcolor
}{cor}

\newtcbtheorem[number within=section]{prob}{Problem}{
  colframe=probcolor, colback=probcolor!15!bgcolor
}{prob}

\newtcbtheorem[number within=section]{ex}{Example}{
  colframe=excolor, colback=excolor!15!bgcolor
}{ex}

% Proof environment 
\renewenvironment{proof}[1][\proofname]{%
  \par\pushQED{\qed}\normalfont\topsep6pt \trivlist
  \item[\hskip\labelsep\itshape #1.]\ignorespaces
}{%
  \popQED\endtrivlist\addvspace{6pt}
}

% Cleveref name formats for tcolorbox environments
\crefname{thm}{theorem}{theorems}
\Crefname{thm}{Theorem}{Theorems}

\crefname{def}{definition}{definitions}
\Crefname{def}{Definition}{Definitions}

\crefname{lem}{lemma}{lemmas}
\Crefname{lem}{Lemma}{Lemmas}

\crefname{cor}{corollary}{corollaries}
\Crefname{cor}{Corollary}{Corollaries}

\crefname{prob}{problem}{problems}
\Crefname{prob}{Problem}{Problems}

\crefname{ex}{example}{examples}
\Crefname{ex}{Example}{Examples}


\title{\huge{Oblig 1}}
\author{\LARGE{Thobias Høivik}}
\date{}

\begin{document}
\maketitle

\newpage
\begin{prob*}[1]
    Let $(X,d)$ be a metric space and $K \subseteq X$ a subset. 
    A function $f : K \to \mathbb R$ is \emph{lower semicontinuous}
    if for all $x \in K$ and $\varepsilon > 0$, there exists 
    $\delta > 0$ such that 
    \[
        y \in K, \: d(x,y) < \delta \Rightarrow f(x) < 
        f(y) + \varepsilon
    \]
    The goal of this problem is to show that if $f : K \to \mathbb R$
    is lower semicontinuous and $K$ is compact, then $f$ attains a 
    minimum, i.e. there is some $\overline x\in K$ such that 
    \[
        f(\overline x) \leq f(x) \qquad \forall x \in K.
    \] 
    You may proceed as follows: 

    (a) Let $m := \inf\{f(x) : x \in K\}$. Argue that there exists 
    a \emph{minimizing sequence} for $f$, i.e. a sequence 
    $(x_n)_{n \in \mathbb N}$ in $K$ such that $f(x_n) \to m$ as 
    $n \to \infty$.

    Assume from now that $K$ is compact. 

    (b) Show that there is a subsequence $(x_{n_k})_{k\in\mathbb N}$ 
    and some $\overline x \in K$ such that $x_{n_k}\to \overline x$ 
    as $k \to \infty$. 

    (c) Show that $m \leq f(\overline x) \leq m + \varepsilon$
    for any $\varepsilon > 0$, and conclude that $\overline x$ is 
    a minimum of $f$.
\end{prob*}
\begin{proof}[Proof of (a)]
    Let $f : K \to \mathbb R$ be some function and let 
    $m := \inf\{f(x) : x \in K\}$. 

    By the definition of the infimum we have that, for every 
    $\varepsilon > 0$ there exists $x \in K$ such that
    \[
        f(x) \leq m + \varepsilon
    \]
    In particular for $\varepsilon := \frac{1}{n}$ this must be the 
    case. Then, for $n \in \mathbb N$ there exists $x_n \in K$ such 
    that 
    \[
        f(x_n) \leq m + \frac{1}{n}
    \]
    but $m$ is a lower bound for all $f(x_n)$ hence 
    \[
        m \leq f(x_n) \leq m + \frac{1}{n}
    \]
    so 
    \[
        f(x_n) \underset{n \to \infty}{\to} m 
    \] 
\end{proof}
\begin{proof}[Proof of (b)]
    $(x_n)_{n\in\mathbb N}$ in $K$ as described in proof of (a). 
    Since $K$ is compact, $(x_n)_{n\in\mathbb N}$ has a convergent 
    subsequence which converges to some point $\overline x \in K$.
\end{proof}
\begin{proof}[Proof of (c)]
    By definition of $m$ as the infimum of all values of $f$ we 
    can see that 
    \[
        m \leq f(\overline x)
    \]
    By the virtue of $f$ being lower semicontinuous we may also 
    conclude that for any $\varepsilon > 0$ there is $\delta > 0$
    such that 
    \[
        d(\overline x, y) < \delta \Rightarrow f(\overline x) < 
        f(y) + \varepsilon
    \]
    In particular, since $x_{n_k} \to \overline x$, we can choose 
    $k$ large enough so that $d(\overline x, x_{n_k}) < \delta$. 
    I.e. 
    \[
        d(\overline x, x_{n_k}) < \delta \Rightarrow f(\overline x) 
        < f(x_{n_k}) + \varepsilon
    \]
    As $k \to \infty$ we know that $f(x_{n_k}) \to m$ so taking 
    the limit we get 
    \[
        m \leq f(\overline x) \leq m + \varepsilon
    \]
    Taking $\varepsilon \to 0$ we conclude that 
    \[
        m \leq f(\overline x) \leq m
    \]
    so $\overline x$ is a minimum of $f$.
\end{proof}

\begin{prob*}[2]
    Let $(X,d)$ be a metric space and $E \subseteq X$ any subset. 
    Let $f : E \to \mathbb R$ be any bounded function (i.e. 
    there is some $M \in \mathbb R$ such that $|f(x)| \leq M$ for 
    all $x \in E$). Define $g : \overline E \to \mathbb R$ by 
    \[
        g(x) := \underset{\underset{y \in E}{y \to x}}{\lim \inf} f(y)
        \qquad \forall x \in \overline E,
    \]
    that is, 
    \[
        g(x) := \lim_{n\to\infty}f_n(x), \qquad \text{where} \ 
        f_n(x) := \inf\{f(y) : y \in E \cap B(x;1/n)\}.
    \]
    (Here, $\overline E := E \cup \partial E$ is the closure of $E$.)
    The function $g$ is called the \emph{lower semicontinuous envelope}
    of $f$. 

    (a) Show that $g$ is well-defined, i.e. that the above limit 
    exists, and that 
    \[
        g(x) \leq f(x)
    \]
    for all $x \in E$. 

    (b) Show that $g$ is lower semicontinuous. 

    (c) Show that if $f$ is lower semicontinuous on $E$, then 
    $g(x) = f(x)$ for all $x \in E$.
\end{prob*}
\begin{proof}[Proof of (a)]
    Denote $\xi_{x,n} := E \cap B(x;1/n)$. It is easy to see that 
    $\xi_{x,n+1} \subseteq \xi_{x,n}$ (and that they are 
    nonempty for all $n\in\mathbb N$). Notice then that 
    \[
        f_n(x) \leq f_{n+1}(x) \leq f_{n+2}(x) \leq \cdots \leq M
    \]
    I.e. we get a monotone increasing sequence which is bounded 
    above, which converges. 

    It is quite clear that for $x \in E$ 
    \[
        \lim_{n\to\infty}f_n(x) \leq f(x)
    \]
    as for every $n$ $x \in B(x;\frac{1}{n})$ because 
    $d(x,x) = 0 < 1/n$ and by the definition of the infimum 
    $f_n(x) = \inf\{f(y) : y \in \xi_{x,n}\}$ is a lower bound for 
    the value of every specific $x \in \xi_{x,n}$. 
\end{proof}
\begin{proof}[Proof of (b)]
    We wish to show that for every $x \in \overline E$ and 
    $\varepsilon > 0$ we can find 
    $\delta > 0$ such that 
    \[
        z \in \overline E, \; d(x,z) < \delta \Rightarrow 
        g(x) < g(z) + \varepsilon
    \]
    Let $x \in \overline E$ and $\varepsilon > 0$. Since 
    $g(x) = \lim_{n\to\infty}f_n(x)$ and the sequence of $f_n(x)$ is 
    non-decreasing, we have $f_n(x) \leq g(x)$ for all 
    $n \in \mathbb N$. Thus there exists $N \in \mathbb N$ such that 
    \[
        g(x) < f_N(x) + \varepsilon
    \]
    Recall that $f_N(x) = \inf\{f(y) : y \in E \cap B(x;1/N)\}$. 
    Choose $\delta > 0$ such that $B(z;\delta) \subseteq B(x;1/N)$. 
    For any $M > 1/\delta$ we have $B(z;1/M) \subseteq B(z;\delta) 
    \subseteq B(x;1/N)$. As we said in (a), the infimum over a larger
    set is less than or equal to the infimum over a subset. 
    Therefore 
    \[
        f_N(x) \leq f_M(z)
    \]
    As $m \to \infty$ we get 
    \[
        f_N(x) \leq g(z)
    \]
    Recalling that 
    \[
        g(x) < f_N(x) + \varepsilon
    \]
    we finally conclude that 
    \[
        g(x) < g(z) + \varepsilon
    \]
\end{proof}
\begin{proof}[Proof of (c)]
    Let $x \in E \subseteq \overline E$. If $f$ is lower 
    semicontinuous then $\forall \varepsilon > 0, \ \exists \delta 
    > 0$ s.t. 
    \[
        z \in E, \: d(x,z) < \delta \Rightarrow f(x) < f(y) +  
        \varepsilon \Rightarrow f(x) \leq f(y) + \varepsilon
    \]
    In particular this holds for every $z \in E \cap B(x;1/N)$ (assume 
    $N \geq 1/\delta$), so it holds for the infimum as well, i.e. 
    \[
        d(x,z) < \delta \Rightarrow f(x) \leq f_n(x) + \varepsilon, \: 
        n \geq N
    \]
    Once again, nondecreasing sequence so take the limit to get 
    \[
        f(x) \leq g(x) + \varepsilon
    \]
    Together with the results from (a) and (b) we conclude the 
    desired result. 
\end{proof}
\newpage 
\begin{prob*}[3]
    Fix some $m \in \mathbb N$. We equip $\mathbb R^m$ with the 
    metric given by the euclidian norm, $d(x,y) = ||x,y||$, where 
    $||z|| = \sqrt{\sum_{n=1}^m|z_n|^2}$ (for $x,y,z \in \mathbb R^m$).
    Define 
    \begin{align*}
        X&:= \text{Lip}([0,1],\mathbb R^m) = \{x:[0,1] \to \mathbb 
        R^m : x \ \text{is Lipschitz continuous}\},\\
            Y&:= C^1([0,1],\mathbb R^m) \\ 
             &= \left\{x:[0,1] \to \mathbb 
        R^m : \text{both} \ x \ \text{and} \ \frac{dx}{dt} \ 
        \text{exist and are continuous}\right\}.
    \end{align*}
    We equip $X$ with the supremum metric, 
    \[
        \rho(x,y) := \sup_{t\in[0,1]}|x(t) - y(t)| \quad 
        \text{for} \ x,y \in X.
    \]
    Fix $x_0, x_1 \in \mathbb R$, let $c > 0$, and define 
    \[
        E_c := \left\{x \in Y : x(0) = x_0, x(1) = x_1, 
        \left|\left|\frac{dx}{dt}(t)\right|\right| \leq c \; \forall 
        t \in [0,1]
        \right\}.
    \]

    (a) Show that 
    \begin{enumerate}
        \item every $x \in E_c$ has Lipschitz constant at most $c$, 
        \item $E_c$ is a subset of $X$, 
        \item $E_c$ is non-empty when $c$ is large enough.
    \end{enumerate}

    (b) Show that $E_c$ is a \emph{precompact} subset of $X$, i.e., 
    every sequence $(x_n)_{n\in\mathbb N}$ in $E_c$ has a subsequence 
    that converges to a point in $X$. 

    (c) For some continuous, bounded function $v : \mathbb R^m 
    \to [0,\infty)$, define $f:E_c \to \mathbb R$ by 
    \[
        f(x) := \displaystyle\int_{0}^{1}v(x(t))\left|\left|
        \frac{dx}{dt}(t) \right|\right| \; dt \qquad \text{for} \ x 
        \in E_c
    \]
    Show that $f$ is bounded on $E_c$. 

    (d) It is a fact that $f$ is not continuous on $E_c$, and 
    $E_c$ is not compact, so we cannot apply the Extreme Value 
    Theorem to deduce the existence of a minimum of $f$. 

    It \emph{can be shown} that $f$ is lower semicontinuous
    (but you are not asked to show this here). Assuming that this is 
    true, show that there exists some $\overline x \in \overline{E_c}$
    which "minimizes" $f$, in the sense that $\overline x$ is a 
    minimum of the lowest semicontinuous envelope $g$ of $f$, 
    and that if $\overline x \in E_c$, then $\overline x$ is 
    a minimum of $f$. 
\end{prob*}
\newpage 
\begin{proof}[Proof of (a)]
    We begin with the second claim. 

    \textbf{Claim 2.}

    We are essentially asked to 
    prove that $x \in E_c$ is Lipschitz. There is, I belive, a theorem 
    in the book which explicitly states that functions from subsets 
    of the reals to the reals which have bounded derivatives 
    are Lipschitz, but that is not entirely applicable. 
    Assume, for a contradiction, that there exists $x \in E_c$ which 
    is not in $X$, i.e. $x$ is not Lipschitz. 

    By definition $x$ and $x'$ are continuous with the latter being 
    bounded by $c \in \mathbb R^+$. We're assuming non-Lipschitz 
    so for every $C > 0$, there are points $\alpha_C, \beta_C \in 
    [0,1]$ such that 
    \[
        ||x(\alpha_C)-x(\beta_C)|| > C|\alpha_C - \beta_C|
    \]
    In particular there exist exist points $\alpha_c, \beta_c$ such  
    that 
    \[
        ||x(\alpha_c)-x(\beta_c)|| > c|\alpha_c - \beta_c|
    \]
    so 
    \[
        \frac{||x(\alpha_c) - x(\beta_c)||}{|\alpha_c - \beta_c|}
        > c
    \]
    but as $dx/dt \leq c$ for every $t \in [0,1]$ it must be the 
    case that 
    \[
        c \geq 
        \frac{||x(\alpha_c) - x(\beta_c)||}{|\alpha_c - \beta_c|}
    \]
    In particular, $c > c$, a contradiction. 
    Therefore our assumption that there exists $x \in E_c$ not 
    Lipschitz is wrong and thus every $x \in E_c$ is Lipschitz. 
    Summing up we get that 
    \[
        E_c \subseteq X
    \]
    
    Now, I believe, \textbf{Claim 1} follows as we just saw that 
    the assumption that $||x(\alpha_c) - x(\beta_c)|| > 
    c |\alpha_c - \beta_c|$ was wrong. 
    If not then the this should suffice: 
    \begin{align*}
        x(t) - x(s) &= \displaystyle\int_{s}^{t}x'(r)\;dr \\   
        |x(t) - x(s)| &\leq \displaystyle\int_s^t |x'(r)|\;dr \\ 
                      &\leq \displaystyle\int_s^t c \; dr \\ 
                      &= c|t-s|
    \end{align*}
    
    \textbf{Claim 3.}

    Let $x(t) = x_0 + t(x_1-x_0)$. Then $x'(t) = x_1 - x_0$ so 
    \[
        ||x'(t)|| = ||x_1-x_0||. 
    \]
    So if $c \geq ||x_1-x_0||$ then this function would lie in 
    $E_c$, which would then be nonempty. 
\end{proof}
\begin{proof}[Proof of (c)]
    We assume $v : \mathbb R^m \to [0,\infty)$ is continuous 
    and bounded, i.e. $0 \leq v(x) \leq M$. 
    \begin{align*}
        f(x) &:= \displaystyle\int_0^1 v(x(t)) 
        \left|\left|\frac{dx}{dt}(t)\right|\right| \; dt \\ 
        &\leq c\displaystyle\int_0^1 v(x(t)) \; dt \\ 
        &\leq Mc \displaystyle\int_0^1 \; dt = Mc
    \end{align*}
\end{proof}


\end{document}
