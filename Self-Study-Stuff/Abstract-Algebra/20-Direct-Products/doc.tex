\documentclass[12pt]{article}
\usepackage{amsmath, amssymb, amsthm, amsfonts, geometry}
\newtheorem{theorem}{Theorem}
\newtheorem{lemma}{Lemma}
\newtheorem{proposition}{Proposition}
\newtheorem{corollary}{Corollary}
\newtheorem{definition}{Definition}

% Page Setup
\geometry{top=1in, bottom=1in, left=1in, right=1in}

\title{Lecture Notes: Abstract Algebra | (Course By: Alvaro Lozano-Robledo)}
\author{Thobias K. Høivik}
\date{\today}

\begin{document}

\maketitle
\subsection*{Example}
\(\mathbb Z/_2 \mathbb Z \times \mathbb Z/_2 \mathbb Z 
= \{(a \text{ mod } 2, b \text{ mod } 2) | a,b \in \{0,1\}\}\)

\begin{proposition}
    Let \(<G, \star_G>, <H, \star_H>\) be groups. 
    Define \(<G\times H, \star> = \{(g, h) | g\in G \land h \in H\}\) with 
    \((g, h) \star (g', h') = (g \star_G g', h \star_H h')\). 
    Then  \(<G\times H, \star>\) is a group (external indirect product of G \& H).
\end{proposition}
\begin{proof}

    \noindent
    \begin{itemize}
        \item \(\star\) is closed: 
    \end{itemize}

    \noindent
    Let \(g, g' \in G \land h, h' \in H\). 
    \[ 
        (g, h) \star (g', h') = (g\star_G g', h \star_H h')
    \]
    \begin{gather*}
        g, g' \in G \land h, h' \in H 
        \Rightarrow g \star_G g' \in G \land h \star_H h' \in H  \\
        \Rightarrow (g \star_G g', h \star_H h') \in G\times H 
    \end{gather*}
    meaning \(G \times H\) is closed under \(\star\).

    \begin{itemize}
        \item Identity: 
    \end{itemize}

    \noindent
    \((e_G, e_H)\) is the identity as \(\star\) is applying the respective 
    binary operations of G and H componentwise and we know 
    \[ 
        (e_G \star_G g, e_H \star_H h) = (g \star_G e_G, h \star_H e_H) = 
        (g, h)
    \]
    So there exists an identity element of \(G \times H\).

    \begin{itemize}
        \item Inverses:
    \end{itemize}

    \noindent
    Since \(G\) and \(H\) are groups, every \(g \in G\) has an inverse \(g^{-1} \in G\), and every \(h \in H\) has an inverse \(h^{-1} \in H\).  
    Define the inverse of \((g, h) \in G \times H\) as \((g^{-1}, h^{-1})\). Then:
    \[
        (g, h) \star (g^{-1}, h^{-1}) = (g \star_G g^{-1}, h \star_H h^{-1}) = (e_G, e_H)
    \]
    \[
        (g^{-1}, h^{-1}) \star (g, h) = (g^{-1} \star_G g, h^{-1} \star_H h) = (e_G, e_H)
    \]
    So every element in \(G \times H\) has an inverse.

    \begin{itemize}
        \item Associativity:
    \end{itemize}

    \noindent
    Since \(\star_G\) and \(\star_H\) are associative, for all \(g, g', g'' \in G\) and \(h, h', h'' \in H\),
    \[
        ((g, h) \star (g', h')) \star (g'', h'') = (g \star_G g', h \star_H h') \star (g'', h'')
    \]
    \[
        = ((g \star_G g') \star_G g'', (h \star_H h') \star_H h'') 
    \]
    \[
        = (g \star_G (g' \star_G g''), h \star_H (h' \star_H h'')) \quad \text{(since \(\star_G\) and \(\star_H\) are associative)}
    \]
    \[
        = (g, h) \star (g' \star_G g'', h' \star_H h'') = (g, h) \star ((g', h') \star (g'', h''))
    \]
    Thus, \(\star\) is associative.

    \noindent
    Since \(\star\) satisfies closure, identity, inverses, and associativity, \(G \times H\) is a group.
\end{proof}

\subsection*{The Order of Elements in \(G \times H\)}
\begin{theorem}
    Let G and H be groups, \(g \in G, h \in H\) of finite orders \(|g| = r, |h| = s\). 
    Then, \(|(g,h)| = lcm(r,s)\).
\end{theorem}
\begin{proof}
    Let \(n = |(g, h)|, m = lcm(r,s)\). 
    Then, 
    \[ 
        (g,h)^m = (g^m, h^m)
    \]
    note that 
    \[ 
        lcm(r,s) = m \Rightarrow m = rk = sj \text{ where } s,j \in \mathbb Z 
    \]
    so 
    \[ 
        (g^m, h^m) = ((g^r)^k, (h^s)^j) = ((e_G)^k, (e_H)^j) = (e_G, e_H)
    \]
    thus \[n \leq m\] 
    Also 
    \[ 
        (g, h)^n = e, g^n = e \land h^s= e \Rightarrow r \mid n \land s \mid n
    \]
    \[ 
        \Rightarrow lcm(r,s) \mid n \rightarrow m \leq n
    \]
    Since \(m \leq n \land n \leq m\) 
    we have \(m = n\), completing the proof.
    
\end{proof}

\end{document}
