\documentclass[12pt]{article}
\usepackage{amsmath, amssymb, amsfonts, geometry}

% Page Setup
\geometry{top=1in, bottom=1in, left=1in, right=1in}

\title{Lecture Notes: Abstract Algebra (Course By: Alvaro Lozano-Robledo)}
\author{Thobias K. Høivik}
\date{\today}

\begin{document}

\maketitle

\section{Fermat's Little Theorem}
Let \(P \leq 2\) be a prime number and let a be an integer relatively prime to p 
\(gcd(a, p) = 1\). Then \(p \mid a^{p-1}-1 
\Leftrightarrow (a^{p-1} \equiv 1 \text{ mod } p)\).
More generally, if b is any integer, then \(p \mid b^{p}-b 
\Leftrightarrow b^p \equiv b \text{ mod } p\).

\subsection*{Example}
p = 5, \(2^4-1 = 16-1 = 15 = 3 \times 5\). 
\(3^4 - 1 = 80 = 16 \times 5\).

\noindent 
p = 7, \(3^6 - 1 = 729 - 1 = 728 = 104 \times 7\).

\noindent 
p = 5, \(2^5-2 = 2(2^4 - 1) = 2 \times (3 \times 5)\).

\subsection*{Proof without A.A}
Let a be an integer, relatively prime to p. 
Consider the map
\(\{1,2,3,\dots, p-1\}\rightarrow \{a \times 1, a \times 2,\dots, a\times (p-1)\}\).
If \(a \times i \equiv a \times j \text{ mod } p 
\Rightarrow i \equiv j \text{ mod } p \text{, } \because \)  a is 
relatively prime to p \(\Rightarrow\) a has an inverse modulo p. 
\(1 \leq i \leq j \leq p-1\), but no pair of distinct elements 
in that set can be congruent to each other 
since p is prime, hence \(i = j\). Then \(\{1,2,3,\dots, p-1\}\) and 
\(\{a\times1,a\times2,\dots,a\times(p-1)\}\) are the same modulo p.
\(1\times2\times3\times\dots\times(p-1) \equiv 
a(a\times2)\dots(a\times(p-1)) \text{ mod } p\). 
\(a(a\times2)\dots(a\times(p-1)) \equiv a^{p-1} \times 
(1\times2\dots\times(p-1) \text{ mod } p\). 
\(1\times2\times\dots\times(p-1) = N \Rightarrow N = a^{p-1} \times N \text{ mod } p\). 
Divide by N on both sides and obtain \(1 \equiv a^{p-1} \text{ mod } p\). \(\square\)

\subsection*{Proof with A.A}
Consider \((Z/pZ)^x = U(p)\) those elements in \(Z/pZ\) that have multiplicative inverses. 
Because p is prime \(U(p) = \{1,2,3,\dots,p-1 \text{ mod } p\}\) and so 
\(|U(p)| = p-1\). Let \(a \in \mathbb Z\) relatively prime to p, and suppose 
\(a \equiv i \text{ mod } p\) with \(1 \leq i \leq p-1\). 
Consider \(H = <a> = \{1, a, a^2,\dots, a^{n-1}\}\) where ord(a)=n. 
By Lagrange's theorem \(|<a>| = n = |<i>| \text{ divides } |U(p)| = p-1\), so \(n \mid p-1\) where 
n is the order of a. If we write \(p-1 = nk, k \in \mathbb Z\), 
then \(a^{p-1} \equiv i^{p-1} \equiv (i^n)^k 
\equiv 1^k \equiv 1 \text{ mod } p, a \equiv 
i \text{ mod } p \Rightarrow a^{p-1} \equiv 1 \text{ mod } p\). \(\square\)

\section{Euler's Theorem}
Let \(n \geq 1\), and let \(a\) be a number that is relatively prime to n. 
Then \(a^{\phi(n)} \equiv 1 \text{ mod } n\) where 
\(\phi(n) = \#\{1 \leq a \leq n : gcd(a,n) = 1\}\) the number 
of numbers up to n relatively prime to n.

\subsection*{Proof}
Consider \((Z/nZ)^x = U(n) = \{1 \leq a \leq n : 
\text{have multiplicative inverses mod n}\}\)
\(\phi(n) = \#U(n)\). Hence if a is relatively prime to n \(\Rightarrow\)
\(a^{\phi(n)} \equiv 1 \text{ mod } n\) by Lagrange's theorem. \(\square\)


\end{document}
