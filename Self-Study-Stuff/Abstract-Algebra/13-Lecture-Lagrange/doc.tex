\documentclass[12pt]{article}
\usepackage{amsmath, amssymb, amsfonts, geometry, amsthm}
\theoremstyle{plain}
\newtheorem{theorem}{Theorem}
% Page Setup
\geometry{top=1in, bottom=1in, left=1in, right=1in}

\title{Lecture Notes: Abstract Algebra (Course by: Alvaro Lozano-Robledo)}
\author{Thobias K. Høivik}
\date{\today}

\begin{document}

\maketitle

\section*{Lagrange's Theorem}
Let G be a finite group and let H be a subgroup of G.

\noindent
Then the order of H is a divisor of G \(\rightarrow |G| = k|H|, k \in \mathbb Z
\Leftrightarrow |G| \equiv 0 \text{ mod } |H|
\)

\subsection*{Examples}
\(\{0, 3\} \leq \mathbb Z / 6\mathbb Z \land 
\{0, 2, 4\} \leq \mathbb Z / 6\mathbb Z
\) 

\noindent 
\(|\{0,3\}| = 2 \land |\{0,2,4\}| = 3\)

\noindent 
\( \{1,6\} \leq (\mathbb Z / 7\mathbb Z)^x\)

\noindent 
and many more.

\subsection*{Coset Definition}
Let \(<G,\star>\) be a group and let H be a subgroup of G. 

\noindent 
A left coset of H with representative \(g \in G\) is \(gH = g\star H = 
\{g \star h | h \in H\}
\)

\noindent 
and a right coset is \(\{h \star g |h \in H\}\). 

\noindent 
Trivially, G abelian \(\Rightarrow gH = Hg \text{ } \forall H \leq G\), 
\(gH = Hg\) means H is a \textbf{normal subgroup}.

\subsection*{Partition Theorem} 
Let \( G \) be a group and \( H \) a subgroup of \( G \). The left cosets of \( H \) in \( G \),
\[
gH = \{ gh \mid h \in H \}, \quad g \in G,
\]
form a partition of \( G \). That is:
\begin{enumerate}
    \item Every element of \( G \) belongs to some left coset of \( H \).
    \item Any two left cosets of \( H \) are either disjoint or identical.
\end{enumerate}

\noindent
\textbf{Proof:}

\noindent
\textbf{(1) Every element of \( G \) belongs to some coset:}  
Let \( g \in G \). Since \( e \in H \), we have
\[
g = g e \in gH.
\]
Thus, every element of \( G \) is contained in some left coset of \( H \).

\noindent
\textbf{(2) Cosets are either disjoint or identical:}  
Suppose \( g_1H \cap g_2H \neq \emptyset \), meaning there exists some \( x \in g_1H \cap g_2H \). Then we can write:
\[
x = g_1 h_1 = g_2 h_2, \quad \text{for some } h_1, h_2 \in H.
\]
Rearranging gives
\[
g_2^{-1} g_1 = h_2 h_1^{-1} \in H,
\]
so \( g_1 \) and \( g_2 \) belong to the same left coset of \( H \), implying
\[
g_1H = g_2H.
\]

\noindent
Therefore, the left cosets of \( H \) form a partition of \( G \). \(\square\)

\subsection*{The Index of H in G Definition}
Let G be a group and H a subgroup. The index of H in G, denoted by \([G:H]\) is 
the number of disjoint left-cosets of H in G.

\subsection*{Example} 
\(G = \mathbb Z/6\mathbb Z \leq \{0,2,4\} = H \text{, }[G:H] = 2 \because 
(0+H)\sqcup (1+H) = G\)

\subsection*{Proposition}
Let H be a subgroup of G. Then every coset of H has the same number of elements.

\noindent
\textbf{Proof: } 

\noindent 
Let H be a subgroup of G, and let \(g \in G\), consider gH.

\noindent 
Let \(\lambda : H -> gH, \lambda (h) = g \star h\)

\noindent 
\begin{enumerate}
  \item Well defined: \(h \in H \Rightarrow g \star h \in gH\)
  \item Injective: \(\lambda (g) = \lambda (g^*) \Rightarrow gh = gh^* 
    \Rightarrow h = h^* 
  \)
  \item Surjective: \(f \in gH \Rightarrow \exists h \in H \text{ s.t } f = gh 
    \Rightarrow \lambda (h) = gh = f\)

\end{enumerate}

\noindent 
Hence \(\lambda\) is a bijection \(\rightarrow |H| = |gH| = |g^*H|\)

\subsection*{Proof of Lagrange's Theorem}
Let G be a finite group and H a subgroup of G. 

\noindent 
Then, \([G:H] = \frac{|G|}{|H|}\) is the number of distinct cosets of H in G.

\noindent 
In particular \(|H| \times [G:H] = |G|\).

\noindent 
Thus \(|H|\) divides \(|G|\).

\subsection*{}
\(H \leq G\). 

\noindent 
Then \(G = g_1H \sqcup g_2H \sqcup \cdots \sqcup g_nH\), where \(n = [H:G]\)

\noindent 
Prop \(\rightarrow |g_1H| = |g_2H| = \cdots |g_nH|\) and 
\(\therefore |G| = \displaystyle\sum_{i=1}^{n} |g_iH| = |H| \times n = |H| 
\times [G:H]\)

\noindent 
In particular \(|G| = |H| \times [G:H] \rightarrow [G:H] = \frac{|G|}{|H|} 
\quad \square\) 



\end{document}
