\documentclass[12pt]{article}
\usepackage{amsmath, amssymb, amsthm, amsfonts, geometry}
\newtheorem{theorem}{Theorem}
\newtheorem{lemma}{Lemma}
\newtheorem{proposition}{Proposition}
\newtheorem{corollary}{Corollary}
\newtheorem{definition}{Definition}

% Page Setup
\geometry{top=1in, bottom=1in, left=1in, right=1in}

\title{Lecture Notes: Abstract Algebra | Homomorphisms (Course by: Alvaro Lozano-Robledo)}
\author{Thobias K. Høivik}
\date{\today}

\begin{document}

\maketitle

\begin{definition}
    Let G and G' be groups. Then a homomorphism from G to G' \(\phi: G \to G'\) 
    is a mapping that respects groups structure such that 
    for \(a,b \in G \), \(\phi(a \star_G b) = \phi(a) \star_{G'} \phi(b)\)
\end{definition}

\noindent 
Notice that how we can then define isomorphisms as bijective homomorphisms.

\noindent 
\textbf{Example 1:}
Let G be any group, \(g\in G\), we can define \(\phi: \mathbb Z \to G\), \(n \to g^n\). 
\[ 
    \phi(n+m) = g^{n+m} = g^n g^m = \phi(n)\phi(m)
\]
Notice how if we take G to be any group this is not guaranteed to be a bijection, 
but it still respects group structure like isomorphisms do.

\noindent 
\textbf{Example 2:} 
\(G = \mathbb Z /_6 \mathbb Z, g = 1 \text{ mod } 6\), \(\phi: \mathbb Z \to G\), 
\(n \to n \text{ mod } 6\) makes a surjective homomorphism. 
Notice how \(\phi(1) = \phi(7)\), making \(\phi\) non-injective.

\noindent 
\textbf{Example 3:} 
\(G = GL(2, \mathbb R), H = \mathbb R^*\). 
\(\psi: GL(2, \mathbb R) \to \mathbb R^*\), \(\psi(A) = det(A)\). 
\(\psi\) is well-defined because \(A \in G \Rightarrow \psi(A) = det(A) \neq 0\). 
\(\psi\) respects structure, because of relationship between matrix multiplication and 
determinants: 
\[ 
    \psi(AB) = det(AB) = det(A)det(B) = \psi(A)\psi(B) 
\]

\section*{Properties of Homomorphisms} 
\begin{proposition}
    Let \(\psi: G \to H\) be a homomorphism. Then 
    \begin{enumerate}
        \item \(\psi(e_G) = e_H\)
        \item \(\forall \psi(g^{-1}) = \psi(g)^{-1}\)
        \item \(J \subseteq G\) is a subgroup, then \(\psi[J] \subseteq H\) is a subgroup
        \item \(K \subseteq H\) is a subgroup, then \(\psi^{-1}[K] \subseteq G\)
        \item If \(K \triangleleft H\) normal subgroup, 
            then \(\psi^{-1}[K] \triangleleft G\)
    \end{enumerate}
\end{proposition}
\begin{proof}[Proof (1)]
    \(\psi: G \to H\) homomorphism. 
    \begin{gather*}
        h = \psi(e_G) = \psi(e_G \star_G e_G \\ 
        = \psi(e_G) \star_H \psi(e_G) = h\star_H h \\ 
        \therefore h = h \star_H h \\ 
        \Rightarrow e_H = h 
    \end{gather*}
    Thus \(\psi(e_G) = e_H\). 
\end{proof}
\begin{proof}[Proof (2)]
    Let \(g\in G\), 
    \begin{gather*}
        e_H = \psi(e_G) = \psi(g \star_G g^{-1}) \\ 
        = \psi(g) \star_H \psi(g^{-1}) = e_H \\ 
        \Rightarrow \psi(g)^{-1} = \psi(g^{-1})
    \end{gather*}
\end{proof}
\begin{proof}[Proof (3)]
    \(J \subseteq G\) a subgroup. Let \(K = \psi[J]\). 
    J being a subgroup of G means: 
    \[ 
        \forall j_1, j_2 \in J : j_1 j_2^{-1} \in J 
    \]
    Let \(k_1, k_2 \in K = \psi[J]\), thus \(\exists j_1, j_2 \in J\) such that 
    \(k_i = \psi(k_i)\). Then  
    \[ 
        k_1k_2^{-1} = \psi(j_1)\psi(j2)^{-1} = \psi(j_1)\psi(j_2^{-1})
    \]
    \[ 
        = \psi(j_1j_2^{-1}) \in \psi[J] = K
    \]
\end{proof}
\begin{proof}[Proof (4)]
    Define \(J = \psi^{-1}[K] = \{ g \in G \mid \psi(g) \in K \}\).

    \noindent 
    Identity: Since \(e_H \in K\) (as \(K\) is a subgroup), 
    and \(\psi(e_G) = e_H\), it follows that \(e_G \in \psi^{-1}[K]\).

    \noindent 
    Closure: Let \(g_1, g_2 \in \psi^{-1}[K]\), 
    meaning that \(\psi(g_1), \psi(g_2) \in K\). Since \(K\) is a subgroup, 
    it is closed under multiplication, so \(\psi(g_1) \psi(g_2) \in K\). 
    Since \(\psi\) is a homomorphism, we have:
    \[
        \psi(g_1 g_2) = \psi(g_1) \psi(g_2) \in K.
    \]
    Thus, \(g_1 g_2 \in \psi^{-1}[K]\), proving closure.

    \noindent 
    Inverses: Let \(g \in \psi^{-1}[K]\), so that \(\psi(g) \in K\). 
    Since \(K\) is a subgroup, it contains inverses, meaning that \(\psi(g)^{-1} \in K\). 
    By the homomorphism property, we know \(\psi(g^{-1}) = \psi(g)^{-1}\), 
    and thus \(\psi(g^{-1}) \in K\). This implies \(g^{-1} \in \psi^{-1}[K]\).

    \noindent 
    Since \(\psi^{-1}[K]\) satisfies the identity, closure, and inverses, 
    it is a subgroup of \(G\).
\end{proof}
\begin{proof}[Proof (5)]
    Let \(x \in \psi^{-1}[K]\), so that \(\psi(x) \in K\). 
    Since \(K\) is normal in \(H\), we know that for any \(h \in H\) 
    and \(k \in K\), we have \(h k h^{-1} \in K\). Applying this to \(h = \psi(g)\) 
    and \(k = \psi(x)\), we get:
    \[
        \psi(g) \psi(x) \psi(g)^{-1} \in K.
    \]
    Using the homomorphism property, we rewrite this as:
    \[
        \psi(g x g^{-1}) \in K.
    \]
    This means \(g x g^{-1} \in \psi^{-1}[K]\), proving that \(\psi^{-1}[K]\) 
    is normal in \(G\).
\end{proof}


\end{document}
