\documentclass[12pt]{article}
\usepackage{amsmath, amssymb, amsfonts, geometry, array}

\usepackage[
  separate-uncertainty = true,
  multi-part-units = repeat
]{siunitx}

% Page Setup
\geometry{top=1in, bottom=1in, left=1in, right=1in}

\title{Lecture Notes: Abstract Algebra (Course by: Alvaro Lozano-Robledo)}
\author{Thobias K. Høivik}
\date{\today}

\begin{document}

\maketitle

\section*{Hamilton's Quaternions}

\subsection*{Definition}
Let Q8 
\[ 
  = \{\pm1, \pm i, \pm j, \pm k. | i^2 = j^2 = k=2 = -1, 
  ij = k, jk = i, ki = j, ji = -k, kj = -i, ik = -j\}
\]

\noindent
Then, \(<Q8, \times>\) is called Hamilton's Quaternions or the Quaternion group.

\noindent
\textbf{Q1)} Is \(<Q8, \times>\) a group?

\noindent 
\textbf{Closure:} We can see by the given definition of the set above that 
when an element of Q8 is multiplied with another element of Q8, we get another 
element of Q8, so it is indeed closed.

\noindent 
\textbf{Associative:} We know multiplication is associative and we can experimentally 
test every combination of elements in Q8, multiply them, and find that Q8 is indeed 
associative. Note also that Q8 can be represented as matrices, 
\[
I = \begin{bmatrix} 1 & 0 \\ 0 & 1 \end{bmatrix}, \quad
- I = \begin{bmatrix} -1 & 0 \\ 0 & -1 \end{bmatrix}
\]

\[
i = \begin{bmatrix} i & 0 \\ 0 & -i \end{bmatrix}, \quad
j = \begin{bmatrix} 0 & 1 \\ -1 & 0 \end{bmatrix}, \quad
k = \begin{bmatrix} 0 & i \\ i & 0 \end{bmatrix}
\]

\[
- i = \begin{bmatrix} -i & 0 \\ 0 & i \end{bmatrix}, \quad
- j = \begin{bmatrix} 0 & -1 \\ 1 & 0 \end{bmatrix}, \quad
- k = \begin{bmatrix} 0 & -i \\ -i & 0 \end{bmatrix}
\]

\noindent 
These matrices satisfy the defining relations:

\[
i^2 = j^2 = k^2 = ijk = -I.
\]

\noindent 
and observe also that Q8 then must be a subset of \(GL(2, \mathbb C)\).

\noindent 
\textbf{Identity:} 1 is the identity since \(q \times 1 = 1 \times q = q\). 

\noindent 
\textbf{Inverses:} 4 of the elements in Q8 are just negative versions of the others since 
\((\pm 1)^{-1} = \pm 1, (\pm i)^{-1} = \mp i,\) etc. Thus each element of Q8 has an inverse, and all group axioms are satisfied for Q8 under multiplication. 

\noindent 
\textbf{Q2)} What is the order of Q8? \(|Q8| = 8 = 2^3\), Q8 is a 2-group. If the order 
of G is \(p^n\) for p prime, then G is a p-group, making Q8 a p-group.

\noindent 
\textbf{Q3)} Is Q8 abelian? No it is not abelian because \(ij = k \neq ji = -k\). 

\noindent
\textbf{Q4)} What are the orders of each element of Q8? 

\begin{itemize}
    \item ord(1) = 1 
    \item ord(-1) = 2
    \item ord(i) = ord(-i) = 4 
    \item ord(j) = ord(-j) = 4 
    \item ord(k) = ord(-k) = 4
\end{itemize}

\noindent 
Here we get to see a corollary of Lagrange's theorem as the order of elements in Q8 
divide the order of Q8.

\noindent 
\textbf{Q5)} What does the Caley table look like?  
\[
\begin{array}{c|cccccccc}
    \times & 1 & -1 & i & -i & j & -j & k & -k \\
    \hline
    1  & 1  & -1  & i  & -i  & j  & -j  & k  & -k  \\
    -1 & -1 & 1   & -i & i   & -j & j   & -k & k   \\
    i  & i  & -i  & -1 & 1   & k  & -k  & -j & j   \\
    -i & -i & i   & 1  & -1  & -k & k   & j  & -j  \\
    j  & j  & -j  & -k & k   & -1 & 1   & i  & -i  \\
    -j & -j & j   & k  & -k  & 1  & -1  & -i & i   \\
    k  & k  & -k  & j  & -j  & -i & i   & -1 & 1   \\
    -k & -k & k   & -j & j   & i  & -i  & 1  & -1  
\end{array}
\]

\noindent 
\textbf{Q6)} What are the elements of \(<i>\)? 
\(<i> = \{e, i, i^2,\cdots\} = \{1, i, -1, -i\} = < -i> \).

\noindent 
\textbf{Q7)} What do the other cyclic subgroups of Q8 look like? 
Well, all elements in Q8 not equal to \(\pm 1\) will look the same as \(<i>\), while 
\(<1> = \{1\}, <-1> = \{-1, 1\}\). 

\noindent 
\textbf{Q8)}
What would other subgroups of Q8 look like?
\begin{gather*}
  H \leq G \text{ } \land \text{ } i,j \in H \text{ then } \\ 
  <i> \subseteq H \text{ } \land \text{ } <j> \subseteq H \\ 
  i \times j = h \in H, -1 \in H, 1 \in H, i \times j = k \in H \\ 
  \text{so } \pm 1, \pm i, \pm j, \pm k \in H = Q8
\end{gather*}

\noindent 
This means that the subgroups of Q8 are \(Q8, <1>, <-1>, <i>, <j>, <k>\).

\end{document}
