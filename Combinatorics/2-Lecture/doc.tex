\documentclass[12pt]{article}
\usepackage{amsmath, amssymb, amsfonts, geometry}

% Page Setup
\geometry{top=1in, bottom=1in, left=1in, right=1in}

\title{Lecture Notes: Combinatorics}
\author{Thobias K. Høivik}
\date{\today}

\begin{document}

\maketitle

\section{Counting Principles}
\subsection*{Multiplication principle}

If there are  $a$ ways of performing task $A$,
and b ways of performing task $B$,
\textbf{regardless of the outcome of A},
there are $ab$ ways of performing \(A\) then \(B\)

\subsection*{Addition Principle}
If there are \(a\) ways of performing task \(A\),
and \(b\) ways of performing task \(B\),
then there are \(a+b\) ways of performing either \(A\) or \(B\)

\subsection*{Example}
\noindent Consider a map with 4 sectors $A, B, C, D$, 
all adjacent except \(B\) and \(D\). 
We have \(q\) colors and we want to color the adjacent sectors
different colors.
How many ways are there to do this?
How many \textbf{proper} colorings?
Note that while this looks like the four color question, it isn't quite the same.

\begin{enumerate}
  \item Color A (\(q\) posibilities) 
  \item Color B (\(q - 1\) posibilities)
  \item Color C (\(q - 2\) posibilities)
  \item Color D (\(q - 2\) posibilities)
\end{enumerate}
By the multiplication principle there are 
\( 
  q(q-1)(q-2)^2
\) posibilities as long as \(q \ge 3\) since we cannot satisfy 
the condition of the color of \(C \neq D\) otherwise.

\begin{enumerate}
  \item Color B (\(q\) posibilities) 
  \item Color C (\(q - 1\) posibilities, any but \(B\))
  \item Color D (\(q - 1\) posibilities, any but \(C\))
  \item Color A (\(q - 2\) or \(q - 3\) posibilities, any but \(B,C,D\))
\end{enumerate}

\noindent \textbf{Case 1:} B,C,D have 3 different colors \(\Rightarrow \)
\# of colorings = \(q(q-1)(q-2)(q-3)\). 

\noindent\textbf{Case 2:} B,C,D don't have 3 different colors. \(\Rightarrow \) 
\# of colorings = \(q(q-1)(q-2)\).

\noindent The answer then becomes \(q(q-1)(q-2)(q-3)+q(q-1)(q-2) 
= q(q-1)(q-2) \times (q-3 + 1) = q(q-1)(q-2) \times (q-2) = q(q-1)(q-2)^2\)

\section{Permutations}
Let \(S_n\) be the set of permutations of \([n] = {1,2,3,...,n}\).  

\noindent \textbf{Proposition:} \(|S_n| = n!\)

\noindent \textbf{Proof:} To choose a permutation of 
\(\pi_1, \pi_2, \pi_3, ..., \pi_n \),
\[
\begin{array}{|c|l|}
\hline
1 & \text{Choose } \pi_1 \text{ (} n \text{ possibilities)} \\
\hline
2 & \text{Choose } \pi_2 \text{ (} n - 1 \text{ possibilities)} \\
\hline
3 & \text{Choose } \pi_3 \text{ (} n - 2 \text{ possibilities)} \\
\hline
\vdots & \vdots \\
\hline
n & \text{Choose } \pi_n \text{ (} 1 \text{ possibility)} \\
\hline
\end{array}
\]

\noindent We have to make a choice for each element and each choice is independent 
of all the other choices, 
\(
  \therefore
  S_n = n \times (n-1) \times (n-2)
  \times (n-3) \times \cdots \times 1 
  = n!   
\)
\(\square\)

\section{Subsets}
Let $S$ be a finite set of size $n$.

\noindent Let \(2^s = \{ \) subsets of set S \(\}\), 
\(\binom{S}{k} = \{k\) -subsets of S\(\}\)

\subsection*{}
\textbf{Proposition:} 
\(|2^s| = 2^n\)

\noindent
\textbf{Proof:} \(S = \{a_1, a_2, a_3, ..., a_n\}\)

\[
\begin{array}{|l|l|}
\hline
a_1 \in S? & 2 \text{ posibilities}\\
\hline
a_2 \in S? & 2 \text{ posibilities}\\
\hline
a_3 \in S? & 2 \text{ posibilities}\\
\hline
\vdots & \vdots \\
\hline
a_n \in S? & 2 \text{ posibilities}\\
\hline
\end{array}
\]

\noindent 
Thus we have to make independent binary choices, meaning \(2\) posibilities, \(n\) times.  

\noindent
\(\therefore |2^s| = 2 \times 2 \times 2 \times \cdots \times 2 = 2^n \)
\(\square\)

\noindent Notice this gives a bijection between the \{subsets of \(S\)\} and 
the \{sqeuences \((e_1, e_2,...,e_n), e = 0 \lor e = 1\)\}. 
So now we know how many posibilities there are when we have to make \(n\) 
binary independent choices.

\subsection*{}
\textbf{Definition: } \(\binom{n}{k} = 
|\binom{S}{k}| \text{ for } |S| = n\)

\noindent 
\textbf{Proposition: } \(\binom{n}{k}\) =  

\noindent To choose a k-subset \(\{b_1,...,b_k\} \subseteq S\)

\[
\begin{array}{|l|l|}
\hline
\text{Choose } b_1 & n \text{ posibilities}\\
\hline
\text{Choose } b_2 & n-1 \text{ posibilities}\\
\hline
\text{Choose } b_3 & n-2 \text{ posibilities}\\
\hline
\vdots & \vdots \\
\hline
\text{Choose } b_k & n-k+1 \text{ posibilies}\\
\hline
\end{array}
\]

\noindent Yielding \(n(n-1)(n-2)...(n-k+1) = \frac{n!}{k!(n-k)!}\). 
Note \(k!\) making it the unordered choosing of \(k\) elements from \(n\) elements.
If it were ordered it would be \(\frac{n!}{(n-k)!}\).

\section{Subsets \& Generating Functions}
The multivariant generating function for subets of \([n]\).
\[
  \displaystyle\sum_{A\subseteq[n]}^{}\prod_{i\in A} x_i
  = (1+x_1)(1+x_2)...(1+x_n)
\] 
\( 
  \text{Example: } n = 2 \rightarrow x_1 x_2 + x_1 + x_2 + 1 
  \text{, } \leftrightarrow \{1,2\}, \{1\}, \{2\}, \emptyset
\)

\noindent 
\(
  x_1 x_2 + x_1 + x_2 + 1 = (x_1 + 1)(x_2 + 1)
\)

\subsection*{}
Plug in \(x_1=x_2=x_3=...=x_n\)
\[ 
  \displaystyle\sum_{A \subseteq [n] }x^{|A|} = (1+x)^n
  = \displaystyle\sum_{k=0}^{n}\binom{n}{k}x^k
\]
We get the binomial theorem for free!

\section{Compositions}
A composition of n is a way of expressing n as an ordered sum of 
of positive integers. 

\noindent
\textbf{Example: } The compositions of 3 are \(1+1+1 = 2+1 = 1+2 = 3\). 

\noindent


\end{document}
