
\documentclass[12pt]{article}
\usepackage{amsmath, amssymb, amsthm, amsfonts, geometry}
\newtheorem{theorem}{Theorem}
\newtheorem{lemma}{Lemma}
\newtheorem{proposition}{Proposition}
\newtheorem{corollary}{Corollary}
\newtheorem{definition}{Definition}
\usepackage{tikz}
\usetikzlibrary{graphs, arrows.meta}
% Page Setup
\geometry{top=1in, bottom=1in, left=1in, right=1in}

\title{Lecture Notes: Enumerative Combinatorics (Lecture 4)}
\author{Thobias K. Høivik}
\date{\today}

\begin{document}

\maketitle

\section{Counting Permutations by Cycle Type}

In this lecture, we explore the enumeration of permutations based on their cycle structure.

\subsection{Cycle Type}

The \textbf{cycle type} of a permutation is a partition of \( n \) that describes the lengths of its disjoint cycles. For example, the permutation \( (1\,2)(3\,4\,5) \) has cycle type \( (2,3) \).

\subsection{Counting Permutations with a Given Cycle Type}

To count the number of permutations with a specific cycle type \( \lambda = (\lambda_1, \lambda_2, \ldots, \lambda_k) \), where \( \lambda_i \) are the lengths of the cycles, we use the formula:

\[
\frac{n!}{z_\lambda}
\]
where \( z_\lambda = \prod_{i=1}^k m_i! \lambda_i^{m_i} \) and \( m_i \) is the number of parts of \( \lambda \) equal to \( i \).

\section{Records in Permutations}

A \textbf{record} in a permutation \( \pi = (\pi_1, \pi_2, \ldots, \pi_n) \) is an element \( \pi_i \) that is larger than all previous elements \( \pi_1, \pi_2, \ldots, \pi_{i-1} \).

\subsection{Counting Records}

The expected number of records in a random permutation of \( n \) elements is \( \sum_{i=1}^n \frac{1}{i} \), which approximates \( \ln(n) + \gamma \) for large \( n \), where \( \gamma \) is the Euler-Mascheroni constant.

\section{Inversions in Permutations}

An \textbf{inversion} in a permutation \( \pi = (\pi_1, \pi_2, \ldots, \pi_n) \) is a pair \( (i, j) \) such that \( i < j \) and \( \pi_i > \pi_j \).

\subsection{Counting Inversions}

The number of inversions in a permutation is a measure of its "sortedness." A permutation with zero inversions is the identity permutation, while the maximum number of inversions is \( \binom{n}{2} \), corresponding to the reverse permutation.

\end{document}
