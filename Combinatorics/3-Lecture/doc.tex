\documentclass[12pt]{article}
\usepackage{amsmath, amssymb, amsfonts, geometry}

% Page Setup
\geometry{top=1in, bottom=1in, left=1in, right=1in}

\title{Lecture Notes: Combinatorics (Course by: Federico Ardila)}
\author{Thobias K. Høivik}
\date{\today}

\begin{document}

\maketitle

\section{Multisets}
A multiset is a set with possibly repeated elements.

\noindent 
\(\{1,2,2,2,4,5,5,7\} =: \{1^1, 2^3, 4^1, 5^2, 7^1\}\) is a multiset, for instance.

\noindent
For multisets we count repetitions so the cardinality here would be 8 rather than 5.

\noindent 
\( (\binom{S}{k}) \) is the multisets of size k on S, 
\( (\binom{n}{k} = |(\binom{S}{k})|) \). \((\binom{n}{k})\) could be called 
"n multi choose k".

\subsection*{Proposition}
\( (\binom{n}{k}) = \binom{n+k-1}{k} \)

\noindent
\textbf{Proof: }

\noindent 
If k multiset S on \([n]\) has \(a_i\) parts equal to i then if I add up 
\(a_1 + a_2 + \cdots \ a_n = k\), making it a weak n-composition of k 
(since 0s are allowed). \( \quad \square \)

\section{Multisets and GFs}
The multivariant generating function for multisets on 
\([n]\) is given by 
\begin{gather*} 
  \displaystyle\sum_{v: [n] \rightarrow \mathbb N}
  x_1^{v(1)} x_2^{v(2)} \cdots x_n^{v(n)} \\
  =
  (1+x_1+x_1^2+\cdots)(1+x_2+x_2^2+\cdots)\cdots(1+x_n+x_n^2+\cdots) \\
  = 
  \frac{1}{1-x_1} \cdots \frac{1}{1-x_n}
\end{gather*}

\noindent 
Let \(x_1 = x_2 = \cdots = x_n = x\) 

\begin{gather*} 
  \displaystyle\sum_{v:[n]\rightarrow \mathbb N}
  x^{v(1) + v(2) + \cdots + v(n)}
  = (\frac{1}{1-x})^n \\
  = \displaystyle\sum_{\text{Multisets on} [n]} x^{|Multiset|}
  = (1-x)^n \\ 
  = \displaystyle\sum_{k=0} (\binom{n}{k})x^k
  = \displaystyle\sum_{k=0}\binom{-n}{k}(-1)^k x^k
\end {gather*}

\noindent 
The binomial theorem states \((1+x)^\alpha = \sum_{k=0} \binom{\alpha}{k}x^k\),
\(\binom{\alpha}{k} := \frac{\alpha(\alpha - 1) \cdots (\alpha-k+1)}{k!} \)
because if \(\alpha\) is an integer it works and if it is real it still makes sense.
Doing a taylor expansion will show why the 
binomial theorem holds for \(\alpha \in \mathbb R\) so we don't have to worry 
about \(\binom{-n}{k} = (-1)^k (\binom{n}{k})\).
This is an example for what we call a "combinatorial resipositry theorem". 
What this shows us that even though it sounds nonsensical to ask how many 
k-subsets of a set sized -n exist, it still means something in combinatorics.

\section{Multinomial coefficients}
The multinomial coefficient \(\binom{n}{a_1, a_2, \cdots, a_k}\)
is the number of ways of splitting a set of size n into a set of size 
\(a_1, a_2, \cdots, a_k \). 

\subsection*{Example}
\(\binom{n}{k, n-k} = \binom{n}{k}\)

\subsection*{Proposition}
The number of permutations of a multiset 
\(\{1^{a_1}, 2^{a_2}, \cdots, k^{a_k}\}\) of size n is 
\(\binom{n}{a_1, \cdots, a_k}\)

\noindent 
\textbf{Proof:}

\noindent 
To make a permutation of size n I must choose $a_1$ positions for 1 to go,  
$a_2$ positions for 2 to go and so on. Split \([n]\) into sets | where the 1s go, 
2s go, etc. \(\quad \square\)

\subsection*{Proposition}
\(
\binom{n}{a_1,\cdots,a_k} = \frac{n!}{a_1! a_2! \cdots a_k!}
\)

\noindent
\textbf{Proof:}

\noindent 
The multinomial coefficient \(\binom{n}{a_1,\cdots, a_k}\) represents the number of ways to partition a set of \(n\) elements into \(k\) subsets of sizes \(a_1, a_2, \dots, a_k\), where each subset contains identical elements. Alternatively, it also counts the number of distinct permutations of a multiset with elements occurring with frequencies \(a_1, a_2, \dots, a_k\).

\noindent
We proceed by considering the step-by-step placement of the elements:

\noindent 
1. First, choose \(a_1\) positions for the first type of element (e.g., the ones). Since the order among these elements does not matter, the number of ways to do this is:
   \[
   \binom{n}{a_1} = \frac{n!}{a_1!(n - a_1)!}.
   \]

\noindent 
2. Next, from the remaining \(n - a_1\) positions, choose \(a_2\) positions for the second type of element. Again, the order among these elements does not matter, so we have:
   \[
   \binom{n - a_1}{a_2} = \frac{(n - a_1)!}{a_2!(n - a_1 - a_2)!}.
   \]

   \noindent 
3. This process continues for all \(k\) types of elements until all positions are filled. The last group of elements is automatically placed, leaving us with:

   \[
   \binom{n - a_1 - \dots - a_{k-1}}{a_k} = \frac{(n - a_1 - \dots - a_{k-1})!}{a_k!(n - a_1 - \dots - a_k)!}.
   \]

\noindent
Multiplying all these binomial coefficients together, we obtain:

\[
\binom{n}{a_1} \binom{n - a_1}{a_2} \binom{n - a_1 - a_2}{a_3} \dots \binom{n - a_1 - \dots - a_{k-1}}{a_k}.
\]

\noindent
Expanding each binomial coefficient in factorial form, we get:

\[
\frac{n!}{a_1!(n - a_1)!} \cdot \frac{(n - a_1)!}{a_2!(n - a_1 - a_2)!} \cdots \frac{(n - a_1 - \dots - a_{k-1})!}{a_k!(n - a_1 - \dots - a_k)!}.
\]

\noindent
Observing that all intermediate factorial terms cancel out, we are left with:

\[
\frac{n!}{a_1! a_2! \cdots a_k!}.
\]

\noindent
Thus, we have shown that:

\[
\binom{n}{a_1,\cdots,a_k} = \frac{n!}{a_1! a_2! \cdots a_k!}.
\]

\noindent
\(\square\)

\subsection*{Example}
How many distinct ways can we rearrange the letters in the word \texttt{MISSISSIPPI}?

\noindent
\textbf{Solution:}  
Since some letters repeat, we cannot simply use the factorial formula for permutations of distinct objects. Instead, we must account for repeated letters using the multiset method.

\noindent 
\textbf{Step 1: Identifying the Frequencies}  
The word \texttt{MISSISSIPPI} consists of 11 letters with the following frequencies:
\[
M: 1, \quad I: 4, \quad S: 4, \quad P: 2.
\]
Since some letters are identical, the number of unique rearrangements is given by the multinomial coefficient:
\[
\frac{11!}{1! 4! 4! 2!}.
\]

\noindent
\textbf{Step 2: Computing the Multinomial Coefficient}  
\begin{gather*}
11! = 39916800, \\
4! = 24, \quad 2! = 2, \quad 1! = 1, \\
4! 4! 2! 1! = (24 \cdot 24 \cdot 2 \cdot 1) = 1152, \\
\frac{39916800}{1152} = 34650.
\end{gather*}

\noindent
Thus, the number of distinct ways to rearrange the letters in \texttt{MISSISSIPPI} is 
\[
\mathbf{34650}.
\]

\subsection*{Multinomial Theorem} 
\[
(x_1+\cdots+x_k)^n = 
\displaystyle\sum_{a_1,\cdots,a_k} \binom{n}{a_1\cdots a_k}x_1^{a_1}\cdots x_k^{a_k}
\]

\subsection*{Proposition}
In a box of size \(a_1 \times a_2 \times \cdots \times a_k\),
the number of shortes lattice paths 
from one corner to the opposite corner is 
\(\binom{a_1+a_2+\cdots +a_k}{a_1, a_2 \cdots, a_k}\). 

\noindent 
\textbf{Proof:}

\noindent 
A shortest lattice path from one corner \((0,0,\dots,0)\) to the opposite corner \((a_1, a_2, \dots, a_k)\) consists of exactly \(a_1 + a_2 + \dots + a_k\) total steps, where:
- \(a_1\) steps move in the first coordinate direction,
- \(a_2\) steps move in the second coordinate direction,
- \(\dots\),
- \(a_k\) steps move in the \(k\)th coordinate direction.

\noindent
Each shortest path is uniquely determined by the sequence of moves taken. Since we must choose \(a_1\) of the \(a_1 + a_2 + \dots + a_k\) total steps to move in the first direction, then \(a_2\) steps to move in the second direction, and so on, the number of such sequences is given by the multinomial coefficient:

\[
\binom{a_1 + a_2 + \dots + a_k}{a_1, a_2, \dots, a_k} = \frac{(a_1 + a_2 + \dots + a_k)!}{a_1! a_2! \dots a_k!}.
\]

\noindent
Thus, the number of shortest lattice paths is exactly \(\binom{a_1+a_2+\cdots +a_k}{a_1, a_2, \dots, a_k}\). \(\quad \square\)

\section{Combinatorial Identities}
\(\binom{n}{k} = \binom{n}{n-k}\), which is easy to prove algebraically. 
Combinatorially we ask why is the number of k-subsets of n equal to the number 
of (n-k)-subsets of n? Well there must exist a bijection between k-subsets and 
(n-k)-subsets which is to take the k-subset's complement which would be the 
(n-k)-subset. 
\(\binom{[n]}{k} \leftrightarrow \binom{[n]}{n-k} \Leftrightarrow 
A \leftrightarrow [n]-A\) in other words.

\noindent 
\(\binom{n+1}{k+1} = \binom{n}{k+1}+\binom{n}{k}\). 
Combinatorially we say we want to count (k+1) subsets of [n+1]. 
If n+1 is not n: \(\binom{n}{k+1}\). If n+1 is n: \(\binom{n}{k}\). Since either 
happens we add them.

\noindent
\textbf{Pascal's Triangle}

\noindent 
In row 1 there is one element; 1. In the second there are two 1s and so on. 
In each row n the first element is \(\binom{n}{0} = 1\) then n choose 2, and so 
on until you reach the middle element then go backwards until you get n choose 
0 again. The previous identity is why Pascal's Triangle works, 
making it a nice visual for the identity.

\subsection*{}
\(\binom{m+n}{k} = \displaystyle\sum_{i=0}^{k} \binom{m}{i} \binom{n}{k-i}\). 
Count the subsets of a set of m blue and n red elements, or some other distinction.
If my subset has i blue elements it has k-i red elements 
\(\rightarrow \binom{m}{i}\binom{n}{k-i}\). I could be any value so we add over all 
possible values of i, giving us the sum. 
 
\end{document}
