\documentclass[11pt]{article}
\usepackage[a4paper,margin=1in]{geometry}
\usepackage{fourier} % Fourier font
\usepackage{xcolor}
\usepackage{tikz}
\usepackage[most]{tcolorbox}
\usepackage{amsthm, amsmath, amssymb}
\usepackage{enumitem}
\usepackage{hyperref}
\usepackage[nameinlink,noabbrev]{cleveref}
\usepackage{titling} 

% Dark mode colors
\definecolor{bgcolor}{HTML}{FFFFFF}
\definecolor{textcolor}{HTML}{000000}
\definecolor{defcolor}{HTML}{E86873}
\definecolor{thmcolor}{HTML}{0A9396}
\definecolor{lemcolor}{HTML}{94D2BD}
\definecolor{corcolor}{HTML}{9B4AF7}
\definecolor{probcolor}{HTML}{EE9B00}
\definecolor{excolor}{HTML}{21E933}

% Background and text color
\pagecolor{bgcolor}
\color{textcolor}

% No paragraph indentation
\setlength{\parindent}{0pt}
\setlength{\parskip}{0.7em}

% Theorem box styles
\tcbset{
  enhanced,
  colback=bgcolor,
  colframe=thmcolor,
  coltext=white,
  coltitle=white,
  fonttitle=\bfseries,
  boxrule=0.7pt,
  left=1em,
  right=1em,
  top=0.7em,
  bottom=0.7em,
  before skip=10pt,
  after skip=10pt,
}

% Theorem environments with colored boxes
\newtcbtheorem[number within=section]{thm}{Theorem}{
  colframe=thmcolor, colback=thmcolor!15!bgcolor
}{thm} % The 'thm' here is the *prefix* for the label

\newtcbtheorem[number within=section]{defn}{Definition}{
  colframe=defcolor, colback=defcolor!15!bgcolor
}{def} % The 'def' here is the *prefix* for the label

\newtcbtheorem[number within=section]{lem}{Lemma}{
  colframe=lemcolor, colback=lemcolor!15!bgcolor
}{lem}

\newtcbtheorem[number within=section]{cor}{Corollary}{
  colframe=corcolor, colback=corcolor!15!bgcolor
}{cor}

\newtcbtheorem[number within=section]{prob}{Problem}{
  colframe=probcolor, colback=probcolor!15!bgcolor
}{prob}

\newtcbtheorem[number within=section]{ex}{Example}{
  colframe=excolor, colback=excolor!15!bgcolor
}{ex}

% Proof environment 
\renewenvironment{proof}[1][\proofname]{%
  \par\pushQED{\qed}\normalfont\topsep6pt \trivlist
  \item[\hskip\labelsep\itshape #1.]\ignorespaces
}{%
  \popQED\endtrivlist\addvspace{6pt}
}

% Cleveref name formats for tcolorbox environments
\crefname{thm}{theorem}{theorems}
\Crefname{thm}{Theorem}{Theorems}

\crefname{def}{definition}{definitions}
\Crefname{def}{Definition}{Definitions}

\crefname{lem}{lemma}{lemmas}
\Crefname{lem}{Lemma}{Lemmas}

\crefname{cor}{corollary}{corollaries}
\Crefname{cor}{Corollary}{Corollaries}

\crefname{prob}{problem}{problems}
\Crefname{prob}{Problem}{Problems}

\crefname{ex}{example}{examples}
\Crefname{ex}{Example}{Examples}


\title{\huge{Computability and Complexity}}
\author{\LARGE{Thobias Høivik}}
\date{\large{Spring 2026}}

\begin{document}
\maketitle

\newpage
\tableofcontents

\newpage
\section{Introduction}
The following will be my notes for computability and complexity, which
I shall hopefully be taking at the University of Oslo during 
the spring of 2026 (Course-code: IN2080). If this document 
is renamed to IN2080 - Computability and Complexity, then I 
was allowed to take the course. 
Like with the rest of my electives at UiO, I have to learn 
everything by myself which will undoubtedly manifest 
in my notes being a bit scattered at times. 
If you are someone who stumbled over these notes keep this in mind. 

Furthermore I have dabbled in some automata theory earlier so while 
these notes will be heavily Sipser-flavoured, I will not go 
very into detail on DFA's. 

\newpage
\section{Nondeterministic Finite Automata}
\begin{defn}
    \label{defn:nfa}
    A nondeterministic finite automaton $N$ is a $5$-tuple 
    $$ 
        (Q,\Sigma,\delta,q_0, F),
    $$
    the same as a DFA, except: 
    \begin{align*}
        \delta : Q \times \Sigma_\varepsilon \to \mathcal P(A) 
        &= \{R : R \subseteq Q\}, \\
        \Sigma_\varepsilon &= \Sigma \cup \{\varepsilon\} \text{ and}\\
        \mathcal P(Q) &= \text{the powerset of } Q
    \end{align*}

    We say that an NFA $N$ accepts a string $w$ if at least one 
    computation path reads all of $w$ and ends in a state in $F$. 
\end{defn}

This is a bit difficult to grasp with the definition alone so we look 
at an example. 

\begin{ex}
    Consider an NFA over $\Sigma = \{a,b\}$ which accepts strings 
    ending in $ab$ constructed as follows: 
    \begin{itemize}
        \item States: $Q = \{q_0, q_1, q_2\}$
        \item Start: $q_0$
        \item Accepting: $q_2$ 
    \end{itemize}
    with the $\delta$ defined such that 
    \begin{itemize}
        \item $q_0 \overset{a}{\mapsto} q_0$
        \item $q_0 \overset{b}{\mapsto} q_0$
        \item $q_0 \overset{a}{\mapsto} q_1$
        \item $q_1 \overset{b}{\mapsto} q_2$
    \end{itemize}
    Take $w = aab$. From $q_0$ with the middle symbol 
    $a$ we got to $q_0$ or $q_1$. From $q_1$ we go to $q_2$ so there 
    is one branch where we end in an accept state so 
    $w$ gets accepted. 
\end{ex}

\newpage
\begin{thm}
    If an NFA recognizes a language $A$, then $A$ is regular. 
\end{thm}
\begin{proof}
    Let the NFA $N = (Q, \Sigma, \delta, q_0, F)$ recognize $A$, 
    construct the DFA $N' = (Q', \Sigma', \delta', q_0', F')$ with 
    \begin{itemize}
        \item $Q' = \mathcal P(Q), R \in Q'$
        \item $\delta'(R,a) = \{q : q \in \delta(r,a), r \in R\}$
        \item $q_0' = \{q_0\}$
        \item $F' = \{R \in Q' : R \cap F \neq \emptyset\}$
    \end{itemize}
    At least one computation path in $N$ ends in an accept state. 
    We have all possible computations $R \in Q'$. At least one 
    $R$ accepts $w$ and $R \cap F'$ is nonempty so $N'$ accepts $w$.
\end{proof}

\end{document}
