\documentclass[11pt]{article}
\usepackage[a4paper,margin=1in]{geometry}
\usepackage{fourier} % Fourier font
\usepackage{xcolor}
\usepackage{tikz}
\usepackage[most]{tcolorbox}
\usepackage{amsthm, amsmath, amssymb}
\usepackage{enumitem}
\usepackage{hyperref}
\usepackage[nameinlink,noabbrev]{cleveref}
\usepackage{titling} 

% Dark mode colors
\definecolor{bgcolor}{HTML}{FFFFFF}
\definecolor{textcolor}{HTML}{000000}
\definecolor{defcolor}{HTML}{E86873}
\definecolor{thmcolor}{HTML}{0A9396}
\definecolor{lemcolor}{HTML}{94D2BD}
\definecolor{corcolor}{HTML}{9B4AF7}
\definecolor{probcolor}{HTML}{EE9B00}
\definecolor{excolor}{HTML}{21E933}

% Background and text color
\pagecolor{bgcolor}
\color{textcolor}

% No paragraph indentation
\setlength{\parindent}{0pt}
\setlength{\parskip}{0.7em}

% Theorem box styles
\tcbset{
  enhanced,
  colback=bgcolor,
  colframe=thmcolor,
  coltext=white,
  coltitle=white,
  fonttitle=\bfseries,
  boxrule=0.7pt,
  left=1em,
  right=1em,
  top=0.7em,
  bottom=0.7em,
  before skip=10pt,
  after skip=10pt,
}

% Theorem environments with colored boxes
\newtcbtheorem[number within=section]{thm}{Theorem}{
  colframe=thmcolor, colback=thmcolor!15!bgcolor
}{thm} % The 'thm' here is the *prefix* for the label

\newtcbtheorem[number within=section]{defn}{Definition}{
  colframe=defcolor, colback=defcolor!15!bgcolor
}{def} % The 'def' here is the *prefix* for the label

\newtcbtheorem[number within=section]{lem}{Lemma}{
  colframe=lemcolor, colback=lemcolor!15!bgcolor
}{lem}

\newtcbtheorem[number within=section]{cor}{Corollary}{
  colframe=corcolor, colback=corcolor!15!bgcolor
}{cor}

\newtcbtheorem[number within=section]{prob}{Problem}{
  colframe=probcolor, colback=probcolor!15!bgcolor
}{prob}

\newtcbtheorem[number within=section]{ex}{Example}{
  colframe=excolor, colback=excolor!15!bgcolor
}{ex}

% Proof environment 
\renewenvironment{proof}[1][\proofname]{%
  \par\pushQED{\qed}\normalfont\topsep6pt \trivlist
  \item[\hskip\labelsep\itshape #1.]\ignorespaces
}{%
  \popQED\endtrivlist\addvspace{6pt}
}

% Cleveref name formats for tcolorbox environments
\crefname{thm}{theorem}{theorems}
\Crefname{thm}{Theorem}{Theorems}

\crefname{def}{definition}{definitions}
\Crefname{def}{Definition}{Definitions}

\crefname{lem}{lemma}{lemmas}
\Crefname{lem}{Lemma}{Lemmas}

\crefname{cor}{corollary}{corollaries}
\Crefname{cor}{Corollary}{Corollaries}

\crefname{prob}{problem}{problems}
\Crefname{prob}{Problem}{Problems}

\crefname{ex}{example}{examples}
\Crefname{ex}{Example}{Examples}

\usepackage{pgfplots}
\usepackage{mathtools,amssymb}
\usepackage{tikz}
\usepackage{xcolor}
\pgfplotsset{compat=1.7}
\pgfmathdeclarefunction{gauss}{2}{\pgfmathparse{1/(#2*sqrt(2*pi))*exp(-((x-#1)^2)/(2*#2^2))}%
}

\title{\huge{Statistics for Engineering}}
\author{\LARGE{Thobias Høivik}}
\date{Fall 2025}

\begin{document}
\maketitle

\newpage
\tableofcontents

\newpage 
\section{Introduction}
This course covers introductory statistics and is taught at 
the Western Norway University of Applied Science. 
This will be among my most scattered and improvised 
notes, probably mostly in norwegian. 

\section{Lecture 1}
\emph{Mål:}
\begin{itemize}
    \item Finne matematiske størrelser som beskriver data i eit 
        datasett
    \item Sentralmål 
    \item Spredningsmål
\end{itemize}

\subsection{Sentralmål}
\begin{itemize}
    \item Finne ein verdi som representerer ein "typisk" enhet 
        i ei mengde.
\end{itemize}
\begin{ex}{}{}
    Lønna til $13$ personer oppgitt som: 
    $110, 125, 125, 300, 350, 370, 375, 380, 390, 410, 430, 435, 440$.
\end{ex}

\emph{Modus/typetal.}

Verdien som dukkar opp flest gangar.
Fra eksempellet har at vi moduslønn er 125.

\emph{Median.}
Median er den midterste verdien i ei datamengde. I dette tilfellet 
fra Example 2.1 har vi at medianlønn ligger på $375$. I tilfellet 
hvor det er et partall antall verdier tar vi gjennomsnittet 
av de to midterste verdiene.

\emph{Gjennomsnitt.}

Gjennomsnittet av en datamengde $X$ er gitt ved 
$$ 
    \overline X = \frac{1}{|X|} \displaystyle\sum_{i=1}^{|X|} x_i
$$ 
I dette tilfellet, fra Example 2.1, har vi gjennomsnitt på 
$326$.

Anta nå at en $14$-ende tjener blir lagt til i mengden som tjener 
$30,000$. Da er gjennomsnittslønnen på de $14$ personene 
rundt $2445$ som ligger langt over den nest høgste tjeneren. 
Med andre ord drar person nummer $14$ gjennomsnittet opp vanvittigt.

\subsection{Spredningsmål.}

Målet med et spredningsmål er å beskrive hvor stor variasjon
det er i datasettet.

\begin{ex}{}{}
    Gitt to mengder $X, Y$. 

    $X = \{80,90,100,110,120\}$
    
    $\overline X = 100$

    $Y = \{20, 60, 100, 140, 180\}$

    $\overline Y = 100$
\end{ex}

\emph{Variasjonsbredde.}

Variasjonsbredda er største verdi minus minste verdi: 
$$ 
    \max X - \min X
$$ 
Fra Example 2.2 har vi at at variasjonsbredda til $X$ er $40$ og 
variasjonsbredda til $Y$ er 160.

\emph{Varians og standardavik.}

Varians: 
$$
\text{var}(X) = s_X^2 = \frac{1}{|X|-1}
\displaystyle\sum_{i=1}^{|X|} \left(x_i - \overline X\right)^2
$$

Standardavik: 
$$ 
\text{std}(X) = \sqrt{s_X^2} = \sqrt{\text{var}(X)}
$$ 

Viss vi ser på Example 2.3 får vi 
$$ 
    \text{var}(X) = 
    \frac{(80-100)^2 + (90-100)^2 + \dots + (120 - 100)^2}{4}
    = 250 = 15.8^2
$$ 
og 
$$ 
    \text{std}(X) = 15.8 
$$ 
mens for $Y$
$$
    \text{var}(Y) = 
    \frac{(20-100)^2 + (40-100)^2 + \dots + (180 - 100)^2}{4}
    = 4000 = 63.25^2
$$ 
$$ 
    \text{std}(Y) = 63.25
$$ 

\subsection{Tsjebytsjevs regel}
"Sammenhengen mellom ei datamengde, gjennomsnitt og standardavik."
 
Minimum $75\%$ av observasjonane har verdi i intervallet
$$ 
    \left[\overline X - 2s_X, \overline X + 2s_X\right]
$$ 

\newpage
\section{Lecture 2}
Vi begynner med begrepet sannsyn.

Mål:
\begin{itemize}
    \item kjennskap til sentrale begrep 
    \item rekne ut sannsyn 
    \item kjenne til grunnleggande begrep frå mengdelære 
\end{itemize}

\subsection{Stokastisk forsøk, utfallsrom og hending}
\begin{defn}{Stokastisk forsøk}{}
    \begin{itemize}
        \item Vi veit kva utfall som er mulig 
        \item Berre eitt utfall kan skje i kvart forsøk 
        \item Veit ikkje utfallet vil bli
    \end{itemize}
\end{defn}
\begin{ex}{}{}
    Terningskast er eit eksempel på eit stokastisk forsøk. 
    Vi veit at vi får $1-6$. Vi kan kunn få eit utfall. 
    Vi veit ikkje på førhånd kva vi får.

    Vi kaster $2$ terninger. Dei mulige utfalla er 
    $$ 
    \begin{matrix}
        1,1 & 1,2 & 1,3 & 1,4 & 1,5 & 1,6 \\ 
        2,1 & 2,2 & 2,3 & 2,4 & 2,5 & 2,6 \\ 
        3,1 & 3,2 & 3,3 & 3,4 & 3,5 & 3,6 \\ 
        4,1 & 4,2 & 4,3 & 4,4 & 4,5 & 4,6 \\ 
        5,1 & 5,2 & 5,3 & 5,4 & 5,5 & 5,6 \\ 
        6,1 & 6,2 & 6,3 & 6,4 & 6,5 & 6,6 \\ 
    \end{matrix}
    $$
    som utgør $6^2 = 36$ mulige utfall.
\end{ex}
\begin{ex}{}{}
    Kaster eit kronestykke $3$ gongar. 
    Dei mulige utfalla er 
    $$
    \begin{matrix}
        KKK & KKM & KMK & KMM \\ 
        MKK & MKM & MMK & MMM 
    \end{matrix}
    $$
    som utgjør $2^3 = 8$ mulige utfall.
\end{ex}
\begin{defn}{Utfallsrom og hending}{}
    Utfallsrommet $S$ er mengda av alle mulige utfall.
     
    Hending er eitt eller fleire utfall som oppfyller ein 
    gitt betingelse.
\end{defn}
\begin{ex}{}{}
    Viss vi tar utgangspunkt i Example 3.1 
    kan vi se på hendinga 
    $A = \{\text{sum er } 4\} = \{(3,1),(2,2), (1,3)\}$. 
    $B = \{\text{minst eit partal}\} = \{(1,2), (1,4), (1,6), \dots\}$
    som gir $|B| = 27$.
\end{ex}
\begin{ex}{}{}
    Viss ein målar høgda til personer er tilstandsrommet 
    $S = \{\text{alle reelle tal mellom } 0.5 \text{ og } 2.5\}$.
\end{ex}

\subsection{Sannsyn for handling}
\begin{itemize}
    \item Bruke modell for sannsyn 
    \item Empiri
    \item Magefølelse
\end{itemize}

\emph{Notasjon.} 

La $A$ være ei hending: 

\begin{itemize}
    \item sannsyn for $A : P(A)$
    \item $0 \leq P(A) \leq 1$
\end{itemize}


\begin{defn}{Uniform sannsynlighetsmodell}{}
    La $S$ være utfallsrommet til eit stokastisk forsøk, 
    og la $A$ være ei hending.

    Viss alle utfall er like sannsynlige, 
    då er 
    $$ 
        P(A) = \frac{\text{antal utfall der } A \text{ er gunnstig}}
        {\text{antal mulige utfall}}
    $$ 
\end{defn}

\begin{ex}{}{}
    Betrakt terningskast eksempellet fra tidligere. 

    La $S$ være terningskast med to terninger. 

    $$ 
        A = \{\text{sum er } 4\}
    $$ 
    $$ 
        P(A) = \frac{3}{6^2} = \frac{3}{36} = \frac{1}{12}
    $$ 
    $$ 
        B = \{\text{minst eitt partal}\} 
    $$ 
    $$ 
        P(B) = \frac{27}{36} = \frac{3}{4}
    $$ 
\end{ex}

\newpage
\emph{Empirisk sannsyn.}

Kva er sannsynligheita for at ein person 
er mellom $175$cm og $180$cm?

Dette er ikkje ein uniform modell siden det ikkje er like stor 
sannsynligheit for ein person å ha ein gitt høgde.

Det vi kan gjere da er å velge $1000$ personer, måle høgda 
og lage ein frekvenstabell.

Da ville vi hatt 
$$ 
    P(\text{mellom } 175 \text{ og } 180) = \text{ relativ frekvens }
$$ 

\emph{Store tals lov.}

Viss eit stokastisk forsøk blir gjentatt mange 
gangar så vil den empiriske sannsynligheita 
nærme seg den teoretiske sannsynligheita.

\subsection{Mengdelære}
\begin{ex}{}{}
    $A = \{\text{sum} = 4\}$, $P(A) = \frac{1}{12}$. 

    $B = \{\text{minst eitt partal}\}$, $P(B) = \frac{3}{4}$. 

    $C = \{\text{sum}\geq 9\}$, $P(C) = \frac{5}{18}$. 

    $P(\text{sum} = 4 \text{ og minst eitt partal})$? 

    $P(\text{sum} ) 4 \text{ og sum} \geq 9)$?
\end{ex}

\begin{defn}{Mengde}{}
    Ei mengde $A$ er ei samling element 
    (fra ei utvalgsmengde $S$).
\end{defn}

\begin{defn}{Union, snitt, komplement, disjunkt}{}
    La $A,B$ være mengder med element fra ei utvalgsmengde $S$.

    \emph{Union.}

    $$ 
        A \cup B = \{x : x \in A \lor x \in B\}
    $$ 

    \emph{Snitt.}

    $$
        A \cap B = \{x : x\in A \land x \in B\} 
    $$ 

    \emph{Komplement.}

    $$ 
        \overline A = A^c = S \setminus A = \{x \in S : x \not\in A\}
    $$ 

    \emph{Disjunkt.}

    $A$ og $B$ er disjunkte viss 
    $$ 
        A \cap B = \emptyset
    $$ 
\end{defn}

\newpage 
\section{Lecture 3}

\subsection{Grunnreglar for sannsynlighet}
La $S$ være utfallsrommet til eit stokastisk forsøk, 
og la $A,B$ være hendingar.
\begin{enumerate}
    \item $0 \leq P(A) \leq 1$
    \item $P(S) = 1, P(\emptyset) = 0$ 
    \item Viss $A$ og $B$ er disjunkte hendingar 
        $(A\cap B = \emptyset)$, då er 
        $P(A \cup B) = P(A) + P(B)$
    \item $P(\overline A) = 1 - P(A)$
\end{enumerate}
\begin{ex}{}{}
    Betrakt kastingen av to terninger. 

    $A = \{\text{summen er }4\}$

    $B = \{\text{minst } 1 \text{ terning er partal}\}$

    $C = \{\text{sum } \geq 9\}$

    $P(A) = \frac{3}{36}, P(B) = \frac{27}{36}, P(C) = \frac{10}{36}$

    $P(A \cup C) = P(A) + P(C) = \frac{13}{36}$ siden 
    $A$ og $C$ er disjunkte (ingen tall $x$ 
    oppfyller $x=4 \land x \geq 9$ samtidig).

    $P(A\cup B)$. Her kan vi ikke bruke 
    regel $3$, siden $2 + 2 = 4$ og $(2,2)$ oppfyller at minst 
    en er partal, så $A \cap B = \{(2,2)\} \neq \emptyset$.
    I dette tilfellet har vi 
    $P(A\cup B) = P(A\cup B) - P(A\cap B) = \frac{29}{36}$
    (som addisjonsregelen for ikke-disjunkte mengder fra MAT210).

    $P(\overline{A\cup B}) = 1 - P(A\cup B) = \frac{7}{36}$
\end{ex}

\subsection{Betinget sannsynlighet}
\begin{ex}{}{}
    Eit bilfirma har $2000$ tilsette. 
    
    La 

    \quad $A = \{\text{stemmer Ap}\}$

    \quad $B = \{\text{er bilmekaniker}\}$

    La oss si at vi vet at en person er bilmekaniker, 
    hva er da sjansen for at de stemmer Ap? 

    Viss vi vet at blant bilmekanikere så stemmer 
    $500$ Ap og $380$ ikke da er sjansen 
    for at, gitt en bilmekaniker, de stemmer Ap 
    $$ 
        P(A | B) ) \frac{500}{500+380} = \frac{500}{880}
    $$ 
\end{ex}

Den betinget sannsynligheten for at $A$ skjer viss $B$ har skjedd 
$$
    P(A | B) = \frac{P(A\cap B)}{P(B)}
$$ 

Fra dette følger det et par formler 
\begin{enumerate}
    \item $P(A\cap B) = P(A|B)\cdot P(B)$
    \item $P(A) = P(A\cap B) + P(A \cap \overline B)$
    \item $P(A) = P(A|B)\cdot P(B) + 
        p(A|\overline B)\cdot P(\overline B)$
\end{enumerate}

\begin{thm}{Bayes teorem}{}
    La $A,B$ være hendingar 
    da har vi at 
    $$ 
        P(A|B) = \frac{P(B|A)P(A)}{P(B)}
    $$ 
\end{thm}
\begin{proof}[Kort bevis]
    \begin{align*}
        P(A|B) &= \frac{P(B|A)P(A)}{P(B)}
            \\ &= \frac{\frac{P(B\cap A)}{P(A)}P(A)}{P(B)} 
            \\ &= \frac{P(B\cap A)}{P(B)}
            \\ &= \frac{P(A \cap B)}{P(B)} 
            \\ &= P(A|B)
    \end{align*}
\end{proof}

\emph{Sensitivitet, spesifisitet, basisrate.}

La  

\quad $B = \{\text{pasient er syk}\}$

\quad $A = \{\text{testmetode er positiv}\}$

Vi har som mål å finne sjansen for at en person er syk når 
testen er positiv, $P(B|A)$.

\begin{itemize}
    \item Sensitivitet: $P(A|B)$, sannsynligheten for at 
        test er positiv når pasient er syk
    \item Spesifisitet: $P(\overline A | \overline B)$, 
        sannsynlighet for at test er negativ når pasient 
        er frisk
    \item Basisrate: $P(B)$, sannsynlighet for at gitt 
        pasient er syk
\end{itemize}

\begin{align*}
    P(B|A) &= P(A|B) \frac{P(B)}{P(A)} \\ 
    P(A)   &= P(A|B)\cdot P(B) + P(A|\overline B) P(\overline B) \\ 
    P(\overline B) 
           &= 1 - P(B) \\ 
    P(A|\overline B) 
           &= 1 - P(\overline A | \overline B) \\ 
    \Rightarrow P(B|A) 
           &= P(A|B) \frac{P(B)}{P(A|B) \cdot P(B) + 
           (1 - P(\overline A | \overline B))(1-P(B))}
\end{align*}

\newpage 
\section{Lecture 4}
\subsection{Diskret synnsynlighetsmodell}
Mål: 
\begin{itemize}
    \item kva er ein stokastisk variabel
    \item innholdet i sannsynlighetsmodell
    \item fordelingsfunksjonar og anvendelse        
    \item sentralmål og spredningsmål for diskret sannsynlighetsmodell
\end{itemize}
\begin{defn}{Stokastisk variabel}{}
    Ein stokastisk variabel $X$ er 
    ein tilfeldig variabel som knytter alle utfall i 
    utfallsrommet til ein verdi.
\end{defn}

\begin{defn}{Verdimengde}{}
    Dei moglege verdiane av eit stokastisk forsøk kalles 
    verdimengda $V_X$.
\end{defn}

\begin{ex}{}{}
    Anta at vi kaster $2$ terninger.
    
    La $X = \{\text{summen av terningene}\}$. 
     
    $X = 4 = \{13, 22, 31\}$. 

    $X \leq 4 = \{11, 12, 21, 13, 22, 31\}$.

    $V_X = \{2, 3, 4, \dots, 9, 10, 11, 12\}$.

    $Y = \{\text{antall partall}\}$. 

    $V_Y = \{0,1,2\}$.

    I eit pengespill taper du $10$ viss 
    sum $\leq 6$, vinner $8$ viss sum $ = 7$, og vinner 1 
    viss sum $\geq 8$.

    Viss vi lar $Z$ betegne den stokastiske variabelen i pengespillet
    får vi: 

    $V_Z = \{-10,8,1\}$.
\end{ex}

\begin{defn}{Sannsynlighetsfordeling}{}
    Ein sannsynlighetsfordeling til ein stokastisk 
    variabel $X$ inneholder: 
    \begin{enumerate}
        \item verdimengden $V_X$
        \item sannsynligheten for alle mulige utfall i $V_X$
    \end{enumerate}
\end{defn}
$$ 
    \begin{array}{c|c|c|c|c}
        V_X & P(X = x) & F(x) & x\cdot P(X=x) & x^2 \cdot P(X = x)\\ 
        \hline  
        2 & 1/36 & 1/36  & 2/36  & 4/36   \\
        3 & 2/36 & 3/36  & 6/36  & 18/36  \\
        4 & 3/36 & 6/36  & 12/36 & 43/36  \\
        5 & 4/36 & 10/36 & 20/36 & 100/36 \\
        6 & 5/36 & 15/36 & 30/36 & 180/36 \\
        7 & 6/36 & 21/36 & 42/36 & 294/36 \\
        8 & 5/36 & 26/36 & 40/36 & 320/36 \\
        9 & 4/36 & 30/36 & 36/36 & 324/36 \\
        10& 3/36 & 33/36 & 30/36 & 300/36 \\
        11& 2/36 & 35/36 & 22/36 & 242/36 \\
        12& 1/36 & 36/36 & 12/36 & 144/36 \\
          &      &       & E(X) = 252/36 = 7
                         & E(X^2) = 1974/36  \\ 

    \end{array}
$$ 
er hvordan synnsynlighetsfordelingen kan se ut for 
$X = \{\text{summen av terningene}\}$.
$F(x)$ er definert som sannsynligheten for at $X \leq x$. 
Med andre ord 
$$
    F(x) = P(X \leq x) = 
    \displaystyle\sum_{y \in V_X, y \leq x}P(X = y) 
$$

\begin{ex}{}{}
    Finn sannsynlighet for at sum er $7$ eller mindre. 
    Da skal vi finne $P(X \leq 7)$ som næmlig er 
    $$F(7) = \frac{21}{36}$$ 

    Finn sannsynlighet for at sum er større enn $8$. 
    
    $$ 
        P(X > 8) = 1 - P(x \leq 8) = 1 - F(8) = \frac{10}{36}
    $$ 
\end{ex}

\emph{Sentralmål og spredningsmål.}

Forventningsverdi: 
$$ 
    E(X) = \mu_X = \displaystyle\sum_{x \in V_X}^{}xP(X=x)
$$ 
$$ 
    Var(X) = \sigma_X^2 = 
    \left(\displaystyle\sum_{x\in V_X}x^2P(X=x)\right) - \mu_X^2
$$ 
$$
    std(X) = \sigma_X 
$$ 

\newpage 
\section{Lecture 5}
\subsection{Fleire stokastiske variabler}
I eit terningspel kaster ein to terningar. 
Det er $2$ måtar å vinne.

I spel $1$ vinner ein $2$ viss summen av terningane er $7$, 
vinner $1$ viss sum $\geq 8$ og ein vinner $0$ viss 
sum $\leq 6$.

I spel $2$ vinner ein $2$ viss begge terningane er partal, 
$1$ viss ein av terningane er partal og $0$ viss begge er 
oddetal.

La $X = \{\text{gevinst spel } 1\}$, $V_X = \{0,1,2\}$. 
$Y = \{\text{gevinst spel } 2\}$, $V_Y = \{0,1,2\}$.

\begin{align*}
\begin{array}{c|c|c|c|c}
    & y = 0 & y = 1 & y = 2 & P(X=x)
    \\
    \hline 
    x = 0 & 1/6 & 1/6 & 1/12 & 5/12
    \\ 
    \hline 
    x = 1 & 1/12 & 1/6 & 1/6 & 5/12
    \\ 
    \hline 
    x = 2 & 0 & 1/6 & 0 & 1/6
    \\ 
    \hline 
    P(Y=y) & 1/4 & 1/2 & 1/4 & 1
\end{array}
\end{align*} 

$E(X), E(Y), Var(X), Var(Y)$

$$ 
    E(X) = \displaystyle\sum_{}^{}xP(X=x) = \mu_X
$$ 
$$ 
    Var(X) = \displaystyle\sum_{}^{}x^2P(X=x) - \mu_X^2
$$ 
$$ 
    E(X) = 0 \cdot \frac{5}{12} + 1 \cdot \frac{5}{12} 
    + 2 \cdot \frac{1}{6} = \frac{3}{4} = \mu_X
$$ 
$$ 
    Var(X) = 0 + \frac{5}{12} + 4 \cdot \frac{1}{6} - \mu_X^2 
    = \frac{13}{12} - \mu_X^2 = \frac{13}{12} - \frac{9}{16}
    = \frac{25}{48} = \left(\frac{5 \sqrt 3}{12}\right)^2
    = \sigma_X^2
$$ 
$$ 
    E(Y) = \frac{1}{2} + \frac{2}{4} = 1 = \mu_Y
$$ 
$$ 
    Var(Y) = \frac{1}{2} + \frac{4}{4} - \mu_Y^2
    = \frac{1}{2} = \frac{1}{\sqrt 2} = \sigma_Y^2
$$ 

\emph{Interessante spørsmål.}

\begin{itemize}
    \item Viss eg spelar på begge spela, 
        kva er forventa verdi og varians?
    \item Viss eg vinner på spel $1$, 
        aukar eg eller minkar eg sjansen for å vinne på 
        spel $2$?
    \item Er spela uavhengige av kvarandre? 
\end{itemize}

\newpage 
La $Z = X + Y$ (den samla gevinsten ved å spele på begge spela).
Verdimengda $V_Z = \{0,1,2,3,4\}$.

\begin{align*}
    \begin{array}{c|c|c|c}
        z & P(Z=z) & zP(Z=z) & z^2P(Z=z) \\ 
        \hline 
        0 & 1/6 & 0 & 0 \\ 
        \hline 
        1 & 1/4 & 1/4 & 1/4 \\
        \hline 
        2 & 1/4 & 1/2 & 1\\
        \hline 
        3 & 1/3 & 1 &  3\\
        \hline 
        4 & 0 & 0 & 0\\
        \hline 
          & & \mu_Z = 7/4 & E(Z^2) = 17/4
    \end{array}
\end{align*}

$E(Z) = \mu_Z = 7/4$ 

$Var(Z) = 17/4 - (7/4)^2 = (\sqrt 19 / 4)^2 = \sigma_Z^2$

\begin{thm}{}{}
    La $X$ og $Y$ vere stokastiske variablar. 
    La $Z = aX + bY + c$. 

    $$ 
        E(Z) = aE(X) + bE(X) + c
    $$ 
\end{thm}

\subsection{Kovarians og korrelasjon}
\begin{itemize}
    \item Er det samanheng mellom $X$ og $Y$? 
    \item Er $X$ og $Y$ uavhengige? 
    \item Vil seier i $X$ påvirke $Y$?
\end{itemize}

\emph{Kovarians}
$$ 
    Cov(X,Y) = E(X \cdot Y) - \mu_X \cdot \mu_Y
    = 
    \left(\displaystyle\sum_{x,y} x \cdot y P(X=x \land Y=y)\right) 
    - \mu_X\mu_Y
$$ 

\emph{Korrelasjon}
$$ 
    Corr(X,Y) = \rho(X,Y) = \frac{Cov(X,Y)}{\sigma_X\sigma_Y}
$$ 

Viss vi betrakter eksempellet fra tidligere får vi 
$$ 
E(X \cdot Y) = \frac{1}{6} + 2 \frac{1}{6} + 2\frac{1}{6} 
= \frac{5}{6}
$$ 
så 
$$ 
    Cov(X,Y) = \frac{5}{6} - \mu_X \mu_Y = \frac{1}{12} 
$$ 

Deretter finner vi korrelasjon ved 
$$ 
    Corr(X,Y) = \frac{1/12}{\sigma_X\sigma_Y} = 
    \frac{1/12}{\left(\frac{5\sqrt 3}{12}\right)
    \left(\frac{1}{\sqrt 2}\right)} = \frac{\sqrt 2}{5\sqrt 3} = 0.164
$$ 

For korrelasjon har vi at 
$$ 
    -1 \leq Corr(X,Y) \leq 1
$$ 
hvor nærmere $-1$ betyr mer negativ relasjon og nærmere
$1$ betyr mer positiv relasjon. 
Med andre ord $\rho$ nær $1$ betyr ein lineær samanheng mellom 
$X$ og $Y$ (altså en økning i $X$ betyr en økning i $Y$ og vice-versa).
På samme måte betyr $\rho$ nær $-1$ at en økning i en er en minking
i den andre.
$\rho = 0$ betyr ingen lineær samanheng.

\begin{thm}{}{}
    For stokastiske variabler $X$ og $Y$ og konstanter 
    $a,b,c$ har vi:
    $$ 
    Var(aX + bY + c) = a^2 Var(X) + b^2 Var(Y) + 2ab Cov(X,Y)
    $$
\end{thm}

\begin{defn}{Uavhengighet}{}
    To stokastiske variablar $X$ og $Y$ er uavhengige 
    dersom 
    $$
        P(X = a \land Y = b) = P(X=a)\cdot P(Y=b)
    $$ 
\end{defn}

\begin{thm}{}{}
    Viss to stokastiske variablar $X$ og $Y$ er uavhengige  
    så er 
    $$ 
        Corr(X,Y) = 0
    $$ 

    Denne implikasjonen går bare en veg.
\end{thm}

\newpage 
\section{Lecture 6}
\subsection{Binomiske forsøk}
\begin{ex}{}{}
    Ei skål inneholder $20$ kuler. $8$ av kulene er raude og 
    $12$ er ikkje raude. Vi trekker $5$ kuler ut av skåla, 
    og legger tilbake kula.
    Kva er sannsynligheita for å trekke $3$ raude kuler?

    La $R$ - kula raud, $P(R) = \frac{2}{5}$. 

    $\overline R$ - kula ikkje raud, $P(\overline R) = \frac{3}{5}$.

    Sannsynligheten for å trekke $3$ raude er då sannsynligheita 
    for å trekke raud $3$ gongar og sannsynligheita for å 
    trekke ikkje raud $2$ gongar, siden hendelsane ikkje 
    er relaterte (tilbakelegging). 

    Dermed er 
    $$ 
        P(RRR\overline R \overline R) = \left(\frac{2}{5}\right)^3 
        \left(\frac{3}{5}\right)^2 = \frac{72}{3125}
    $$ 
    men dette er bare en av måtene å trekke $3$ raude. 

    Vi kunne også tatt 
    $$ 
        P(RR\overline R R \overline R) = \frac{72}{3125}
    $$ 
    
    Som vi ser, sannsynligheten for å trekke $3$ raude er 
    uavhengig av rekkefølgen. 
    Dermed kan vi legge sammen alle sannsynligheitene for 
    alle måtene å velge $3$ raude og $2$ ikkje raude. 

    Måtene å velge $3$ raude er 
    $$ 
        \binom{5}{3} = 20
    $$ 

    Dermed er sjansen for å trekke $3$ raude er 
    $$ 
        20 \cdot \frac{72}{3125}
    $$ 
\end{ex}
\begin{defn}{Binomisk forsøk}{}
    \begin{enumerate}
        \item Det utføres $n$ forsøk
        \item Kvart forsøk har to utfall $A$ eller $\overline A$
        \item Sannsynet $P(A)$ er likt i kvart forsøk 
        \item Forsøka er uavhengige 
    \end{enumerate}
    
    Viss $X$ - antal ganger $A$ skjer er 
    $X$ \textasciitilde binomial $(n,p)$.

    $$ 
    P(X = x) = \binom{n}{x}P^x(1-p)^{n-x}
    $$ 
    $$ 
        E(X) = np
    $$ 
    $$ 
        Var(X) = np(1-p)
    $$ 
\end{defn}

\begin{ex}{}{}
    Trekker kuler ut av ei skål. 
    Legger kula tilbake. 
    $8$ raude, $12$ ikkje raude. 
    Trekker $5$ kuler. 

    $X$ \textasciitilde antall raude kuler, så er det 
    binomial($5$, $\frac{2}{5}$). 

    $E(X) = 5 \cdot \frac{2}{5} = 2$

    $Var(X) = 5 \cdot \frac{2}{5}(1-\frac{2}{5}) = 1.2$. 

    $$
    p(X=0) = \binom{5}{0}\left(\frac{2}{5}\right)^0 
    \left(\frac{3}{5}\right)^5 = 0.074
    $$   
    $$ 
        P(X=1) = \binom{5}{1} \left(\frac{2}{5}\right)^1 
        \left(\frac{3}{5}\right)^4 = 0.259
    $$ 
    $$ 
        P(X=2) = \binom{5}{2} \left(\frac{2}{5}\right)^2 
        \left(\frac{3}{5}\right)^3 = 0.346
    $$ 
\end{ex}

\begin{prob}{Oppgave 5.2}{}
    Du er sulten, og kjøper åtte tilfeldig valgte pølser. 
    Sannsynet for at pølsa er sprukken er $0.2$ for hver av 
    pølsene. La $X$ være antall sprukne pølser blant de $8$. 
    Hvilken sannsynlighetsfordeling for $X$? Finn $P(X=2)$.
\end{prob}
\begin{proof}[Løsning]
    $X$ er et binomisk forsøk siden hver utfallet er sprukken 
    eller ikke sprukken. Det at en pølse er sprukken påvirker ikke 
    sannsynet for at en annen pølse er sprukken, så forsøka 
    er uavhengige. Dermed er $X$ \textasciitilde binomial($8$,$0.2$).

    $$ 
        P(X=2) = \binom{8}{2} \left(0.2\right)^2\left(0.8\right)^6 
        = 0.29360128
    $$ 
    litt mindre en $1/3$ gongar.
\end{proof}

Med utgangspunkt i Oppgave 5.2, hva er da sjansen for at mellom 
$2$ og $5$ av pølsene sprekker? 

\begin{align*}
    P(2 \leq X \leq 5) &= \displaystyle\sum_{x=2}^{5} P(X=x)
                    \\ &= \binom{8}{2}0.2^2 0.8^6 + \cdots + 
                         \binom{8}{5}0.2^5 0.8^3  
\end{align*}
som blir en ganske lang utregning. Kan vi være mer effektive?

Fra \emph{Kapittel 4.1} har vi 
$$ 
    P(a \leq X \leq b) = P(X \leq b) - P(X < a)
$$ 
og 
$$ 
    P(X \geq a) = 1 - P(X < a)
$$ 

Så 
$$ 
    P(2 \leq X \leq 5) = P(X \leq 5) - P(X \leq 1) 
    = 0.999 - 0.503 = 0.496
$$ 

\subsection{Poissonfordeling}
\begin{defn}{Poissonprosess}{}
    La $A = \{\text{antall førekomster}\}$. 
    Da er $A$ ein poissonprosess viss 
    \begin{enumerate}
        \item Antall gongar $A$ intreffer i eit tidsinterval 
            er uavhengig av andre førekomster. 

        \item Forventa antal førekomster i ei gitt tidsenhet er 
            konstant.

        \item To førekomster kan ikkje skje samtidig.
    \end{enumerate}
\end{defn}

\begin{thm}{}{}
    $X$ - antal gongar $A$ skjer. 

    $X$ \textasciitilde poisson($\lambda t$) der 
    $\lambda$ er en konsentrasjon og $t$ er eit tidsinterval. 

    Da er 
    $$ 
        E(X) = Var(X) = \lambda t
    $$ 
    $$ 
        P(X = x) = \frac{(\lambda t)^x}{x!}e^{-\lambda t}
    $$ 
\end{thm}
\begin{ex}{}{}
    Ved en bensinstasjon er det i snitt $10$ bilar som fyller bensin 
    per time. Da er $\lambda = \frac{10 \text{ bilar}}{t}$. 
    
    La $Y = \{\text{antall tankingar på } 30 \text{ min}\}$.  
    Vi antar at raten ikke endrer seg, at forventet antal biler 
    per time er konstant og at to biler ikke kan tanke på same tid. 

    Da er $Y$ en poissonprosess med $t = 0.5$, så 
    $Y$ \textasciitilde poisson($5$). 

    \begin{align*}
        P(Y = 10) &= \frac{5^{10}}{10!}e^{-5}
        = 0.018
        \\ 
        P(Y = 4) &= \frac{5^4}{4!}e^{-5} = 0.175
        \\ 
        P(Y \leq 8) &= \displaystyle\sum_{y = 0}^{8}P(Y=y) 
        \\ 
    \end{align*}
    Dette er langt å rekne ut så vi bruker en tabell for å finne
    \begin{align*}
        P(Y \leq 8) &= 0.932 
    \end{align*}
\end{ex}

Et viktig punkt er at $\lambda$ og $t$ ikkje treng å være 
tidsavhengig. For eksempel kan 
$\lambda$ være antall mol/liter og $t$ være volum. 
Eller $\lambda$ sauar/$m^2$ og $t$ være $m^2$, osv.

\begin{ex}{}{}
    I prøvetaking av avløpsvatn kan du spore molekyla i vatnet. 
    Vi forventer $100$ molekyl fentanyl pr.liter.
    Ein tank inneholder $40mL$.

    $X =$ antall molekyl $=$ poisson($100\cdot 0.04$) $=$ 
    possion($4$) 

    $E(X) = 4$ molekyl med fentanyl. 

    $Var(X) = 4$, så $90\%$ av tiden har vi mellom $2$ og $6$
    molekyl med fentanyl. 

    \begin{align*}
        P(X=8) = \frac{(\lambda t)^x}{x!} e^{-\lambda t} 
        = \frac{4^8}{8!}e^{-4} = 0.030
    \end{align*} 
    så det er $3\%$ sjanse å finne $8$ molekyl fentanyl i en 
    $40mL$ prøve.

    \begin{align*}
        P(3 \leq X \leq 5) 
        &= P(X \leq 5) - P(X \leq 2)
        \\
        &= 0.547 
    \end{align*}
    via poisson-tabell. 

\end{ex}

\newpage 
\section{Lecture 7 | Kontinurelige sannsynlighetsmodellar}
\begin{defn}{Sannsynlighetstetthet}{}
    Sannsynlighetstettheten $f(x)$ beskriver 
    fordelinga av ein kontinuerlig variabel, og 
    oppfyller: 
    \begin{enumerate}
        \item Arealet under kurven er $1$. 
            $$ 
                \displaystyle\int_{-\infty}^{\infty} f(x) dx = 1
            $$ 

        \item Sannsynligheten for utfall mellom $a$ og $b$ er 
            gitt ved 
            $$ 
                P(a \leq X \leq B) = \displaystyle\int_{a}^{b} f(x)dx
            $$ 

        \item $f(x) \geq 0$
    \end{enumerate}
\end{defn}

\begin{defn}{Fordelingsfunksjon}{}
    La 
    $$
        F(x) = \displaystyle\int_{-\infty}^{x}f(x)dx
    $$

    Da er $F(a)$ sannsynligheten for at $X$ er mindre enn eller 
    like $a$, i.e. $P(X \leq a)$.

    Da er $1 - F(a)$ sannsynligheten for at $X$ er større enn 
    $a$, i.e. $P(a < X)$.

    $F(B) - F(A)$ er sannsynligheten for $P(a < X \leq b)$. 
\end{defn}

\begin{defn}{Normalfordelt}{}
    Ein variabel $X$ er normalfordelt dersom med middelverdi 
    $E(X) = \mu$ og standardavik $\sigma$. 

    $$ 
    X \text{\textasciitilde} \mathcal N(\mu, \sigma)
    $$ 
    viss tetthetsfunksjonen er 
    $$ 
        f(x) = \frac{1}{\sqrt{2\pi} \sigma}
        e^{-\frac{(x-\mu)^2}{2\sigma^2}}
    $$ 

\end{defn}

Tetthetsfunksjonen beskrivd i $8.3$ ser slik ut: 

\begin{tikzpicture}

    \begin{axis}[no markers, domain=0:10, samples=100,
        axis lines*=left, xlabel=Normalfordeling, ylabel=axis $y$,
        height=6cm, width=\textwidth,
        enlargelimits=false, clip=false, axis on top,
        grid = major]
        \addplot [fill=cyan!20, draw=none, domain=-3:3] {gauss(0,1)} \closedcycle;
        \addplot [fill=orange!20, draw=none, domain=-3:-2] {gauss(0,1)} \closedcycle;
        \addplot [fill=orange!20, draw=none, domain=2:3] {gauss(0,1)} \closedcycle;
        \addplot [fill=blue!20, draw=none, domain=-2:-1] {gauss(0,1)} \closedcycle;
        \addplot [fill=blue!20, draw=none, domain=1:2] {gauss(0,1)} \closedcycle;
    \end{axis}
\end{tikzpicture}

Her har vi $\mu = 0$ som fører til at høydepunktet på kurven er 
ved $0$.

\emph{Regel.}

Anta at $X$ \textasciitilde $\mathcal N(\mu, \sigma)$. 

Da er 
$$
P(X \leq b) = G\left(\frac{b-\mu}{\sigma}\right) 
$$ 

$$ 
    P(a \leq X \leq b) = G\left(\frac{b-\mu}{\sigma}\right) - 
    G\left(\frac{a - \mu}{\sigma}\right)
$$ 
og 
$$ 
    P(X > b) = 1 - G\left(\frac{b-\mu}{\sigma}\right)
$$ 

Hvor vi leter opp  
$$ 
    G\left(\frac{b - \mu}{\sigma}\right)
$$ 
i en tabell.

\begin{ex}{}{}
    Anta at høgda av $18$ åringar er 
    $\mathcal N(180, 7)$.

    \begin{itemize}
        \item Kva er sannsynlighet for å velge en person mellom 
            $175$- og $185$cm?
    \end{itemize}

    \begin{align*}
        P(175 \leq X \leq 185) 
        &= 
        G\left(\frac{5}{7}\right) - G\left(-\frac{5}{7}\right)
        \\ 
        &= 
        G(0.71) - G(-0.71)
        \\
        &= 
        0.7611 - 0.2389 
        \\ 
        &= 
        0.52
    \end{align*}

    \begin{itemize}
        \item Kva er sannsynligheten for å velge ein person 
            høgare enn $200$cm?
    \end{itemize}
    \begin{align*}
        P(X \geq 200)  
        &= 
        1 - G\left(\frac{20}{7}\right) 
        \\ 
        &= 
        1 - G(2.86)
        \\ 
        &= 1 - 0.9979 
        \\ 
        &= 
        0.0021
    \end{align*}

    \begin{itemize}
        \item Finn eit intervall $[180-x, 180+x]$ slik 
            at det er $90\%$ sjanse for at ein gitt person 
            har høgde innafor intervallet. 
    \end{itemize}
    \begin{align*}
        P(180 - x \leq X \leq 180 + x) 
        &= 0.9  
        \\ 
        G\left(\frac{x}{7}\right) - G\left(-\frac{x}{7}\right)
        &=
        0.9 
        \\ 
        x/7 &= 1.65
        \\ 
        x &= 11.55
    \end{align*}

    Så intervallet blir $[168.45, 191.55]$. 
\end{ex}

\begin{thm}{Spredningsinterval}{}
    La $X$ \textasciitilde $\mathcal N(\mu, \sigma)$. 

    \begin{itemize}
        \item Da er det $100(1-\alpha)\%$ sikkert at utfallet 
            av $X$ er mindre enn $\mu + Z_\alpha \sigma$.

        \item Det er $100(1-\alpha)\%$ sjanse for at utfallet av 
            $X$ er større enn $\mu -Z_\alpha \sigma$. 

        \item Det er $100(1-\alpha)\%$ sikkert at utfallet av 
            $X$ er mellom $\mu - Z_{\alpha/2} \sigma$ og 
            $\mu + Z_{\alpha/2} \sigma$. 
    \end{itemize}
\end{thm}

\newpage
\emph{Regel.}
La $X_1, X_2, \dots, X_n$ være uavhengige. 
Anta $X_i$ \textasciitilde $\mathcal N(\mu, \sigma)$. 
Da er summen av forsøka $X_1 + X_2 +\cdots + X_n$ 
\textasciitilde $\mathcal N(n\mu, \sqrt{n}\sigma$. 
Da er gjennomsnittet 
$\frac{1}{n} \left(X_1 + \cdots X_n\right)$ \textasciitilde 
$\mathcal N\left(\mu, \frac{\sigma}{\sqrt n}\right)$

\begin{ex}{}{}
    Anta øretvekt er $\mathcal N(300, 50)$. 

    En dag fekk eg $50$ fisk.
    Kva er sannsynligheten for at totalvekta er større en $15750$g?

    $Y=$ totalvekt $= \sum_{i=1}^{50}X_i$ \textasciitilde 
    $\mathcal N(50 \cdot 300, \sqrt{50} \cdot 50) = 
    \mathcal N(15000, 354)$. 

    \begin{align*}
        P(Y > 15750) 
        &= 1 - G\left(\frac{15750-15000}{354}\right)
        \\ 
        &= 1 - G(2.12)
        \\ 
        &= 1 - 0.983 
        \\ 
        &= 0.017
    \end{align*}
\end{ex}

\begin{thm}{Sentralgrenseteoremet}{}
    La $X_1, \dots, X_n$ være uavhengige forsøk frå samme 
    sannsynlighets-fordeling.
    Anta $E(X_i) = \mu$, $Var(X_i) = \sigma^2$. 
    Viss $n$ er tilstrekkelig stor (tommelregel $n \geq 30$), 
    $$ 
        \frac{\sum_{i=1}^{n}X_i}{n} 
        \text{\textasciitilde} \mathcal 
        N\left(\mu, \frac{\sigma}{\sqrt n}\right)
    $$ 
    og 
    $$ 
        \displaystyle\sum_{i=1}^{n}X_i \text{\textasciitilde}
        \mathcal N(n\mu, \sqrt n \sigma)
    $$ 
\end{thm}

\begin{ex}{Galton}{}
    La $X_i = \text{antall høgrehopp}$ \textasciitilde 
    binomial(30, 0.5). 

    $E(X) = n \cdot p = 15$. 

    $Var(X) = np(1-p) = 2.74^2$.

    Anta at vi slipper $1000$ kuler. La $Y = \sum_{i=1}^{1000}X_i$ 
    \textasciitilde $\mathcal N\left(15000, 86.6\right)$. 
\end{ex}
 
\newpage
\section{Lecture 8 | Konfidensinterval}
\begin{ex}{}{}
    La oss si at vi har to fjell hvor vi vil måle avstanden mellom 
    toppene av desse fjellene.

    Da vil det være mange faktorer som påvirker nøyaktigheten 
    av målingen som f.eks feil i måleutstyr, feil 
    av den som måler avstanden, og mer.

    Vi øsnker å kunne finne et intervalt slik at vi kan være 
    $x$ antall $\%$ sikker på at høyden faller 
    innenfor det intervalet.
    I vårt eksempell ønsker vi å finne et interval slik at det 
    er $95\%$ sikkerheit for at den faktiske høgda ligger 
    i intervalet.

    \begin{itemize}
        \item Eg veit usikkerheita til måleutstyret av målinga 
            $X$ \textasciitilde $\mathcal N(\mu, 30)$, 
            hvor $\mu$ (middelverdien) forventer vi at 
            er den nøytaktige verdien og $30$ (standardaviket) 
            er usikkerheita.
    \end{itemize}

    \emph{Regel 5.16 (Repitisjon).}

    Viss $X$ \textasciitilde $\mathcal N(\mu, \sigma)$ veit vi at 
    det er $100(1-\alpha)\%$ sikker på at $X$ er i intervallet 
    $\mu \pm Z_{\alpha/2} \cdot \sigma$.

    \begin{itemize}
        \item $\sigma = 30$  

        \item $100(1-\alpha) = 95 \Rightarrow \alpha = 0.05 
            \Rightarrow \alpha/2 = 0.025$
            
            $Z_{0.025} = 1.960$
    \end{itemize}

    Vi trenger da eit estimat for $\mu$.
    Vi måler avstanden, og det beste estimatet er resultatet 
    av målinga.
    $$ 
        X_1 = 3245m
    $$ 

    Eit $95\%$-Konfidensinterval er gitt ved 
    $$ 
        \widehat{\mu} = \pm Z_{\alpha/2} \cdot \sigma 
    $$ 
    $$ 
        3245 \pm 1.960 \cdot 30 = [3186, 3303] 
    $$
    når vi runder av til næreste meter.

    For å forbedre estimatet tar vi fleire målingar.
    
    Anta totalt $9$ målingar, der gjennomsnittet ble 
    $$ 
        Z = \frac{X_1 + \cdots + X_9}{9} = 3204m
    $$

    Sentralgrenseteoremet seier då at 
    $$ 
        Z \text{ \textasciitilde } \mathcal N(\mu, 30/\sqrt 9) 
        = \mathcal N(\mu, 10)
    $$ 
    så usikkerheten har gått ned ved fleire målinger. 

    Eit $95\%$-konfidensinterval blir da 
    $$ 
        \widehat{\mu} \pm Z_{\alpha/2} \cdot (\sigma/\sqrt n)
    $$ 
    $$ 
        3204 \pm 1.96 \cdot 10 = [3184, 3223]
    $$ 

\end{ex}

\newpage 
\begin{defn}{Estimator}{}
    Ein estimator $\widehat{\theta}$ er ein stokastisk variabel 
    knytta til ei forsøksrekke $\theta_1, \theta_2, \dots$ som 
    estimerer $\theta$.

    \begin{enumerate}
        \item $E(\widehat\theta) = \theta$ 

        \item $Var(\widehat\theta)$ skal være minst mulig.    

        \item $Var(\widehat\theta)$ skal gå mot $0$ viss 
            antal forsøk auker.
    \end{enumerate}
\end{defn}

\textbf{Estimator for $\mu$.}

\[ 
    \widehat\mu = \overline X = 
    \frac{\displaystyle\sum_{i=1}^{n}X_i}{n}
\]

\textbf{Estimator for $\sigma$.} 

\[ 
    \widehat{\sigma^2} = S^2 = 
    \frac{1}{n-1}\displaystyle\sum (X_i-\overline X)^2
\]

\textbf{Estimator for $P$.}

\[ 
    \widehat P = \frac{X}{n}
\]
der $X$ er positive utfall og $n$ er antall forsøk.

\begin{prob}{}{}
    $\theta$ er ein ukjent stokastisk variabel.

    Mål: Finne eit interval $[A,B]$ slik at 
    $$ 
        P(A \leq \theta \leq B) = 1 - \alpha
    $$ 

\end{prob}

\textbf{$Z$-interval.}  

\begin{itemize}
    \item $\mu$ - ukjent

    \item $\sigma$ - kjent
\end{itemize}

Da er eit $100(1-\alpha)\%$ konfidensinterval gitt ved 
$$ 
    X \pm Z_{\alpha/2} \frac{\sigma}{\sqrt n}
$$ 

\textbf{$P$-interval.}

Da er eit $100(1-\alpha)\%$ konfidensinterval gitt ved
$$
\widehat P \pm Z_{\alpha/2} \sqrt{\frac{\widehat P(1-\widehat P)}{n}} 
$$ 

\newpage
\textbf{$T$-interval.}
\begin{itemize}
    \item $\mu$ - ukjent 
    \item $\sigma$ - ukjent 
\end{itemize}

Eit $100(1-\alpha)\%$-konfidensinterval for $\mu$ er gitt ved 
\[ 
    \overline X \pm t_{\alpha/2}\cdot \frac{S}{\sqrt n}
\]
der $S$ er estimatet for standardaviket, og $t_{\alpha/2}$ har 
$(n-1)$ frihetsgrader. 

\begin{itemize}
    \item Siden $\sigma$ er ukjent er eit $T$-interval større 
        enn eit $Z$-interval.

    \item Viss $n > 30$ er $T$-interval tilnærma lik $Z$-interval.
\end{itemize}

\begin{prob}{5.13}{}
    $X = $ diametere \textasciitilde $\mathcal N(\mu, \sigma)$, 
    $n = 6$. 


    Vi ønsker $95\%$-konfidensinterval $\Rightarrow \alpha/2 = 0.025$.
\end{prob}
$t_{0.025}$ med $5$ frihetsgrad $= 2.571$.

\(\overline X = (31 + 32 + 30 + 31 + 29 + 30)/6 = 30.5\) 

\(S^2 = \frac{1}{5}((31 - 30.5)^2 + (32 - 30.5)^2 + 
\cdots + (30-30.5)^2) = 1.1 \approx 1.05^2 \)

Da blir konfidensintervalet 
\[ 
    30.5 \pm 2.571 \cdot \frac{1.05}{\sqrt 6} = [29.4, 31.6] 
\]

\textbf{Konfidens for $P$.}

Eit \(100(1-\alpha)\%\)-konfidensinterval for $P$ er gitt ved 
\[ 
    \widehat P \pm Z_{\alpha/2}
    \sqrt{\frac{\widehat P (1 - \widehat P)}{n}}
\]

\begin{prob}{6.16}{}
    $n = 36$ 

    $X = 29$


    Da er $\widehat P = \frac{29}{36}$.

    Vi vil finne $90\%$-konfidensinterval.
\end{prob}

\begin{align*}
    \alpha/2 &= 0.05
    \\ 
    Z_{0.05} &= 1.645
\end{align*}

Da blir konfidensintervalet 
\[ 
    \frac{29}{36} \pm \sqrt{\frac{29/36(1-29/36)}{36}} 
    = [0.70, 0.91]
\]

\newpage 
\section{Lecture 9 | Hypotesetesting}
\begin{ex}{}{}
    Politiet utfører ei hastighetsmåling i ei \(80\) sone.  
    Dei bruker lasermåler \(X \sim \mathcal N(\mu, 3)\).
    Dei utfører \(5\) uavhengige målinger.
    \begin{align*}
        x_1 &= 79.9 \\ 
        x_2 &= 81.5 \\ 
        x_3 &= 83.2 \\ 
        x_4 &= 83.1 \\
        x_5 &= 82.0 \\
    \end{align*}

    Tar ein gjennomsnittet av desse målingane får vi 
    \[ 
        \overline X \sim \mathcal N\left(\mu, \frac{3}{\sqrt 5}\right)
    \]

    \textbf{Spørsmål.}

    \[ 
        \text{Kjører personen for fort?}
        \begin{cases}
            \text{ja, og han skal dommast} \\ 
            \text{nei han blir frikjent}
        \end{cases}
    \]

    Vi må ha rom for tvil: 
    
    Enten: bevise med tilstrekkelig sikkerhet at dei har 
    kjørt for fort. 

    Eller: frikjennast.

    Tilstrekkelig \(=\) signifikansnivå. Viss vi krever 
    signifikansnivå \(\alpha = 0.05\) betyr det i praksis at 
    viss vi har \(100\) folk som kjøyrer nøyaktig fartsgrensa 
    skal \(5\) personar dømmast urettferdig.

    \textbf{Hypoteser.}

    Nullhypotese \(H_0\) (må ikkje bevisast). 
    \[ 
        H_0 : \mu \leq 80
    \]

    Alternativ hypotese \(H_1\) må bevisast. 

    \[ 
        H_1 : \mu > 80
    \]
\end{ex}

\begin{defn}{}{}
    \textbf{Testobservatør.}

    Ein stokastisk variabel som vi baserer beslutning på.
    \begin{align*}
        \widehat\mu &= \overline X = \frac{X_1 + \cdots + X_n}{n}
        \\ 
        \widehat P &= \frac{X}{n}
    \end{align*}

    \textbf{From av forkastningsområde.}
    
    Høgresidig: Forkast \(H_0\) viss observatør er for stor. 

    Venstresidig: Forkast \(H_0\) viss observatør er for liten.

    Tosidig: Forkast \(H_0\) viss observatlør enten 
    er for stor eller viss den er for liten.

    \textbf{Signifikansnivå.} 

    Vi må fastsette kor stor sannsynlighet vi kan tillate 
    for å feilaktig forkaste \(H_0\). 

    \textbf{Styrkefunksjon.} 

    Styrkefunksjon \(\gamma(x)\) angir sannsynet for 
    å forkaste \(H_0\). 

    \textbf{p-verdi.}

    p-verdien av ein test er sannsynet \(P\) for at ein 
    foraster \(H_0\) (og påstår \(H_1\)) viss 
    \(H_0\) er sann.
\end{defn}

\textbf{Hypotesetest(Z-test, \(\mu\) ukjent, \(\sigma\) kjent).}

La 
\[ 
    Z = \frac{\overline X - \mu_0}{\sigma/\sqrt n}
\]

Høgresidig test: 
\[ 
    \begin{cases}
        H_0 : \mu \leq \mu_0 \\ 
        H_1 : \mu > \mu_0
    \end{cases}
\]

Forkast \(H_0\) viss \(Z > Z_\alpha\). 

Styrkefunksjon: \(\gamma(x) = 1 - 
G\left(Z_\alpha - \frac{X - \mu_0}{\sigma/\sqrt n}\right)\)

Venstresidig test: 
\[ 
    \begin{cases}
        H_0 : \mu \geq \mu_0 \\ 
        H_1 : \mu < \mu_0
    \end{cases}
\]

Forkast \(H_0\) viss \(Z < -Z_\alpha\).

Dobbeltsidig test: 
\[ 
    \begin{cases}
        H_0 : \mu = \mu_0 \\ 
        H_1 : \mu \neq \mu_0
    \end{cases}
\]

Forkast \(H_0\) viss \(|Z| > Z_{\alpha/2}\).

\newpage
\textbf{Hypotesetest(T-test, \(\mu\) ukjent, \(\sigma\)ukjent).}
La 
\[
    T = \frac{\overline X - \mu_0}{S/\sqrt n}
\]

\(t_\alpha\) has \(n-1\) frihetsgrader. 

Høgresidig: 
\[ 
    \begin{cases}
        H_0 : \mu \leq \mu_0 \\ 
        H_1 : \mu > \mu_0 
    \end{cases}
\]

Forkast \(H_0\) dersom \(T > t_\alpha\).

Dette fortsetter helt likt som med \(Z\)-test bare med 
\(T\) isteden.

\begin{ex}{6.21}{}
    Hypotese 
    
    \(H_0 : \mu \leq 100\)

    \(H_1 : \mu > 100\)

    Vi velger å bruke signifikansnivå \(\alpha = 0.05\).

    Testobservatør: \(\overline X = 
    \frac{103 + \cdots + 103}{9} = 103.8\).

    Estimerer \(S^2 = \frac{1}{9-1}
    \left((103-103.8)^2 + \cdots + (103-103.8)^2\right) = 
    8.695 = 2.95^2\).

    \(t_0.05\) med \(8\) frihetsgrader \(= 1.86\). 

    \[ 
        T = \frac{103.8-100}{2.95/\sqrt 9} = 3.86
    \]
    \[ 
        \Rightarrow T > t_\alpha
    \]
    
    Dermed forkaster vi \(H_0\).
\end{ex}

\textbf{Hypotesetest for sannsyn P.}
La 
\[ 
    Z = \frac{\widehat P - p_0}{\sqrt{\frac{p_0(1-p_0)}{n}}}
    = \frac{X - np_0}{\sqrt{np_0(1-p_0)}}
\]

Høgresidig: 
\[ 
    \begin{cases}
        H_0 : p \leq p_0 \\ 
        H_1 : p > p_0
    \end{cases}
\]

Forkaster \(H_0\) dersom \(Z > z_\alpha\).

Venstresidig: 
\[ 
    \begin{cases}
        H_0 : p \geq p_0 \\ 
        H_1 : p < p_0
    \end{cases}
\]

Forkaster \(H_0\) dersom \(Z < -z_\alpha\).

Dobbeltsidig: 
\[ 
    \begin{cases}
        H_0 : p = p_0 \\ 
        H_1 : p \neq p_0
    \end{cases}
\]

Forkaster \(H_0\) dersom \(|Z| > z_{\alpha/2}\).

\newpage 
\begin{ex}{6.29}{}
    \(n = 134\)

    \(X = 6\)

    Estimator/testobservatør: \(\widehat P = \frac{6}{134} = 0.045\)

    Hypotese: 
    \[ 
        \begin{cases}
            H_0: p = 0.1 
            \\ 
            H_1: p \neq 0.1
        \end{cases}
    \]

    Signifikansnivå \(\alpha = 0.05\), \(z_{0.025} = 1.960\)

    Finner 
    \[ 
        Z = \frac{6-134\cdot 0.1}{\sqrt{134\cdot 0.1 \cdot 0.9}} 
        = -2.131 
    \]

    \[ 
        |Z| = 2.131 > z_{0.025} = 1.960
    \]

    Dermed forkaster vi \(H_0\).
\end{ex}

\textbf{Konfidensinterval og hypotesetest.}

Gitt hypotesa 
\[ 
    \begin{cases}
        H_0 : \mu \leq \mu_0 \\ 
        H_1 : \mu > \mu_0
    \end{cases}
\]

Test med signifikansnivå \(\alpha\).

Da er \((-\infty,\mu]\) \(H_0\) sitt gyldighetsområde og 
\((\mu, \infty)\) \(H_1\) sitt gyldighetsområde.

Anta at vi har eit \(100(1-\alpha)\) konfidensinterval 
\([x_0, x_1]\). 

Dersom 
\[ 
    [x_0,x_1] \subseteq (-\infty, \mu]
    \text{ og } [x_0,x_1] \cap (\mu, \infty) = \emptyset
\]
forkaster vi ikke \(H_0\).

Dersom 
\[ 
    [x_0,x_1] \subseteq (\mu, \infty)
    \text{ og } [x_0,x_1] \cap (-\infty, \mu] = \emptyset
\]
forkaster vi \(H_0\).

Ellers viss 
\[ 
    [x_0, x_1] \subseteq (-\infty, \mu]
    \text{ og } [x_0,x_1] \subseteq (\mu, \infty) 
\]
beholder vi fortsatt \(H_0\).

\newpage 
\section{Lecture 10 | Regresjon}
Er det ein samanhenge mellom to stokastiske variabler 
\(X\) og \(Y\)?

\((X_1,Y_1), \dots, (X_n,Y_n)\). 

\textbf{Spørsmål:} 

Er det ein lineære samanhgen mellom \(X\) og \(Y\)?

\textbf{Spredningsplott.}
Ved å plotte datapunkt i planet, viss det er en sterk lineære  
samanheng mellom \(x\)- og \(y\)-verdiene finne ligninga for 
ei linjer som korrelerer \(x\)- og \(y\)-verdiene.

\begin{defn}{Empirisk korrelasjon}{}
    Gitt \(n\) observasjonar \((x_1,y_1),(x_2,y_2),\dots,(x_n,y_n)\).

    La 
    \begin{align*}
        S_x &= \sqrt{\frac{1}{n-1}\displaystyle\sum_{i=1}^n 
        (X_i - \overline X)^2}
        \text{ standardavvike for x-ane}
        \\
        S_y &= \sqrt{\frac{1}{n-1}\displaystyle\sum_{i=1}^n 
        (Y_i - \overline Y)^2}
        \text{ standardavvik for y-ane}
        \\ 
            S_{XY} &= \frac{1}{n-1}\displaystyle\sum_{i=1}^{n}
        (X_i-\overline X)(Y_i - \overline Y)
        \text{ kovarians}
    \end{align*}

    Da er korrelasjonen 
    \[ 
        r=\frac{S_{XY}}{S_X S_Y}
    \]
\end{defn}

\begin{ex}{}{}
    \[
        \begin{array}{c|ccccc}
            x & 4&2&3&5&2 \\
            y & 3&2&2&4&1 \\
        \end{array}
    \]

    \(\overline X = 3.2\) 

    \(\overline Y = 2.4\)

    \(S^2_X = 1.31^2 \Rightarrow S_X = 1.31\)

    \(S^2_Y = 1.14^2 \Rightarrow S_Y = 1.14\)

    \(S_{XY} = 1.4\)

    \[ 
        r = \frac{S_{XY}}{S_X S_Y} = \frac{1.4}{1.31 \cdot 1.14} = 
        0.94
    \]
\end{ex}

\textbf{Egenskaper av korrelasjon.}

\begin{itemize}
    \item \(|r| \leq 1\)
    \item Jo nærmare \(r\) er \(1\), jo meir indikerer det at 
        observasjonane ligger langs ei aukande linje.
    \item Jo nærmare \(r\) er \(-1\), jo meir indikerer det 
        at observasjonane ligger langs ei minkande linje.
    \item Jo nærmare \(r\) er \(0\), jo meir indikerer det 
        ingen lineær samanheng.
\end{itemize}

\newpage 
\subsection{Lineær regresjon}
Ved ein høg korrelasjon vil vi finne ei linje som best 
beskriver samanhengen mellom x og y. 

\textbf{Minste kvadratets rette linje.} 

\(\widehat y = \widehat \alpha + \widehat \beta x\) 

\[ 
    \widehat \beta = r \frac{S_Y}{S_X}
\]

\[ 
    \widehat \alpha = \overline y - \widehat \beta \overline X
\]

Med utgangspunkt i eksempelet over 
får vi 
\[ 
    \widehat \beta = 0.94 \cdot \frac{1.14}{1.31} = 0.82 
\]
\[ 
    \widehat \alpha = 2.4 - 0.82 \cdot 3.2 = -0.224
\]

Da får vi 
\[ 
    \widehat y = -0.224 + 0.82 x
\]

\textbf{Modellens godhet.}

Del av variasjon blant observasjonane som blir 
forklart av modell \(r^2\).

Del av variasjon som skyldes tilfeldige avvik 
er \(1-r^2\).

\end{document}

