\documentclass[11pt]{article}
\usepackage[a4paper,margin=1in]{geometry}
\usepackage{fourier} % Fourier font
\usepackage{xcolor}
\usepackage{tikz}
\usepackage[most]{tcolorbox}
\usepackage{amsthm, amsmath, amssymb}
\usepackage{enumitem}
\usepackage{hyperref}
\usepackage[nameinlink,noabbrev]{cleveref}
\usepackage{titling} 

% Dark mode colors
\definecolor{bgcolor}{HTML}{FFFFFF}
\definecolor{textcolor}{HTML}{000000}
\definecolor{defcolor}{HTML}{E86873}
\definecolor{thmcolor}{HTML}{0A9396}
\definecolor{lemcolor}{HTML}{94D2BD}
\definecolor{corcolor}{HTML}{9B4AF7}
\definecolor{probcolor}{HTML}{EE9B00}
\definecolor{excolor}{HTML}{21E933}

% Background and text color
\pagecolor{bgcolor}
\color{textcolor}

% No paragraph indentation
\setlength{\parindent}{0pt}
\setlength{\parskip}{0.7em}

% Theorem box styles
\tcbset{
  enhanced,
  colback=bgcolor,
  colframe=thmcolor,
  coltext=white,
  coltitle=white,
  fonttitle=\bfseries,
  boxrule=0.7pt,
  left=1em,
  right=1em,
  top=0.7em,
  bottom=0.7em,
  before skip=10pt,
  after skip=10pt,
}

% Theorem environments with colored boxes
\newtcbtheorem[number within=section]{thm}{Theorem}{
  colframe=thmcolor, colback=thmcolor!15!bgcolor
}{thm} % The 'thm' here is the *prefix* for the label

\newtcbtheorem[number within=section]{defn}{Definition}{
  colframe=defcolor, colback=defcolor!15!bgcolor
}{def} % The 'def' here is the *prefix* for the label

\newtcbtheorem[number within=section]{lem}{Lemma}{
  colframe=lemcolor, colback=lemcolor!15!bgcolor
}{lem}

\newtcbtheorem[number within=section]{cor}{Corollary}{
  colframe=corcolor, colback=corcolor!15!bgcolor
}{cor}

\newtcbtheorem[number within=section]{prob}{Problem}{
  colframe=probcolor, colback=probcolor!15!bgcolor
}{prob}

\newtcbtheorem[number within=section]{ex}{Example}{
  colframe=excolor, colback=excolor!15!bgcolor
}{ex}

% Proof environment 
\renewenvironment{proof}[1][\proofname]{%
  \par\pushQED{\qed}\normalfont\topsep6pt \trivlist
  \item[\hskip\labelsep\itshape #1.]\ignorespaces
}{%
  \popQED\endtrivlist\addvspace{6pt}
}

% Cleveref name formats for tcolorbox environments
\crefname{thm}{theorem}{theorems}
\Crefname{thm}{Theorem}{Theorems}

\crefname{def}{definition}{definitions}
\Crefname{def}{Definition}{Definitions}

\crefname{lem}{lemma}{lemmas}
\Crefname{lem}{Lemma}{Lemmas}

\crefname{cor}{corollary}{corollaries}
\Crefname{cor}{Corollary}{Corollaries}

\crefname{prob}{problem}{problems}
\Crefname{prob}{Problem}{Problems}

\crefname{ex}{example}{examples}
\Crefname{ex}{Example}{Examples}

\usepackage{listings}
    % In the preamble, define settings for listings
    \lstset{
        basicstyle=\ttfamily\small, % Font style
        numbers=left,              % Line numbers on the left
        numberstyle=\tiny,         % Style of line numbers
        frame=single,              % Frame around the code block
        breaklines=true,           % Enable line breaks
        caption=Kode,      % Caption for the listing
        label=lst:examplecode      % Label for referencing
    }

\title{\huge{MAT122 Oblig}}
\author{\LARGE{Birk Tangen, Gunnar Salbu, Holly Storøy, Thobias Høivik}}
\date{}

\begin{document}
\maketitle

\newpage
\section*{Oppgave 1}
\emph{Løsning for a)}

La $M = \{\text{studerer matematikk}\}$ og 
$I = \{\text{studerer informatikk}\}$.
\begin{align*}
    |M| 
    &= 120 
    \\ 
    |I|
    &= 60
    \\ 
    |M \cap I| 
    &= 30
\end{align*}

Da blir mengden av de som studerer matematikk, men ikke 
informatikk \(M \setminus I = M \setminus (M \cap I)\) 
og kardinaliteten av denne mengden 
blir \(90\).

Dermed er sjansen for at en student studerer matematikk, 
men ikke informatikk 
\[ 
    P(\text{matematikk, men ikke informatikk}) 
    = \frac{90}{200} = 45\%
\]

For å finne sjansen for at en studerer informatikk, men ikke matematikk
bruker vi samme fremgansmåte: 
\begin{align*}
    I \setminus M 
    &= I \setminus (M \cap I)
    \\ 
    P(\text{informatikk, men ikke matematikk}) 
    &= \frac{30}{200} = 15\%
\end{align*}

\emph{Løsning for b)}
\[  
    P(\text{matematikk eller informatikk}) 
    = \frac{150}{200} = 75\%
\]

Med andre ord: 
\[ 
    \frac{|M\cup I| \setminus |M\cap I|}{200}
\]
for å ungå dobbeltelling.

\emph{Løsning for c)} 
\[ 
    P(\text{verken mat eller inf}) = 1-P(\text{mat eller inf}) 
    = 25\%
\]


\newpage 
\section*{Oppgave 2}
\emph{Løsning for a)}

Ved bruk av Bayes' teorem får vi at 
\[ 
    P(\text{Korona}\mid \text{Positiv test})
    = \frac{
    P(\text{Positiv Test} \mid \text{Korona})P(\text{Korona})
    }{P(\text{Positiv test})}
\]

\emph{Løsning for b)}
\begin{align*}
    P(\text{Positiv test}) 
    &= 
    P(\text{Positiv test} \mid \text{Korona})P(\text{Korona}) 
    + 
    P(\text{Positiv test} \mid \overline{\text{Korona}})
    P(\overline{\text{Korona}})
    \\ 
    P(\text{Positiv test}) 
    &= 
    0.911 \cdot 0.01 + 0.004 \cdot (0.99) 
    = 0.01307
\end{align*}

\emph{Løsning for c)}

\(P(\text{Korona}) = 0.1\) og 
\(P(\text{Positiv} \mid \overline{\text{Korona}}) = 0.004\).
\begin{align*}
    P(\text{Positiv}) &= 0.911 \cdot 0.1 + 0.004 \cdot (1-0.1) \\ 
                     &= 0.0911 + 0.0036 \\ 
                     &= 0.0947
\end{align*}

\emph{Løsning for d)}

\begin{enumerate}
    \item \textbf{1.} For \(P(\text{Korona}) = 0.01\): 
        \[ 
            P(\text{Korona}\mid \text{Positiv}) = 
            \frac{0.911 \cdot 0.01}{0.01307} \approx 69.7\%
        \]
    \item \textbf{2.} For \(P(\text{Korona}) = 0.1\): 
        \[ 
            P(\text{Korona}\mid \text{Positiv}) = 
            \frac{0.911 \cdot 0.1}{0.0947} \approx 96.2\%
        \]
\end{enumerate}

\emph{Løsning for e)}

Selv om testen har høg sensitivitet på \(91.1\%\) og lav 
falsk-positiv rate på \(0.4\%\) så vil en lav prevalens 
på \(1\%\) føre til en lavere positivt prediktiv verdi 
på \(69.7\%\) kontra den \(96.2\%\) vi får når 
en høg andel er smittet \((10\%)\).


\newpage 
\section*{Oppgave 3}
\emph{Løsning for a)}

Vi snakker om antall forekomster av en hendelse i et tidsinterval. 
Vi blir gitt at hendelsene er tilfeldig fordelt i dette intervallet 
og uavhengige. Vi kan også gjør en simplifiserende antagelse 
at ingen hendelser skjer på akkurat samme tid. 
Dermed får vi at 

\[
    X \sim \text{poisson}(\lambda t)
\]   

Siden antall forekomster er tilfeldig fordelt iløpet av et 
\(8\) timer tidsinterval kan vi fint anta 
at vi da forventer \(32/8 = 4\). 
Så det er 
naturlig å snakke om \(t\) timer, også siden de følgende oppgavene 
snakker om individuelle timer. 

\emph{Løsning for b)}

Hva er sjansen for \(0\) forespørseler på en time? 

For \(t = 8\) har forventer vi \(32\). Dermed har vi at på en time: 
\[ 
    E(X) = \lambda t = 4
\]
så \(\lambda = 4\).

\[ 
    P(X=0) = \frac{(\lambda t)^x}{x!}e^{-\lambda t} 
    = e^{-4} = 0.01831563888 
    \approx 1.8\%
\]

\emph{Løsning for c)}

Det minste antallet forespørseler vi kan ha er \(0\). 
En negativ verdi gir ikke mening. 

Dermed får vi at 
\[ 
    P(X \geq 1) = 1 - P(X = 0)
\]

\begin{align*}
    P(X \geq 1) &= 1 - P(X = 0) \\ 
                &= 1 - 0.01831563888 \\ 
                &= 0.98168436112 \\ 
                &\approx 98\%
\end{align*}

\emph{Løsning for d)}

Vi ønsker å finne \(P(X \geq 32)\) for \(t = 8\). 
Det blir \(1 - P(X \leq 31)\). 
\(P(X \leq 31)\) er nøyaktig 
\[ 
    \displaystyle\sum_{x=0}^{31}P(X=x)
\]
som er plagsomt å finne for hand. Vi kan merke at poissonfordelingen 
vår konvergerer her mot en normalfordeling (som er symmetrisk) og 
siden \(32\) er middelverdien vil \(P(X \leq 31)\) være ca. halvparten
av arealet under normalfordelingen (\(50\%\))
, men en gøyere måte å finne 
den er å ta integralet  
\[ 
    \displaystyle\int_{0}^{31}\frac{(\lambda t)^x}{x!}e^{-\lambda t}dx
\]

Dette integralet blir nødvendighvis mindre enn eller lik den faktiske 
sannsynligheten siden 
\[ 
    \displaystyle\int_{0}^{N} f(x)dx = 
    \displaystyle\sum_{n=1}^{N} \displaystyle\int_{n-1}^{n}f(x)dx 
    \leq \displaystyle\sum_{n=0}^{N}f(n)
\]
for en \(f : [0, \infty) \to [0, \infty)\).

De betyr at \(1\) minus dette integralet vil være høyere enn 
eller lik \(1 - P(X \leq 31)\).

Dermed kan vi gi en lavere-bund for \(P(X \leq 31) = I\) hvor 
\[ 
    I = \displaystyle\int_{0}^{31}\frac{32^x}{\Gamma(x + 1)} e^{-32}dx
\]
hvor \(\Gamma(x + 1) = \int_{0}^{\infty} e^{-z} z^{x-1} dz = x!\), 
når \(x \in \mathbb N\).

Vi får at 
\[ 
    I \approx 0.441226\dots \leq P(X \leq 31)
\]

Dermed får vi at 
\[ 
    P(X \geq 32) \leq 1 - 0.441226\dots = 0.558774
\]

\emph{Løsning for e)}

\(Y \sim \text{bin}(10, 0.05)\).

\emph{Løsning for f)}

\begin{align*}
    P(Y = 0) = \binom{10}{0}p^0(1-p)^10 = 0.5987\dots \approx 60\% 
\end{align*}

\emph{Løsning for g)}
\begin{align*}
    E(Y) &= 10 \cdot 0.05 = 0.5 \\ 
    Var(Y) &= 10 \cdot 0.05(0.95) = 0.475 \\ 
    \sigma &\approx 0.69
\end{align*}

\newpage 
\section*{Oppgave 4}
Lavere standardavvik betyr høyere konsentrasjon ved middelverdien. 
Dermed er den blå kurven \(\sigma = 0.5\) og den oransje 
er \(\sigma = 0.9\).

\section*{Oppgave 5}
\(X \sim \mathcal N(3510, 385)\).

\emph{Løsning for a)}
\[  
    P(X \geq 4500) = P(X > 4500) = 1 - 
    G\left(\frac{4500-3510}{385}\right)
    = 0.0051
\]

\emph{Løsning for b)}
\[ 
    P(2500 < X < 4500) = 
    G \left(\frac{4500-3510}{385}\right) - 
    G\left(\frac{2500-3510}{385}\right)
    = 0.9906
\]

\newpage 
\section*{Oppgave 6}

\emph{Løsning for a)}

\[ 
    \begin{cases}
        H_0: \mu \geq = 400 \\ 
        H_1: \mu < 400
    \end{cases}
\]

Siden vi ikke er misnøgde med en batterilengde på mer en det 
som er reklamert.

Vi har et estimat for \(X\) og et estimat for standardavviket 
\(S\). Vi bruker en \(T\)-test med \(19\)-frihetsgrader. 

Forkaster \(H_0\) viss \(T < -t_{\alpha}\).

\emph{Løsning for b)}

\(\alpha = 0.05\)

\[ 
    T = \frac{\overline X - \mu_0}{S/\sqrt{n}}
    = \frac{380 - 400}{10-\sqrt{20}}
    \approx -3.618 < -2.433 = -t_{0.025}
\]

\emph{Løsning for c)}
\begin{align*}
    [a,b] &= \overline X \pm t_{\alpha/2} \cdot \frac{S}{\sqrt n}
    \\ 
          &= 380 \pm 2.433 \cdot \frac{10}{\sqrt{20}}
          \\ 
          &= 380 \pm 5.436 
          \\ 
          &= [374.564, 385.436]
\end{align*}

Eit \(90\%\) konfidensinterval vil være smalere (eller like stort)
siden et mindre "vindu" for korrekte verdier gir en lavere sjanse 
for at vi treffer.

\newpage 
\section*{Oppgave 7}
\emph{Løsning for a)}

\[ 
    \begin{cases}
        H_0 : \mu = 26 \\ 
        H_1 : \mu \neq 26
    \end{cases}
\]

Vi kjenner \(\sigma\) og kjenner ikkje \(\mu\) så vi gjøre 
en \(Z\)-test. 
\[ 
    Z = \frac{\overline X - 26}{\sigma / \sqrt n}
\]

Forkast \(H_0\) viss \(|Z| > z_{\alpha/2}\).

\emph{Løsning for b)}
\(\alpha = 0.05 \Rightarrow \alpha/2 = 0.025\).

\[ 
    Z = \frac{28 - 26}{4 / \sqrt{10}} = 1.58 < 1.96 = z_{\alpha / 2} 
\]


\emph{Løsning for c)}
\begin{align*}
    [a,b] &= 28 \pm 1.96 \frac{4}{\sqrt{10}} \\ 
          &=28 \pm 2.48 
          \\ 
          &= [25.52, 30.48]
\end{align*}

Igjen, \(90\%\) gir et smalere interval.

\emph{Løsning for d)}

\(9\) frihetsgrader.
\begin{align*}
    28 \pm t_{\alpha/2} \frac{4}{\sqrt{10}} 
    &= 28 \pm 2.262 \frac{4}{\sqrt{10}} \\ 
    &= 28 \pm 2.86 \\ 
    &= [25.14, 30,86]
\end{align*}

Intervallet blir større fordi det er mer usikkerhet 
(\(\sigma\) ukjent).

\newpage 
\section*{Oppgave 8}
\emph{Løsning for a)}

\(X \sim \text{bin}(n,p)\) hvor \(n\) er antall forsøk 
(\(1200\)) og \(p\) er sannsynet for positivt utfall 
(\(1/6\)). 

\[ 
    E(X) = np = 1200/6 = 200 
\]
\[ 
    Var(X) = np(1-p) = 200 \cdot \frac{5}{6} = 166.\overline6
\]

\emph{Løsning for b)}

\[ 
    \begin{cases}
        H_0: \mu = \frac{1}{6} \\ 
        H_1: \mu \neq \frac{1}{6}
    \end{cases}
\]

\(\alpha = 0.05 \Rightarrow \alpha/2 = 0.025\).

Vi får \(\widehat p = 172/1200 = 0.14\overline3\).

\[ 
    Z = \frac{0.14\overline3 - 
    \frac{1}{6}}{\sqrt{\frac{\frac{1}{6}(\frac{5}{6})}{n}}}
    \approx -2.17 
\]

\[ 
    Z_{0.025} = 1.96 < |Z|
\]

Dermed forkaster vi \(H_0\) og sier med \(95\%\) sikkerhet at 
terningen ikke er rettferdig.

\newpage 
\section*{Oppgave 9}
\emph{Løsning for a)}

Siden vi snakker om antall hendelser innen et tidsrom og 
det er rimelig å anta at om en gitt pasient blir henvist eller ikke
har ingen påvirkning på om en annen pasient blir henvist samt at 
å anta at ingen to pasienter blir henvist på akkurat samme tid, 
kan vi anta at antall henvisninger er poissonfordelt.

\[ 
    p = \frac{24300}{650000} = 0.0373846
\]

For et fastlegekontor med \(n=6200\) pasienter har vi 
\[ 
    \lambda = np = 231.7846
\]

\begin{align*}
    E(X) = Var(X) = \sigma^2 = \lambda = 231.7846 \\ 
    \sigma = 15.22
\end{align*}

\emph{Løsning for b)}
\[ 
    \begin{cases}
        H_0 : p = 0.0373846 \\ 
        H_1 : p \neq 0.0373846  
    \end{cases}
\]
Viss \(H_0\) ikke blir forkastet forventer vi rundt \(\lambda\) 
henvisninger.

\(X = 276\).

For stor \(\lambda\) går det mot normalfordelt.

\[ 
    Z = \frac{276 - 231.78}{15.22} \approx N(0,1)
\]


\emph{Løsning for c)}
\(\alpha = 0.05 \Rightarrow \alpha/2 = 0.025\).

\[ 
    |Z| = \left|Z\right| \approx 2.9 > 1.96
\]

Forkast \(H_0\).

\newpage 
\section*{Oppgave 10}
\[ 
    \widehat y = 19.83 + 0.85x
\]

\(180\) tonn fordelt tilfeldig utover \(20\) mål gir i gjennomsnitt 
\(9\) tonn per mål. 

\[ 
    \widehat y(9) = 27.48 \text{ tonn per mål} 
\]


Til sammen over \(20\) mål har vi 
\[ 
    20\cdot 27.48 = 549.6 \text{ tonn avlinger}
\]

\begin{lstlisting}[language=Python]
import numpy as np
import matplotlib.pyplot as plt
from sklearn.linear_model import LinearRegression


X = np.array([8, 7, 13, 9, 6, 5, 11, 9, 14, 13, 6, 10, 12, 13, 12, 10, 7, 8, 12, 10]).reshape(-1, 1)  
Y = np.array([27, 30, 26, 32, 26, 18, 36, 27, 36, 31, 34, 29, 21, 25, 35, 26, 19, 22, 37, 25])        


model = LinearRegression()
model.fit(X, Y)

alpha = model.intercept_   
beta = model.coef_[0]      
print(f"a:{alpha:.2f}")
print(f"b:{beta:.2f}")


Y_pred = model.predict(X)

plt.scatter(X,Y,color='blue',label='Data-punkt')

plt.plot(X,Y_pred,color='red',label=f'Y={alpha:.2f}+{beta:.2f}X')

plt.xlabel('X(gjodsel)')
plt.ylabel('Y(avling)')
plt.title('Y=a+bX+e')
plt.legend()
plt.show()
\end{lstlisting}

\begin{figure}[ht]
\centering
\includegraphics[width=1.0\linewidth]{./PLOT.png}
\end{figure}
\end{document}




\end{document}

