\documentclass[12pt]{article}
\usepackage{amsmath, amssymb, amsthm, amsfonts, geometry}
\newtheorem{theorem}{Theorem}
\newtheorem{lemma}{Lemma}
\newtheorem{proposition}{Proposition}
\newtheorem{corollary}{Corollary}
\newtheorem{definition}{Definition}

% Page Setup
\geometry{top=1in, bottom=1in, left=1in, right=1in}

\title{Lecture Notes: Design and Analysis of Algorithms \\ 
Fast Fourier Transform (Course: MIT-6.046J)}
\author{Thobias K. Høivik}
\date{\today}

\begin{document}

\maketitle
\section*{Fast Fourier Transform}
\begin{definition}
    A polynomial of degree \(n-1\) can be written as 
    \[A(x) = a_0 + a_1x + a_2x^2 + \cdots + a_{n-1}x^{n-1} 
    = \displaystyle\sum_{k = 0}^{n-1} a_{k}x^{k}, a_k \in \mathbb R \]
    \[ 
        = \langle a_0, a_1, a_2, \dots, a_{n-1} \rangle
    \]
    \[ 
        = \{(x_k, A(x_k)) \quad|\quad k = 1,2,\dots,n-1\}
    \]
    \[ 
        = c(x-r_0)(x-r_1)\cdots(x-r_{n-1})
    \]

\end{definition}

\subsection*{Operations on polynomials}
\begin{enumerate}
    \item Evaluation: \(A(x) \text{ \& } x_0 \rightarrow A(x_0)?\)
        \\ Horner's Rule: \(A(x) = a_0 + x(a_1 + x(a_2 + \cdots x(a_{n-1})))\)
        \(\Rightarrow \mathcal O (n)\)
    \item Add: \(A(x) \text{ \& } B(x) \rightarrow C(x) = A(x) + B(x), \forall x\) 
        \\ \(c_k = a_k + b_k \Rightarrow \mathcal O(n)\)
    \item Multiplication: \(A(x) \text{ \& } B(x) 
        \rightarrow C(x) = A(x) \cdot B(x), \forall x\)
        \[ 
            c_k = \displaystyle\sum_{k}^{j=0}a_j b_{k-j}
        \]
        \[ 
            C(x) = \displaystyle\sum^{n-1}_{k=0}\displaystyle\sum^{k}_{j=0}a_jb_{k-j}
        \]
        \(\Rightarrow \mathcal O(n^2)\) which is very bad.
\end{enumerate}

\subsection*{Convolution}
Suppose we have a vector representing a wave where each entry in the vector represents 
the amplitude at that frequency. We then want to take a gaussian distribution and shift 
it along each possible frequency of the wave and take the dot product of the wave 
and gaussiang distribution to "smooth out" the wave.
This is more or less finding \(c_k\) for all \(k\). In other words it's finding 
\(C(x)\) which we know is \(\mathcal O(n^2)\) (very bad).

\end{document}
