\documentclass[11pt]{article}
\usepackage[a4paper,margin=1in]{geometry}
\usepackage{fourier} % Fourier font
\usepackage{xcolor}
\usepackage{tikz}
\usepackage[most]{tcolorbox}
\usepackage{amsthm, amsmath, amssymb}
\usepackage{enumitem}
\usepackage{hyperref}
\usepackage[nameinlink,noabbrev]{cleveref}
\usepackage{titling} 

% Dark mode colors
\definecolor{bgcolor}{HTML}{FFFFFF}
\definecolor{textcolor}{HTML}{000000}
\definecolor{defcolor}{HTML}{E86873}
\definecolor{thmcolor}{HTML}{0A9396}
\definecolor{lemcolor}{HTML}{94D2BD}
\definecolor{corcolor}{HTML}{9B4AF7}
\definecolor{probcolor}{HTML}{EE9B00}
\definecolor{excolor}{HTML}{21E933}

% Background and text color
\pagecolor{bgcolor}
\color{textcolor}

% No paragraph indentation
\setlength{\parindent}{0pt}
\setlength{\parskip}{0.7em}

% Theorem box styles
\tcbset{
  enhanced,
  colback=bgcolor,
  colframe=thmcolor,
  coltext=white,
  coltitle=white,
  fonttitle=\bfseries,
  boxrule=0.7pt,
  left=1em,
  right=1em,
  top=0.7em,
  bottom=0.7em,
  before skip=10pt,
  after skip=10pt,
}

% Theorem environments with colored boxes
\newtcbtheorem[number within=section]{thm}{Theorem}{
  colframe=thmcolor, colback=thmcolor!15!bgcolor
}{thm} % The 'thm' here is the *prefix* for the label

\newtcbtheorem[number within=section]{defn}{Definition}{
  colframe=defcolor, colback=defcolor!15!bgcolor
}{def} % The 'def' here is the *prefix* for the label

\newtcbtheorem[number within=section]{lem}{Lemma}{
  colframe=lemcolor, colback=lemcolor!15!bgcolor
}{lem}

\newtcbtheorem[number within=section]{cor}{Corollary}{
  colframe=corcolor, colback=corcolor!15!bgcolor
}{cor}

\newtcbtheorem[number within=section]{prob}{Problem}{
  colframe=probcolor, colback=probcolor!15!bgcolor
}{prob}

\newtcbtheorem[number within=section]{ex}{Example}{
  colframe=excolor, colback=excolor!15!bgcolor
}{ex}

% Proof environment 
\renewenvironment{proof}[1][\proofname]{%
  \par\pushQED{\qed}\normalfont\topsep6pt \trivlist
  \item[\hskip\labelsep\itshape #1.]\ignorespaces
}{%
  \popQED\endtrivlist\addvspace{6pt}
}

% Cleveref name formats for tcolorbox environments
\crefname{thm}{theorem}{theorems}
\Crefname{thm}{Theorem}{Theorems}

\crefname{def}{definition}{definitions}
\Crefname{def}{Definition}{Definitions}

\crefname{lem}{lemma}{lemmas}
\Crefname{lem}{Lemma}{Lemmas}

\crefname{cor}{corollary}{corollaries}
\Crefname{cor}{Corollary}{Corollaries}

\crefname{prob}{problem}{problems}
\Crefname{prob}{Problem}{Problems}

\crefname{ex}{example}{examples}
\Crefname{ex}{Example}{Examples}


\title{\huge{My Math Notes}}
\author{\LARGE{Thobias Høivik}}
\date{\Large{\today}}

\begin{document}
\maketitle

\newpage
\tableofcontents

\newpage
\section{Introduction}

\begin{defn}[Sets]
\label{def:sets}
A \emph{set} is a collection of distinct elements.
\end{defn}

As discussed in \cref{def:sets}, elements must be distinct.

\begin{thm}[Sum of evens]\label{thm:sum}
The sum of two even numbers is even.
\begin{proof}
Let the even numbers be \(2a\) and \(2b\). Their sum is
\[
2a + 2b = 2(a+b),
\]
which is even.
\end{proof}
\end{thm}

As described in \cref{thm:sum}. 

\begin{lem}[Parity of sum]\label{lem:parity}
If \(x\) and \(y\) are integers, then \(x + y\) is even if and only if \(x\) and \(y\) have the same parity.
\end{lem}

As seen in \cref{lem:parity}, you are a very nice person.

\begin{proof}
Assume \(x\) and \(y\) are both even or both odd.

- If both even, \(x=2m\), \(y=2n\), then \(x+y=2(m+n)\) is even.

- If both odd, \(x=2m+1\), \(y=2n+1\), then
\[
x+y = 2m + 1 + 2n + 1 = 2(m+n+1),
\]
also even.

Conversely, if \(x+y\) is even and \(x\) is even, then \(y\) must be even; similarly for odd.
\end{proof}

\section{Model Theory}

\begin{prob}[Satisfiability]
Let \(\mathfrak{M} \models \phi\). Show that \(\phi\) is satisfiable.
\end{prob}

\begin{proof}[Proof of Problem 2.1]
Since \(\mathfrak{M} \models \phi\), by definition \(\phi\) is true in some model (namely \(\mathfrak{M}\)). Therefore, \(\phi\) is satisfiable.

More explicitly,
\[
\phi = \psi \wedge \theta,
\]
where \(\psi\) and \(\theta\) are formulas satisfied by \(\mathfrak{M}\). Hence \(\phi\) is satisfiable.
\end{proof}

\begin{cor}[Consequence of satisfiability]
If \(\phi\) is satisfiable, then \(\neg \phi\) is not valid.
\end{cor}

\begin{proof}
If \(\neg \phi\) were valid, then \(\phi\) would be false in every model. This contradicts the satisfiability of \(\phi\).
\end{proof}

\section{Fourier Analysis}

\begin{defn}[Fourier Transform]
The Fourier transform of a function \(f \in L^1(\mathbb{R})\) is defined as
\[
\hat{f}(\xi) = \int_{\mathbb{R}} f(x) e^{-2\pi i x \xi} \, dx.
\]
\end{defn}

\begin{thm}[Fourier Inversion]
If \(f\) and \(\hat{f}\) are both in \(L^1(\mathbb{R})\), then for almost every \(x\),
\[
f(x) = \int_{\mathbb{R}} \hat{f}(\xi) e^{2\pi i x \xi} \, d\xi.
\]
\end{thm}

\begin{proof}[Sketch of proof]
This theorem follows from Plancherel's theorem and properties of the Fourier transform on Schwartz functions. The full proof is beyond this note.
\end{proof}


\end{document}
