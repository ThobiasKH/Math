\documentclass[11pt]{article}
\usepackage[a4paper,margin=1in]{geometry}
\usepackage{fourier} % Fourier font
\usepackage{xcolor}
\usepackage{tikz}
\usepackage[most]{tcolorbox}
\usepackage{amsthm, amsmath, amssymb}
\usepackage{enumitem}
\usepackage{hyperref}
\usepackage[nameinlink,noabbrev]{cleveref}
\usepackage{titling} 

% Dark mode colors
\definecolor{bgcolor}{HTML}{FFFFFF}
\definecolor{textcolor}{HTML}{000000}
\definecolor{defcolor}{HTML}{E86873}
\definecolor{thmcolor}{HTML}{0A9396}
\definecolor{lemcolor}{HTML}{94D2BD}
\definecolor{corcolor}{HTML}{9B4AF7}
\definecolor{probcolor}{HTML}{EE9B00}
\definecolor{excolor}{HTML}{21E933}

% Background and text color
\pagecolor{bgcolor}
\color{textcolor}

% No paragraph indentation
\setlength{\parindent}{0pt}
\setlength{\parskip}{0.7em}

% Theorem box styles
\tcbset{
  enhanced,
  colback=bgcolor,
  colframe=thmcolor,
  coltext=white,
  coltitle=white,
  fonttitle=\bfseries,
  boxrule=0.7pt,
  left=1em,
  right=1em,
  top=0.7em,
  bottom=0.7em,
  before skip=10pt,
  after skip=10pt,
}

% Theorem environments with colored boxes
\newtcbtheorem[number within=section]{thm}{Theorem}{
  colframe=thmcolor, colback=thmcolor!15!bgcolor
}{thm} % The 'thm' here is the *prefix* for the label

\newtcbtheorem[number within=section]{defn}{Definition}{
  colframe=defcolor, colback=defcolor!15!bgcolor
}{def} % The 'def' here is the *prefix* for the label

\newtcbtheorem[number within=section]{lem}{Lemma}{
  colframe=lemcolor, colback=lemcolor!15!bgcolor
}{lem}

\newtcbtheorem[number within=section]{cor}{Corollary}{
  colframe=corcolor, colback=corcolor!15!bgcolor
}{cor}

\newtcbtheorem[number within=section]{prob}{Problem}{
  colframe=probcolor, colback=probcolor!15!bgcolor
}{prob}

\newtcbtheorem[number within=section]{ex}{Example}{
  colframe=excolor, colback=excolor!15!bgcolor
}{ex}

% Proof environment 
\renewenvironment{proof}[1][\proofname]{%
  \par\pushQED{\qed}\normalfont\topsep6pt \trivlist
  \item[\hskip\labelsep\itshape #1.]\ignorespaces
}{%
  \popQED\endtrivlist\addvspace{6pt}
}

% Cleveref name formats for tcolorbox environments
\crefname{thm}{theorem}{theorems}
\Crefname{thm}{Theorem}{Theorems}

\crefname{def}{definition}{definitions}
\Crefname{def}{Definition}{Definitions}

\crefname{lem}{lemma}{lemmas}
\Crefname{lem}{Lemma}{Lemmas}

\crefname{cor}{corollary}{corollaries}
\Crefname{cor}{Corollary}{Corollaries}

\crefname{prob}{problem}{problems}
\Crefname{prob}{Problem}{Problems}

\crefname{ex}{example}{examples}
\Crefname{ex}{Example}{Examples}

\usepackage{algorithm}
\usepackage{algorithmic}

\title{\huge{Oblig 1}}
\author{\LARGE{Thobias Høivik}}
\date{\Large{}}

\begin{document}
\maketitle

\newpage
\begin{prob*}{Oppgave 1}{}
    Hva er verdien av $k$ etter at følgende script er utført.
    \[
        \begin{aligned}
  &k = 0 \\
  &\text{for } i_{1} = 1 \text{ to } n_{1} \\
  &\quad \text{for } i_{2} = 1 \text{ to } n_{2} \\
  &\quad \quad \vdots \\
  &\quad \quad \text{for } i_{k} = 1 \text{ to } n_{k} \\
  &\quad \quad \quad k = k + 1
        \end{aligned}
    \]
\end{prob*}
\begin{proof}[Løsning]
    Vi antar for denne oppgaven at loop-en er inklusiv på siste 
    variabel (i.e. for $i_k$ to $n_k$ tar $n_k$ iterasjoner).

    Viss vi jobber baklengs har vi at, i den siste loop-en, 
    $k$ ender opp med å være $n_k$. 
    Loop-en som gjør dette blir utført $n_{k-1}$ ganger som gir 
    $k = n_{k-1}n_k$.

    Vi kunne lagd en formel og vist korrekthet via induksjon, men 
    vi kan tydelig se at summen for $k$ blir antal iterasjoner av 
    den siste loop-en, multiplisert med antall iterasjoner av 
    den nest siste loop-en og så videre ned til vi når den første 
    iterasjonen. 

    Da får vi at 
    $$ 
        k = n_1n_2n_3\cdots n_p = \displaystyle\prod_{i=1}^p n_i
    $$ 
    viss vi antar $p$ loops.
\end{proof}

\begin{prob*}{Oppgave 2}{}
    a) Teikn grafane til alle bijektive funksjonar mellom 
    $A = \{1,2,3\}$ og $B = \{a,b,c\}$. Kva seier dette om 
    kardinaliteten?

    b) Anta at mengdene $A = \{1,2,3,4,5,6,7\}$ og 
    $B = \{a,b,c,d,e,f,g\}$. Kor mange bijektive funksjonar er 
    det mellom $A$ og $B$? Bruk multiplikasjonsprinsippet 
    for å grunngje svaret.
\end{prob*}
\begin{proof}[Løsning av a)]
    \[
        \begin{tikzpicture}[node distance=2cm, every node/.style={minimum width=1cm, align=center}]
  % Set positions for clarity
            \foreach \y/\label in {1/1,2/2,3/3} {
                \node (A\y) at (0,-\y) {$\label$};
            }
            \foreach \y/\label in {1/a,2/b,3/c} {
                \node (B\y) at (3,-\y) {$\label$};
            }

  % Bijection 1: 1->a, 2->b, 3->c
  \draw[->] (A1) -- (B1);
  \draw[->] (A2) -- (B2);
  \draw[->] (A3) -- (B3);
  \node at (1.5,0.5) {1};

  % Bijection 2: 1->a, 2->c, 3->b
  \begin{scope}[yshift=-4cm]
    \foreach \y/\label in {1/1,2/2,3/3} {
      \node (A\y) at (0,-\y) {$\label$};
    }
    \foreach \y/\label in {1/a,2/b,3/c} {
      \node (B\y) at (3,-\y) {$\label$};
    }
    \draw[->] (A1) -- (B1);
    \draw[->] (A2) -- (B3);
    \draw[->] (A3) -- (B2);
    \node at (1.5,0.5) {2};
  \end{scope}

  % Bijection 3: 1->b, 2->a, 3->c
  \begin{scope}[yshift=-8cm]
    \foreach \y/\label in {1/1,2/2,3/3} {
      \node (A\y) at (0,-\y) {$\label$};
    }
    \foreach \y/\label in {1/a,2/b,3/c} {
      \node (B\y) at (3,-\y) {$\label$};
    }
    \draw[->] (A1) -- (B2);
    \draw[->] (A2) -- (B1);
    \draw[->] (A3) -- (B3);
    \node at (1.5,0.5) {3};
  \end{scope}

  % Bijection 4: 1->b, 2->c, 3->a
  \begin{scope}[yshift=-12cm]
    \foreach \y/\label in {1/1,2/2,3/3} {
      \node (A\y) at (0,-\y) {$\label$};
    }
    \foreach \y/\label in {1/a,2/b,3/c} {
      \node (B\y) at (3,-\y) {$\label$};
    }
    \draw[->] (A1) -- (B2);
    \draw[->] (A2) -- (B3);
    \draw[->] (A3) -- (B1);
    \node at (1.5,0.5) {4};
  \end{scope}

  % Bijection 5: 1->c, 2->a, 3->b
  \begin{scope}[yshift=-16cm]
    \foreach \y/\label in {1/1,2/2,3/3} {
      \node (A\y) at (0,-\y) {$\label$};
    }
    \foreach \y/\label in {1/a,2/b,3/c} {
      \node (B\y) at (3,-\y) {$\label$};
    }
    \draw[->] (A1) -- (B3);
    \draw[->] (A2) -- (B1);
    \draw[->] (A3) -- (B2);
    \node at (1.5,0.5) {5};
  \end{scope}

  % Bijection 6: 1->c, 2->b, 3->a
  \begin{scope}[yshift=-20cm]
    \foreach \y/\label in {1/1,2/2,3/3} {
      \node (A\y) at (0,-\y) {$\label$};
    }
    \foreach \y/\label in {1/a,2/b,3/c} {
      \node (B\y) at (3,-\y) {$\label$};
    }
    \draw[->] (A1) -- (B3);
    \draw[->] (A2) -- (B2);
    \draw[->] (A3) -- (B1);
    \node at (1.5,0.5) {6};
  \end{scope}

\end{tikzpicture}
\]

Det er $6 = 3! = |A|! = |B|!$ bijeksjoner.
\end{proof}

\begin{proof}[Løsning av b)]
    Mellom to mengder med kardinalitet $7$ fins det 
    $7!$ ulike bijeksjoner.

    Viss vi betrakter en vilkårlig bijeksjon $\phi: A \to B$, der 
    $|A| = |B| = 7$, har vi at $\phi(a_1) = b_1$ for en 
    $a_1 \in A, b_1 \in A$ der $b_1$ er en av $7$ elementer i $B$.
    Så $a_1$ har $7$ elementer å "velge" mellom. 
    Deretter for $a_2 \in A$ må $\phi(a_2) \neq b_1$ så $a_2$ 
    kan sendes til $1$ av $7-1 = 6$ elementer.
    Dette fortsettes helt til $a_7$ kun har et element det kan 
    være pre-bildet av. For å finne antallet mulige bijeksjoner 
    tar vi produktet av alle mulige "valg". 
    Dette betyr at antallet bijeksjoner 
    er 
    $$ 
        7 \cdot 6 \cdot 5 \cdots 1 = 7! = |A|! = |B|!
    $$ 

    Generellt blir formelen for kardinaliteten til mengden av 
    bijeksjoner mellom $A$ og $B$ (gitt at $|A| = |B|$) lik 
    $|A|!$ når $|A| < \aleph_0$.
    Viss $|A| \geq \aleph_\alpha$, så er 
    $|\{\phi:A \to B : \phi \text{ bijektiv}\}| = 
    2^{\aleph_\alpha}$.

    Så det er like mange bijeksjoner fra $\mathbb N$ til 
    $\mathbb N$ som det er reelle tall!
\end{proof}

\begin{prob*}{Oppgave 3}
    I forelesningsrommet i MAT210 er det $100$ sitteplassar. 
    I ei forelesning møtte $20$ studentar. Kor mange ulike 
    måtar kunne studentane plassere seg i forelesninga?
\end{prob*}
\begin{proof}[Løsning]
    Det er $100$ plassar og $20$ av plassene skal velges. 
    Da får vi 
    $$ 
        \binom{100}{20}
    $$ 
    konfigurasjoner.

    Men, studentene er unike, så for hver av  
    måtene å velge $20$ seter fra $100$ er det 
    $20!$ studentene kan sette seg (siden student nr.$1$ har 
    $20$ valg, student nr.$2$ har $19$ valg, osv.). 

    Så det endelige antall måter $20$ studenter kan 
    fordele seg på $100$ seter er 
    $$ 
        \binom{100}{20}\cdot 20!
    $$ 
    som er ca. antall stjerner i universet i andre.
\end{proof}

\newpage 
\begin{prob*}{Oppgave 4}{}
    a) Vis at $|\mathbb N| = |\{5n + 1 : n \in \mathbb N\}|$.

    b) Vis at $|(0,1)| = |(0,10)|$.
\end{prob*}
\begin{proof}[Bevis av a)]
    La $f: \mathbb N \to \{5n+1 : n \in \mathbb N\}$ definert ved 
    $$ 
        f(n) = 5n+1
    $$ 

    Vi ser at $f$ er veldefinert og fortsetter med 
    å vise at $f$ utgjør en bijeksjon. 

    \emph{Injektiv.}

    La $n,m \in \mathbb N$ og anta at 
    $f(n) = f(m)$. 
    \begin{align*}
        f(n) &= f(m) \\ 
        5n + 1 &= 5m + 1 \\ 
        5n &= 5m \\ 
        n &= m \\ 
    \end{align*}
    $\Rightarrow f$ injektiv. 

    \emph{Surjektiv.}

    La $m \in \{5n+1 : n \in \mathbb N\}$. 
    Da er pre-bildet til $m$ 
    $$ 
        n = \frac{m - 1}{5}
    $$ 
    siden 
    $$ 
        f(n) = f\left(\frac{m - 1}{5}\right) = 5 \frac{m-1}{5} + 1 = m
    $$ 
    så $f$ er surjektiv. 

    Med dette har vi vist at $f$ utgjør en bijeksjon 
    fra de to mengdene og de har da samme kardinalitet ($\aleph_0$).
\end{proof}

\begin{proof}[Bevis av b)]
    La $f : (0,1) \to (0, 10)$ definert ved 
    $$ 
        f(x) = 10x
    $$ 
    da har vi, for en vilkårlig $x \in (0,1)$, at
    $$
        0 < x < 1 \Rightarrow f(0) < f(x) < f(10)
        \Rightarrow 0 < f(x) < 10
    $$
    så $f(x) \in (0,10)$. 

    \emph{Injektiv.}

    La $x,y \in (0,1)$ og anta at $f(x) = f(y)$. 
    Da får vi 
    \begin{align*} 
        f(x) &= f(y) \\ 
        10x &= 10y \\ 
        x &= y
    \end{align*}

    \emph{Surjektiv.}
    La $y \in (0,10)$. Da er pre-bildet til $y$
    $$ 
        x = \frac{y}{10}
    $$ 
    som vi vet er i $(0,1)$ siden $0/10 = 0$ og $10/10 = 1$. 
    Som vi ser har vi at 
    $$ 
        f(x) = f\left(\frac{y}{10}\right) = 10\frac{y}{10} = y
    $$ 

    $f$ utgjør en bijeksjon mellom mengdene og de har derfor 
    lik kardinalitet ($\aleph_1$).
\end{proof}

\begin{prob*}{Oppgave 5}{}
    a) Kor mange ord med $3$ bokstavar startar med $A$ eller $B$?

    b) Kor mange to-bokstav-ord startar med ein vokal i det norske 
    alfabetet?

    c) Du har $5$ par sokkar og $3$ par sandalar. 
    Kor mange måtar kan ein kombinere sokkar og sandalar? 
    Kor mange er det viss du ikkje har sokkar og sandalar på 
    samtidig?
\end{prob*}
\begin{proof}[Løsning av a)]
    Viss vi antar at vi har tilgang til det norske alfabetet, 
    og at vi ignorerer store- og små bokstaver,
    får vi ord av formen 
    $$ 
        XYZ,\quad Y,Z \in \{A,B,C,\dots,\text{Æ},\text{Ø},\text{Å}\}
    $$ 
    og 
    $$ 
        X \in \{A,B\}
    $$ 

    Valgene for bokstavene er uavhengige 
    så vi begynner med bokstav nummer $1$ hvor vi har 
    $n_1 = 2$ valg. 
    
    Deretter for $Y$ og $Z$ har vi $n_2 = n_3 = 29$ valg. 

    Da er det $$n_1 \cdot n_2 \cdot n_3 = 2 \cdot 29^2 = 1682$$
    muligheter.
\end{proof}
\begin{proof}[Løsning av b)]
    \begin{align*}
        &AB, \\   
        A &\in \{A,E,I,O,U,Y,\text{Æ},\text{Ø},\text{Å}\} \\
        B &\in \{A,B,C,\dots,\text{Æ},\text{Ø},\text{Å}\} \\
        |\text{vokaler}| \cdot |\text{alfabetet}| 
          &= 9 \cdot 29 = 261 
    \end{align*}
\end{proof}
\begin{proof}[Løsning av c)]
    Først, viss vi antar at du må velge et par (kan ikke 
    ha på to mismatchende sandaler/sokker), 
    da får vi at du velger blant $5$ par sokkar og 
    $3$ par sandaler. 

    $\Rightarrow 3 \cdot 5 = 15$ kombinasjoner.

    Viss du må ha enten sokkar på eller sandaler 
    kan du ta eit valg blant $5$ sokkar og $3$ sandaler. 

    $\Rightarrow 5 + 3 = 8$ kombinasjoner.
\end{proof}

\begin{prob*}{Oppgave 6}{}
    Ei klasse med informatikkstudentar tar tre emne: 
    diskret matematikk, statistikk og
    programmering. Under er ein tabell med kor mange som 
    strauk kvart emne:

    \begin{tabular}{|c|c|c|c|c|c|c|c|}
        \hline 
        Emne: & Stat & Mat & Prog & Stat \& Mat & Prog \& Stat & 
        Mat \& Prog & Mat \& Prog \& Stat \\
        \hline
        Stryk: & 12 & 5 & 8 & 2 & 6 & 3 & 1 \\
        \hline
    \end{tabular}

    \vspace{1em}
    a) Kor mange studentar strauk på eit emne? 

    b) Kor mange studentar strauk berre diskret matematikk?
\end{prob*}
\begin{proof}[Løsning av a)]
    La $S = \{\text{studenter som trauk Stat}\}$, 
    $M = \{\text{studenter som strauk Mat}\}$, osv. 
    hvor vi da kan finne de som strauk Mat \& Prog med 
    $$ 
        M \cap P = \{\text{studenter som strauk Mat \& Prog}\}
    $$ 

    Da har vi 
    \begin{align*}
        |M \cap P \cap S| &= 1 \\ 
        |M \cap P| &= 3 \\ 
        |P \cap S| &= 6 \\ 
        |S \cap M| &= 2 \\ 
        |P| &= 8 \\ 
        |M| &= 5 \\ 
        |S| &= 12 
    \end{align*}

    Da har vi at 
    \begin{align*}
    |\{\text{studenter som strauk ett emne}\}|
        &= |S| + |M| + |P| - 2(|S\cap M| + |S\cap P|+ |M \cap P|)
        + 3(|M\cap P \cap S|) \\ 
        &= 12 + 5 + 8 - 2(2 + 6 + 3) + 3(1) \\ 
        &= 25 - 22 + 3 \\
        &= 6
    \end{align*}
\end{proof}
\begin{proof}[Løsning av b)]
    \begin{align*}
        6 - |S| - |P| = 6 - 5 = 1
    \end{align*}
\end{proof}

\newpage 
\begin{prob*}{}{}
    a) Kor mange måtar kan ein skrive bokstavane i ordet 
    DISKRET? 

    b) Kor mange måtar kan ein skrive bokstavane i ordet 
    DISKRET viss bokstavane IS må stå saman. 

    c) Kor mange måtar er det for å ordne $5$ jenter og 
    $5$ gutar i ei rad på kino med $10$ stoler slik at ingen 
    gutar sitter ved sidan av kvarrandre?
\end{prob*}
\begin{proof}[Løsning av a)]
    Vi velger bokstav for bokstave. 

    Velg en bokstav fra 
    $$ 
        \{D,I,S,K,R,E,T\}
    $$ 
    for eksempel D. 

    Da tar vi neste valget fra 
    $$ 
    \{D,I,S,K,R,E,T\} \setminus \{D\}
    $$ 
    og så videre. 

    Dermed får vi
    \begin{align*}
        = 7 \cdot 6 \cdot 5 \cdots 1 = 7!  
    \end{align*}
    muligheter.
\end{proof}
\begin{proof}[Løsning av b)]
    Samme prosedyre som sist, men 
    mengden vi velger fra er 
    $$ 
        \{D, IS, K, R, E, T\}
    $$
    som burde bety $6!$ muligheter, men 
    vi kan også velge $SI$ siden bokstavene fortsatt stå sammen. 
    Så for hvert ord 
    kan vi bytte plass på I og S som gir et nytt valg for 
    hvert ord som kan lages med metoden over. 

    Dermed får vi  
    $$ 
        6! \cdot 2
    $$ 
    muligheter.
\end{proof}
\begin{proof}[Løsning av c)]
    Hver gutt må ha en jente på hver side, eller 
    en jente på en side viss de sitter på enden, så 
    jentene må plasseres uniformt på følgende måte: 
    $$ 
        F\_F\_F\_F\_F\_
    $$ 
    eller 
    $$ 
        \_F\_F\_F\_F\_F
    $$ 
    så vi har to konfigurasjoner. 
    For hver av disse er det $5!$ måter å arangere jentene.

    Da må vi plassere gutane på $5!$ måter i hvert tilfelle. 

    Alt i alt 
    $$ 
        2 \cdot 2 \cdot 5! \cdot 5!
    $$ 
\end{proof}

\newpage 
\begin{prob*}{Oppgave 8}{}
    a) Kor mange måtar kan vi trekke $2$ ess og $3$ kongar fra 
    ein kortstokk med $52$ kort? 

    b) Kor mange løysinger finns det til likninga som oppfyller 
    betingelsen: 
    $$ 
        a + b + c + d + e = 500
    $$ 
    Betingelse: $a,b,c,d,e$ er heiltal som er minst $10$?
\end{prob*}
\begin{proof}[Løsning av a)]
    Vi antar at vi bruker en standard kortstokk $4$ av hvert 
    kort og at vi ikke betrakter 
    $$ 
        \text{ess}, \text{1}, \text{kong}
    $$ 
    som å være forskjellig fra 
    $$ 
        \text{ess}, \text{kong}
    $$ 

    Da får vi 
    $$ 
        \binom{4}{2} \cdot \binom{4}{3} = 6 \cdot 4 = 24 
    $$ 
\end{proof}
\begin{proof}[Løsning av b)]
    Siden hver variabel er større en $10$ kan vi skrive 
    \begin{align*}
        a &= \alpha + 10 \\ 
        b &= \beta + 10 \\ 
        c &= \gamma + 10 \\ 
        d &= \delta + 10 \\ 
        e &= \epsilon + 10 \\ 
    \end{align*}

    Da har vi 
    $$ 
        \alpha + \beta + \gamma + \delta + \epsilon = 450
    $$ 

    Antall ikke-negative heltalsløsninger til 
    $$ 
        x_1 + x_2 + \cdots + x_n = k 
    $$ 
    er 
    $$ 
        \binom{n + k - 1}{n - 1}
    $$ 

    Så svaret er 
    $$ 
        \binom{454}{4}
    $$ 
\end{proof}

\end{document}
