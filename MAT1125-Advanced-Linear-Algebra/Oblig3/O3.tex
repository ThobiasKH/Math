\documentclass[11pt]{article}
\usepackage[a4paper,margin=1in]{geometry}
\usepackage{fourier} % Fourier font
\usepackage{xcolor}
\usepackage{tikz}
\usepackage[most]{tcolorbox}
\usepackage{amsthm, amsmath, amssymb}
\usepackage{enumitem}
\usepackage{hyperref}
\usepackage[nameinlink,noabbrev]{cleveref}
\usepackage{titling} 

% Dark mode colors
\definecolor{bgcolor}{HTML}{FFFFFF}
\definecolor{textcolor}{HTML}{000000}
\definecolor{defcolor}{HTML}{E86873}
\definecolor{thmcolor}{HTML}{0A9396}
\definecolor{lemcolor}{HTML}{94D2BD}
\definecolor{corcolor}{HTML}{9B4AF7}
\definecolor{probcolor}{HTML}{EE9B00}
\definecolor{excolor}{HTML}{21E933}

% Background and text color
\pagecolor{bgcolor}
\color{textcolor}

% No paragraph indentation
\setlength{\parindent}{0pt}
\setlength{\parskip}{0.7em}

% Theorem box styles
\tcbset{
  enhanced,
  colback=bgcolor,
  colframe=thmcolor,
  coltext=white,
  coltitle=white,
  fonttitle=\bfseries,
  boxrule=0.7pt,
  left=1em,
  right=1em,
  top=0.7em,
  bottom=0.7em,
  before skip=10pt,
  after skip=10pt,
}

% Theorem environments with colored boxes
\newtcbtheorem[number within=section]{thm}{Theorem}{
  colframe=thmcolor, colback=thmcolor!15!bgcolor
}{thm} % The 'thm' here is the *prefix* for the label

\newtcbtheorem[number within=section]{defn}{Definition}{
  colframe=defcolor, colback=defcolor!15!bgcolor
}{def} % The 'def' here is the *prefix* for the label

\newtcbtheorem[number within=section]{lem}{Lemma}{
  colframe=lemcolor, colback=lemcolor!15!bgcolor
}{lem}

\newtcbtheorem[number within=section]{cor}{Corollary}{
  colframe=corcolor, colback=corcolor!15!bgcolor
}{cor}

\newtcbtheorem[number within=section]{prob}{Problem}{
  colframe=probcolor, colback=probcolor!15!bgcolor
}{prob}

\newtcbtheorem[number within=section]{ex}{Example}{
  colframe=excolor, colback=excolor!15!bgcolor
}{ex}

% Proof environment 
\renewenvironment{proof}[1][\proofname]{%
  \par\pushQED{\qed}\normalfont\topsep6pt \trivlist
  \item[\hskip\labelsep\itshape #1.]\ignorespaces
}{%
  \popQED\endtrivlist\addvspace{6pt}
}

% Cleveref name formats for tcolorbox environments
\crefname{thm}{theorem}{theorems}
\Crefname{thm}{Theorem}{Theorems}

\crefname{def}{definition}{definitions}
\Crefname{def}{Definition}{Definitions}

\crefname{lem}{lemma}{lemmas}
\Crefname{lem}{Lemma}{Lemmas}

\crefname{cor}{corollary}{corollaries}
\Crefname{cor}{Corollary}{Corollaries}

\crefname{prob}{problem}{problems}
\Crefname{prob}{Problem}{Problems}

\crefname{ex}{example}{examples}
\Crefname{ex}{Example}{Examples}

\usepackage{tikz}

\title{\huge{Oblig 3}}
\author{\LARGE{Thobias Høivik}}
\date{}

\begin{document}
\maketitle
\newpage 
Jeg antar i oppgave 4.1.2 (a) at det er en skrivefeil, siden 
når jeg søker opp additivitet i det andre argumentet på nettet 
får jeg 
$$
    \langle u, v + w \rangle = \langle u, v\rangle + \langle u,w\rangle
$$ 

\newpage
\begin{prob*}{Oppgave 3.5.8}{}
    En diagonalmatrise er en matrise $A \in M_{m\times n}(\mathbb K)$ som har ikke-null elementer 
    kun langs diagonalen: $(A)_{ij} = 0$ for alle $i \neq j$. Vis at 
    $$ 
    \left\|A\right\|_\mathcal L = \max_i|(A_{ij})|
    $$
\end{prob*}
\begin{proof}[Bevis]
    La $A \in M_{m\times n}(\mathbb K)$ være en diagonalmatrise. 
    La $d_i = (A)_{ii}$ ($1\leq i \leq \min\{m,n\}$ for $m \neq n$). 
    
    For enhver $x \in \mathbb K^n$ har vi da 
    \begin{align*}
        \left\|Ax\right\|_{\ell^2}^2 
        &= 
        \displaystyle\sum_{i} |d_ix_i|^2 \\ 
        &\leq (\max_i|d_i|^2) \displaystyle\sum_{i}|x_i|^2 \\ 
        &= (\max_i |d_i|^2)\left\|x\right\|_{\ell^2}^2
    \end{align*}
    men for enhetsvektorer har vi $\left\|x\right\|_{\ell^2}^2 = 1$
    så 
    $$ 
        \left\|Ax\right\|_{\ell^2} \leq \max_i|d_i|
    $$ 
    og dette gjelder for enhetsvektorer, dermed 
    $$ 
        \left\|A\right\|_\mathcal L = 
        \sup_{\left\|x\right\| = 1} \left\|Ax\right\|_{\ell^2} 
        \leq \max_i|d_i|
    $$ 

    La $k$ være slik at $|d_k| = \max_i |d_i|$. Ta basisvektoren 
    $e_k$ med $\left\|e_k\right\|_{\ell^2} = 1$. Da har vi 
    \begin{align*}
        \left\|Ae_k\right\|_{\ell^2} &= 
        |d_k|\left\|e_k\right\|_{\ell^2} \\ 
                                     &= |d_k| \\
                                     &= \max_i|d_i|
    \end{align*}
    så 
    $$ 
    ||A||_\mathcal L \geq \max_i|d_i|
    $$ 

    Det følger da at 
    $$ 
        \left\|A\right\|_\mathcal L = \max_i|d_i|
    $$ 
\end{proof}

\newpage
\begin{prob*}{Oppgave 4.1.2}{}
    Bevis følgende: 

    \quad (a) Indreproduktet er additivt i andre argument: 
    $\langle u, v + w \rangle = \langle u, v \rangle + 
    \langle u, w \rangle$

    \quad (b) Indreproduktet er konjugert-homogen i andre argument: 
    $\langle u, \alpha v \rangle = \overline \alpha \langle u,v\rangle$

    \quad (c) Dersom $\mathbb K = \mathbb R$, er indreproduktet 
    symmetrisk: $\langle u, v \rangle = \langle v, u \rangle$
\end{prob*}
\begin{proof}[Bevis av (a)]
    La $V$ være et indreproduktrom med $u,v,w \in V$.

    \begin{align*}
        \langle u, v + w \rangle 
        &= \overline{\langle v + w, u \rangle}
        \\ 
        &= 
        \overline{\langle v, u \rangle + \langle w, u \rangle}
        \\
        &= 
        \overline{\langle v, u\rangle} + \overline{\langle w, u \rangle}
        \\ 
        &=
        \langle u, v\rangle + \langle u,w\rangle
    \end{align*}
    \begin{proof}[Bevis av (b)]
        La $V$ være et indreproduktrom over $\mathbb K$ med 
        $u,v \in V$ og $\alpha \in \mathbb K$.

        \begin{align*}
            \langle u, \alpha v\rangle
            &= \overline{\langle \alpha v, u\rangle} 
            \\ 
            &= \overline{\alpha \langle v, u\rangle} \\ 
            &= \overline \alpha \overline{\langle v, u\rangle} \\ 
            &= \overline \alpha \langle u, v \rangle
        \end{align*}
    \end{proof}
    \begin{proof}[Bevis av (c)]
        La $V$ være et indreproduktrom over $\mathbb R$ 
        med $u,v \in \mathbb R$. 

        $$
            \langle u,v\rangle = \overline{\langle v, u\rangle}
        $$
        
        Det komplekse konjugatet er 
        $$
            \overline{a + bi} = a - bi 
        $$
        
        For reelle tall har vi $b = 0$ så det komplekse konjugatet 
        gjør ingenting med reelle tall så 
        $$ 
            \overline{\langle v, u \rangle} = \langle v, u\rangle 
        $$ 
    \end{proof}
\end{proof}

\newpage 
\begin{prob*}{Oppgave 4.1.3}{}
    La $u \in U$ være slik at $\langle u,v\rangle = 0$ for alle 
    $v \in U$. Vis at $u = 0$. Er det samme sant dersom 
    $\langle v, u\rangle = 0$ for alle $v \in U$?

    \emph{Hint:} Prøv med $v = u$.
\end{prob*}
\begin{proof}[Bevis]
    La $U$ være et indreproduktrom med $u \in U$ slik som i  
    oppgavebeskrivelsen. 

    Siden $u$ tilfredsstiller 
    $$ 
        \langle u, v\rangle = 0 \quad \forall v \in U
    $$
    så krever vi at 
    $$ 
        \langle u, u \rangle = 0
    $$ 
    siden $u \in U$. 

    $$ 
        \langle u, u \rangle = 0
    $$ 
    kun dersom $u = \textbf{0}$. 

    Dersom $\langle v, u\rangle = 0, \forall v \in U$ 
    så har vi at 
    $$ 
        \overline{\langle v, u\rangle} = \overline 0 = 0 
    $$ 
    så 
    $$ 
        \langle u, v\rangle = 0
    $$ 
\end{proof}
\newpage 
\begin{prob*}{Oppgave 4.1.5}{}
    Vi lar $\mathbb K = \mathbb R$ eller $\mathbb K = \mathbb C$. 
    Vis at følgende er indreprodukter.

    \quad (a) Det kanoniske indreproduktet i $\mathbb K^n$ er 
    prikkproduktet 
    $$ 
    \langle x, y \rangle_{\mathbb K^n} := x \cdot y = 
    \displaystyle\sum_{k = 1}^{n} x_k\overline{y_k}
    $$ 

    \quad (c) La $\ell^2(\mathbb K) := 
    \left\{(a_1,a_2,\dots) : \sum_{k \in \mathbb N}^{}
    |a_k|^2 < \infty\right\}$ (se Eksempel 3.1.2). 

    \quad Vi definerer 
    $\ell^2$-indreproduktet som 
    $$ 
    \langle a,b\rangle_{\ell^2} := \displaystyle\sum_{k=1}^{\infty}
    a_k\overline{b_k}
    $$ 

    \quad (d) For $f,g \in C([a,b], \mathbb K)$ (se Oppgave 1.1.5)
    definerer vi $L^2$-indreproduktet 
    $$ 
    \langle f,g \rangle_{L^2} := \displaystyle\int_{a}^{b}f(t)
    \overline{g(t)}dt
    $$ 
\end{prob*}
\begin{proof}[Bevis av (a)]
    La $U$ være et vektorrom over $\mathbb K = \mathbb R$ eller 
    $\mathbb K = \mathbb C$, med 
    $u,v,w \in U$ og $\alpha \in \mathbb K$. Da vil 
    $\langle \cdot, \cdot \rangle_{\mathbb K^n}$ gi en skalar i 
    $\mathbb K$ siden den er definert til å ta summen av skalar-verdier
    i $\mathbb K$.

    \emph{(i)}

    Vi vet at $\langle u, u \rangle \geq 0$ fordi, i tilfellet 
    der $\mathbb K = \mathbb R$ tar vi summen av kvadraten av hver 
    komponent i vektoren $u$ 
    (i.e. vi tar summen av verdier som er større enn eller lik 0). 
    I tilfellet der $\mathbb K = \mathbb C$ simplifiseres uttrykket 
    til 
    $$ 
        \displaystyle\sum_{k=1}^{n}a_k^2 + b_k^2
    $$ 
    hvor $u_k = a_k + b_ki, a_k,b_k \in \mathbb R$, 
    som er ikke-negativt 
    for samme grunn.

    Dersom $u = (0, 0, \dots, 0)$ blir summen åpenbart $0$. 
    Viss vi antar 
    $$ 
        \displaystyle\sum_{k=1}^{n}x_k\overline{x_k} = 0
    $$ 
    vet vi at hver $x_k = 0$ siden $x_k \overline{x_k} \geq 0$ 
    så eneste måten summen er $0$ er vist alle $x_k = 0$.

    \emph{(ii)}

    \begin{align*}
        \langle u + v, w\rangle 
        &= \displaystyle\sum_{k = 1}^{n} (u_k + v_k)\overline{w_k} \\  
        &= \displaystyle\sum_{k = 1}^{n} u_k\overline{w_k} + 
        v_k\overline{w_k} \\ 
        &= \displaystyle\sum_{k=1}^{n}u_k\overline{w_k} + 
        \displaystyle\sum_{k=1}^{n}v_k\overline{w_k} 
        \\ 
        &= \langle u, w\rangle + \langle v,w\rangle
    \end{align*}

    \emph{(iii)}

    \begin{align*}
        \langle \alpha u, v \rangle 
    &= \displaystyle\sum_{k = 1}^{n} \alpha u_k \overline{v_k} 
    \\ 
    &= \alpha \displaystyle\sum_{k=1}^{n} u_k \overline{v_k}
    \\ 
    &= \alpha \langle u, v\rangle
    \end{align*}

    \emph{(iv)}

    \begin{align*}
        \langle u, v\rangle 
    &= \displaystyle\sum_{k=1}^{n} u_k \overline v_k 
    \\ 
    &= \displaystyle\sum_{k=1}^{n}\overline{v_k}u_k
    = \overline{\displaystyle\sum_{k=1}^{n}v_k\overline{u_k}}
    \\ 
    &= \overline{\langle v,u\rangle}
    \end{align*}

    Dermed oppfyller $\langle \cdot, \cdot 
    \rangle_{\mathbb K^n} : U\times U \to \mathbb K$ egenskapene 
    til et indreprodukt.
\end{proof}
\begin{proof}[Bevis av (c)]
    La $a, b \in \ell^2(\mathbb K)$. Da har vi at 
    \begin{align*}
        \displaystyle\sum_{k=1}^{\infty}|a_k \overline{b_k}|
        &\leq  
        \frac{1}{2}\left( 
        \displaystyle\sum_{i=1}^{\infty}|a_i|^2 
    + \displaystyle\sum_{i = 1}^{\infty}|b_i|^2
\right)
    \end{align*}    
    og de to summene er ved definisjon of $\ell^2(\mathbb K)$ endelige,
    så $\langle a, b \rangle$ gir da en endelig verdi. 

    Videre la også $c \in \ell^2(\mathbb K)$ og $\delta \in \mathbb K$.

    \emph{i)}

    \begin{align*}
        \langle a, a \rangle 
        &= 
        \displaystyle\sum_{k = 1}^{\infty} a_k\overline{a_k} \\ 
        &= 
        \displaystyle\sum_{k=1}^{\infty}\mathbb Re(a_K)^2 + 
        \mathbb Im(a_k)^2 \geq 0 
    \end{align*}
    siden hver $\mathbb Re(a_k)^2 \geq 0$ og 
    $\mathbb Im(a_k)^2 \geq 0$.

    Viss $a = \textbf 0$ får vi $\sum_{k=1}^{\infty} 0 = 0$. 
    Anta da at 
    $\langle a, a \rangle = 0$. 
    \begin{align*}
        \displaystyle\sum_{k=1}^{\infty} 
        \mathbb Re(a_k)^2 + \mathbb Im(a_k)^2 &= 0
    \end{align*}
    
    Som sagt er $\mathbb Re(a_k)^2$ og $\mathbb Im(a_k)^2$ 
    ikke-negative reelle tall. Den eneste løsningen for at summen 
    av ikke-negative reelle tall skal være lik $0$ er den 
    trivielle løsningen hvor 
    $\mathbb Re(a_k) = \mathbb Im(a_k) = 0$ for alle $k$. 
    Med andre ord har vi at hver $a_k = 0$ så $a = \textbf 0$.

    \emph{ii)}
    \begin{align*}
        \langle a + b, c\rangle 
        &= \displaystyle\sum_{k=1}^{\infty}(a_k+b_k) \overline{c_k}
        \\ 
        &= \displaystyle\sum_{k=1}^{\infty}a_k\overline{c_k} + 
        b_k\overline{c_k}
        \\ 
        &= \displaystyle\sum_{k=1}^{\infty} a_k\overline{c_k} + 
        \displaystyle\sum_{k=1}^{\infty} b_k\overline{c_k} 
        \\ 
        &= \langle a, c\rangle + \langle b,c\rangle
    \end{align*}

    \emph{iii)}
    \begin{align*}
        \langle \delta a, b \rangle 
        &= 
        \displaystyle\sum_{k=1}^{\infty}\delta a_k \overline{b_k}
        \\ 
        &= 
        \delta\displaystyle\sum_{k=1}^{\infty}a_k\overline{b_k}
        \\ 
        &= \delta \langle a,b\rangle
    \end{align*}

    \emph{iv)}
    \begin{align*}
        \langle a,b \rangle 
        &=
        \displaystyle\sum_{k=1}^{\infty}a_k \overline{b_k}
        \\ 
        &= 
        \displaystyle\sum_{k=1}^{\infty} \overline{b_k} a_k 
        = 
        \overline{\displaystyle\sum_{k=1}^{\infty} b_k\overline{a_k}} 
        \\ 
        &= \overline{\langle b,a\rangle}
    \end{align*}
     
    Dermed er $\langle \cdot, \cdot \rangle$ et indreprodukt 
    på $\ell^2(\mathbb K)$.
\end{proof}
\begin{proof}[Bevis av (c)]
    La $f,g,h \in C([a,b], \mathbb K)$ med $\alpha \in \mathbb K$,
    indreproduktet 
    $$ 
    \langle f, g \rangle_{L^2} = 
        \displaystyle\int_{a}^{b}f(t)\overline{g(t)}dt
    $$ 

    Viss $g$ er kontinuerlig på $[a,b]$ så er $\overline g$ også 
    kontinuerlig på dette intervallet. Dermed er integralet 
    over $f(t)\overline{g(t)}$ vell-definert.

    \emph{i)}
    \begin{align*}
        \langle f, f \rangle = \displaystyle\int_{a}^{b}|f(t)|^2 dt
    \end{align*}

    $|f(t)|^2 \in [0, \infty)$ for alle $t \in [a,b]$. 
    Siden $|f(t)|^2$ er kontinuerlig og ikke negativ på $[a,b]$ gir 
    inetgralet en ikke-negativ verdi.
    Siden vi tar integralet over ikke-negative verdier er eneste 
    løsningen hvor hele integralet blir $0$ når $f(t) = 0$ for alle 
    $t \in [a,b]$.

    \emph{ii)}
    \begin{align*}
        \langle f + g, h \rangle 
        &= \displaystyle\int_{a}^{b} (f(t) + g(t))\overline{h(t)}dt
        \\ 
        &= \displaystyle\int_{a}^{b}f(t)\overline{h(t)} + 
        g(t)\overline{h(t)} dt
        \\ 
        &= 
        \displaystyle\int_{a}^{b}f(t)\overline{h(t)}dt + 
        \displaystyle\int_{a}^{b}g(t)\overline{h(t)}dt
        \\ 
        &= \langle f, h \rangle + \langle g, h \rangle
    \end{align*}

    \emph{iii)}
    \begin{align*}
        \langle \alpha f, g \rangle 
        &= 
        \displaystyle\int_{a}^{b}\alpha f(t)\overline{g(t)}dt 
        \\ 
        &= 
        \alpha \displaystyle\int_{a}^{b}f(t)\overline{g(t)}dt 
        \\ 
        &= \alpha \langle f, g\rangle
    \end{align*}

    \emph{iv)}
    \begin{align*}
        \langle f,g\rangle 
        &= 
        \displaystyle\int_{a}^{b}f(t)\overline{g(t)}dt 
        \\ 
        &= 
        \displaystyle\int_{a}^{b}\overline{g(t)}f(t)dt 
        = 
        \overline{\displaystyle\int_{a}^{b}g(t)\overline{f(t)}dt} 
        \\ 
        &= \overline{\langle g,f\rangle}
    \end{align*}

    $\langle \cdot, \cdot \rangle_{L^2}$ utgjør et 
    indreprodukt på $C([a,b], \mathbb K)$.
\end{proof}

\newpage 
\begin{prob*}{Oppgave 4.2.2}{}
    Vis at $L^2$-normen $\left\|f\right\|_{L^2} := 
    (\int_{a}^{b}|f(t)|^2 dt)^{1/2}$ (se Eksempel 3.1.2 (d)) er 
    normen indusert av $L^2$-indreproduktet $\langle f,g\rangle_{L^2}
    := \int_{a}^{b}f(t)\overline{g(t)}dt$ (se Oppgave 4.1.5 (d)).

\end{prob*}
\begin{proof}[Bevis]
    Husk at for en $f$ i et indreproduktrom har vi  
    $$ 
        \left\|f\right\|^2 = \langle f, f \rangle
    $$ 
    så
    \begin{align*}
        (\langle f,f \rangle)^{1/2}_{L^2} 
        &= 
        \left\|f\right\|_{L^2} \\ 
        (\langle f,f\rangle)^{1/2}                                   
        &= 
        \left(\displaystyle\int_{a}^{b}f(t)
        \overline{f(t)}dt\right)^{1/2}
        \\ 
        &= 
        \left(\displaystyle\int_{a}^{b}|f(t)|^2 dt\right)^{1/2}
        \\ 
        &= 
        \left\|f\right\|_{L^2}
    \end{align*}

    $L^2$-normen er indusert av $L^2$-indreproduktet.
\end{proof}
\newpage 
\begin{prob*}{Oppgave 4.3.2}{}
    Vi betrakter $\mathbb R^2$ med standardindreproduktet. Vis 
    at listene 
    $$ 
        \mathcal B = 
        \left( 
            \begin{pmatrix}
                1 \\ 0 
            \end{pmatrix}, 
            \begin{pmatrix}
                0 \\ 1
            \end{pmatrix}
        \right)
        \text{ og }
        \mathcal C = 
        \left( 
            \begin{pmatrix}
                1/\sqrt 2 \\ 1/\sqrt 2
            \end{pmatrix}, 
            \begin{pmatrix}
                -1/\sqrt 2 \\ 1 / \sqrt 2
            \end{pmatrix}
        \right)
    $$ 
    er ortonormale. Tegn vektorene i hver av listene og 
    forklar hvordan man kan se at de to listene er ortonormale.
\end{prob*}
\begin{proof}[Bevis]
    La $\langle x, y \rangle : 
    \mathbb R^2 \times \mathbb R^2 \to \mathbb R$ betegne 
    standardindreproduktet $x \cdot y = \sum_{k=1}^{2}x_ky_k$. 

    $\mathcal{B}$)

    \begin{align*}
        \langle e_1, e_2 \rangle 
        \\ 
        &= 1 \cdot 0 + 0 \cdot 1  
        \\ 
        &= 0 
    \end{align*}
    siden $\mathbb R^2$ er over $\mathbb R$ har vi symmetri 
    og trenger derfor ikke sjekke $\langle e_2, e_1 \rangle$ 
    eksplisitt.

    \begin{align*}
        \langle e_1, e_1\rangle
        &= 
        1 \cdot 1 + 0 \cdot 0 
        \\ 
        &= 1
        \\ 
        \langle e_2, e_2\rangle
        &= 
        0 \cdot 0 + 1\cdot 1 
        \\ 
        &= 1
    \end{align*}

    Så $\mathcal B = (e_1, e_2)$ er ortonormal.

    $\mathcal C$)
    La $f_1 = (1/\sqrt 2, 1/\sqrt 2)$ og 
    $f_2 = (-1/\sqrt 2, 1/\sqrt 2)$ så $\mathcal C = (f_1, f_2)$.

    \begin{align*}
        \langle f_1, f_2 \rangle 
        &= 
        1/\sqrt 2 \cdot -1 / \sqrt 2 + 
        1/\sqrt 2 \cdot 1 /\sqrt 2
        \\ 
        &= 
        -1/2 + 1/2 
        \\ 
        &= 0
    \end{align*}
    
    Igjen, vi har symmetri så dette holder for å konkludere 
    at listen er ortogonal. 

    \begin{align*}
        \langle f_1, f_1\rangle
        &= 
        1/\sqrt 2 \cdot 1 / \sqrt 2 + 1 /\sqrt 2 \cdot 1 /\sqrt 2
        \\ 
        &= 
        1/2 + 1/2 
        \\ 
        &= 1 
        \\ 
        \langle f_2, f_2 \rangle 
        &= 
        -1 / \sqrt 2 \cdot -1/\sqrt 2 + 1 / \sqrt 2 \cdot 1 / \sqrt 2
        \\ 
        &= 
        1/2 + 1/2 
        \\ 
        &= 1
    \end{align*}
    
    $\mathcal C$ er ortonormal.
\end{proof}
\begin{tikzpicture}[scale=1]
  % axes
  \draw[->] (-0.5,0) -- (3,0) node[right] {$x$};
  \draw[->] (0,-0.5) -- (0,3) node[above] {$y$};

  % vector
  \draw[->, thick, blue] (0,0) -- (1,0) node[midway, above] {$\vec{e_1}$};

  % another vector
  \draw[->, thick, red] (0,0) -- (0,1) node[midway, left] {$\vec{e_2}$};
\end{tikzpicture}
\begin{tikzpicture}[scale=1]
  % axes
  \draw[->] (-0.5,0) -- (3,0) node[right] {$x$};
  \draw[->] (0,-0.5) -- (0,3) node[above] {$y$};

  % vector
  \draw[->, thick, blue] (0,0) -- (0.7,0.7) node[midway, above] {$\vec{f_1}$};

  % another vector
  \draw[->, thick, red] (0,0) -- (-0.7,0.7) node[midway, left] {$\vec{f_2}$};
\end{tikzpicture}

Vi ser at vektorene i $\mathcal B$ og $\mathcal C$ er 
enhetsvektorer med $90$ grader vinkel i mellom seg.

\newpage 
\begin{prob*}{Oppgave 4.3.3}{}
    Vis at en ortogonal liste $(u_1,\dots,u_n)$ automatisk er 
    linerært uavhengig.
\end{prob*}
\begin{proof}[Bevis]
    Anta at vi har en liste $U = (u_1,\dots,u_n)$ som er 
    ortogonal. 

    $$ 
        \langle u_i, u_j\rangle = 0 \quad \forall i \neq j
    $$ 
    og ingen av vektorene er $0$.

    For at en liste skal være lineært uavhengig trenger vi at 
    $$ 
        \displaystyle\sum_{i=1}^{n} \alpha_i u_i = 0
    $$ 
    kunn har den trivielle løsningen hvor $\alpha_i = 0$ for alle 
    $1 \leq i \leq n$.   

    La $u_j \in U$ være en vilkårlig vektor. 

    \begin{align*}
        \displaystyle\sum_{i=1}^{n}a_iu_i 
        &= 0 \\ 
        \left\langle \displaystyle\sum_{i=1}^{n}a_iu_i,u_j
        \right\rangle 
        &= \langle 0, u_j \rangle = 0 \quad \forall u_j \in U\\ 
        \left\langle \displaystyle\sum_{i=1}^{n}\alpha_iu_i, u_j 
            \right\rangle
        &= 
        \displaystyle\sum_{i=1}^{n}\alpha_i \langle u_i, u_j \rangle
        \\ 
        &= 
        \alpha_1 \langle u_1, u_j\rangle + \cdots + 
        \alpha_n \langle u_n, u_j \rangle
    \end{align*}

    Siden $\langle u_i, u_j\rangle = 0$ for alle $i \neq j$ 
    sitter vi igjen med 
    \begin{align*}
        \alpha_j \langle u_j, u_j \rangle 
    \end{align*}

    Siden ingen $u_i$ kan være $0$ har vi at $\alpha_j = 0$. 
    $u_j$ var valgt vilkårlig så da har vi at $\alpha_j = 0$ for alle 
    $j$. Så $\alpha_1 = \alpha 2 = \cdots = \alpha_n = 0$.
\end{proof}

\end{document}
