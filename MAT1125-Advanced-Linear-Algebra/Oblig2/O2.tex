\documentclass[11pt]{article}
\usepackage[a4paper,margin=1in]{geometry}
\usepackage{fourier} % Fourier font
\usepackage{xcolor}
\usepackage{tikz}
\usepackage[most]{tcolorbox}
\usepackage{amsthm, amsmath, amssymb}
\usepackage{enumitem}
\usepackage{hyperref}
\usepackage[nameinlink,noabbrev]{cleveref}
\usepackage{titling} 

% Dark mode colors
\definecolor{bgcolor}{HTML}{FFFFFF}
\definecolor{textcolor}{HTML}{000000}
\definecolor{defcolor}{HTML}{E86873}
\definecolor{thmcolor}{HTML}{0A9396}
\definecolor{lemcolor}{HTML}{94D2BD}
\definecolor{corcolor}{HTML}{9B4AF7}
\definecolor{probcolor}{HTML}{EE9B00}
\definecolor{excolor}{HTML}{21E933}

% Background and text color
\pagecolor{bgcolor}
\color{textcolor}

% No paragraph indentation
\setlength{\parindent}{0pt}
\setlength{\parskip}{0.7em}

% Theorem box styles
\tcbset{
  enhanced,
  colback=bgcolor,
  colframe=thmcolor,
  coltext=white,
  coltitle=white,
  fonttitle=\bfseries,
  boxrule=0.7pt,
  left=1em,
  right=1em,
  top=0.7em,
  bottom=0.7em,
  before skip=10pt,
  after skip=10pt,
}

% Theorem environments with colored boxes
\newtcbtheorem[number within=section]{thm}{Theorem}{
  colframe=thmcolor, colback=thmcolor!15!bgcolor
}{thm} % The 'thm' here is the *prefix* for the label

\newtcbtheorem[number within=section]{defn}{Definition}{
  colframe=defcolor, colback=defcolor!15!bgcolor
}{def} % The 'def' here is the *prefix* for the label

\newtcbtheorem[number within=section]{lem}{Lemma}{
  colframe=lemcolor, colback=lemcolor!15!bgcolor
}{lem}

\newtcbtheorem[number within=section]{cor}{Corollary}{
  colframe=corcolor, colback=corcolor!15!bgcolor
}{cor}

\newtcbtheorem[number within=section]{prob}{Problem}{
  colframe=probcolor, colback=probcolor!15!bgcolor
}{prob}

\newtcbtheorem[number within=section]{ex}{Example}{
  colframe=excolor, colback=excolor!15!bgcolor
}{ex}

% Proof environment 
\renewenvironment{proof}[1][\proofname]{%
  \par\pushQED{\qed}\normalfont\topsep6pt \trivlist
  \item[\hskip\labelsep\itshape #1.]\ignorespaces
}{%
  \popQED\endtrivlist\addvspace{6pt}
}

% Cleveref name formats for tcolorbox environments
\crefname{thm}{theorem}{theorems}
\Crefname{thm}{Theorem}{Theorems}

\crefname{def}{definition}{definitions}
\Crefname{def}{Definition}{Definitions}

\crefname{lem}{lemma}{lemmas}
\Crefname{lem}{Lemma}{Lemmas}

\crefname{cor}{corollary}{corollaries}
\Crefname{cor}{Corollary}{Corollaries}

\crefname{prob}{problem}{problems}
\Crefname{prob}{Problem}{Problems}

\crefname{ex}{example}{examples}
\Crefname{ex}{Example}{Examples}


\title{\huge{Oblig 2}}
\author{\LARGE{Thobias Høivik}}
\date{}

\begin{document}
\maketitle

\newpage 
\emph{Problem med Newton-basisen.}

I to av oppgavene skal vi se på Newton-basisen over 
tre punkter, men viss jeg forstår forklaringen i 
Seksjon 1.5.2 er ikke $t_2$ relevant siden det basiselementet 
med grad lik $n$ er definert som 
$$\alpha_n\displaystyle\prod_{i=0}^{n-1}(t-t_i)$$ 
så en Newton-basisen for $\mathcal P_2$ vil ikke ha 
noen element som trenger $t_2$?

Kan hende jeg misforstår, men i de oppgavene antar 
jeg at $t_2$ står i oppgaveteksten ved uhell og 
ignorer de.

\newpage
\begin{prob*}{Oppgave 2.5.2}{}
    Bevis følgende korollar: Alle $n$-dimensjonelle vektorrom over 
    $\mathbb K$ er isomorfe med hverrandre.
\end{prob*}
\begin{proof}[Bevis]
    La $\mathbb K$ være en kropp, og $U,V$ n-dimensjonelle 
    vektorrom over 
    $\mathbb K$. 

    Vi ønsker å vise at de er isomorfe, og med det at alle 
    $n$-dimensjonelle vektorrom over $\mathbb K$ er isomorfe 
    med hverrandre (siden $U,V$ er vilkårlige).

    Siden $U$ og $V$ er $n$-dimensjonelle vektorrom over $\mathbb K$
    så har vi $U \cong \mathbb K^n$ og $\mathbb K^n \cong V$ 
    ifølge Teorem 2.5.1 fra boken. 

    Siden $\cong$ utgjør en ekvivalensrelasjon må det 
    være tilfellet at vi har transitivitet: 
    $$ 
        U \cong \mathbb K^n \land \mathbb K^n \cong V \Rightarrow 
        U \cong V
    $$ 
\end{proof}

\begin{prob*}{Oppgave 2.5.3}{}
    I vektorrommet $\mathcal P_2$ lar vi $\mathcal B = (p_0, p_1, p_2)$
    være den kanoniske basisen og $\mathcal C = (q_0,q_1,q_2)$ være 
    Newton-basisen over punktene $t_0 = 0, t_1 = 1, t_2 = 2$. 
    Finn $[p]_\mathcal B$ og $[p]_\mathcal C$, der 
    $p(t) = 2 - t + 3t^2$.
\end{prob*}
\begin{proof}[Løsning]
    La $\mathcal B = (p_0, p_1, p_2)$ være den kanoniske basisen 
    til $\mathcal P_2$, og $\mathcal C = (q_0, q_1, q_2)$ være 
    Newton-basisen over punktene $t_0 = 0, t_1 = 1, t_2 = 2$. 

    Vi ønsker å finne $[p]_\mathcal B$ og $[p]_\mathcal C$, der 
    $p(t) = 2 - t + 3t^2$.

    Med andre ord, vi skal finne $\alpha_0, \alpha_1, \alpha_2$ og 
    $\beta_0, \beta_1, \beta_2$ slik at 
    $$ 
        p(t) = \alpha_0 p_0 + \alpha_1 p_1 + \alpha_2 p_2
    $$ 
    og
    $$ 
        p(t) = \beta_0 q_0 + \beta_1 q_1 + \beta_2 q_2
    $$  
    
    Vi begynner med $[p]_\mathcal B$. 

    \begin{align*}
        p(t) &= \alpha_0 p_0 + \alpha_1 p_1 + \alpha_2 p_2 \\
        2 - t + 3t^2 &= \alpha_0 p_0 + \alpha_1 p_1 + \alpha_2 p_2 \\
    \end{align*} 
    
    Husk at $p_i = t^i$ så 

    \begin{align*}
        2 - t + 3t^2 &= \alpha_0 + \alpha_1 t + \alpha_2 t^2 \\ 
        \alpha_0 &= 2 \\ 
        \alpha_1 &= -1 \\ 
        \alpha_2 &= 3 \\ 
        \Rightarrow \left[p\right]_\mathcal B &= (2, -1, 3)
    \end{align*}


    Deretter tar vi $[p]_\mathcal C$. 

    \begin{align*}
        p(t) &= \beta_0 q_0 + \beta_1 q_1 + \beta_2 q_2 \\ 
             &= \beta_0 \cdot 1 + \beta_1 \cdot t + 
             \beta_2 \cdot t(t-1) \\ 
             &= \beta_0 + \beta_1 t + \beta_2(t^2 - t) \\ 
             &= \beta_2 t^2 + (\beta_1 - \beta_2)t + \beta_0
    \end{align*}

    Vi sammenligner med $p(t) = 2 - t + 3t^2$, og får systemet
    \begin{align*}
        \beta_2 &= 3 \\ 
        \beta_1 - \beta_2 &= -1 \\ 
        \beta_0 &= 2
    \end{align*}

    Dermed
    \begin{align*}
        \beta_2 &= 3 \\ 
        \beta_1 &= 2 \\ 
        \beta_0 &= 2
    \end{align*}

    som gir
    $$
    [p]_\mathcal C = (2, 2, 3).
    $$
\end{proof}

\begin{prob*}{Oppgave 2.5.6}{}
    La $\mathcal B$ og $\mathcal C$ være standardbasisene 
    i henholdsvis $\mathbb K^n$ og $\mathbb K^m$, og la 
    $A \in M_{m\times n}(\mathbb K)$. 
    La $T: \mathbb K^n \to \mathbb K^m$ være den tilhørende 
    avbildingen $T(x) := Ax$. 
    Vis at 
    $$ 
        [T]_\mathcal C^\mathcal B = A
    $$ 
\end{prob*}
\begin{proof}[Bevis]
    La $\mathcal B = (e_1,\dots,e_n)$ være standardbasisen 
    for $\mathbb K^n$, og la $\mathcal C = (f_1,\dots,f_m)$ være 
    standardbasisen for $\mathbb K^m$. 
    La $A \in M_{m\times n}(\mathbb K)$, 
    og definer en lineær avbildning 
    $T : \mathbb K^n \to \mathbb K^m$ ved $T(x) = Ax$.

    Vi skal vise at
    \[
        [T]_{\mathcal C}^{\mathcal B} = A.
    \]

    Per definisjon består matrisen $[T]_{\mathcal C}^{\mathcal B}$ 
    av kolonnene
    \[
        [T]_{\mathcal C}^{\mathcal B}
        = \bigl([T(e_1)]_{\mathcal C}\;\; [T(e_2)]_{\mathcal C}\;\; \dots\;\; [T(e_n)]_{\mathcal C}\bigr).
    \]

    Siden $T(e_j) = A e_j$, får vi
    \[
        [T]_{\mathcal C}^{\mathcal B}
        = \bigl([Ae_1]_{\mathcal C}\;\; [Ae_2]_{\mathcal C}\;\; \dots\;\; [Ae_n]_{\mathcal C}\bigr).
    \]

    Men vektoren $Ae_j$ er nøyaktig den $j$-te kolonnen i $A$, 
    og koordinatene til en standardkolonne i 
    standardbasisen er seg selv. 
    Dermed blir hver kolonne $[Ae_j]_{\mathcal C}$ 
    lik kolonne $j$ i $A$. 

    Altså er
    \[
        [T]_{\mathcal C}^{\mathcal B} = A.
    \]
\end{proof}

\begin{prob*}{Oppgave 2.5.9}{}
    La $U$ og $V$ være endeligdimensjonelle vektorrom av lik dimensjon,
    og la $T \in \mathcal L(U,V)$. La $\mathcal B$ og $\mathcal C$
    være basiser for henholdsvis $U$ og $V$. Vis at 
    $T$ er en isomorfi hvis og bare hvis den kvadratiske matrisen 
    $[T]_\mathcal C^\mathcal B$ er inverterbar.
\end{prob*}
\begin{proof}[Bevis]
    La $U$ og $V$ være vektorrom slik at 
    $\text{dim}U = \text{dim}V = n, n \in \mathbb N$, 
    med basiser $\mathcal B$ for $U$ og $\mathcal C$ for $V$.
    La $T \in \mathcal L(U,V)$. 

    Vi ønsker å vise at $T$ er en isomorfi viss og bare viss 
    $[T]_\mathcal C^\mathcal B$ er inverterbar.
    Dette gjøres enklest med bruk av Proposisjon 2.5.8 fra 
    boken. 

    \emph{Fremover.}

    Anta at $T$ utgjør en isomorfi fra $U$ til $V$. Da finnes 
    $T^{-1} \in \mathcal L (V, U)$ slik at 
    $TT^{-1} = id$. 
    I følge 2.5.8 har vi at 
    $$ 
        \left[TT^{-1}\right]_\mathcal B^\mathcal B = 
        \left[T\right]_\mathcal C^\mathcal B 
        \left[T^{-1}\right]_\mathcal B^\mathcal C
    $$ 

    Betrakt
    \begin{align*}
        \left[TT^{-1}\right]_\mathcal B^\mathcal B 
        &= \left[id\right]_\mathcal B^\mathcal B
    \end{align*}

    Siden $\text{id}: V \to V$ er identitetsavbildingen, 
    er den $j$-te kolonnen i matrisen 
    $\left[\text{id}\right]_\mathcal C^\mathcal B$ 
    koordinatvektoren til $\text{id}(b_j) = b_j$ uttrykt i 
    basis $\mathcal C$.  
    Koordinatvektoren til $b_j$ i basis 
    $\mathcal C$ har 1 i den $j$-te posisjonen og 0 ellers.  

    Derfor er 
    \[
        \left[TT^{-1}\right]_\mathcal B^\mathcal B = 
        \left[\text{id}\right]_\mathcal B^\mathcal B = I,
    \] 
    der $I$ er identitetsmatrisen.

    Så $\left[T\right]_\mathcal C^\mathcal B$ har en matrise 
    $\left[T^{-1}\right]_\mathcal B^\mathcal C$ slik at 
    de multipliserer til identitetsmatrisen. Så 
    $\left[T\right]_\mathcal C^\mathcal B$ er inverterbar.

    \medskip
    \noindent
    \emph{Bakover.} 

    Anta nå at $\left[T\right]_\mathcal C^\mathcal B$ er inverterbar. Da finnes 
    en matrise $A^{-1}$ slik at 
    \[
        \left[T\right]_\mathcal C^\mathcal B \, A^{-1} = I.
    \] 
    I følge Proposisjon 2.5.8 finnes det da en lineær avbildning 
    $S \in \mathcal L(V,U)$ slik at $\left[S\right]_\mathcal B^\mathcal C = A^{-1}$.  

Da har vi 
\[
[T]_\mathcal C^\mathcal B [S]_\mathcal B^\mathcal C = 
I = [\text{id}]_\mathcal B^\mathcal B,
\] 
som igjen impliserer at 
\[
T \circ S = \text{id}_V \quad \text{og} \quad S \circ T = \text{id}_U.
\] 

Dermed er $T$ invertibel som lineær avbildning, altså en isomorfi.

\medskip
\noindent
Vi konkluderer da at $T$ er en isomorfi hvis og bare hvis 
$\left[T\right]_\mathcal C^\mathcal B$ er inverterbar.
\end{proof}

\newpage 
\begin{prob}{Oppgave 2.5.10}{}
    La $T : \mathcal P_3 \to \mathcal P_2$ være gitt ved 
    $$ 
        Tp := \frac{dp}{dt} \quad\quad \forall p \in \mathcal P_3,
    $$ 

    den deriverte til polynomet $p$.

    (a) La $\mathcal B = (p_0,p_1,p_2,p_3)$ og 
    $\mathcal C = (p_0, p_1, p_2)$ være de kanoniske 
    basisene for henholdsvis 
    $\mathcal P_3$ og $\mathcal P_2$. 
    Finn basisrepresentasjonen $[T]_\mathcal C^\mathcal B$. 

    (b) Generaliser den forrige oppgaven til avbildingen 
    $T:\mathcal P_n \to \mathcal P_{n-1}$, definert på samme måte, 
    for en vilkårlig $n \in \mathbb N$.
\end{prob}
\begin{proof}[Løsning til (a)]
    \begin{align*}
        [T]_\mathcal C^\mathcal B  
        &= 
        \begin{pmatrix}
            [Tp_0]_\mathcal C
            &
            [Tp_1]_\mathcal C 
            &
            [Tp_2]_\mathcal C 
            &
            [Tp_3]_\mathcal C
        \end{pmatrix}
        \\ 
        Tp_0 &= \frac{d}{dt}1 = 0 
        \\
        [0]_\mathcal C &= (0, 0, 0)
        \\ 
        Tp_1 &= \frac{d}{dt}t = 1 
        \\ 
        [1]_\mathcal C &= (1, 0, 0)
        \\ 
        Tp_2 &= \frac{d}{dt}t^2 = 2t 
        \\ 
        [2t]_\mathcal C &= (0, 2, 0)
        \\ 
        Tp_3 &= \frac{d}{dt}t^3 = 3t^2 
        \\ 
        [3t^2]_\mathcal C &= (0, 0, 3)
        \\ 
        \Rightarrow 
        [T]_\mathcal C^\mathcal B  
                          &= 
        \begin{pmatrix}
            0 & 1 & 0 & 0 \\ 
            0 & 0 & 2 & 0 \\ 
            0 & 0 & 0 & 3 
        \end{pmatrix}
    \end{align*}
    
    La oss sjekke at matrisen gjør det vi forventer. 

    La $p = t^3 + 2t^2 + 4t + 6$. Da er 
    $[p]_\mathcal B = (6, 4, 2, 1)$. 
    \begin{align*}
        [T]_\mathcal C^\mathcal B [p]_\mathcal B 
        &= 
        \begin{pmatrix}
            0 & 1 & 0 & 0 \\ 
            0 & 0 & 2 & 0 \\ 
            0 & 0 & 0 & 3 \\ 
        \end{pmatrix}
        \begin{pmatrix}
            6 \\ 4 \\ 2 \\ 1
        \end{pmatrix}
        \\ 
        &= 
        \begin{pmatrix}
            4 \\ 4 \\ 3
        \end{pmatrix}
        \\ 
        \Rightarrow 4 + 4t + 3t^2 &= \frac{dp}{dt}
    \end{align*}
\end{proof}

\begin{proof}[Løsning til (b)]
    For $T:\mathcal P_n \to \mathcal P_{n-1}$ har vi 
    en $n \times n+1 = \text{dim}\mathcal P_{n-1} \times 
    \text{dim}\mathcal P_n$ 
    matrise med følgende struktur 
    $$ 
        \begin{pmatrix}
            0 & 1 & 0 & 0 & \cdots & 0 & 0 \\ 
            0 & 0 & 2 & 0 & \cdots & 0 & 0 \\ 
            0 & 0 & 0 & 3 & \cdots & 0 & 0 \\ 
            0 & 0 & 0 & 0 & \cdots & 0 & 0 \\ 
            \vdots & \vdots & \vdots & \vdots & 
            \ddots & \vdots & \vdots \\ 
            0 & 0 & 0 & 0 & \cdots & 0 & n \\
            0 & 0 & 0 & 0 & \cdots & 0 & 0
        \end{pmatrix}
    $$ 

    altså en matrise med en "offset" diagonal med økende 
    naturlige tal som ender med $n$ is nest nederste 
    rad i kolonnen helt til høyre.
\end{proof}

\newpage
\begin{prob*}{Oppgave 2.6.4}{}
    I vektorrommet $\mathbb R^2$ lar vi $\mathcal B = (u_1,u_2)$
    være basisen gitt ved 
    $
    u_1 = 
    \begin{pmatrix}
        1 \\ 2
    \end{pmatrix}
    $,
    $ 
    \begin{pmatrix}
        -2 \\ 2
    \end{pmatrix}
    $ 
    og 
    $\mathcal C = (e_1, e_2)$ være standardbasisen. 
    Finn basisskiftematrisene $[id]_\mathcal B^\mathcal C$
    og $[id]_\mathcal C^\mathcal B$
\end{prob*}

\begin{proof}[Løsning]
    Vi begynner med å finne 
    $$ 
        \left[id\right]_\mathcal C^\mathcal B = 
        \begin{pmatrix}
            [e_1]_\mathcal C & [e_2]_\mathcal C
        \end{pmatrix}
    $$ 

    \begin{align*}
        [e_1]_\mathcal C 
        &= 
        \begin{pmatrix}
            \alpha \\ \beta
        \end{pmatrix},
        \\ 
        \alpha u_1 + \beta u_2
        &= e_1 
        \\ 
        \alpha 
        \begin{pmatrix}
            1 \\ 2 
        \end{pmatrix}
        + 
        \beta
        \begin{pmatrix}
            -2 \\ 2
        \end{pmatrix} 
        &= 
        \begin{pmatrix}
            1 \\ 0
        \end{pmatrix}
        \\ 
        \alpha - 2\beta &= 1
        \\
        2(\alpha + \beta) &= 0
        \\ 
        3\alpha &= 1 
        \Rightarrow \alpha = \frac{1}{3}
        \\ 
        \Rightarrow \beta &= -\frac{1}{3}  
    \end{align*}
    
    så 
    $$ 
        [e_1]_\mathcal C =
        \frac{1}{6}
        \begin{pmatrix}
            2 \\ -2
        \end{pmatrix}
    $$ 

    Deretter 
    \begin{align*}
        [e_2]_\mathcal C 
        &= 
        \begin{pmatrix}
            \gamma \\ \delta
        \end{pmatrix}, 
        \\ 
        \gamma u_1 + \delta u_2 &= e_2 
        \\ 
        \gamma 
        \begin{pmatrix}
            1 \\ 2 
        \end{pmatrix}
        +
        \delta 
        \begin{pmatrix}
            -2 \\ 2
        \end{pmatrix} 
                                &= 
        \begin{pmatrix}
            0 \\ 1
        \end{pmatrix}
        \\ 
        \gamma - 2\delta &= 0 \\
        2(\gamma + \delta) &= 1 \\ 
        3\gamma &= 1 \Rightarrow \gamma = \frac{1}{3} \\ 
        \Rightarrow \delta &= \frac{1}{6}
    \end{align*}    

    så 
    $$ 
        [e_2]_\mathcal C = \frac{1}{6}
        \begin{pmatrix}
            2 \\ 1
        \end{pmatrix}
    $$ 

    Med dette har vi at 
    $$ 
        [id]_\mathcal C^\mathcal B = 
        \frac{1}{6}
        \begin{pmatrix}
            2 & 2 \\
            -2 & 1
        \end{pmatrix}
    $$ 

    For å finne 
    $$ 
        [id]_\mathcal B^\mathcal C
    $$ 

    må vi bare finne inversen til 
    matrisen over siden 
    $$ 
        [id]_\mathcal B^\mathcal C [id]_\mathcal C^\mathcal B 
        = I_n
    $$ 
    
    \begin{align*}
        \begin{pmatrix}
            1 & -2 \\ 
            2 & 2 
        \end{pmatrix}
        \begin{pmatrix}
            \frac{1}{3} & \frac{1}{3} \\ 
            -\frac{1}{3} & \frac{1}{6}
        \end{pmatrix}
        &= 
        \begin{pmatrix}
            1 & 0 \\ 
            0 & 1
        \end{pmatrix}
        \\ 
        \therefore [id]_\mathcal B^\mathcal C &= 
        \begin{pmatrix}
            1 & -2 \\ 
            2 & 2
        \end{pmatrix}
    \end{align*}

\end{proof}

\newpage
\begin{prob*}{Oppgave 2.6.5}{}
    På vektorrommet $\mathcal P_2$ lar vi 
    $\mathcal B = (p_0, p_1, p_2)$ være den kanoniske basisen 
    og $\mathcal C = (q_0, q_1, q_2)$ være Newton-basisen
    over punktene $t_0 = 0, t_1 = 1, t_2 = 2$. 
    Finn basisskiftematrisene $[id]_\mathcal B^\mathcal C$ 
    og $[id]_\mathcal C^\mathcal B$.
\end{prob*}
\begin{proof}[Løsning]
    La $\mathcal P_2$ være rommet av alle polynomer med reelle
    koeffisienter av grad $\leq 2$. 

    La $\mathcal B = (p_0, p_1, p_2)$ være den kanoniske basisen, der $p_0(x)=1, p_1(x)=x, p_2(x)=x^2$. Newton-basisen $\mathcal C = (q_0, q_1, q_2)$ over punktene $t_0 = 0, t_1 = 1, t_2 = 2$ er definert som $q_0(x)=1$, $q_1(x) = x-t_0 = x$, og $q_2(x) = (x-t_0)(x-t_1) = x(x-1) = x^2-x$.

    Vi ønsker å finne basisskiftematrisene.

    Vi begynner med $[id]_\mathcal C^\mathcal B$. Per definisjon er kolonnene i $[id]_\mathcal C^\mathcal B$ koordinatvektorene til basisvektorene i $\mathcal B$ uttrykt i basis $\mathcal C$.
    \begin{align*}
        [id]_\mathcal C^\mathcal B
        &=
        \begin{pmatrix}
            [p_0]_\mathcal C & [p_1]_\mathcal C & [p_2]_\mathcal C
        \end{pmatrix}
    \end{align*}
    Vi finner koordinatvektorene:
    \begin{itemize}
        \item For $p_0(x) = 1$:
        $1 = \alpha_0 q_0(x) + \alpha_1 q_1(x) + \alpha_2 q_2(x) = \alpha_0(1) + \alpha_1(x) + \alpha_2(x^2-x)$.
        Ved å sammenligne koeffisienter får vi $\alpha_0=1, \alpha_1=0, \alpha_2=0$.
        Dermed er
        $$[p_0]_\mathcal C = \begin{pmatrix} 1 \\ 0 \\ 0 \end{pmatrix}$$
        \item For $p_1(x) = x$:
        $x = \beta_0 q_0(x) + \beta_1 q_1(x) + \beta_2 q_2(x) = \beta_0(1) + \beta_1(x) + \beta_2(x^2-x)$.
        Ved å sammenligne koeffisienter får vi $\beta_0=0, \beta_1-\beta_2=1, \beta_2=0$. Dette gir $\beta_0=0, \beta_1=1, \beta_2=0$.
        Dermed er
        $$[p_1]_\mathcal C = \begin{pmatrix} 0 \\ 1 \\ 0 \end{pmatrix}$$
        \item For $p_2(x) = x^2$:
        $x^2 = \gamma_0 q_0(x) + \gamma_1 q_1(x) + \gamma_2 q_2(x) = \gamma_0(1) + \gamma_1(x) + \gamma_2(x^2-x)$.
        Ved å sammenligne koeffisienter får vi $\gamma_0=0, \gamma_1-\gamma_2=0, \gamma_2=1$. Dette gir $\gamma_0=0, \gamma_1=1, \gamma_2=1$.
        Dermed er
        $$[p_2]_\mathcal C = \begin{pmatrix} 0 \\ 1 \\ 1 \end{pmatrix}$$
    \end{itemize}
    Dette gir oss basisskiftematrisen
    $$[id]_\mathcal C^\mathcal B = \begin{pmatrix} 1 & 0 & 0 \\ 0 & 1 & 1 \\ 0 & 0 & 1 \end{pmatrix}$$

    For å finne $[id]_\mathcal B^\mathcal C$ kan vi ta inversen av $[id]_\mathcal C^\mathcal B$.
    \begin{align*}
        \left([id]_\mathcal C^\mathcal B\right)^{-1}
        &=
        \begin{pmatrix}
            1 & 0 & 0 \\
            0 & 1 & 1 \\
            0 & 0 & 1
        \end{pmatrix}^{-1}
        \\
        &=
        \begin{pmatrix}
            1 & 0 & 0 \\
            0 & 1 & -1 \\
            0 & 0 & 1
        \end{pmatrix}
    \end{align*}
    Dermed er
    $$[id]_\mathcal B^\mathcal C = \begin{pmatrix} 1 & 0 & 0 \\ 0 & 1 & -1 \\ 0 & 0 & 1 \end{pmatrix}$$
\end{proof}

\end{document}
