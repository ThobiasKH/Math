\documentclass[11pt]{article}
\usepackage[a4paper,margin=1in]{geometry}
\usepackage{fourier} % Fourier font
\usepackage{xcolor}
\usepackage{tikz}
\usepackage[most]{tcolorbox}
\usepackage{amsthm, amsmath, amssymb}
\usepackage{enumitem}
\usepackage{hyperref}
\usepackage[nameinlink,noabbrev]{cleveref}
\usepackage{titling} 

% Dark mode colors
\definecolor{bgcolor}{HTML}{FFFFFF}
\definecolor{textcolor}{HTML}{000000}
\definecolor{defcolor}{HTML}{E86873}
\definecolor{thmcolor}{HTML}{0A9396}
\definecolor{lemcolor}{HTML}{94D2BD}
\definecolor{corcolor}{HTML}{9B4AF7}
\definecolor{probcolor}{HTML}{EE9B00}
\definecolor{excolor}{HTML}{21E933}

% Background and text color
\pagecolor{bgcolor}
\color{textcolor}

% No paragraph indentation
\setlength{\parindent}{0pt}
\setlength{\parskip}{0.7em}

% Theorem box styles
\tcbset{
  enhanced,
  colback=bgcolor,
  colframe=thmcolor,
  coltext=white,
  coltitle=white,
  fonttitle=\bfseries,
  boxrule=0.7pt,
  left=1em,
  right=1em,
  top=0.7em,
  bottom=0.7em,
  before skip=10pt,
  after skip=10pt,
}

% Theorem environments with colored boxes
\newtcbtheorem[number within=section]{thm}{Theorem}{
  colframe=thmcolor, colback=thmcolor!15!bgcolor
}{thm} % The 'thm' here is the *prefix* for the label

\newtcbtheorem[number within=section]{defn}{Definition}{
  colframe=defcolor, colback=defcolor!15!bgcolor
}{def} % The 'def' here is the *prefix* for the label

\newtcbtheorem[number within=section]{lem}{Lemma}{
  colframe=lemcolor, colback=lemcolor!15!bgcolor
}{lem}

\newtcbtheorem[number within=section]{cor}{Corollary}{
  colframe=corcolor, colback=corcolor!15!bgcolor
}{cor}

\newtcbtheorem[number within=section]{prob}{Problem}{
  colframe=probcolor, colback=probcolor!15!bgcolor
}{prob}

\newtcbtheorem[number within=section]{ex}{Example}{
  colframe=excolor, colback=excolor!15!bgcolor
}{ex}

% Proof environment 
\renewenvironment{proof}[1][\proofname]{%
  \par\pushQED{\qed}\normalfont\topsep6pt \trivlist
  \item[\hskip\labelsep\itshape #1.]\ignorespaces
}{%
  \popQED\endtrivlist\addvspace{6pt}
}

% Cleveref name formats for tcolorbox environments
\crefname{thm}{theorem}{theorems}
\Crefname{thm}{Theorem}{Theorems}

\crefname{def}{definition}{definitions}
\Crefname{def}{Definition}{Definitions}

\crefname{lem}{lemma}{lemmas}
\Crefname{lem}{Lemma}{Lemmas}

\crefname{cor}{corollary}{corollaries}
\Crefname{cor}{Corollary}{Corollaries}

\crefname{prob}{problem}{problems}
\Crefname{prob}{Problem}{Problems}

\crefname{ex}{example}{examples}
\Crefname{ex}{Example}{Examples}

\newcommand\restr[2]{{% we make the whole thing an ordinary symbol
  \left.\kern-\nulldelimiterspace % automatically resize the bar with \right
  #1 % the function
  \littletaller % pretend it's a little taller at normal size
  \right|_{#2} % this is the delimiter
  }}

\newcommand{\littletaller}{\mathchoice{\vphantom{\big|}}{}{}{}}

\title{\huge{Advanced Linear Algebra}}
\author{\LARGE{Thobias Høivik}}
\date{\Large{Fall 2025}}

\begin{document}
\maketitle

\newpage
\tableofcontents

\newpage
\section{Introduction}
The first few sections of this collection of notes follows 
the public advanced linear algebra course from the 
math at andrews university youtube channel. The lecturer 
is Dr. Andrew Bosman. These notes will, later on, be for 
the advanced linear algebra course at the University of Oslo
if I'm allowed to take the course. The University of Oslo's course 
follows the same book as Andrews' does; Advanced Linear and Matrix 
Algebra by Nathaniel Johnston. Thus these notes will be an 
amalgamation of the content of those two courses and the book.

\section{Vector Spaces and Subspaces}
Typically, when vectors are introduced in an introductory course, 
we look at vectors in \(\mathbb R^n\).  
However, vector spaces can be much more abstract and unintuitive. 
For example, \(\mathbb R\) itself, can be understood as a 
vector space over \(\mathbb Q\) with an uncountably infinite 
set of bases. So \(\mathbb R\) is a field, but also a vector space 
(which, as we'll see in the following definition, is a structure 
over a field). The same can be said for \(\mathbb C\), which is  
a field as well as a vector space over \(\mathbb R\). 
We'll also look at vector spaces of functions and all other sorts 
of abstract spaces.

\begin{defn}{Vector Space}{}
A \emph{vector space} over a field \( \mathbb{F} \) is a set \( V \) equipped with two operations:
\begin{itemize}
  \item \textbf{Vector addition:} \( + : V \times V \to V \)
  \item \textbf{Scalar multiplication:} \( \cdot : \mathbb{F} \times V \to V \)
\end{itemize}
These must satisfy 8 axioms such as associativity, distributivity, identity, and existence of additive inverses.
\end{defn}

While we will not rigorously define what a field is 
(that is usually first seen in a first course in 
abstract algebra), we can aid our intuition by 
listing a few interesting fields: 
\begin{itemize}
    \item \(\mathbb R\), as mentioned above
    \item \(\mathbb Q\)
    \item \(\mathbb C\)
    \item \(\mathbb F_2 = \{0,1\}\)
\end{itemize}

\begin{ex}{Vector Spaces}{}
\begin{itemize}
  \item \( \mathbb{R}^n \), \( \mathbb{C}^n \), over \(\mathbb R\)
      and \(\mathbb C\), respectively
  \item Spaces of functions: \( C([a,b]) \), \( \mathbb{R}^{\mathbb{N}} \)
  \item Polynomial spaces: \( \mathbb{R}[x] \)
\end{itemize}
\end{ex}

\begin{defn}{Subspace}{}
A subset \( U \subseteq V \) is a \emph{subspace} if:
\begin{itemize}
  \item \( \overline0 \in U \)
  \item \( u + v \in U \) for all \( u, v \in U \)
  \item \( \lambda u \in U \) for all \( \lambda \in \mathbb{F}, u \in U \)
\end{itemize}
\end{defn}
Note that a subspace has to be a subset which satisfies all 
of the axioms of a vector space, but checking these properties 
gives us the rest for free. In fact, checking the last property 
gives us \(0 \in U\) for free, since \(0 \in \mathbb F\), so 
that check is also technically redundant.  

\begin{ex}{Subspaces of \( \mathbb{R}^3 \)}{}
\begin{itemize}
  \item The zero subspace: \( \{0\} \)
  \item Any line or plane through the origin
  \item The whole space \( \mathbb{R}^3 \)
\end{itemize}
\end{ex}


\newpage
\section{Spans and Linear Independence}

\begin{defn}{Span}{}
Given vectors \( v_1, \ldots, v_k \in V \), the \emph{span} is:
\[
\text{span}(v_1, \ldots, v_k) = \left\{ \sum_{i=1}^k \lambda_i v_i : \lambda_i \in \mathbb{F} \right\}.
\]
It is the smallest subspace containing all the \( v_i \).
\end{defn}

\begin{defn}{Linear Independence}{}
Vectors \( v_1, \ldots, v_k \) are \emph{linearly independent} if:
\[
\sum_{i=1}^k \lambda_i v_i = 0 \quad \Rightarrow \quad \lambda_1 = \cdots = \lambda_k = 0.
\]
Otherwise, they are \emph{linearly dependent}.
\end{defn}

\begin{ex}{Dependence and Independence}{}
\begin{itemize}
  \item In \( \mathbb{R}^3 \), any three vectors lying in the same plane are linearly dependent.
  \item The standard basis vectors \( e_1, e_2, e_3 \) in \( \mathbb{R}^3 \) are linearly independent.
\end{itemize}
\end{ex}

\begin{thm}{Characterization of Dependence}{}
A set \( \{v_1, \ldots, v_k\} \) is linearly dependent if and only if some \( v_j \) lies in the span of the others.
\end{thm}

\newpage
\section{Bases}

\begin{defn}{Basis}{}
A \emph{basis} for a vector space \( V \) is a linearly independent set \( \{v_1, \ldots, v_n\} \subseteq V \) such that
\[
\text{span}(v_1, \ldots, v_n) = V.
\]
\end{defn}

\begin{defn}{Dimension}{}
The \emph{dimension} of a vector space \( V \), written \( \dim V \), is the number of vectors in any basis for \( V \).
\end{defn}

\begin{thm}{Uniqueness of Representation}{}
If \( \{v_1, \ldots, v_n\} \) is a basis for \( V \), then every \( v \in V \) can be written \emph{uniquely} as
\[
v = \lambda_1 v_1 + \cdots + \lambda_n v_n.
\]
\end{thm}

\begin{ex}{Standard Basis}{}
The standard basis of \( \mathbb{R}^n \) is \( \{e_1, \ldots, e_n\} \), where \( e_i \) has a 1 in the \( i \)th position and 0 elsewhere.
\end{ex}

\begin{thm}{All Bases Have Equal Size}{}
If \( V \) has a finite basis, then all bases of \( V \) have the same number of vectors.
\end{thm}

\begin{proof}[Sketch of Proof]
Any set of more than \( \dim V \) vectors is dependent, and any spanning set with fewer than \( \dim V \) vectors cannot span. This leads to the conclusion that all bases must contain exactly \( \dim V \) vectors.
\end{proof}

\newpage
\section{Some problems for section 2--3}
\begin{prob}{Subspace Verification}{}
Let \( V = \mathbb{R}^3 \) and let \( U = \{ (x, y, z) \in \mathbb{R}^3 : x + y + z = 0 \} \). Prove that \( U \) is a subspace of \( V \).
\end{prob}
\begin{proof}[Proof of 5.1]
    Let \(u,v \in U = \{(x,y,z)\in \mathbb R^3 : x+y+z = 0\}\).
    Trivially, the zero vector is in \(U\) since 
    \(\overline 0 = (0, 0, 0)\) and \(0+0+0 = 0\). 
    Now denote \(u = (u_1, u_2, u_3), v = (v_1, v_2, v_3)\)
    and let \(x \in \mathbb R\). 
    Then 
    \begin{align*}
        u_1 + u_2 + u_3 &= 0 \\
        x \cdot u &= x \cdot u_1 + x \cdot u_2 + x \cdot u_3 \\ 
         &= x \cdot 0 = 0\\
         &\Rightarrow x \cdot u \in U
    \end{align*}
    Next we check vector addition is closed: 
    \begin{align*}
        u + v &= (u_1 + v_1, u_2 + v_2, u_3 + v_3) \\
        u_1 + v_1 + u_2 + v_2 + u_3 + v_3 &= 0 + 0 \\ 
                                          &= 0 \\ 
                                          &\Rightarrow u + v \in U
    \end{align*}
    Thus \(U\) is a subspace of \(\mathbb R^3\)
\end{proof}

\begin{prob}{Span Problem}{}
Let \( v_1 = (1, 2, 1) \), \( v_2 = (2, 4, 2) \), and \( v_3 = (0, 1, -1) \) in \( \mathbb{R}^3 \). Determine whether \( \text{span}(v_1, v_2, v_3) = \mathbb{R}^3 \).
\end{prob}
\textit{Solution.} Note that \( v_2 = 2 v_1 \), so \( v_1 \) and \( v_2 \) are linearly dependent and span the same line.

We must check whether \( v_3 \) lies outside the span of \( v_1 \). Suppose \( v_3 = a v_1 \) for some \( a \in \mathbb{R} \):
\[
(0, 1, -1) = a(1, 2, 1) = (a, 2a, a),
\]
which implies \( a = 0 \Rightarrow 1 = 0 \), contradiction.

So \( v_3 \notin \text{span}(v_1) \), and \( \{v_1, v_3\} \) is linearly independent.

But since \( \mathbb{R}^3 \) is 3-dimensional, and \( \text{span}(v_1, v_2, v_3) = \text{span}(v_1, v_3) \), a 2-dimensional subspace, the span is not all of \( \mathbb{R}^3 \).

Therefore,
\[
\text{span}(v_1, v_2, v_3) \subsetneq \mathbb{R}^3.
\]

\newpage
\section{Dimensions of a Vector Space}
Suppose we have the following vectors in $\mathbb R^4$: 
$$ 
    \begin{bmatrix}
        2 \\ 1 \\ 0 \\ -3 
    \end{bmatrix},
    \begin{bmatrix}
        0 \\ 1 \\ 1 \\ 0 
    \end{bmatrix},
    \begin{bmatrix}
        3 \\ 0 \\ 0 \\ 1 
    \end{bmatrix}
$$
Are these linearly independent in $\mathbb R^4$? 
In other words, 
does there not exist non-trivial $c_1,c_2,c_3 \in \mathbb R$
such that the sum of these three vectors is $\vec{0}$?
\begin{align*}
    2c_1 + 3c_3 &= 0 \\ 
    c_1 + c_2 &= 0 \\ 
    c2 &= 0 \\ 
    -3c_1 + c_3 &= 0
\end{align*}
As we can quickly see there is no solution to this system, save for 
the trivial one. Therefore these three vectors are linearly  
independent. 
How about we look at a different space. 

Let $P^3$ be the space 
of polynomials of degree at most $3$.
Consider the following polynomials: 
\begin{align*}
    2+x-3x^3 \\ 
    x + x^2 \\ 
    3 + x^3
\end{align*}
Somehow, the question of these polynomials being linearly independent,
is the same as the one for the vectors in $\mathbb R^4$.
Let us see if we can make this problem again in some different vector 
space. 

Let $M_2(\mathbb R)$ be the space of real-valued $2\times 2$ matrices, 
focusing on the following elements: 
\begin{align*}
    \begin{bmatrix}
        2 & 1 \\ 
        0 & -3
    \end{bmatrix},
    \begin{bmatrix}
        0 & 1 \\ 
        1 & 0
    \end{bmatrix},
    \begin{bmatrix}
        3 & 0 \\ 
        0 & 1
    \end{bmatrix}
\end{align*}

All of these are equivalent so it would be helpful to find a 
way to reduce problems of linear independence to 
an equivalent problem in the familiar $\mathbb R^n$.


\begin{defn}{}{}
    Let \( B = \{\vec{v}_1, \dots, \vec{v}_n \} \) 
    be a finite basis for a vector space \( V \) over a field 
    \( \mathbb{F} \).  
    If a vector \( \vec{v} \in V \) can be written as
    \[
        \vec{v} = c_1 \vec{v}_1 + \cdots + c_n \vec{v}_n
    \]
    for scalars \( c_1, \dots, c_n \in \mathbb{F} \), then the tuple  
    \[
        \left[\vec{v}\right]_B = (c_1, \dots, c_n) \in \mathbb{F}^n
    \]
    is called the \textbf{coordinate vector} of \( \vec{v} \) 
    with respect to the basis \( B \), and the scalars \( c_1, 
    \dots, c_n \) are called the \textbf{coordinates} or 
    \textbf{coefficients/coefficients} of 
    \( \vec{v} \) in the basis \( B \).
\end{defn}
\begin{ex}{}{}
    Consider the example from earlier with the vector space of 
    polynomials of degree at most 4. 
    The natural basis $B = \{1, x, x^2, x^3\}$. 
    Then we have that  
    \begin{align*}
        \left[2 + x -3x^3\right]_B &= (2, 1, 0, -3) \\
        \left[x + x^2\right]_B &= (0, 1, 1, 0) \\
        \left[3 + x^3\right]_B &= (3, 0, 0, 1) \\
    \end{align*}
\end{ex}

\begin{thm}{}{}
    If $B = \{\vec{v_1},\dots,\vec{v_n}\}$ is a finite basis 
    for some vector space $V$ over a field $\mathbb F$,
    then any basis for $V$ will have $n$ vectors.

    \begin{proof}
        Assume, seeking contradiction, there exists some 
        basis $B'$ which does not have $n$ elements. 
        
        \emph{Case 1: $B'$ has more than $n$ elements.}

        Suppose $B' = \{\vec{w_1},\dots,\vec{w_m}\} \subseteq V$ 
        has $m > n$ elements. We show 
        $B'$ is linearly dependent in $V$.
        We want to find a non-trivial solution to 
        $$ 
            \displaystyle\sum_{i=1}^{m} a_i\vec{w_i} = \vec{0}
        $$
        Let us express these as coefficient vectors 
        $$ 
            \displaystyle\sum_{i=1}^{m} 
            a_i\left[\vec{w_i}\right]_{B}
            = \left[\vec{0}\right]_B
        $$ 
        $$ 
            \left[\vec{w_i}\right]_B 
            = (c_{i1}, \dots, c_{in}) 
            \in \mathbb F^n 
        $$
        Then the the expression becomes 
        \begin{align*}
            \displaystyle\sum_{i=1}^{m} 
            a_i
            \begin{bmatrix}
                c_{i1} \\ 
                \vdots \\ 
                c_{in}
            \end{bmatrix}
            = 
            \begin{bmatrix}
                0 \\
                \vdots \\ 
                0
            \end{bmatrix}
        \end{align*}
        Here we get $n$ equations with $m > n$ variables.
        By a standard result from linear algebra, 
        any homogeneous system 
        of linear equations with more unknowns than equations 
        has at least one non-trivial solution, 
        regardless of the field. 
        Hence, the vectors \( \vec{w}_1, \dots, \vec{w}_m \) 
        are linearly dependent.

        \emph{Case 2: B' has less than n elements.}

        Suppose there is some other basis 
        $B' = \{\vec{w_1}, \dots, \vec{w_m}\} \subseteq V$
        where $m < n$. If it were true that $B'$ constitutes a basis, 
        i.e. spans $V$ and is linearly independent, then 
        since $m < n$ we must have that $B$ is linearly dependent, 
        as shown in \emph{case 1}. In other words if $B'$ was a 
        basis with $m < n$ elements then $B$ cannot be a basis for 
        $V$ since we would, through the same method as above, get 
        that there exists a non-trivial solution for 
        the sum of $n$ coefficients multiplied by 
        $\vec{v_1}, \dots, \vec{v_n}$ equaling the zero vector. 
        This contradicts our assumption that $B$ is a basis for 
        $V$ and therefore $B'$ cannot be a basis for $V$.

    \end{proof}
\end{thm}

\begin{defn}{}{}
    $V$ has a basis $\{\vec{v_1},\dots,\vec{v_n}\}$, 
    we say that the dimension $dim(V) = n$. 
    Note by the theorem above that $n$ is unique since 
    we cannot have a basis with $m \neq n$ elements.
    This means that $V$ is a finite-dimensional vector space. 
    If there does not exist a finite basis we say $V$ is  
    infinite-dimensional, $dim(V) = \infty$.
\end{defn}

\newpage 
\section{Tasks from MAT1125 book chapter 2}
The following section covers some tasks from the 
main book used in my course and will therefore be 
in Norwegian.

\begin{prob}{Oppgave 2.1.3}{}
    Vis at følgende funksjoner er lineære. 
    
    \quad (a) $T:\mathbb R^n \to \mathbb R^m$ gitt ved 
    $Tx := Ax$ for en matrise $A \in M_{m\times n}(\mathbb R)$. 

    \quad (b) La $a,b \in \mathbb R, a < b$, og definer
    $T:C^0([a,b], \mathbb R) \to \mathbb R$ ved 
    $$ 
        Tf := \displaystyle\int_{a}^{b}f(t)dt
        \quad \quad 
        \forall f \in C^0([a,b], \mathbb R)
    $$ 

    \quad (c) La $g\in C^0(\mathbb R, \mathbb R) \to \mathbb R$
    være en gitt funksjon, og definer 
    $T:C^0([0,1],\mathbb R) \to \mathbb R$ ved 
    $$ 
        Tf := \displaystyle\int_{0}^{1}f(t)g(t)dt
        \quad\quad 
        \forall f \in C^0([0,1],\mathbb R)
    $$  

    \quad (d) $T:\mathcal P \to C^0([a,b],\mathbb R)$ gitt ved 
    $$ 
        Tp:=\restr{p}{[a,b]}
    $$ 
    alstå restriksjonen av $p$ til intervallet $[a,b]$.
\end{prob}
\begin{proof}[Bevis av (a)]
    La $T:\mathbb R^n \to \mathbb M^m$ definert slik 
    som i oppgavebeskrivelsen.

    Vi ønsker å vise at funksjonen er lineær 
    (bevarer vektor-addisjon og skalar-multiplikasjon).

    La $x,y \in \mathbb R^n$, la $r \in \mathbb R$, 
    og la $A \in M_{m\times n}(\mathbb R)$.
    Da er $x$ og $y$ av formen: 
    $$ 
        x = 
        \begin{pmatrix}
            x_1 \\ 
            \vdots \\ 
            x_n
        \end{pmatrix}, 
        \quad
        y = 
        \begin{pmatrix}
            y_1 \\ 
            \vdots \\ 
            y_n
        \end{pmatrix},
        \quad 
        x_y,y_i \in \mathbb R \quad(1\leq i \leq n)
    $$ 
    mens 
    $$ 
        A = 
        \begin{pmatrix}
            a_{11} & \dots & a_{1n} \\ 
            \vdots & \ddots & \vdots \\ 
            a_{m1} & \dots & a_{mn}
        \end{pmatrix}
    $$
    som betyr at 
    \begin{align*}
        Tx = Ax &= 
        \begin{pmatrix}
            a_{11} & \dots & a_{1n} \\ 
            \vdots & \ddots & \vdots \\ 
            a_{m1} & \dots & a_{mn}
        \end{pmatrix}
        \begin{pmatrix}
            x_1 \\ \vdots \\ x_n
        \end{pmatrix}
        \\ 
           &= 
           \begin{pmatrix}
            \displaystyle\sum_{i=1}^n a_{1i}x_i \\ 
            \vdots \\
            \displaystyle\sum_{i=1}^n a_{mi}x_i
           \end{pmatrix}
           \in \mathbb R^m
    \end{align*} 
    så $T$ er vell-definert.

    For så å sjekke lineæritet av vektor addisjon: 
    \begin{align*}
        T(x+y) 
        &= 
        \begin{pmatrix}
            a_{11} & \dots & a_{1n} \\ 
            \vdots & \ddots & \vdots \\ 
            a_{m1} & \dots & a_{mn}
        \end{pmatrix}
        \begin{pmatrix}
            x_1 + y_1 \\ \vdots \\ x_n + y_i 
        \end{pmatrix} 
        \\
        &= 
        \begin{pmatrix}
            \displaystyle\sum_{i=1}^n a_{1i}x_i 
            +
            \displaystyle\sum_{i=1}^n a_{1i}y_i 
            \\ 
            \vdots \\
            \displaystyle\sum_{i=1}^n a_{mi}x_i
            +
            \displaystyle\sum_{i=1}^n a_{mi}y_i
        \end{pmatrix}
        \\ 
        &= 
        \begin{pmatrix}
            \displaystyle\sum_{i=1}^n a_{1i}x_i 
            \\ 
            \vdots \\
            \displaystyle\sum_{i=1}^n a_{mi}x_i
        \end{pmatrix}
        +
        \begin{pmatrix}
            \displaystyle\sum_{i=1}^n a_{1i}y_i 
            \\ 
            \vdots \\
            \displaystyle\sum_{i=1}^n a_{mi}y_i
        \end{pmatrix}
        \\ 
        &= 
        \begin{pmatrix}
            a_{11} & \dots & a_{1n} \\ 
            \vdots & \ddots & \vdots \\ 
            a_{m1} & \dots & a_{mn}
        \end{pmatrix}
        \begin{pmatrix}
            x_1 \\ \vdots \\ x_n 
        \end{pmatrix}
        + 
        \begin{pmatrix}
            a_{11} & \dots & a_{1n} \\ 
            \vdots & \ddots & \vdots \\ 
            a_{m1} & \dots & a_{mn}
        \end{pmatrix}
        \begin{pmatrix}
            y_1 \\ \vdots \\ y_i 
        \end{pmatrix}
        \\ 
        &= 
        Tx + Ty
    \end{align*}

    Deretter sjekker vi lineæritet av skalar-multiplikasjon: 
    \begin{align*}
        T(rx) &= 
        \begin{pmatrix}
            a_{11} & \dots & a_{1n} \\ 
            \vdots & \ddots & \vdots \\ 
            a_{m1} & \dots & a_{mn}
        \end{pmatrix}
        \begin{pmatrix}
            rx_1 \\ \vdots \\ rx_n 
        \end{pmatrix}
    \\ 
    &= 
    \begin{pmatrix}
        \displaystyle\sum_{i=1}^n a_{1i}rx_i 
        \\ 
        \vdots \\
        \displaystyle\sum_{i=1}^n a_{mi}rx_i
    \end{pmatrix}
    = 
    \begin{pmatrix}
        r\displaystyle\sum_{i=1}^n a_{1i}x_i 
        \\ 
        \vdots \\
        r\displaystyle\sum_{i=1}^n a_{mi}x_i
    \end{pmatrix}
    = 
    r 
    \begin{pmatrix}
        \displaystyle\sum_{i=1}^n a_{1i}x_i 
        \\ 
        \vdots \\
        \displaystyle\sum_{i=1}^n a_{mi}x_i
    \end{pmatrix}
    \\
    &= rTx
    \end{align*}

    Med det har vi vist at $T$ er en lineær.
\end{proof}

\begin{proof}[Bevis av (b)]
    La $T:C^0([a,b], \mathbb R) \to \mathbb R$ være definert 
    som i oppgavebeskrivelsen.

    Vi ønsker å vise at $T$ er lineær.

    $T$ er vell-definert siden å ta integralet fra $a$ til $b$
    av en funksjon, kontinuerlig på det intervalet, som 
    har $\mathbb R$ som co-domene, gir et reelt tal.

    Vi sjekker så lineæritet av vektor-addisjon. 
    La $f,g \in C^0([a,b], \mathbb R)$. Da har vi 
    \begin{align*}
        T(f+g) &= \displaystyle\int_{a}^{b}(f+g)(t) dt \\ 
               &= \displaystyle\int_{a}^{b}f(t)+g(t)dt \\ 
               &= \displaystyle\int_{a}^{b}f(t)dt 
               +  \displaystyle\int_{a}^{b}g(t)dt      \\ 
               &= Tf + Tg
    \end{align*}

    Deretter sjekker vi lineæritet av skalar-multiplikasjon. 
    La $f \in C^0([a,b],\mathbb R)$ og $r \in \mathbb R$.
    Da har vi 
    \begin{align*}
        T(rf) &= \displaystyle\int_{a}^{b}(rf)(t)dt \\ 
              &= \displaystyle\int_{a}^{b}rf(t)dt \\ 
              &= r\displaystyle\int_{a}^{b}f(t)dt \\ 
              &= rTf
    \end{align*}
    som ønsket. 

    Med dette har vi vist at $T$ er lineær fra 
    $C^0([a,b],\mathbb R) \to \mathbb R$.
\end{proof}

\begin{proof}[Bevis av (c)]
    La $g \in C^0([0,1], \mathbb R)$ være gitt og 
    $T:C^0([0,1],\mathbb R$ definert som i oppgavebeskrivelsen.

    Vi ønsker å vise at $T$ er lineær.

    Først merk at produktet av to funksjoner kontinuerlige på 
    $[0,1]$ blir en funksjon kontinuerlig på $[0,1]$ 
    (kan vises med å bruke definisjonen, med grenseverdier).
    Så integralet fra $[0,1]$ av denne funksjonen er vell-definert.

    La $f,h \in C^0([0,1], \mathbb R)$. Da har vi 
    \begin{align*}
        T(f+h) &= \displaystyle\int_{0}^{1}(f+h)(t)g(t)dt \\ 
               &= \displaystyle\int_{0}^{1}(f(t)+h(t))g(t)dt \\ 
               &= \displaystyle\int_{0}^{1}f(t)g(t) + h(t)g(t)dt \\ 
               &= \displaystyle\int_{0}^{1}f(t)g(t)dt  
               +  \displaystyle\int_{0}^{1}h(t)g(t)dt \\ 
               &= Tf + Th
    \end{align*}

    La $f \in C^0([0,1], \mathbb R)$ og $r \in \mathbb R$. 
    Da har vi 
    \begin{align*}
        T(rf) &= \displaystyle\int_{0}^{1}(rf)(t)g(t)dt \\ 
              &= \displaystyle\int_{0}^{1}rf(t)g(t)dt \\ 
              &= r\displaystyle\int_{0}^{1}f(t)g(t)dt \\ 
              &= rTf
    \end{align*}
    som ønsket.
\end{proof}
\begin{proof}[Bevis av (d)]
    La $T:\mathcal P \to C^0([a,b],\mathbb R)$ definert som i 
    oppgavebeskrivelsen.

    Merk at for alle $p \in \mathcal P$, $p$ er kontinuerlig. 
    Så restriksjonen av $p$ til $[a,b]$ gir en kontinuerlig 
    funksjon med domene $[a,b]$, så $T$ er vell-definert. 

    For lineæritet av vektor-addisjon bruker vi faktumet at 
    for to funksjoner $p,q$ med domene $\mathbb R$ 
    (som polynomene våre) 
    $$ 
        (p+q)(x) = p(x) + q(x)
    $$ 

    Siden $[a,b] \subset \mathbb R$ må dette også 
    være tilfellet for alle $x \in [a,b]$. 

    Et lignende argument gjelder for skalar-multiplikasjon.
\end{proof}

\begin{prob}{Oppgave 2.2.5}{}
    La $T:\mathcal P_8 \to \mathbb R$ være den lineære avbildingen 
    $$ 
        T(p):= \displaystyle\int_{0}^{1}p(t)dt 
        \quad\quad  
        \forall p \in \mathcal P_8
    $$ 

    \quad (a) Vis at $\text{im}T = \mathbb R$. 

    \quad (b) Bruk dimensjonssatsen til å finne dimensjonen til 
    kjernen til $T$. Vil du si at kjernen til $T$ 
    er "stor" eller "liten"?
\end{prob}
\begin{proof}[Bevis av (a)]
    La $T: \mathcal P_8 \to \mathbb R$ være definert som 
    i oppgavebeskrivelsen.

    Vi ønskjer å vise at $\text{im}T = \mathbb R$. 

    For en hver $\alpha \in \mathbb R$, betrakt konstant-polynomet 
    $p(t) = \alpha \in \mathcal P_8$.
    Merk da at 
    $$ 
        T(p) = \displaystyle\int_{0}^{1}\alpha dt
        = \alpha \displaystyle\int_{0}^{1} 1 dt
        = \alpha \left[t\right]_0^1 = \alpha \cdot 1 = \alpha
    $$ 

    Så for hver $\alpha \in \mathbb R$ har vi polynomet 
    $p(t) = \alpha$ slik at $T(t) = \alpha$. 
    Så $\text{im}T = \mathbb R$.

\end{proof}
\begin{proof}[Løsning av (b)]
    Ifølge dimensjonssatsen har vi at viss $U,V$ er vektorrom 
    med $T \in \mathcal L(U,V)$ så er 
    $$ 
        \text{dim}(\text{im}T) + \text{dim}(\text{ker}T) = \text{dim}U
    $$ 

    Vi vet at $\text{dim}\mathcal P_8 = 9$ og 
    $\text{dim}(\text{im}T) = \text{dim}\mathbb R = 1$ så da vet vi 
    at dimensjonen til kjernen av $T$ er $8$, alle polynom 
    utenom konstant-polynomene. 

    Et underrom med dimensjon $n - 1$ er ganske stort.
\end{proof}

\begin{prob}{Oppgave 2.5.2}{}
    Bevis at alle $n$-dimensjonelle vektorrom over $\mathbb K$
    er isomorfe med hverandre.
\end{prob}
\begin{proof}[Bevis]
    La $U,V$ være $n$-dimensjonelle vektorrom over $\mathbb K$.

    Vi ønsker å vise at  
    $$ 
        U \cong V
    $$ 

    Siden $U,V$ er $n$-dimensjonelle har vi at 
    det finnes basiser $\mathcal B = (u_1,\dots,u_n)$ for $U$, og 
    $\mathcal C = (v_1,\dots,v_n)$ for $V$. 
    Siden $U,V$ er isomorfe med $\mathbb K^n$, så bør vi ha 
    at komposisjonen av isomorfien fra $U$ til $\mathbb K^n$ med 
    isomorfien fra $\mathbb K^n$ til $V$ gir en isomorfi 
    fra $U$ til $V$.
\end{proof}

\newpage
\section{Analysis in vector spaces}
\begin{defn}{Norm}{}
    Let $U$ be a vector space over either $\mathbb F = \mathbb R$ 
    or $\mathbb F = \mathbb C$. A norm on $U$ is a 
    function $\left\|\cdot\right\|:U \to \mathbb R$ such that 
    \begin{enumerate}
        \item $\left\|u\right\| \geq 0$ for all $u \in U$,
            and $\left\|u\right| = 0$ if any only if 
            $u = \overline 0$ 
        
        \item $\left\|\alpha u \right\| = |\alpha|\left|u\right|$ 
            for all $\alpha \in \mathbb K$ and $u \in U$

        \item $\left\|u + v \right\| \leq 
            \left\|u\right\| + \left\|v\right\|$ for all 
            $u,v \in U$
    \end{enumerate}

    A normed vector space is a pair 
    $\left(U, \left\|\cdot\right\|\right)$ consisting of a 
    vector space $U$ and a norm $\left\|\cdot\right\|$ on $U$. 
    The "length" of a vector $u$ is $\left\|u\right\|$. 
    The "distance" between two vectors $u,v$ is 
    $\left\|u - v\right\|$.
\end{defn} 

The "distance" we are most familiar with in a first course 
in linear algebra is the euclidian norm 
$$ 
    \left\|x\right\|_{\ell^2} = 
    \sqrt{\sum_{k=1}^{n}|x_k|^2}
$$ 
on $\mathbb R^n$ or $\mathbb C^n$.

In general for $p \in [1, \infty)$, the $\ell^p$-norm on 
$\mathbb R^n$ or $\mathbb C^n$ is given by 
$$ 
    \left\|x\right\|_{\ell^p(\mathbb F^n)} := 
    \left(\sum_{k=1}^{n}|x_k|^p\right)^{\frac{1}{p}}
$$ 
where $|x| = \sqrt{x\overline x}$ for a complex $x$, and 
absolute value if real.

When we set $p = \infty$ we get the so-called "max-norm"
$$ 
    \left\|x\right\|_{\ell^\infty} := \max_{k=1,\dots,n}|x_k|
$$ 

On the other extreme, when $p=1$, is often called the 
"manhattan-metric". 

\begin{prob}{Oppgave 3.1.3}{}
    La $x, y \in \mathbb R^2$, given by 
    $$ 
        x := 
        \begin{pmatrix}
            0 \\ 3 
        \end{pmatrix}, 
        y := 
        \begin{pmatrix}
            1 \\ 1
        \end{pmatrix}
    $$ 

    Calculate the length of $x$ and $y$ in the following norms: 
    $\left\|\cdot\right\|_{\ell^1}$
    $\left\|\cdot\right\|_{\ell^2}$
    and 
    $\left\|\cdot\right\|_{\ell^\infty}$
\end{prob}
\begin{proof}[Solution]
    Let $x,y$ be given as in the problem description. 

    \emph{$\ell^1$)}

    \begin{align*}
        \left\|x\right\| 
        &= 
        \displaystyle\sum_{k=1}^{2}|x_k| 
        \\ 
        &= 
        |0| + |3| = 0+3 = 3   
        \\ 
        \left\|y\right\|
        &=  
        \displaystyle\sum_{k=1}^{2}|y_k|
        \\
        &= |1| + |1| = 1+1 = 2
    \end{align*}

    \emph{$\ell^2$)}

    \begin{align*}
        \left\|x\right\| 
        &= 
        \sqrt{\displaystyle\sum_{k=1}^{2}|x_k|}
        \\ 
        &= 
        \sqrt{|0| + |3|} = \sqrt{|3|} = \sqrt 3
        \\ 
        \left\|y\right\|
        &= 
        \sqrt{\displaystyle\sum_{k=1}^{2}|y_k|}
        \\ 
        &= 
        \sqrt{|1| + |1|} = \sqrt{1 + 1} = \sqrt{2}
    \end{align*}

    \emph{$\ell^\infty$)}

    \begin{align*}
        \left\|x\right\| 
        &= 
        \max_{k=1,2}\{|x_k|\}
        \\ 
        &= \max\{0,3\} = 3 
        \\ 
        \left\|y\right\|
        &= 
        \max_{k=1,2}\{|y_k|\}
        \\ 
        &= 
        \max\{1\} = 1
    \end{align*}
\end{proof}

\begin{prob}{Oppgave 3.2.7}{}
    Vis at 

    \quad\quad (a) alle kuler er begrensede 

    \quad\quad (b) alle åpne kuler $B_r(u)$ er åpne mengder 
\end{prob}
\begin{proof}[Bevis av (a)]
    Vi ønsker å vise at det finnes en $r' > 0$ slik at 
    $B_r(u) \subseteq \overline B_r(u) \subseteq B_{r'}(0)$.

    La $r' = \left\|u\right\| + r + 1$. 
    Da er 
    $$ 
        B_{r'}(0) = \{v \in U : \left\|v\right\| < 
        \left\|u\right\| + r + 1\}
    $$

    Viss vi betrakter en $v \in \overline B_r(u)$ så 
    får vi at 
    \begin{align*}
        \left\|v\right\| &= \left\|v - u + u\right\| 
        \leq \left\|v - u\right\| + \left\|u\right\| \\ 
                        &\leq r + \left\|u\right\| < r + 
                        \left\|u\right\| + 1
    \end{align*}

    så $v \in B_{r'}(0)$ og siden $B_r(u) \subseteq \overline B_r(u)$,
    så gjelder er både den åpne- og lukkede kulen 
    begrensede.
\end{proof}

\begin{proof}[Bevis av (b)]
    Vi ønsker å vise at alle åpne kuler er åpne mengder.

    La $B_r(u) = \{v \in U : \left\|v - u\right\| < r\}$ for 
    et vektorrom $U$ og en vilkårlig $r > 0$. Vi må så 
    vise at det finnes en $s > 0$ slik at 
    $B_s(v) \subseteq B_r(u)$ for enhver $v \in B_r(u)$.

    For en vilkårlig $v \in B_r(u)$, la 
    $s = r - \left\|v - u\right\| > 0$.
    Da har vi for enhver $v' \in B_s(v)$ at 
    \begin{align*}
        \left\|v' - u\right\| &\leq \left\|v - v'\right\| + 
        \left\|v - u\right\| < s + \left\|v - u\right\| = r\\ 
                             &\Rightarrow 
        v' \in B_r(u) \Rightarrow B_s(v) \subseteq B_r(u)
    \end{align*}


\end{proof}

\begin{prob*}{Oppgave 3.4.3}{}
    Vi betrakter normene $\left\|\cdot\right\|_{\ell^1}$ og 
    $\left\|\cdot\right\|_{\ell^2}$ på $\mathbb R^2$.

    Vis at 
    $$ 
        \left\|u\right\|_{\ell^1} \leq 2\left\|u\right\|_{\ell^2}
        \text{ og }
        \left\|u\right\|_{\ell^2} \leq \left\|y\right\|_{\ell^1}
    $$ 
    for alle $u \in \mathbb R^2$
\end{prob*}
\begin{proof}[Bevis av (a)]
    La $u \in \mathbb R^2$.

    Da har vi 
    \begin{align*}
        \left\|u\right\|_{\ell^2} 
        &= \sqrt{u_1^2 + u_2^2} \\ 
        &\leq \sqrt{u_1^2} + \sqrt{u_2^2} \\ 
        &= u_1 + u_2 = \left\|u\right\|_{\ell^1} 
    \end{align*}

    For den andre ulikheten anta at 
    $$ 
        \left\|u\right\|_{\ell^1} > 2\left\|u\right\|_{\ell^2}
    $$ 

    Da får vi 
    \begin{align*}
        u_1 + u_2 &> 2\sqrt{u_1^2 + u_2^2} \\ 
        \frac{u_1 + u_2}{2} &> \sqrt{u_1^2 + u_2^2} \\ 
        \frac{u_1^2 + 2u_1u_2 + u_2^2}{4} &> u_1^2 + u_2^2 \\ 
        \frac{u_1^2 - 4u_1^2 + 2u_1u_2 + u_2^2 - 4u_2^2}{4} &> 0 \\
        -3u_1^2 - 3u_2^2 + 2u_1u_2 &> 0 \\ 
        2u_1u_2 &> 3(u_1^2u_2^2)
    \end{align*}
    som er usant for alle reelle $u_1, u_2$. 
    Da var antagelsen feil og dermed har vi 
    $$ 
        \left\|u\right\|_{\ell^1} \leq \left\|u\right\|_{\ell^2}
    $$ 
\end{proof}

\end{document}
