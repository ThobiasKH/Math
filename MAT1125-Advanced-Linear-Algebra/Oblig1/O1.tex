\documentclass[11pt]{article}
\usepackage[a4paper,margin=1in]{geometry}
\usepackage{fourier} % Fourier font
\usepackage{xcolor}
\usepackage{tikz}
\usepackage[most]{tcolorbox}
\usepackage{amsthm, amsmath, amssymb}
\usepackage{enumitem}
\usepackage{hyperref}
\usepackage[nameinlink,noabbrev]{cleveref}
\usepackage{titling} 

% Dark mode colors
\definecolor{bgcolor}{HTML}{FFFFFF}
\definecolor{textcolor}{HTML}{000000}
\definecolor{defcolor}{HTML}{E86873}
\definecolor{thmcolor}{HTML}{0A9396}
\definecolor{lemcolor}{HTML}{94D2BD}
\definecolor{corcolor}{HTML}{9B4AF7}
\definecolor{probcolor}{HTML}{EE9B00}
\definecolor{excolor}{HTML}{21E933}

% Background and text color
\pagecolor{bgcolor}
\color{textcolor}

% No paragraph indentation
\setlength{\parindent}{0pt}
\setlength{\parskip}{0.7em}

% Theorem box styles
\tcbset{
  enhanced,
  colback=bgcolor,
  colframe=thmcolor,
  coltext=white,
  coltitle=white,
  fonttitle=\bfseries,
  boxrule=0.7pt,
  left=1em,
  right=1em,
  top=0.7em,
  bottom=0.7em,
  before skip=10pt,
  after skip=10pt,
}

% Theorem environments with colored boxes
\newtcbtheorem[number within=section]{thm}{Theorem}{
  colframe=thmcolor, colback=thmcolor!15!bgcolor
}{thm} % The 'thm' here is the *prefix* for the label

\newtcbtheorem[number within=section]{defn}{Definition}{
  colframe=defcolor, colback=defcolor!15!bgcolor
}{def} % The 'def' here is the *prefix* for the label

\newtcbtheorem[number within=section]{lem}{Lemma}{
  colframe=lemcolor, colback=lemcolor!15!bgcolor
}{lem}

\newtcbtheorem[number within=section]{cor}{Corollary}{
  colframe=corcolor, colback=corcolor!15!bgcolor
}{cor}

\newtcbtheorem[number within=section]{prob}{Problem}{
  colframe=probcolor, colback=probcolor!15!bgcolor
}{prob}

\newtcbtheorem[number within=section]{ex}{Example}{
  colframe=excolor, colback=excolor!15!bgcolor
}{ex}

% Proof environment 
\renewenvironment{proof}[1][\proofname]{%
  \par\pushQED{\qed}\normalfont\topsep6pt \trivlist
  \item[\hskip\labelsep\itshape #1.]\ignorespaces
}{%
  \popQED\endtrivlist\addvspace{6pt}
}

% Cleveref name formats for tcolorbox environments
\crefname{thm}{theorem}{theorems}
\Crefname{thm}{Theorem}{Theorems}

\crefname{def}{definition}{definitions}
\Crefname{def}{Definition}{Definitions}

\crefname{lem}{lemma}{lemmas}
\Crefname{lem}{Lemma}{Lemmas}

\crefname{cor}{corollary}{corollaries}
\Crefname{cor}{Corollary}{Corollaries}

\crefname{prob}{problem}{problems}
\Crefname{prob}{Problem}{Problems}

\crefname{ex}{example}{examples}
\Crefname{ex}{Example}{Examples}

\usepackage{mathrsfs}

\title{\huge{Oblig 1}}
\author{\LARGE{Thobias Høivik}}
\date{}

\begin{document}
\maketitle

\newpage
\begin{prob*}{Oppgave 1.4.2}{}
    Vis at standardbasisen $\left(e_1,\dots,e_n\right)$ er en basis 
    for $\mathbb K^n$. 
\end{prob*}
\begin{proof}[Bevis]
    La $\mathbb K$ være en kropp og la 
    $E = (e_1,\dots,e_n)$ hvor
    $$ 
        e_1 = 
        \begin{pmatrix}
            1 \\
            \vdots \\
            0 \\ 
            \vdots \\ 
            0
        \end{pmatrix},
        e_i = 
        \begin{pmatrix}
            0 \\
            \vdots \\
            1 \\ 
            \vdots \\ 
            0
        \end{pmatrix},
        e_n = 
        \begin{pmatrix}
            0 \\
            \vdots \\
            0 \\ 
            \vdots \\ 
            1
        \end{pmatrix}
    $$ 
    For å vise at 
    $E$ utgjør en basis for 
    $\mathbb K^n$ må vi vise at $E$ er 
    lineært uavhengig og spenner $\mathbb K^n$.

    \emph{Lineær uavhengighet.}

    Vi ønsker å vise at det ikke finnes en 
    ikke-triviel løsning til 
    $$ 
        \displaystyle\sum_{i=1}^{n}k_ie_i = \vec{0} 
    $$
    hvor $k_i \in \mathbb K$. 
    Summen av alle $k_ie_i$ er 
    $$
        \displaystyle\sum_{i=1}^{n}k_ie_i 
        = 
        \begin{pmatrix}
            k_1 \\ 
            0 \\ 
            \vdots \\ 
            0
        \end{pmatrix} 
        +
        \begin{pmatrix}
            0 \\ 
            k_2 \\ 
            \vdots \\
            0
        \end{pmatrix}
        +
        \dots 
        +
        \begin{pmatrix}
            0 \\ 
            0 \\
            \vdots \\ 
            k_n
        \end{pmatrix}
        =
        \begin{pmatrix}
            k_1 \\ 
            k_2 \\ 
            \vdots \\ 
            k_n
        \end{pmatrix} 
    $$ 
    Vi ser at eneste måten 
    $$ 
        \begin{pmatrix}
            k_1 \\ 
            \vdots \\ 
            k_n
        \end{pmatrix}  
        = 
        \vec{0}
        =
        \begin{pmatrix}
            0 \\ 
            \vdots \\ 
            0
        \end{pmatrix} 
    $$ 
    er den trivielle løsningen 
    $k_1 = k_2 = \dots = k_n = 0$.
    Dette betyr at $E$ er lineært uavhengig. 

    \emph{Utspenning av rommet.}

    Vi ønsker å vise at $\text{span}(E) = \mathbb K^n$.
    Husk at 
    $$ 
        \mathbb K^n = 
        \left\{ 
            \begin{pmatrix}
                k_1 \\ 
                k_2 \\ 
                \vdots \\ 
                k_n
            \end{pmatrix}
            \middle| 
            k_i \in \mathbb K, i \in \{1,2,\dots, n\}
        \right\} 
    $$ 
    Spennrommet $\text{span}(E)$ er underrommet av alle vektorer 
    av formen 
    $$ 
        k_1e_1 + k_2e_2 + \dots + k_ne_n = 
        \begin{pmatrix}
            k_1 \\ 
            k_2 \\ 
            \vdots \\ 
            k_n
        \end{pmatrix}
    $$ 
    hvor $k_i \in \mathbb K$. Med andre ord: 
    $$ 
        \text{span}E = 
        \left\{
            \begin{pmatrix}
                k_1 \\ 
                k_2 \\ 
                \vdots \\ 
                k_n
            \end{pmatrix}
            \middle| 
            k_i \in \mathbb K, i \in \{1,2,\dots, n\}
        \right\}
        = \mathbb K^n
    $$ 
    som ønsket. Med dette har vi vist at $E$ spenner ut 
    hele rommet. 

    Siden vi har vist at $E$ er lineært uavhengig 
    og spenner ut $\mathbb K^n$
    konkluderer vi med at $E$ utgjør en basis for $\mathbb K^n$.
\end{proof}

\newpage
\begin{prob*}{Oppgave 1.4.6}{}
    (a) 
    La $\mathscr C = (e_1,\dots,e_n)$ være standardbasisen i 
    $\mathbb K^n$. Vis at 
    $$ 
        \left[ x \right]_{\mathscr C} = x 
        \quad 
        \forall x \in \mathbb K^n
    $$ 

    (b)
    La $U$ være et vektorrom over $\mathbb K$ med en basis 
    $\mathscr B = (u_1,\dots,u_n)$. Vis at 
    $$ 
        \left[u_j\right]_{\mathscr B} = e_j \quad \forall j = 1,\dots,n
    $$ 
\end{prob*}

\begin{proof}[Bevis av (a)]
    La $\mathbb K$ være en kropp og la 
    $\mathscr C = (e_1,\dots,e_n)$ være 
    standardbasisen i $\mathbb K^n$.
    
    Vi ønsker å vise at basisrepresentasjonen av $x$ i basisen 
    $\mathscr C$, $[x]_{\mathscr C} = x$ for alle $x \in \mathbb K^n$.

    La $x \in \mathbb K^n$. Da har vi at det finnes en 
    unik n-tuppel $(b_1, b_2, \dots, b_n) \in \mathbb K^n$ slik at 
    $x = b_1e_1 + b_2e_2 + \dots + b_ne_n$. Denne tuppelen er 
    basisrepresentasjonen $[x]_\mathscr C = (b_1, b_2, \dots, b_n)$.
    Med standardbasisen har vi at 
    \begin{align*}
        x &= b_1e_1 + b_2e_2 + \dots + b_ne_n \\ 
          &= 
          \begin{pmatrix}
            1 \\ 
            0 \\ 
            \vdots \\ 
            0
          \end{pmatrix} 
          b_1
          + 
          \begin{pmatrix}
            0 \\ 
            1 \\ 
            \vdots \\ 
            0   
          \end{pmatrix}
          b_2
          +
          \dots 
          + 
          \begin{pmatrix}
            0 \\
            0 \\ 
            \vdots \\ 
            1
          \end{pmatrix}
          b_n
          \\ 
          &= 
          \begin{pmatrix}
            b_1 \\ 
            b_2 \\
            \vdots \\ 
            b_n
          \end{pmatrix} 
          = (b_1, b_2, \dots, b_n) = \left[x\right]_{\mathscr C}
    \end{align*}
    som er det vi ønsket å vise.
\end{proof}

\begin{proof}[Bevis av (b)]
    La $U$ være et vektorrom over kroppen $\mathbb K$ med 
    en basis $\mathscr B = (u_1,\dots,u_n)$. 

    Vi ønsker å vise at basisrepresentasjonen, 
    $\left[u_j\right]_\mathscr B = e_j$ hvor 
    $e_j$ refererer til den $j$-ende vektoren 
    i standard basisen (vektoren med 0 i alle index-er utenom j, 
    hvor vi har 1) 
    og $j \in \{1,2,\dots,n\}$.
    La $u_j$ være en vilkårlig vektor i basisen. 
    Da er $u_j$ også en vektor i $U$, næmlig vektoren 
    $$ 
        0 \cdot u_1 + \dots + 1 \cdot u_j + \dots + 0 \cdot u_n
    $$ 
    Da finens det en n-tuppel $(k_1, k_2, \dots, k_n) \in \mathbb K^n$
    slik at  
    $$ 
        k_1 u_1 + \dots + k_j u_j + \dots + k_n j_n = u_j
    $$ 
    Dette er basisrepresentasjonen av $u_j$ i basisen. 
    Spesifikt, er det tuppelen $(0, \dots, 1, \dots, 0)$, hvor 
    alle oppføringer er $0$ utenom den $j$-ende index-en, som er $1$. 
    Dette er $e_j = (0, \dots, 1, \dots, 0)$.
    Med andre ord, 
    $$ 
        \left[u_j\right]_\mathscr B = e_j
    $$ 
    som ønsket.
\end{proof}

\newpage 
\begin{prob*}{Oppgave 1.5.2}{}
    Vis at 

    \quad (a) dim$\mathscr P^n = n + 1$

    \quad (b) dim$\mathscr P = \infty$

    \quad (c) dim$C^0(\mathbb R, \mathbb R) = \infty$
\end{prob*}

\begin{proof}[Bevis av (a)]
    La $\mathscr P^n$ være rommet av polynomer med 
    koeffisienter i $\mathbb R$.

    Vi ønsker å vise at dim$\mathscr P^n = n+1$. 
    Vi vet at, viss et vektorrom har en 
    basis $(b_1, b_2, \dots, b_n)$, 
    så er rommet et endelig-dimensjonelt rom med dimensjon $n$.
    La oss se på den kanoniske basisen til $\mathscr P^n$, 
    altså $(1, x, x^2, \dots, x^n)$. 
    Denne basisen består av alle $x^i$ hvor $i = 0,1,\dots,n$. 
    Med andre ord består den av $|\{0,1,2,\dots,n\}| = n+1$ elementer. 
    Derfor kan vi konkludere at dim$\mathscr P^n = n+1$.
\end{proof}

\begin{proof}[Bevis av (b)]
    La $\mathscr P$ være rommet av alle polynomer med 
    koeffisienter i $\mathbb R$.

    Vi ønsker å vise at dim$\mathscr P = \infty$. Vi gjør dette ved 
    å vise at det ikke finnes en endelig basis for $\mathscr P$.

    Anta, i søk om kontradiksjon, at det finnes en endelig basis 
    $B = \{p_1, p_2, \dots, p_n\}$ for $\mathscr P$, 
    med $n$ elementer. 
    La $d_i$ være graden av polynomet $p_i$ for $i = 1,\dots,n$. 
    La $d = \max\{d_1,\dots,d_n\}$. Da er graden av hver $p_i$
    mindre enn eller lik $d$. 
    Hvis vi nå ser på den lineære kombinasjonen 
    $$ 
        q(x) = \displaystyle\sum_{i=1}^{n} k_ip_i(x), 
        k_i \in \mathbb K
    $$
    så får vi at graden av $q$ er mindre enn eller lik $d$. 
    Så for hvert polynom i spennrommet til $B$ har grad høyst $d$, 
    men $x^{d + 1} \in \mathscr P$, som motsier antagelsen vår 
    at det finnes en endelig basis for rommet. 

    Med det kan vi konkludere at dim$\mathscr P = \infty$.
\end{proof}

\begin{proof}[Bevis av (c)] 
    La $C^0(\mathbb R, \mathbb R)$ være rommet av kontinuerlige 
    $f:\mathbb R \to \mathbb R$.

    Hvert polynom er kontinuerlig, så 
    $\mathscr P \subset C^0(\mathbb R, \mathbb R)$.

    Siden $C^0(\mathbb R, \mathbb R)$ har et uendelig-dimensjonelt 
    underrom, kan ikke $C^0(\mathbb R, \mathbb R)$ 
    være endelig-dimensjonelt. Som sagt i Proposisjon 1.4.11 fra 
    boken, for et underrom $V$ av rommet $U$ har vi at 
    $\text{dim}(V) \leq \text{dim}(U)$.

    Konklusjon: $\text{dim}(C^0(\mathbb R, \mathbb R)) = \infty$.
\end{proof}

\newpage 
\begin{prob*}{Oppgave 1.6.5}{}
    (a) Vis at $V \oplus W$ er et vektorrom.  

    (b) Vis at dersom $(v_1,\dots,v_n)$ er en basis 
    for $V$ og $(w_1,\dots,w_n)$ er en basis for $W$, er 
    $$ 
        \mathscr B := 
        \left(
            \left( 
                v_1, 0_W
            \right),
            \dots, 
            \left( 
                v_n, 0_W
            \right),
            \left( 
                0_V, w_1
            \right),
            \dots, 
            \left( 
                0_V, w_m
            \right)
        \right)
    $$ 
    en basis for $V \oplus W$.

    (c) Vis at $\text{dim}(V \oplus W) = \text{dim}V + \text{dim}W$.
\end{prob*}
\begin{proof}[Bevis av (a)]
    La $\mathbb K$ være en kropp og $V,W$ rom over 
    $\mathbb K$.

    Vi ønsker å vise at $V \oplus W = \{(v,w):v\in V, w \in W\}$ 
    er et vektorrom med 
    $$ 
        \text{Null element: } \left(0_V, 0_W\right)
    $$ 
    og
    $$ 
        \left(u,v\right) + \left(u',v'\right)
        = 
        \left(u+u',v+v'\right)
    $$ 
    og 
    $$ 
        k\left(u,w\right) = \left(ku,kw\right) 
    $$ 
    for $v,v' \in V, w,w' \in W$ og $k \in \mathbb K$.
    
    Vi begynner med å vise at $(V \oplus W, +)$ er en abelsk gruppe.
    Da må vi vise følgende
    \begin{enumerate}
        \item Assosiativitet av addisjon
        \item Identitetselement 
        \item Inverser
        \item Kommutativitet
    \end{enumerate}
    
    Merk først at addisjonen, som vi har definert komponentvis, 
    er lukket, siden addisjonsoperasjonene for 
    $V$ og $W$ er lukket.

    \emph{Assosiativitet.}

    La $v_1,v_2,v_3 \in V$ og $w_1,w_2,w_3 \in W$.
    \begin{align*}
        \left(
            \left(v_1, w_1\right) 
            +  
            \left(v_2, w_2\right) 
        \right)
        + 
        \left(v_3, w_3\right) 
        &= 
        \left( 
            v_1 + v_2, w_1 + w_2
        \right)
        + 
        \left(v_3, w_3\right)
        \\ 
        &= 
        \left(
            v_1 + v_2 + v_3, w_1 + w_2 + w_3
        \right)
        \\ 
        &= 
        \left(v_1, w_1\right) 
        + 
        \left( 
            v_2 + v_3, w_2 + w_3 
        \right)
        \\ 
        &= 
        \left(v_1, w_1\right) 
        + 
        \left( 
            \left(v_2, w_2\right) 
            + 
            \left(v_3, w_3\right)
        \right)
    \end{align*}

    \emph{Identitetselement.}

    Vi viser at null-elementet utgjør 
    en identitet for denne gruppen. 
    Vi har, for en vilkårlig $(v,w) \in V \oplus W$
    $$ 
        (v,w) + \left(0_V, 0_W\right) 
        = 
        \left(0_V, 0_W\right) + (v,w)
        = 
        \left(v + 0_V, w + 0_W\right) 
        = (v, w)
    $$ 

    \emph{Inverser.}
    For en vilkårlig $(v,w) \in V \oplus W$ har vi inversen 
    $$ 
        (-v, -w) \in V \oplus W 
    $$ 
    $$ 
        (v,w) + (-v,-w) = (v - v, w - w) = 
        (0_V, 0_W) = (-v, -w) + (v, w)
    $$ 

    \emph{Kommutativitet.}

    Addisjonsoperasjonene til $V$ og $W$ er 
    kommutative, så siden addisjon i $V \oplus W$ er komponentvis 
    er addisjonen kommutativ.
    $$ 
        (v_1,w_1) + (v_2, w_2) = (v_1+v_2, w_1 + w_2)
        = (v_2 + v_1, w_2, + w_1) 
        = (v_2, w_2) + (v_1, w_1)
    $$ 

    Deretter viser vi egenskapene til multiplikasjon, 
    næmlig følgende 
    \begin{enumerate}
        \item $0(v,w) = (0_V, 0_W)$
        \item $1(v,w) = (v, w)$
    \end{enumerate}

    Den første egenskapen følger fra hvordan 
    multiplikasjon er definert med $k \in \mathbb K$.
    Det samme gjelder for den andre egenskapen. 

    Til slutt har vi di distributative egenskapene. 

    For $k \in \mathbb K$ og $(v_1, w_1), (v_2,w_2) \in V \oplus W$
    så har vi 
    \begin{align*}
        k((v_1, w_1) + (v_2, w_2)) 
        &= 
        k(v_1+v_2, w_1 + w_2) = (k(v_1 + v_2), k(w_1, w_2)) 
        \\ 
        &= 
        (kv_1 + kv_2, kw_1 + kw_2) = (kv_1, kw_1) + (kv_2, kw_2)
        \\ 
        &= k(v_1, w_1) + k(v_2, w_2)
    \end{align*}
    Med det har vi vist at multiplikasjon er distributativt
    over vektor-addisjon. 

    For $k_1, k_2 \in \mathbb K$ og $(v,w) \in V \oplus W$ 
    så har vi 
    \begin{align*}
        (k_1 + k_2)(v,w) 
        &= 
        ((k_1 + k_2) v, (k_1 + k_2) w) = (k_1v + k_2v, k_1w + k_2w)
        \\                      
        &=
        (k_1v, k_1w) + (k_2v, k_2w) = k_1(v,w) + k_2(v,w)
    \end{align*}
    Med dette kan vi konludere at multiplikasjon er distributativt 
    over skalar-addisjon. 

    Da har vi vist at $V \oplus W$ oppfølger alle kravene 
    for å være et vektorrom.
\end{proof}

\begin{proof}[Bevis av (b)]
    La $(v_1,\dots,v_n)$ være en basis for $V$ og 
    $(w_1,\dots,w_m)$ være en basis for $W$.
    
    Vi ønsker å vise at  
    $$ 
        \mathscr B := 
        \left(
            \left( 
                v_1, 0_W
            \right),
            \dots, 
            \left( 
                v_n, 0_W
            \right),
            \left( 
                0_V, w_1
            \right),
            \dots, 
            \left( 
                0_V, w_m
            \right)
        \right)
    $$
    er en basis for $V \oplus W$.
    Vi må da vise at $\mathscr B$ er lineært uavhengig og 
    spenner ut $V \oplus W$.

    \emph{Lineær uavhengighet.}
    
    Vi må vise at den eneste løsningen til 
    $$ 
        \displaystyle\sum_{s=1}^{|\mathscr B|} k_sb_s = 
        \left(0_V, 0_W\right), 
        k_s \in \mathbb K,
        b_s \in \mathscr B 
    $$ 
    er den trivielle løsningen.

    Summen over blir 
    \begin{align*}
        &
        k_1(v_1, 0_W) + \dots + k_n(v_n, 0_W) + 
        k_{n+1}(0_V, w_1) + \dots + k_{m+n}(0_V, w_m)
        \\
        &= 
        (k_1v_1 + \dots k_nv_n + k_{n+1}0_V + \dots + k_{m+n}0_V, 
        k_10_W + \dots + k_n0_W + k_{n+1}w_1 + \dots + k_{m+n}w_m)
        \\ 
        &=
        (k_1v_1 + \dots + k_nv_n , k_{n+1}w_1 + \dots + k_{m+n}w_m )
        \\
        &=
        \left(
            \sum_{i=1}^n k_iv_i, 
            \sum_{j=1}^m k_jw_j 
        \right)
    \end{align*}
    Nå sitter vi igjen med en 2-tuppel der første 
    er summen av alle $v_i$ i basisen til $V$, ganget 
    med koeffisient, og andre 
    er summen av alle $w_j$ i basisen til $W$, ganget med koeffisient. 
    Siden $v_i, w_j$ er elementer av basisen til sine 
    rom, vet vi at den eneste måten 
    $$ 
        \displaystyle\sum_{i=1}^{n}k_iv_i 
        = 0_V
        \text{ og }
        \displaystyle\sum_{j=1}^{m}k_jw_j
        = 
        0_W
    $$ 
    er når alle koeffisientene er $0$; 
    Den trivielle løsningen.

    \emph{Spenner hele rommet.}

    Vi vet at for en vilkårlig $v \in V$ så har vi at 
    $v$ er en lineær kombinasjon av $(v_1, \dots, v_n)$  
    hvor $v_i$ er multiplisert med en $k_i \in \mathbb K$
    og det samme gjelder for en $w \in W$ med $W$ sin basis 
    og koeffisienter i $\mathbb K$.

    Viss $v = \sum_{i=1}^nk_iv_i$ og $w = \sum_{j=1}^m k_jw_j$, 
    så er 
    \begin{align*}
        (v,w) 
        &= 
        \displaystyle\sum_{i=1}^n k_i\left(v_i, 0_W\right)
        + 
        \displaystyle\sum_{j=1}^{m} k_j\left(0_V, w_j\right)
        \\      
        &= 
        \displaystyle\sum_{s=1}^{n+m = |\mathscr B|} 
        k_s b_s 
    \end{align*}
    hvor $k_s \in \{k_1,\dots,k_n, k_{n+1}, \dots, k_m\}$ 
    og $b_s = (v_s, 0_W)$ for $s = 1,2,\dots,n$ og 
    $b_s = (0_V, w_{s-n})$ 

    for $s = n + 1, n + 2, \dots, n+m$. 

    Så vi kan, for en vilkårlig $(v,w) \in V \oplus W$, 
    lage $(v,w)$ med en lineær kombinasjon av vektorer i basisen 
    $\mathscr B$. Med andre ord har vi at 
    $\text{span}(\mathscr B) = V \oplus W$. 
    
    Siden $\mathscr B$ er lineært uavhengig og spenner ut $V \oplus W$
    så har vi at $\mathscr B$ utgjør en basis til dette rommet.
\end{proof}

\begin{proof}[Bevis av (c)]
    Som vi peket til i \emph{Bevis av (b)}, så har vi at 
    $|\mathscr B| = n+m$. Siden $\mathscr B$ utgjør en basis 
    for $V \oplus W$, har vi at 
    $\text{dim}V\oplus W = n+m = \text{dim}V + \text{dim}W$ på 
    grunn av at det er antatt at basisen til $V$ har $n$ elementer
    og basisen til $W$, $m$.
\end{proof}

\newpage 
\begin{prob*}{Oppgave 1.8.2}{}
    (a) Vis at vektorrommet $\mathbb K^n$ har dimensjon $n$.

    (b) Vis at vektorrommet $M_{m\times n}(\mathbb K)$ har  
    dimensjon $mn$.
\end{prob*}

\begin{proof}[Bevis av (a)]
    La $\mathbb K$ være en kropp. 

    Vi ønsker å vise at $\text{dim}(\mathbb K^n) = n$.

    Betrakt basisen 
    $$
        E = (e_1, e_2, \dots, e_n)
    $$ 
    hvor $e_i$ er tuppelen med $0$ alle steder utenom 
    den $i$-ende index hvor det er en $1$.
    I Oppgave 1.4.2 viste vi at dette utgjør en basis 
    for $\mathbb K^n$. Siden $E$ har $n$ elementer 
    og utgjør en basis for $\mathbb K^n$ så har 
    $\mathbb K^n$ dimensjon $n$.
\end{proof}

\begin{proof}[Bevis av (b)]
    La $\mathbb K$ være en kropp og 
    $M_{m\times n}(\mathbb K)$ være rommet av 
    $m \times n$ matriser med koeffisienter 
    i $\mathbb K$.

    Vi ønsker å vise at $\text{dim}(M_{m\times n}(\mathbb K)) = mn$.
    
    La $\mathscr B$ være den kanoniske basisen 
    $$
        \mathscr B = \{E_{ij} : 1 \leq i \leq m, 1 \leq j \leq n\}
    $$
    hvor $E_{ij}$ er matrisen hvor med $1$ i posisjon $(i,j)$  
    og $0$ i alle de andre. Det er tydlig at 
    $|\mathscr B| = m \cdot n$, så hvis den utgjør en basis for 
    $M_{m \times n}(\mathbb K)$ kan vi konkludere at rommet 
    er $mn$-dimensjonelt.

    \emph{Lineær uavhengighet.}

    La $c_{ij} \in \mathbb K$, for alle $1 \leq i \leq m$ og 
    $1 \leq j \leq n$.
    \begin{align*}
        \displaystyle\sum_{i=1}^{m}\displaystyle\sum_{j=1}^{n}
        c_{ij}E_{ij} 
        = 
        \begin{pmatrix}
            c_{11} & \dots & c_{1n} \\
            \vdots & \ddots & \vdots \\
            c_{m1} & \dots & c_{mn}
        \end{pmatrix} 
        &= 
        \begin{pmatrix}
            0 & \dots & 0 \\
            \vdots & \ddots & \vdots \\
            0 & \dots & 0
        \end{pmatrix} 
        \\ 
        & 
        \text{\Large$\Updownarrow$}
        \\ 
        c_{11} = \dots = c_{mn} &= 0
    \end{align*}

    \emph{Spenner rommet.}

    La $A=(a_{ij}) \in M_{m \times n}(\mathbb K)$. 
    \begin{align*}
        A 
        &= 
        \begin{pmatrix}
            a_{11} & \dots & a_{1n} \\
            \vdots & \ddots & \vdots \\
            a_{m1} & \dots & a_{mn}
        \end{pmatrix}
        = 
        \displaystyle\sum_{i=1}^{m}\displaystyle\sum_{j=1}^{n}
        a_{ij}E_{ij}
    \end{align*}

    $\mathscr B$ utgjør en basis for rommet og inneholder 
    $m \cdot n$ elementer, derfor er 
    $M_{m \times n}(\mathbb K)$ $mn$-dimensjonelt.
\end{proof}

\newpage 
\begin{prob*}{Oppgave 1.8.3}{}
    La $U$ være et vektorrom. Bevis at følgende utsagn er ekvivalente: 
    
    \quad (a) dim$U = \infty$

    \quad (b) for enhver $n \in \mathbb N$ finnes et underrom 
    $U_n$ av $U$ med dimensjon $n$ 
\end{prob*}
\begin{proof}[Bevis]
    La $U$ være et vektorrom over $\mathbb K$. 

    Vi ønsker å vise at 
    $$
        \text{dim}(U) = \infty 
        \Leftrightarrow \exists U_n \text{ underrom av } U, 
        \quad
        \text{dim}(U_n) = n
    $$ 

    \emph{$\Rightarrow$}

    Basistilfelle: Velg $u_1 \in U, u_1 \neq 0$. 

    Anta at det er valgt $u_1,\dots,u_n$ lineært uavhengige. 
    $u_1,\dots,u_n$ kan ikke spenne $U$ siden da ville 
    ikke $\text{dim}U = \infty$.

    Velg så $u_{n+1} \in U\setminus\{\{\vec{0}\} 
    \cup \text{span}(u_1,\dots,u_n)\} \neq \emptyset$.
    Da er $(u_1,\dots,u_{n+1})$ lineært uavhengig.

    Det finnes da, for hver $n\in\mathbb N$ en lineært uavhengig 
    $(u_1,\dots,u_n)$. 

    La $U_n = \text{span}(u_1,\dots,u_n)$.
    $U_n$ er et $n$-dimensjonelt underrom av $U$.

    \emph{$\Leftarrow$}

    Anta at for enhver $n \in \mathbb N$, så finnes det et underrom 
    $U_n \subset U$ hvor $\text{dim}(U_n) = n$. 
    La oss anta, for en kontradiksjon, at det finnes en endelig 
    basis $(b_1, \dots, b_m)$. 
    Betrakt underrommet $U_{m+1}$ som nødvendighvis har 
    en basis $B'$ med $m+1$ elementer. $B'$ er, per definisjon, 
    lineært uavhengig, men vi vet at i et $m$-dimensjonelt 
    vektorrom kan det ikke eksistere en samling 
    med vektorer på størrelse mer en $m$, lineært uavhengig. 
    Så $B$ kan ikke utgjøre en basis for $U$. 
    Vår antagelse at det finnes en endelig basis for $U$ var feil 
    og derfor har vi at $\text{dim}(U) = \infty$.

    Siden vi har vist implikasjonen begge veier har 
    vi at de to utsagnene er ekvivalente.

\end{proof}

\end{document}
