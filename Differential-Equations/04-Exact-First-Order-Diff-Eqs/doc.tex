\documentclass[12pt]{article}
\usepackage{amsmath, amssymb, amsthm, amsfonts, geometry}
\newtheorem{theorem}{Theorem}
\newtheorem{lemma}{Lemma}
\newtheorem{proposition}{Proposition}
\newtheorem{corollary}{Corollary}
\newtheorem{definition}{Definition}

% Page Setup
\geometry{top=1in, bottom=1in, left=1in, right=1in}

\title{Lecture Notes: Professor Dave's Differential Equations  Exact First-Order Differential Equations}
\author{Thobias K. Høivik}
\date{\today}

\begin{document}

\maketitle
\noindent 
Suppose \(f: \mathbb R^2 \to \mathbb R\). The function 
can be plotted as a surface in \(\mathbb R^3\) 
where \(Z = f(x,y)\) is the height of the shape above the x-y-plane. 
If we take a horizonal slice of the surface at some height, then the intersection 
of a plane at that height with the shape will give a countour line.
\(Z = f(x,y)\) remains constant along the countour line. 
Suppose we move in the direction h on the countour line so that we are taken from 
\((x,y, f(x,y))\) to \((x+dx,y+dy,f(x,y))\). Then, clearly, the directional derivative 
\[ 
    \nabla f \cdot h = 
    \begin{bmatrix}
        \partial f / \partial x \\ 
        \partial f / \partial y
    \end{bmatrix}
    \cdot 
    \begin{bmatrix}
        dx \\ dy  
    \end{bmatrix}
    = 
    \frac{\partial f}{\partial x} dx + \frac{\partial f}{\partial y} dy = 0
\]
since there is no change in f. In other words, a step h on the contour line will have no change 
in height Z.
From here if we name the partial derivatives as functions we can get 
\[ 
    df = \frac{\partial f}{\partial x} dx + \frac{\partial f}{\partial y}dy = 0 
\]
\[ 
    M(x,y) dx + N(x,y) dy = 0
\]
If we get a differential equation in this form we can try to find the 
multivariable function it could have come from. Then, any contour line of \(f(x)\)
will then be a solution to the differential equation.
\[ 
    \frac{dy}{dx} = - \frac{M(x,y)}{N(x,y)}
\]
First we must check if the function f exists. 
\[ 
    \frac{\partial f}{\partial x} = M, \frac{\partial f}{\partial y} = N
\]
\[ 
    \Rightarrow \frac{\partial^2 f}{\partial x \partial y} = \frac{\partial M}{\partial y},
\]
\[
    \frac{\partial^2 f}{\partial y \partial x} = \frac{\partial N}{\partial x}
\]
By Clairaut's theorem: 
\[ 
    \frac{\partial^2 z}{\partial x \partial y} = \frac{\partial^2 z}{\partial y \partial x}
\]
Thus 
\[ 
    \frac{\partial M}{\partial y} = \frac{\partial N}{\partial x}
\]
must be satisfied. We call this the condition for exactness, when faced with a differential 
equation in the form above. F is called the potential function for the differential equation.

\noindent 
\textbf{Example: }

\noindent 
Consider the DE: 
\[ 
    \frac{dy}{dx} = \frac{x^2 - 4y^2}{8xy + y^4}
\]
\[ 
    \Rightarrow (4y^2 - x^2)dx + (8xy + y^4)dy = 0
\]
Now check for exactness: 
\[ 
    \frac{\partial }{\partial y}\left[ 4y^2 - x ^2\right] = 8y 
\]
\[ 
    \frac{\partial}{\partial x}\left[ 8xy + y^4\right] = 8y
\]
So we know we have an exact differential equation.
We know then that  
\[
    \exists F:\mathbb R^2 \to R : F(x,y) = \displaystyle\int (4y^2 - x^2)dx = 
    \displaystyle\int (8xy+y^4)dy
\]
\[ 
    \Rightarrow 4xy^2 - \frac{x^3}{3} + f(y) = 4xy^2 + \frac{y^5}{5}+g(x)
\]
\[ 
    -\frac{x^3}{3} + f(y) = \frac{y^5}{5} + g(x)
\]
Notice that if \(f(y) = \frac{y^5}{5}\) and \(g(x) = -\frac{x^3}{3}\) everything works out. 
Thus 
\[
    F(x,y) = 4xy^2 - \frac{x^3}{3} + \frac{y^5}{5}
\]
All contour lines of \(F\) will give a solution to the differential equation so 
an implicit general solution would be 
\[ 
    4xy^2 - \frac{x^3}{3}+\frac{y^5}{5} = C
\]


\end{document}
