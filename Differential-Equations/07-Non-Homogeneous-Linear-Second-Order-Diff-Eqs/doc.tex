\documentclass[12pt]{article}
\usepackage{amsmath, amssymb, amsthm, amsfonts, geometry}
\usepackage{float}
\newtheorem{theorem}{Theorem}
\newtheorem{lemma}{Lemma}
\newtheorem{proposition}{Proposition}
\newtheorem{corollary}{Corollary}
\newtheorem{definition}{Definition}

% Page Setup
\geometry{top=1in, bottom=1in, left=1in, right=1in}

\title{Lecture Notes: Linear Second-Order Differential Equations 
| Non-Homogeneous \\ (Course by: Professor Dave on YouTube)
}
\author{Thobias K. Høivik}
\date{\today}

\begin{document}

\maketitle
\[ 
    a\frac{d^2y}{dx^2} + b\frac{dy}{dx} + cy = f(x)
\]
Is non-homogeneous because instead of being equal to 0, it is 
equal to a function of x.
We start by finding the general solution to the equivalnent 
homogeneous case.
\[ 
    a\lambda^2 + b\lambda + c = 0
\]
This is known as the complementary solution.
Now we must find the \textbf{particular integral} (solution 
compatible with the right side \(f(x)\)).

\noindent 
Suppose we want to solve 
\[ 
    y'' - 5y' + 6y = e^x
\]
If we pretend that the left hand side equals 0 instead, we get a 
homogeneous differential equation with general solution 
\[ 
    y_{CF} = Ae^{2x}+Be^{3x}
\]

\break
\section*{Trial Functions}
We find a corresponding trial function from the table below and proceed.

\begin{table}[H]
\centering
\renewcommand{\arraystretch}{1.5}
\begin{tabular}{|c|c|}
\hline
\textbf{Form of \( f(x) \)} & \textbf{Trial Solution \( y_p(x) \)} \\
\hline
\( P_n(x) \) (polynomial of degree \( n \)) & \( A_0 + A_1 x + 
\cdots + A_n x^n \) \\
\hline
\( e^{\alpha x} \) & \( A e^{\alpha x} \) \\
\hline
\( \sin(\beta x),\ \cos(\beta x) \) & \( A \cos(\beta x) + B 
\sin(\beta x) \) \\
\hline
\( e^{\alpha x} \cdot P_n(x) \) & 
\( (A_0 + A_1 x + \cdots + A_n x^n) e^{\alpha x} \) \\
\hline
\( e^{\alpha x} \cdot \cos(\beta x) \) or 
\( e^{\alpha x} \cdot \sin(\beta x) \) & \( e^{\alpha x}
(A \cos(\beta x) + B \sin(\beta x)) \) \\
\hline
\( x^ne^{\alpha x} \sin(p x) \) or 
\( x^ne^{\alpha x}\cos(p x) \) &  
\( (C_nx^n + C_{n-1}x^{n-1} + \cdots + C_0)
(C_s\sin px + C_c \cos px) e^{\alpha x} \) \\
\hline
\end{tabular}
\caption{Choosing Trial Functions for \( y'' + ay' + by = f(x) \)}
\end{table}

\noindent 
We must choose a trial function from this table based on the 
form of \(f(x)\). We have \(f(x) = e^x\) so 
we get the trial function 
\[ 
    y_{PI}(x) = Ce^x
\]
The function must be linearly independent of our complementary function,
which this function is.
We then substitute in our trial function for \(y\) in the original differential 
equation to get: 
\[ 
    Ce^x - 5Ce^x + 6Ce^x = e^x \Rightarrow 2Ce^x = e^x \Rightarrow C=\frac{1}{2}
\]
\[ 
    y_{PI} = \frac{1}{2}e^x
\]
Our general solution then becomes 
\[ 
    y = y_{CF} + y_{PI}Ae^{2x} + Be^{3x} + \frac{1}{2} e^x
\]
If we find that our trial function is not linearly independent of the 
complementary solution we multiply the constant term by x.

\break
\section*{Variation of Parameters}
Trial functions is not the only way to find the solution to the differential 
equation.
We begin by calculating the \textbf{Wronskian}: 
\[ 
    W(x) = \det 
    \begin{vmatrix}
        y_1  & y_2 \\ 
        y_1' & y_2'
    \end{vmatrix}
\]
In the case of our earlier example we have 
\[ 
    W(x) = \det
    \begin{vmatrix} 
        e^{2x} & e^{3x} \\ 
        2e^{2x} & 3e^{3x}
    \end{vmatrix} 
    = 3e^{5x} - 2e^{5x} = e^{5x}
\]
To find the particular integral we use the formula 
\[ 
    y_{PI} = y_2 \displaystyle\int \frac{y_1 \cdot f(x)}{W(x)}dx - 
    y_1 \displaystyle\int \frac{y_2 \cdot f(x)}{W(x)}dx
\]
\[ 
    y_{PI} = e^{3x} \displaystyle\int \frac{e^{3x} \cdot e^x}{e^{5x}}dx - 
    e^{2x} \displaystyle\int \frac{e^{3x} \cdot e^x}{e^{5x}}dx
\]
\[ 
    = e^{3x}\left(-\frac{1}{2}e^{-2x}\right)-e^{2x}\left(-e^{-x}\right)
    = \frac{1}{2}e^{x}
\]
So we see that we get the same solution as before.

\end{document}
