\documentclass[12pt]{article}
\usepackage{amsmath, amssymb, amsthm, amsfonts, geometry}
\newtheorem{theorem}{Theorem}
\newtheorem{lemma}{Lemma}
\newtheorem{proposition}{Proposition}
\newtheorem{corollary}{Corollary}
\newtheorem{definition}{Definition}

% Page Setup
\geometry{top=1in, bottom=1in, left=1in, right=1in}

\title{Lecture Notes: Linear First-Order Differential Equations (Course By: Professor Dave Explains on Youtube)} 
\author{Thobias K. Høivik}
\date{\today}

\begin{document}

\maketitle

\noindent 
The general form of a linear first-order differential equation is 
\[ 
    \frac{dy}{dx} + P(x)y = Q(x)
\]

\noindent 
We introduce the integrating factor 
\[ 
    I(x) = e^{\int P(x) dx} = exp \displaystyle\int P(x) dx
\]

\noindent 
If we take the differential equation in it's general form and multiply all terms 
with the integrating factor we get: 
\[ 
    e^{\int P(x) dx} \frac{dy}{dx} + e^{\int P(x) dx}P(x)y = e^{\int P(x) dx}Q(x)
\]

\noindent 
Consider now the product rule of differentiation 
\[ 
    (f \cdot g)' = f \cdot g' + g \cdot f'
\]

\noindent 
Note how if this resembles the general form when multiplied by the integrating factor. 
Thus it can be replaced by: 
\[ 
    \frac{d}{dx}\left( e^{\int P(x) dx}y \right) = e^{\int P(x) dx}Q(x)
\]

\[ 
    e^{\int P(x) dx}y = \displaystyle\int e^{\int P(x) dx}Q(x)dx
\]
\[ 
    I(x)y = \displaystyle\int I(x)Q(x)dx
\]

\break 
\noindent 
\textbf{Example 1: }

\noindent 
Consider the differential equation: 
\[ 
    \frac{dy}{dx} + \frac{y}{x} = 3x
\]

\noindent 
It is not separable, but it is linear with an order of one.

\[ 
    I(x) = e^{\int \frac{1}{x} dx} = e^{\ln x} = x
\]
\[ 
    xy = \displaystyle\int x \cdot 3x dx = x^3 + C
\]
Giving us the general solution. Dividing by x yields the explicit general solution: 
\[ 
    \boxed{y = x^2 + \frac{C}{x}}
\]
 
\noindent 
\textbf{Example 2:}
Consider the differential equation, valid \(\forall x : 0 < x < \frac{\pi}{2}\), 
with initial condition \(y = \frac{1}{2\sqrt2}\)
when \(x = \frac{\pi}{4}\): 
\[ 
    \frac{dy}{dx} + y \cot x = \sin x
\]
\[ 
    I(x) = e^{\int \cot(x) dx} = e^{\ln (\sin x)} = \sin x
\]
\[ 
    y\sin x = \displaystyle\int \sin^2 x = \frac{x}{2} - \frac{\sin 2x}{4} + C 
\]
Initial conditions: 
\[
    \frac{1}{2\sqrt{2}}\frac{\sqrt2}{2} = \frac{pi}{8}-\frac{1}{4} + C
\]
\[ 
    \Rightarrow C = \frac{4 - \pi}{8}
\]
Explicit particular solution: 
\[ 
    \boxed{ y = \frac{1}{\sin x} 
    \left(  \frac{x}{2}-\frac{\sin 2x}{4}+\frac{4-\pi}{8}\right)}
\]



\end{document}
