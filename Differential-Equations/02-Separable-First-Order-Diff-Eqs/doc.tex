\documentclass[12pt]{article}
\usepackage{amsmath, amssymb, amsthm, amsfonts, geometry}
\newtheorem{theorem}{Theorem}
\newtheorem{lemma}{Lemma}
\newtheorem{proposition}{Proposition}
\newtheorem{corollary}{Corollary}
\newtheorem{definition}{Definition}

% Page Setup
\geometry{top=1in, bottom=1in, left=1in, right=1in}

\title{Professor Dave's Differential Equations}
\author{Thobias K. Høivik}
\date{\today}

\begin{document}

\maketitle
\section*{Separable First-Order Differential Equations}
This is the simplest type of differential equation. 
Separable differential equations can be written in the form 
\[ 
    g(y)dy = f(x)dx
\]
where y is a function of x, and f and g are general functions.
\[ 
    \frac{dy}{dx} = \frac{f(x)}{g(y)} \Rightarrow g(y)dy = f(x)dx
\]
\[ 
    \displaystyle\int g(y)dy = \displaystyle\int f(x)dx
\]

\subsection*{Example} 
Consider the differential equation 
\[ 
    \frac{dy}{dx} = \frac{1 + \sin x}{15y^4}
\]
cross multiply to get
\[ 
    15y^4 dy = (1+\sin x) dx
\]
\[ 
    \displaystyle\int15y^4 dy = \displaystyle\int(1+\sin x) dx
\]
\[ 
    3y^5 + C_1 = x - \cos x + C_2  
\]
\[ 
    3y^5 = x - \cos x + C, \quad C = C_2 - C_1
\]
an implicit general solution of the differential equation. 
The explicit general solution would be 
\[ 
    \boxed {y = \sqrt[5]{\frac{x-\cos x + C}{3}}}
\]
Assume now that we are given the boundary conditions \(x = 0 \land y(0)=0\).
\[ 
    3(0)^5 = 0 - \cos 0 + C \Rightarrow 0 = -1 + C \Rightarrow C = 1
\]
\[ 
    \boxed {y = \sqrt[5]{\frac{x-\cos x + 1}{3}}}
\]
If we were given the boundary condition earlier we could instead 
\[ 
    \displaystyle\int_{0}^{y}15y^4 dy = \displaystyle\int_{0}^{x} (1 + \sin x)dx
\]
where the lower bound is the boundary of the variable in the upper bound. 
This works well when boundary conditions are simple. 
 
\subsection*{Example} 
We are given the DE 
\[ 
    \frac{dy}{dx} = \frac{2x}{y(1+x^2)} \text{, } y = 0 \text{ when } x = 1
\]
\begin{gather*}
    y dy = \frac{2x}{1+x^2} dx \\ 
    \displaystyle\int_{0}^{y} y dy = \displaystyle\int_1^x \frac{2x}{1+x^2} dx \\ 
    \text{Let } u = 1+x^2 \Rightarrow du = 2xdx \\ 
    \Rightarrow \displaystyle\int_{2}^{1+x^2} \frac{du}{u} = \ln(1+x^2)-\ln 2 \\
    \frac{y^2}{2} = \ln \frac{1+x^2}{2} \\ 
    \boxed {y^2 = 2\ln \frac{1+x^2}{2} }
\end{gather*}


\end{document}
