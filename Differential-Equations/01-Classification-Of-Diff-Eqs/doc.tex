\documentclass[12pt]{article}
\usepackage{amsmath, amssymb, amsfonts, geometry}

% Page Setup
\geometry{top=1in, bottom=1in, left=1in, right=1in}

\title{Professor Dave's Differential Equations}
\author{Thobias K. Høivik}
\date{\today}

\begin{document}

\maketitle
\section*{Classification of Differential Equations}
Algebraic functions such as \(x^2 + 4x + 4 = 0\) have an unknown variable which
we are solving for. 
Differential equations such as \(\frac{dy}{dx} = x^2\), on the other hand, 
have an unknown function.
When we find indefinite integrals, we are technically finding the function 
of a differential equation: \(\frac{dy}{dx} = x^2 
\Rightarrow y = \displaystyle\int x^2 dx = \frac{x^3}{3} + C\). 

\subsection*{Terminology Pertaining to the Classification of Differential Equations}
\begin{enumerate}
    \item 
    y is the dependent variable and x the independent variable in the differential 
    equation above.

    \item
    \textbf{ODE:} An ordinary differential equation 
    contains only \textbf{one} independent variable, 
    for instance: \[\frac{dy}{dx} + xy = \tan x\]

    \item 
    \textbf{PDE:}
    A partial differential equation contains more than one independent variable, 
    for instance: 
    \[\frac{\partial y}{\partial x} - 2\frac{\partial y}{\partial t} = t^2 y\]

    \item 
    Linear differential equations are DE's where the dependent variables, and it's 
    derivatives, are alone (no exponent or function acting on y or it's derivatives): 
    \[\frac{dx}{dy} + y\sin x = 0\]

    \item Nonlinear differential equations are DE's not explicitly satisfying 
    the conditions that make a DE linear: 
    \[\frac{dy}{dx}+y^2 = 0\]

    \item 
    A homogenous differntial equation is a DE where there are no terms with just the 
    independent variable or constant: 
    \[ 
        \frac{dy}{dx} + y = 0
    \]

    \item 
    A nonhomegnous differential equation has terms with just the independent variable 
    or a constant: 
    \[ \frac{dy}{dx} + y = x \lor \frac{dy}{dx} + y = 1 \]

\item 
    The order of a DE is the highest derivative present: 
    \[\frac{d^2y}{dx^2} + \frac{dy}{dx} = x\]
    has order = 2. 

\item 
    The degree is the exponent on the highest derivative present: 
    \[\left(\frac{d^2y}{dx^2} \right)^3 + \frac{dy}{dx} = x\]
    would have order = 2 and degree = 3.

\item 
    Autonomous DE's are DE's where the independent variable only appears as a derivative:
    \[\frac{dy}{dx} + y = 0\]
    has no term with x other than the derivative. 

\item 
    Nonautonomous have the independent variable appearing in other terms: 
    \[\frac{dy}{dx} + xy = 0\]
\end{enumerate}

\subsection*{Solving Differential Equations} 
We want to solve for the dependent variable y. Getting y as a function of x would be 
an \textbf{explicit solution}.

\noindent 
If we can't express y in terms of x we would call it an \textbf{implicit solution}.

\noindent 
Consider \[y = \frac{x^3}{3} + C\] the solution to a differential equation. 
This solution is what we would call a general solution as it contains the constant 
term \(+ C\). Boundary conditions tell us information about the solution.  
A common type of boundary condition is an initial condition, for example 
given \(y(0) = 1\) we could find the value for \(C\) in the genral solution above.
This number of boundary conditions required to reach a particular solution is 
equal to the order of the DE. First derivative yields one C term, second two C terms 
and so on.
 
\end{document}
