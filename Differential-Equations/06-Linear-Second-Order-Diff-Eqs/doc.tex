\documentclass[12pt]{article}
\usepackage{amsmath, amssymb, amsthm, amsfonts, geometry}
\newtheorem{theorem}{Theorem}
\newtheorem{lemma}{Lemma}
\newtheorem{proposition}{Proposition}
\newtheorem{corollary}{Corollary}
\newtheorem{definition}{Definition}

% Page Setup
\geometry{top=1in, bottom=1in, left=1in, right=1in}

\title{Lecture Notes: Professor Dave's Differential Equations \\ Linear Second-Order Differential Equations \\ Homogeneous Case}
\author{Thobias K. Høivik}
\date{\today}

\begin{document}

\maketitle
\section*{Second-Order Differential Equations}
\[ 
    \frac{d^2y}{dx^2} + \frac{dy}{dx} = x 
\]
Second-order differential equations are differential equations with a second derivative 
in them. They are very commonm in physics. For instance: 
\[ 
    F = ma \Leftrightarrow m\frac{d^2x}{dt^2} = \displaystyle\sum F
\]

\section*{Linear Homogeneous Second-Order Differential Equations}
General form: 
\[ 
    \frac{d^2y}{dx^2} + P(x)\frac{dy}{dx} + Q(x)y = 0
\]
Constant coefficients case: 
\[ 
    a\frac{d^2y}{dx^2} + b\frac{dy}{dx} + cy = 0
\]
Characteristic/auxiliary equation: 
\[ 
    a\lambda^2 + b\lambda + c = 0
\]
where we can solve for \(\lambda\) using the quadratic equation.
Ansatz solution: 
\[ 
    e^{\lambda x}
\]
\noindent 
\textbf{Superposition principle: }

\noindent 
Given linearly independent solutions: \(y_ 1\) and \(y_2\) 
yields a general solution: \(y = Ay_1 + By_2\)

\noindent 
Now, if we return to our characteristic roots \(\lambda_1, \lambda_2\) we have the following
possibilities: 
\begin{enumerate}
    \item Values of lambda are real and distinct:
        \noindent 
        Ansatz solutions: \[e^{\lambda_1 x}, e^{\lambda_2 x}\] (linearly independent)
        \(\Rightarrow\) general solution: \[y = Ae^{\lambda_1x} + Be^{\lambda_2x}\]

    \item Values of lambda are complex conjugates; \(\lambda_{1,2} = \alpha \pm \beta i\). 
        Then 
        \[ 
            y = e^{\alpha x}\cos\beta x + e^{\alpha x}\sin \beta x
        \]
        yielding, via the superposition principle, the general solution: 
        \[ 
            y = e^{\alpha x}(A\cos \beta X + B \sin \beta x)
        \]

    \item Lambda is a single repeated real root. 
        \[ 
            y = Ae^{\lambda x} + B x e^{\lambda x} = e^{\lambda x} (A + Bx)
        \]
\end{enumerate}
\subsection*{Boundary conditions}
\textbf{Initial value problem:} values of \(y,dy/dx,y'\) at \(x\). 

\noindent 
\textbf{Boundary value problem:} two different values of y at two different values 
of x.

\section*{Application of Second-Order Differential Equations}
Suppose we have a trolley with mass \(m\) \(kg\) moving on frictionless horizontal rails. 
Attatched to the side the trolley is a spring with stiffness \(k\) \(N/m\), connecting the trolley to a wall, 
as well as a dashpot with a damping rate of \(c\) \((N\cdot s)/m\). 
We know the following laws hold in this system: 
\begin{gather*}
    F = -kx \text{ /Hooke's law} \\ 
    F = -cv = -c\frac{dx}{dt} \text{ /Dashpot} \\ 
    \displaystyle\sum F = ma = m\frac{d^2x}{dt^2} \text{ /Newton's second law}
\end{gather*}
The spring- and dashpot forces constitute the net force on the system so we know 
that 
\[ 
    \displaystyle\sum F = -kx - c\frac{dx}{dt}
\]
\[ 
    ma - \displaystyle\sum F = 0 
\]
\[ 
    m\frac{d^2x}{dt^2} + c\frac{dx}{dt} - kx = 0
\] 

\end{document}
