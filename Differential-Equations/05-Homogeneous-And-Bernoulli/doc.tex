\documentclass[12pt]{article}
\usepackage{amsmath, amssymb, amsthm, amsfonts, geometry}
\newtheorem{theorem}{Theorem}
\newtheorem{lemma}{Lemma}
\newtheorem{proposition}{Proposition}
\newtheorem{corollary}{Corollary}
\newtheorem{definition}{Definition}

% Page Setup
\geometry{top=1in, bottom=1in, left=1in, right=1in}

\title{Lecture Notes: Professor Dave's Differential Equations  Homogeneous Differential Equations and Bernoulli Differential Equations}
\author{Thobias K. Høivik}
\date{\today}

\begin{document}

\maketitle
\section*{Homogeneous Functions}
\begin{definition}    
    A function \(f(a,b)\) is said to be homogeneous if 
    scaling the arguments by some parameter \(t\)
    results in f being scaled by some exponent of 
    \(t\): \(f(ta,tb) = t^nf(a,b)\). 
\end{definition}

\noindent 
\textbf{Example:}
\(f(a,b)=a^2+ab\) is homogeneous because 
\(f(ta,tb) = (ta)^2 + (ta)(tb) = t^2a^2+t^2ab = t^2f(a,b)\).
\begin{definition}
    The order of the homogeneous function is the constant exponent \(n\)
    which \(t\) is raised to: \(f(ta,tb) = t^nf(a,b)\).
\end{definition}

\noindent 
So in the example above the order is \(2\).

\section*{Homogeneous Differential Equations}
\[ 
    f(x,y)dx = g(x,y)dy
\]
\(f,g\) are homogeneous of the same order. 
When we encounter homogeneous differential equation like this 
we can use a substitution \(y = ux\) or \(u = y/x\). Functions 
that can be written in terms of \(\frac{dy}{dx} = f(\frac{y}{x})\)
are also in this form.

\noindent 
\textbf{Example:} 
\[ 
    \frac{dy}{dx} = \frac{x^2 + y^2}{xy}
\]
This function is non separable nor linear nor is it exact.
By cross-multiplying we get 
\[ 
    (x^2 + y^2)dx = (xy) dy
\]
Now we have a function in the form 
\[ 
    f(x,y) dx = g(x,y) dy 
\]
and since 
\[ 
    f(tx,ty) = ((tx)^2 + (ty)^2) = t^2(x^2 + y^2)   
\]
\[ 
    g(tx,ty) = (txty)dy = t^2(xy)dy 
\]
They are both homogeneous functions of order 2. 
Thus we have a homogeneous differential equation.
We could also manipulate the original expression as such: 
\[ 
    \frac{dy}{dx} = \frac{x^2+y^2}{xy} = \frac{x}{y}+\frac{y}{x} 
    = \left(\frac{y}{x}\right)^{-1} + \frac{y}{x} =
    f\left(\frac{y}{x}\right)
\]
to show that the function is a homogeneous differential equation.
We solve this differential equation by using the homogeneous 
substitution \(y = ux\).
\begin{gather*}
    \frac{dy}{dx} = \frac{x^2 + y^2}{xy} \\
    y = ux \\ 
    \frac{dy}{dx} = \frac{d}{dx}ux = x \cdot \frac{du}{dx} 
    + \frac{dx}{dx} \cdot u =
    x\frac{du}{dx} + u \\ 
    x\frac{du}{dx} + u = \frac{x^2 + (ux)^2}{x(ux)}
    = \frac{1}{u} + u \\ 
    x\frac{du}{dx} = \frac{1}{u} \\ 
    \frac{du}{dx} = \frac{1}{ux}
\end{gather*}
giving us a separable differential equation with \(u\) as the dependent 
variable, instead of \(y\).
\begin{gather*}
    (u)du = \frac{1}{x}dx \Rightarrow \displaystyle\int (u) du 
    = \displaystyle\int \frac{1}{x}dx \\ 
    \Rightarrow \frac{1}{2}u^2 + C_1 = \ln|x|+C_2 \\ 
    \frac{1}{2}u^2 = \ln|x| + C 
    = \ln|x| + \ln|e^C| = \ln|x| + \ln|k| = \ln|kx| \\ 
    u^2 = 2\ln|kx| \\ 
    y = ux \Leftrightarrow u = \frac{y}{x} \\ 
    \left(\frac{y}{x}\right)^2 = 2\ln|kx| \\ 
    y^2 = 2x^2\ln|kx|
\end{gather*}

\break
\section*{Bernoulli Differential Equations}
While homogeneous differential equations can be reduced to separable 
differential equations (via substitution), Bernoulli differential 
equations can be reduced to linear differential equations (via 
substituion).
Bernoulli differential equations can be written in the form:
\[ 
    \frac{dy}{dx} + P(x)y = Q(x)y^n
\]
where \(P,Q\) are functions and \(n\) is constant (making this a 
nonlinear DE).
The appropriate substitution for these types of differential 
equations is to let \(u = y^{1-n}\). 

\noindent 
\textbf{Example:}
\[ 
    \frac{dy}{dx} + x^5y = x^5y^7
\]
\[ 
    u=y^{1-n} = y^{-6} \Rightarrow y = u^{-1/6}
\]
if we differentiate both sides of the substitution with respect to x 
we get 
\begin{gather*}
    y = u^{-1/6} \\ 
    \frac{dy}{dx} = -\frac{1}{6} u^{-7/6} 
    \frac{du}{dx} \text{ /via chain- and power rule} 
\end{gather*}
We can then put this into our equation 
\begin{gather*}
    -\frac{1}{6}u^{-7/6}\frac{du}{dx} + x^5u^{-1/6} = 
    x^5u^{-7/6} \\ 
    \cdot -6u^{7/6} \\ 
    \frac{du}{dx}-6x^5u = -6x^5
\end{gather*}
yielding a linear differential equation.
\begin{gather*}
    \mu(x) = e^{\int -6x^5\,dx }= e^{-x^6} \\ 
    \mu(x)\frac{du}{dx} - 6x^5u \cdot \mu(x) = -6x^5 \mu(x) \\
    \frac{d}{dx}(u\cdot \mu(x)) = -6x^5 \mu(x) \\
    \int \frac{d}{dx}(u \cdot \mu(x))\,dx = \int -6x^5 \mu(x)\,dx \\
    u \cdot \mu(x) = \int -6x^5 e^{-x^6} \, dx \\
    \text{Let } w = -x^6 \Rightarrow dw = -6x^5\,dx \\
    u \cdot e^{-x^6} = \int e^w \, dw = e^w + C = e^{-x^6} + C \\
    u(x) = 1 + C e^{x^6} \\ 
    y^{-6} = 1 + Ce^{x^6} \\ 
    y = \left( 1 + Ce^{x^6} \right)^{-1/6}
\end{gather*}

\end{document}
