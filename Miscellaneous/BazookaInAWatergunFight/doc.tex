\documentclass[11pt]{article}
\usepackage[a4paper,margin=1in]{geometry}
\usepackage{fourier} % Fourier font
\usepackage{xcolor}
\usepackage{tikz}
\usepackage[most]{tcolorbox}
\usepackage{amsthm, amsmath, amssymb}
\usepackage{enumitem}
\usepackage{hyperref}
\usepackage[nameinlink,noabbrev]{cleveref}
\usepackage{titling} 

% Dark mode colors
\definecolor{bgcolor}{HTML}{FFFFFF}
\definecolor{textcolor}{HTML}{000000}
\definecolor{defcolor}{HTML}{E86873}
\definecolor{thmcolor}{HTML}{0A9396}
\definecolor{lemcolor}{HTML}{94D2BD}
\definecolor{corcolor}{HTML}{9B4AF7}
\definecolor{probcolor}{HTML}{EE9B00}
\definecolor{excolor}{HTML}{21E933}

% Background and text color
\pagecolor{bgcolor}
\color{textcolor}

% No paragraph indentation
\setlength{\parindent}{0pt}
\setlength{\parskip}{0.7em}

% Theorem box styles
\tcbset{
  enhanced,
  colback=bgcolor,
  colframe=thmcolor,
  coltext=white,
  coltitle=white,
  fonttitle=\bfseries,
  boxrule=0.7pt,
  left=1em,
  right=1em,
  top=0.7em,
  bottom=0.7em,
  before skip=10pt,
  after skip=10pt,
}

% Theorem environments with colored boxes
\newtcbtheorem[number within=section]{thm}{Theorem}{
  colframe=thmcolor, colback=thmcolor!15!bgcolor
}{thm} % The 'thm' here is the *prefix* for the label

\newtcbtheorem[number within=section]{defn}{Definition}{
  colframe=defcolor, colback=defcolor!15!bgcolor
}{def} % The 'def' here is the *prefix* for the label

\newtcbtheorem[number within=section]{lem}{Lemma}{
  colframe=lemcolor, colback=lemcolor!15!bgcolor
}{lem}

\newtcbtheorem[number within=section]{cor}{Corollary}{
  colframe=corcolor, colback=corcolor!15!bgcolor
}{cor}

\newtcbtheorem[number within=section]{prob}{Problem}{
  colframe=probcolor, colback=probcolor!15!bgcolor
}{prob}

\newtcbtheorem[number within=section]{ex}{Example}{
  colframe=excolor, colback=excolor!15!bgcolor
}{ex}

% Proof environment 
\renewenvironment{proof}[1][\proofname]{%
  \par\pushQED{\qed}\normalfont\topsep6pt \trivlist
  \item[\hskip\labelsep\itshape #1.]\ignorespaces
}{%
  \popQED\endtrivlist\addvspace{6pt}
}

% Cleveref name formats for tcolorbox environments
\crefname{thm}{theorem}{theorems}
\Crefname{thm}{Theorem}{Theorems}

\crefname{def}{definition}{definitions}
\Crefname{def}{Definition}{Definitions}

\crefname{lem}{lemma}{lemmas}
\Crefname{lem}{Lemma}{Lemmas}

\crefname{cor}{corollary}{corollaries}
\Crefname{cor}{Corollary}{Corollaries}

\crefname{prob}{problem}{problems}
\Crefname{prob}{Problem}{Problems}

\crefname{ex}{example}{examples}
\Crefname{ex}{Example}{Examples}


\title{\huge{Killing an ant with an ICBM}}
\author{\LARGE{Thobias Høivik}}
\date{\Large{\today}}

\begin{document}
\maketitle

\newpage
The following is a proof that the $n$-th root of $2$ 
is irrational for $n \geq 2$, inspired by a similar 
proof that $\sqrt[3]2$ is irrational I saw on YT shorts.

\begin{proof}
    Let $n \geq 2$. Then 
    $$ 
        \sqrt[n]2 \text{ is irrational }
    $$

    \textbf{Case 1 $(n = 2)$}.

    We proceed in the usual way. Assume, for a contradiction, 
    that the square-root of $2$ is rational. 
    \begin{align*}
        \sqrt 2 &= \frac{a}{b} \\ \text{ where } a,b &\in \mathbb Z, 
        \text{ and } gcd(a,b) = 1\\ 
        2 &= \frac{a^2}{b^2} \\ 
        2b^2 &= a^2 \\ 
        \Rightarrow a &= 2k, k \in \mathbb Z \\ 
        2b^2 &= 4k^2 \\ 
        b^2 &= 2k^2 \\ 
        \Rightarrow b &= 2j, j\in\mathbb Z
    \end{align*}
    Now we see that $a$ and $b$ share a factor larger than $1$, 
    contradicting 
    our assumption. Hence $\sqrt 2$ cannot be a rational 
    number. 

    \textbf{Case 2 $(n > 2)$}. 

    Proceed as above, assuming the $n$-th root is rational. 
    \begin{align*}
        \sqrt[n]2 &= \frac{a}{b}, a,b\in \mathbb Z \\ 
        2 &= \frac{a^n}{b^n} \\ 
        2b^n &= a^n \\ 
        b^n + b^n &= a^n
    \end{align*}

    Which has no non-zero integer solutions 
    (which we would require since $b\neq0$) 
    by Fermat's 
    Last Theorem \cite{Wiles95}, a 
    contradiction. 
    Thus $\sqrt[n]2$ cannot be rational.
     
\end{proof}

\newpage
\begin{thebibliography}{9}

\bibitem{Wiles95} Wiles, A. (1995). Modular elliptic curves and Fermat's Last Theorem. \textit{Annals of Mathematics}, 141(3), 443--551.

\end{thebibliography}

\end{document}
