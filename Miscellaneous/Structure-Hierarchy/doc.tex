\documentclass{article}
\usepackage{amsmath, amssymb, amsthm}
\usepackage{hyperref}

\title{Mathematical Structures and Their Applications}
\author{}
\date{}

\begin{document}

\maketitle

\section{Introduction}
Mathematical structures are sets equipped with additional operations, 
relations, or properties that enable mathematical 
reasoning within different fields. 
This document provides an overview of various structures, 
their hierarchical relationships, how they extend or combine axioms, 
and their applications across mathematics and related disciplines.

\section{Hierarchy of Mathematical Structures}
Below is an outline of major mathematical structures 
and their interrelations.

\subsection{Sets and Basic Structures}
A set is the most fundamental mathematical structure, 
serving as a foundation for all others. From sets, 
additional structures arise by introducing operations or relations:
\begin{itemize}
    \item \textbf{Set}: \\ 
        A collection of elements.
    \item \textbf{Binary Operation}:  \\
        A function \( *: S \times S \to S \) that combines 
        elements of a set.
    \item \textbf{Relations}: \\ 
        A subset of \( S \times S \) defining order, 
        equivalence, or other properties.
\end{itemize}

\subsection{Algebraic Structures}
Structures with one or more binary operations:
\begin{itemize}
    \item \textbf{Semigroup}: \\
        A set with an associative binary operation.
    \item \textbf{Monoid}: \\
        A semigroup with an identity element.
    \item \textbf{Group}: \\
        A monoid where every element has an inverse.
    \item \textbf{Abelian Group}:\\ 
        A group where the binary operation is commutative.
    \item \textbf{Ring}: \\
        A set with two operations (addition and multiplication) 
        satisfying distributive laws.
    \item \textbf{Commutative Ring}:\\ 
        A ring where multiplication is commutative.
    \item \textbf{Field}: \\
        A commutative ring where every nonzero 
        element has a multiplicative inverse.
    \item \textbf{Module}: \\
        A generalization of a vector space where 
        scalars come from a ring.
    \item \textbf{Vector Space}: \\ 
        A module where scalars form a field.
    \item \textbf{Algebra}: \\
        A vector space with a bilinear multiplication operation.
\end{itemize}

\subsection{Topological and Geometric Structures}
Structures that introduce notions of continuity and distance:
\begin{itemize}
    \item \textbf{Topological Space}: \\
        A set with a collection of open sets 
        satisfying closure properties.
    \item \textbf{Metric Space}: \\
        A topological space with a metric defining distances.
    \item \textbf{Normed Vector Space}: \\
        A vector space with a norm inducing a metric.
    \item \textbf{Banach Space}: \\
        A complete normed vector space.
    \item \textbf{Hilbert Space}: \\
        A Banach space with an inner product.
    \item \textbf{Manifold}: \\
        A topological space locally resembling Euclidean space.
    \item \textbf{Lie Group}: \\
        A group that is also a differentiable manifold.
\end{itemize}

\subsection{Analytic and Functional Structures}
Spaces of functions with additional structure:
\begin{itemize}
    \item \textbf{Function Space}: \\
        A set of functions forming a space with 
        algebraic or topological properties.
    \item \textbf{$C^\infty$ Space}: \\
        The space of infinitely differentiable functions.
    \item \textbf{$L^p$ Spaces}: \\
        Function spaces with integrability constraints.
    \item \textbf{Sobolev Space}: \\
        Functions with weak derivatives satisfying 
        certain conditions.
\end{itemize}

\subsection{Measure and Probability Structures}
Structures based on measures and probability:
\begin{itemize}
    \item \textbf{Measure Space}: \\
        A set equipped with a measure function.
    \item \textbf{Probability Space}:\\ 
        A measure space where total measure is 1.
    \item \textbf{Sigma-Algebra}: \\
        A collection of subsets closed under countable 
        unions and complements.
\end{itemize}

\section{Extension and Combination of Structures}
Many mathematical structures extend or combine axioms from others:
\begin{itemize}
    \item \textbf{Rings Extend Groups}: \\  
        A ring extends an abelian group by adding a second operation.
    \item \textbf{Fields Extend Rings}: \\ 
        Fields introduce multiplicative inverses for nonzero elements.
    \item \textbf{Vector Spaces Extend Modules}: \\  
        Vector spaces restrict scalars to a field.
    \item \textbf{Normed Spaces Extend Vector Spaces}: \\ 
        Norm introduces a topology.
    \item \textbf{Hilbert Spaces Extend Banach Spaces}: \\ 
        Inner product structures enable geometric reasoning.
    \item \textbf{Lie Groups Combine Groups and Manifolds}: \\
        Differentiable structure adds smooth transformations.
\end{itemize}

\section{Applications of Mathematical Structures}
Different structures appear in various branches of 
mathematics and applied sciences:
\begin{itemize}
    \item \textbf{Group Theory}: \\  
        Symmetry in algebra, physics, and cryptography.
    \item \textbf{Ring and Field Theory}: \\  
        Number theory, coding theory, algebraic geometry.
    \item \textbf{Topology}: \\ 
        Continuity in analysis, physics, and differential geometry.
    \item \textbf{Functional Analysis}: \\  
        Quantum mechanics, PDEs, and approximation theory.
    \item \textbf{Measure and Probability}: \\  
        Statistics, machine learning, stochastic processes.
\end{itemize}

\section{Conclusion}
Mathematical structures form a rich hierarchy, 
with many extensions and combinations of axioms 
leading to new insights. Their applications span pure mathematics, 
theoretical physics, and computational sciences.

\end{document}

