\documentclass[11pt]{article}
\usepackage[a4paper,margin=1in]{geometry}
\usepackage{fourier} % Fourier font
\usepackage{xcolor}
\usepackage{tikz}
\usepackage[most]{tcolorbox}
\usepackage{amsthm, amsmath, amssymb}
\usepackage{enumitem}
\usepackage{hyperref}
\usepackage[nameinlink,noabbrev]{cleveref}
\usepackage{titling} 

% Dark mode colors
\definecolor{bgcolor}{HTML}{FFFFFF}
\definecolor{textcolor}{HTML}{000000}
\definecolor{defcolor}{HTML}{E86873}
\definecolor{thmcolor}{HTML}{0A9396}
\definecolor{lemcolor}{HTML}{94D2BD}
\definecolor{corcolor}{HTML}{9B4AF7}
\definecolor{probcolor}{HTML}{EE9B00}
\definecolor{excolor}{HTML}{21E933}

% Background and text color
\pagecolor{bgcolor}
\color{textcolor}

% No paragraph indentation
\setlength{\parindent}{0pt}
\setlength{\parskip}{0.7em}

% Theorem box styles
\tcbset{
  enhanced,
  colback=bgcolor,
  colframe=thmcolor,
  coltext=white,
  coltitle=white,
  fonttitle=\bfseries,
  boxrule=0.7pt,
  left=1em,
  right=1em,
  top=0.7em,
  bottom=0.7em,
  before skip=10pt,
  after skip=10pt,
}

% Theorem environments with colored boxes
\newtcbtheorem[number within=section]{thm}{Theorem}{
  colframe=thmcolor, colback=thmcolor!15!bgcolor
}{thm} % The 'thm' here is the *prefix* for the label

\newtcbtheorem[number within=section]{defn}{Definition}{
  colframe=defcolor, colback=defcolor!15!bgcolor
}{def} % The 'def' here is the *prefix* for the label

\newtcbtheorem[number within=section]{lem}{Lemma}{
  colframe=lemcolor, colback=lemcolor!15!bgcolor
}{lem}

\newtcbtheorem[number within=section]{cor}{Corollary}{
  colframe=corcolor, colback=corcolor!15!bgcolor
}{cor}

\newtcbtheorem[number within=section]{prob}{Problem}{
  colframe=probcolor, colback=probcolor!15!bgcolor
}{prob}

\newtcbtheorem[number within=section]{ex}{Example}{
  colframe=excolor, colback=excolor!15!bgcolor
}{ex}

% Proof environment 
\renewenvironment{proof}[1][\proofname]{%
  \par\pushQED{\qed}\normalfont\topsep6pt \trivlist
  \item[\hskip\labelsep\itshape #1.]\ignorespaces
}{%
  \popQED\endtrivlist\addvspace{6pt}
}

% Cleveref name formats for tcolorbox environments
\crefname{thm}{theorem}{theorems}
\Crefname{thm}{Theorem}{Theorems}

\crefname{def}{definition}{definitions}
\Crefname{def}{Definition}{Definitions}

\crefname{lem}{lemma}{lemmas}
\Crefname{lem}{Lemma}{Lemmas}

\crefname{cor}{corollary}{corollaries}
\Crefname{cor}{Corollary}{Corollaries}

\crefname{prob}{problem}{problems}
\Crefname{prob}{Problem}{Problems}

\crefname{ex}{example}{examples}
\Crefname{ex}{Example}{Examples}


\title{\huge{Shift Space of First Order Language Symbols}}
\author{\LARGE{Thobias Høivik}}
\date{\Large{\today}}

\begin{document}
\maketitle


\newpage
\section{Cool Beans}
\begin{defn}{First Order Language}{}
    A first-order language $\mathfrak L$ is a set of symbols 
    composed of two disjoint subsets: 
    \begin{enumerate}
        \item 
        The first part, which is common to all languages, 
        consists of "(" and ")" together with the following 
        symbols: 
        the variables $\mathcal V = \{v_n : n \in \mathbb N\}$, 
        the equality 
        symbol "$=$", connectives "$\lnot$", "$\land$" and the 
        existential quantifier "$\exists$".

        \item 
        The non-logical symbols, consisting of 
        \begin{itemize}
            \item a set of constant symbols $\mathcal C^\mathcal L$.
            \item a sequence of sets \(\mathcal F^\mathfrak L_n\), 
                $n\in \mathbb Z^+$, where elements 
                of this set are called n-ary functions symbols.
            \item a sequence of sets $\mathcal R_n^\mathfrak L$, 
                $n\in \mathbb Z^+$, where elements of this set 
                are called n-ary relation symbols or predicates 
                depending on the context. 
        \end{itemize}
    \end{enumerate}
\end{defn}


\begin{defn}{The Language Alphabet}{}
    Let $\mathfrak L$ be a first-order language as defined above. 
    We define the alphabet $\mathcal A_\mathfrak L$ of the 
    dynamical system to be the complete set of symbols comprising 
    the language $\mathfrak L$:
    $$\mathcal A_\mathfrak L = \text{Set of all symbols in } 
    \mathfrak L$$
    Since $\mathfrak L$ contains a countably infinite set of 
    variables $\mathcal V$ (and potentially infinite sets of 
    non-logical symbols), the alphabet $\mathcal A_\mathfrak L$ 
    is countably infinite. We equip $\mathcal A_\mathfrak L$ 
    with the discrete topology $\tau_{\text{disc}}$.
\end{defn}

\begin{defn}{The Full Shift Space over $\mathfrak L$}{}
    The full shift space over the first-order language 
    $\mathfrak L$, denoted $\Sigma_\mathfrak L$, is the set of all two-sided infinite sequences (or words) formed by symbols from 
    $\mathcal A_\mathfrak L$:
    $$\Sigma_\mathfrak L = \mathcal A_\mathfrak L^{\mathbb Z} = 
    \{ s = (\dots, s_{-1}, s_0, s_1, \dots) : 
    s_i \in \mathcal A_\mathfrak L \text{ for all } i 
    \in \mathbb Z \}$$
    We equip $\Sigma_\mathfrak L$ with the product topology, 
    where a basis for the open sets is given by the 
    cylinder sets $C_W = \{ s \in \Sigma_\mathfrak L : 
    s_{[i, i+k-1]} = W \}$ for any finite word (block) $W$ of 
    length $k$ occurring at position $i$.
\end{defn}

\begin{defn}{The Shift Map}{}
    The shift map (or shift automorphism) $\sigma: 
    \Sigma_\mathfrak L \to \Sigma_\mathfrak L$ is the function 
    defined by shifting every sequence one position to the left:
    $$\sigma(s)_i = s_{i+1} \quad \text{for all } s \in 
    \Sigma_\mathfrak L \text{ and } i \in \mathbb Z$$
    The pair $(\Sigma_\mathfrak L, \sigma)$ forms a 
    topological dynamical system. Since the alphabet 
    $\mathcal A_\mathfrak L$ is infinite, the space 
    $\Sigma_\mathfrak L$ is not compact, which contrasts 
    with classical symbolic dynamics over finite alphabets.
\end{defn}

\begin{defn}{The Subshift of Well-Formed Terms}{}
    A subshift $X \subseteq \Sigma_\mathfrak L$ is a closed and 
    $\sigma$-invariant subset of $\Sigma_\mathfrak L$. 
    We can define a specific subshift $X_{\text{Term}}$ by
    imposing a logical constraint:
    \begin{align*}
        &X_{\text{Term}} = \{ s \in \Sigma_\mathfrak L : 
        \text{no block in } s \\ &\text{ is a finite sequence of 
            symbols that violates the } 
            \mathfrak L \text{-grammar for a 
        well-formed term} \}
    \end{align*}
    This definition means that every finite word $W$ 
    appearing in a sequence $s \in X_{\text{Term}}$ must be a 
    $\mathfrak L$-grammatically valid string (not necessarily a 
    complete term, but one that doesn't contain a forbidden block 
    $\mathcal F$ that violates the rules of $\mathfrak L$-term 
    formation, such as $x(\land$). The set of forbidden blocks 
    $\mathcal F$ is typically context-free or higher in 
    complexity due to the recursive nature of term formation.
\end{defn}

\end{document}
