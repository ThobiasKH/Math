\documentclass[12pt]{article}
\usepackage{amsmath, amssymb, amsthm, amsfonts, geometry}
\newtheorem{theorem}{Theorem}
\newtheorem{lemma}{Lemma}
\newtheorem{proposition}{Proposition}
\newtheorem{corollary}{Corollary}
\newtheorem{definition}{Definition}

% Page Setup
\geometry{top=1in, bottom=1in, left=1in, right=1in}

\title{Lecture Notes: Real Analysis | Supremum and Infimum (From: Wrath of Math on Youtube)}
\author{Thobias K. Høivik}
\date{\today}

\begin{document}

\maketitle
\section*{Definitions}
\begin{definition}[Upper Bound]
    Let \( S \) be a subset of a field \( F \). 
    An element \( b \in F \) is called an \textit{upper bound} 
    of \( S \) if for all \( x \in S \), \( x \leq b \).
\end{definition}

\begin{definition}[Lower Bound]
    Let \( S \) be a subset of a field \( F \). 
    An element \( a \in F \) is called a \textit{lower bound} of 
    \( S \) if for all \( x \in S \), \( x \geq a \).
\end{definition}

\begin{definition}[Supremum]
    Let \(F\) be an ordered field and \(S \subseteq F\) be nonempty.
    The \textbf{supremum} of S, if it exists, is some \(b_0 \in F\) 
    such that 
    \begin{enumerate}
        \item \(b_0\) is an upper bound of S 
        \item if b is any other upper bound of S then \(b_0 \leq b\)
    \end{enumerate}
    If it exists, the supremum of S is denoted \(sup(S)\).
    
\end{definition}
\begin{definition}[Infimum]
    Let \(F\) be an ordered field and \(S \subseteq F\) be nonempty.
    The \textbf{infimum} of S, if it exists, is some \(b_0 \in F\) 
    such that 
    \begin{enumerate}
        \item \(b_0\) is a lower bound of S 
        \item if b is any other lower bound of S then \(b_0 \geq b\)
    \end{enumerate}
    If it exists, the infimum of S is denoted \(inf(S)\).
    
\end{definition}

\noindent 
We may also call the supremum and infimum the \textbf{least upper bound} and 
the \textbf{greates lower bound}, respectively.
As always when we say field we are usually refering to \(\mathbb R\) in real 
analysis as that is the premier field of interest in the subject.
If we take S to be a subset of the reals we may visualize S as an interval 
on the real number line. The supremum could then be thought of as 
the closest you could get to the interval from the right without being in it 
and the same is true for the infimum, but on the other side.

\section*{In Practice}
Observe the concept of Supremums and Infimums in these examples: 

\noindent 
Let \( S = \{1, 2, 3, 5, 8\} \subset \mathbb{R} \). The supremum of \( S \) is:
\[
\sup(S) = 8
\]
because \( 8 \) is the greatest element in the set and there are no elements in \( S \) greater than \( 8 \).

\noindent 
Let \( S = \{1, 2, 3, 5, 8\} \cup (10, 12) \subset \mathbb{R} \). The supremum of \( S \) is:
\[
\sup(S) = 12
\]
because \( 12 \) is the least upper bound of the set, even though \( 12 \) is not an element of the set (since \( S \) contains the interval \( (10, 12) \) but not \( 12 \)).

\noindent 
Let \( S = (0, 1) \subset \mathbb{R} \). The supremum of \( S \) is:
\[
\sup(S) = 1
\]
because \( 1 \) is the least upper bound of the set, even though \( 1 \) is not an element of the set.

\noindent 
Let \( S = [0, 1] \subset \mathbb{R} \). The supremum of \( S \) is:
\[
\sup(S) = 1
\]
because \( 1 \) is the greatest element of the set, and there is no element in \( S \) greater than \( 1 \).
\end{document}
