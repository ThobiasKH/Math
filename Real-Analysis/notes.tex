\documentclass[11pt]{article}
\usepackage[a4paper,margin=1in]{geometry}
\usepackage{fourier} % Fourier font
\usepackage{xcolor}
\usepackage{tikz}
\usepackage[most]{tcolorbox}
\usepackage{amsthm, amsmath, amssymb}
\usepackage{enumitem}
\usepackage{hyperref}
\usepackage[nameinlink,noabbrev]{cleveref}
\usepackage{titling} 

% Dark mode colors
\definecolor{bgcolor}{HTML}{FFFFFF}
\definecolor{textcolor}{HTML}{000000}
\definecolor{defcolor}{HTML}{E86873}
\definecolor{thmcolor}{HTML}{0A9396}
\definecolor{lemcolor}{HTML}{94D2BD}
\definecolor{corcolor}{HTML}{9B4AF7}
\definecolor{probcolor}{HTML}{EE9B00}
\definecolor{excolor}{HTML}{21E933}

% Background and text color
\pagecolor{bgcolor}
\color{textcolor}

% No paragraph indentation
\setlength{\parindent}{0pt}
\setlength{\parskip}{0.7em}

% Theorem box styles
\tcbset{
  enhanced,
  colback=bgcolor,
  colframe=thmcolor,
  coltext=white,
  coltitle=white,
  fonttitle=\bfseries,
  boxrule=0.7pt,
  left=1em,
  right=1em,
  top=0.7em,
  bottom=0.7em,
  before skip=10pt,
  after skip=10pt,
}

% Theorem environments with colored boxes
\newtcbtheorem[number within=section]{thm}{Theorem}{
  colframe=thmcolor, colback=thmcolor!15!bgcolor
}{thm} % The 'thm' here is the *prefix* for the label

\newtcbtheorem[number within=section]{defn}{Definition}{
  colframe=defcolor, colback=defcolor!15!bgcolor
}{def} % The 'def' here is the *prefix* for the label

\newtcbtheorem[number within=section]{lem}{Lemma}{
  colframe=lemcolor, colback=lemcolor!15!bgcolor
}{lem}

\newtcbtheorem[number within=section]{cor}{Corollary}{
  colframe=corcolor, colback=corcolor!15!bgcolor
}{cor}

\newtcbtheorem[number within=section]{prob}{Problem}{
  colframe=probcolor, colback=probcolor!15!bgcolor
}{prob}

\newtcbtheorem[number within=section]{ex}{Example}{
  colframe=excolor, colback=excolor!15!bgcolor
}{ex}

% Proof environment 
\renewenvironment{proof}[1][\proofname]{%
  \par\pushQED{\qed}\normalfont\topsep6pt \trivlist
  \item[\hskip\labelsep\itshape #1.]\ignorespaces
}{%
  \popQED\endtrivlist\addvspace{6pt}
}

% Cleveref name formats for tcolorbox environments
\crefname{thm}{theorem}{theorems}
\Crefname{thm}{Theorem}{Theorems}

\crefname{def}{definition}{definitions}
\Crefname{def}{Definition}{Definitions}

\crefname{lem}{lemma}{lemmas}
\Crefname{lem}{Lemma}{Lemmas}

\crefname{cor}{corollary}{corollaries}
\Crefname{cor}{Corollary}{Corollaries}

\crefname{prob}{problem}{problems}
\Crefname{prob}{Problem}{Problems}

\crefname{ex}{example}{examples}
\Crefname{ex}{Example}{Examples}


\title{\huge{Real Analysis}}
\author{\LARGE{Thobias Høivik}}
\date{\Large{Spring 2026}}

\begin{document}
\maketitle

\newpage
\tableofcontents

\newpage
\section{Introduction}
The following is intended for anyone who stumbles over these notes. 
This is intended to be my personal notes in real analysis. 
I will hopefully be attending MAT2400 Real Analysis at 
the University of Oslo in the spring of 2026. However, 
as I am not a program student at this institution, but 
just someone who takes individual courses there 
of my own volition, I do not and cannot attend lectures and 
therefore I have to learn the material on my own. 
As far as I understand, while this course is called Real Analysis, 
it is a bit different than a first course from what I understand. 
The earlier exams give a hint of functional analysis and 
also include topics such as fourier analysis, meassure- and integration
theory among other things. Thus the different supplementary 
coursematerial I will use to aid myself in learning the content 
of this course will most likely be a bit scattered and all over the 
place, which these notes will undoubtedly reflect. I will 
do my best to keep things organized for my own sake, but keep 
this in mind if you are someone who intends to use these 
notes to learn Real Analysis. 

\newpage 
\section{Basic Banch Space theory}
The following section of notes is derived from the first video 
in the lecture series MIT 18.102 Introduction to Functional 
Analysis, Spring 2021 (found on youtube). 

\begin{defn}[Vector Space]
    \label{defn:vector_space}
    A vector space $V$ over a field $\mathbb F$ is a nonempty 
    set of elements called "vectors" together with 
    a binary operation $+$ on $V$ and a binary function 
    $\cdot$ which maps elements of $V, \mathbb F$ to $V$ 
    satisfying: 

    \begin{enumerate}
        \item Associativity of vector addition: 
            $$ 
                u + (v+w) = (u+v)+w, \forall u,v,w \in V
            $$
        \item Commutativity of vector addition: 
            $$ 
                u + v = v + w, \forall u,v \in V
            $$ 
        \item Identity element: 
            $$ 
                \exists 0 \in V : v + 0 = 0 + v = v, \forall v \in V
            $$ 
        \item Each $v \in V$ has an inverse $-v$ under the 
            vector-addition operation.
        \item Scalar multiplication is compatible with field 
            multiplication: 
            $$ 
                a(bv) = (ab)v
            $$ 
            where $a,b \in \mathbb F$ and $v \in V$. 
        \item The multiplicative identity $1 \in \mathbb F$ 
            satisfies: 
            $$ 
                1v = v, \forall v \in V
            $$ 
        \item Distributivity of scalar multiplication with respect 
            to vector addition: 
            $$ 
                a(u+v) = au + av
            $$ 
            where $a \in \mathbb F$ and $u,v \in V$. 
        \item Distributivity of scalar multiplication with 
            respect to field addition: 
            $$ 
                (a+b)v = av + bv
            $$ 
            where $a,b \in \mathbb F$ and $v \in V$. 
    \end{enumerate}

    When proving that something is a vector space, most 
    of these follow naturally from showing closure under 
    addition and scalar multiplication and those two properties, 
    are generally enough to show that it is indeed a vector space. 

    A subspace $U$ of $V$ is a set $U \subseteq V$ which is also 
    a vector space. It is enough to show that $U \subseteq V$ and 
    that it is closed under the two operations. 
\end{defn}

\newpage 
Some typical examples of vector spaces are 
$\mathbb F^n$ where $\mathbb F$ is the reals or the complex 
numbers. We also have spaces like the space of real polynomials 
of degree $\leq n$, i.e. $\mathcal P_n = 
\{\sum_{i=0}^n \alpha_i x^i : \alpha_i \in \mathbb R\}$, 
which is itself a subspace of the space of continuous real-valued 
functions $C(\mathbb R)$. 

So $\mathbb R^2$ and $C(\mathbb R)$ 
are both vector spaces over $\mathbb R$, but 
they have one really big difference, that being the dimension. 

\begin{defn}
    \label{defn:linear_independence}
    Let $V$ be a vector space. A set $\{v_1,\ldots, v_n\} \subseteq V$
    is linearly independent if  
    $$ 
        \displaystyle\sum_{i=1}^n \alpha_i v_i = 0 
        \Leftrightarrow \alpha_1 = \dots = \alpha_n = 0 \in \mathbb F
    $$ 

    Note: the right-to-left direction of this implication is 
    always true. 
\end{defn}

The two spaces discussed above are different in dimension, 
$\mathbb R^2$ being $2$-dimensional and the other being 
infinite-dimensional. 
One definition of finite-dimensional is that every linearly 
independent set in the space is finite. 
I however like the definition using bases more. 
Both of these definitions are equivalent. 

We won't give a rigorous definition of a basis, but in short 
a basis of $V$ is a linearly independent set of vectors which spans 
$V$, i.e. every vector in $V$ can be expressed as a linear combination
of basis-vectors. If the basis is finite then $V$ 
is finite dimensional. Moreover, if the basis is finite then the 
dimension of $V$ is the number of basis-vectors. Note that 
if a finite dimensional space $V$ has a basis with $n$ elements 
then every basis of $V$ has $n$ elements. A space is  
infinite-dimensional if no finite set of linearly independent vectors 
spans the space. 

\newpage 
\subsection{Norms and Metrics}

\begin{defn}[Norm]
    \label{defn:norm}
    Let $V$ be a vector space. 
    A norm $|| \cdot || $ is a function 
    from $V \to [0,\infty)$ satisfying: 
    \begin{enumerate}
        \item $|| v || = 0$ if and only if 
            $v = 0$. 
        \item $|| \alpha v || = |\alpha| \cdot ||v||$ where 
            $\alpha$ is an element of the ground field. 
        \item $||v + w|| \leq ||v|| + ||w||$.
    \end{enumerate}
    
    The tuple $(V, ||\cdot||)$ is called a normed space. 
\end{defn}

\begin{ex}
    \label{ex:lp_norms}
    $||x||_2 = \sqrt{x_1^2 + x_2^2 + \cdots + x_n^2}$ defines 
    a norm on $\mathbb R^n$.

    In fact it constitutes a norm on $\mathbb C^2$ as well. 
    Formally, if we take $\mathbb F = \mathbb R$ or  
    $\mathbb F = \mathbb C$ the p-norm of a vector 
    $v \in \mathbb F^n$ ($p \in [1,\infty]$) is
    $$ 
        ||v||_p := 
        \begin{cases}
            \left(\displaystyle\sum_{i=1}^n |v_i|^p\right)^{1/p} & 
            p < \infty\\ 
            \max_{i=1,\dots,n}|v_i| & p = \infty
        \end{cases}
    $$ 
\end{ex}

Norms give us a notion of the "length" of a vector.  
Now all we need to do analysis on spaces is a notion of 
distance. Intuitively, norms already give us a notion of 
distance from $0$. 

\begin{defn}
    \label{defn:metric}
    Let $X$ be a set. 

    A metric is a function 
    $d:X^2 \to [0,\infty)$ satisfying: 
    \begin{enumerate}
        \item $d(x,y) = 0$ if and only if $x = y$. 
        \item $d(x,y) = d(y,x)$. 
        \item $d(x,z) \leq d(x,y) + d(y,z)$. 
    \end{enumerate}
\end{defn}

The metric gives us a notion of distance. 
In a typical first course in analysis where we work on 
the reals, $d(a,b) = |a - b|$ is the metric we deal with. 

\newpage 
\begin{prop}
    Let $V$ be a normed space with norm $||\cdot||$. 
    Then we can define the distance (a metric) between two vectors 
    by 
    $$ 
        d(x,y) := \left|\left| x - y \right|\right|
    $$ 

    In other words you can define a metric in terms of the norm 
    in any normed space. This metric is usually refered to 
    as the metric induced by the norm. 
\end{prop}

We won't provide a proof of this as it's fairly intuitive. 
Now we can get a sense of convergence and continuity in 
vector spaces by saying that a sequence $\{a_n\}_{n \in \mathbb N}$
converges to a value $a$ if 
$$ 
    \forall \varepsilon > 0, \exists N \in \mathbb N : 
    n \geq N \Rightarrow ||a_n - a || < \varepsilon 
$$ 
and a linear transformation (look up definition if necessary) 
$T \in \mathcal L(U,V)$ is (uniformly) continuous if 
$$ 
    \forall \varepsilon > 0, \exists \delta > 0 : 
    ||x - y||_U < \delta \Rightarrow 
    ||Tx - Ty||_V < \varepsilon
$$ 
for every $x,y \in U$. Notice that if you replace $||x-y||$ with 
$d(x,y)$ it looks like the standard definitions in terms 
of metric spaces. 

\newpage 
\subsection{Banach Spaces}
\begin{defn}[Banach Space]
    \label{defn:banach_space}
    A normed space $V$ is a Banach Space if it is 
    complete with respect to the metric induced by the norm, meaning
    that every cauchy sequence converges to a value in the 
    space. 
\end{defn}

\begin{ex} 
    $\mathbb R^n$ or $\mathbb C^n$ form Banach Spaces 
    with respect to the $\ell^p$ norms (see \Cref{ex:lp_norms}).
\end{ex}

\begin{thm}
    If $X$ is a complete metric space, then 
    $C_\infty(X)$ is a Banach Space. 
\end{thm}

Recall that $C_\infty(X)$ is the space of bounded continuous 
functions on $X$.

\begin{proof}
    We show that every Cauchy sequence in $C_\infty(X)$ converges 
    to an element
    of $C_\infty(X)$.

    Let $\{u_n\}_{n=1}^\infty \subseteq C_\infty(X)$ be a Cauchy 
    sequence with
    respect to the supremum norm. Then for every $\varepsilon > 0$ 
    there exists
    $N \in \mathbb{N}$ such that
    \[
        \|u_n - u_m\|_\infty < \varepsilon \quad \text{for all } 
        n,m \ge N.
    \]
    Equivalently,
    \[
        |u_n(x) - u_m(x)| < \varepsilon
        \quad \text{for all } x \in X \text{ and all } n,m \ge N.
    \]

    Fix $x \in X$. Then $\{u_n(x)\}_{n=1}^\infty$ is a 
    Cauchy sequence in
    $\mathbb{R}$ (or $\mathbb{C}$), since
    \[
        |u_n(x) - u_m(x)| \le \|u_n - u_m\|_\infty.
    \]
    Because $\mathbb{R}$ (or $\mathbb{C}$) is complete, the limit
    \[
        u(x) := \lim_{n\to\infty} u_n(x)
    \]
    exists. This defines a function $u : X \to \mathbb{R}$.

    We now show that $u_n \to u$ uniformly on $X$. Let 
    $\varepsilon > 0$ and
    choose $N$ such that $\|u_n - u_m\|_\infty < \varepsilon$ 
    for all $n,m \ge N$.
    Fix $n \ge N$ and $x \in X$. Taking the limit $m \to \infty$ gives
    \[
        |u_n(x) - u(x)|
        = \lim_{m\to\infty} |u_n(x) - u_m(x)|
        \le \varepsilon.
    \]
    Since $x \in X$ was arbitrary, it follows that
    \[
        \|u_n - u\|_\infty \le \varepsilon \quad \text{for all } 
        n \ge N.
    \]
    Thus $u_n \to u$ uniformly on $X$.

    Since each $u_n$ is bounded and the convergence is uniform, 
    the limit
    function $u$ is bounded. Moreover, since each $u_n$ 
    is continuous and
    uniform limits of continuous functions are continuous, 
    $u$ is continuous
    on $X$.

    Therefore $u \in C_\infty(X)$ and $\{u_n\}$ converges to $u$ in the
    supremum norm. Hence $C_\infty(X)$ is complete, 
    and thus a Banach space.
\end{proof}

\newpage 
\section{Exercises from 2.1 in the textbook}
\begin{prob}[Problem 1]
    Show that if $\{x_n\} \to a$ then the sequence $\{Mx_n\}$, 
    where $M$ is some constant, converges to $Ma$. 
\end{prob}
\begin{proof}
    Suppose $\{x_n\} \to a$, i.e. 
    $\forall \varepsilon_0 > 0, \exists N_0 \in \mathbb N$ such that 
    $$ 
        ||x_n - a|| < \varepsilon_0
    $$ 
    whenever $n \geq N_0$. 
    
    Let $\varepsilon > 0$ and pick some $N$ such that 
    $$ 
        ||x_n - a|| < \varepsilon/M
    $$ 

    Recall that such an $N$ can be found since $\varepsilon / M > 0$. 
    Then 
    \begin{align*}
        M||x_n-a|| &< \varepsilon \\ 
        ||M(x_n - a)|| &< \varepsilon \\ 
        ||Mx_n - Ma|| &< \varepsilon
    \end{align*}
    so $\{Mx_n\} \to Ma$. 
\end{proof}



\end{document}
