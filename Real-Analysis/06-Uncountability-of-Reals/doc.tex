\documentclass[12pt]{article}
\usepackage{amsmath, amssymb, amsthm, amsfonts, geometry}
\newtheorem{theorem}{Theorem}
\newtheorem{lemma}{Lemma}
\newtheorem{proposition}{Proposition}
\newtheorem{corollary}{Corollary}
\newtheorem{definition}{Definition}

% Page Setup
\geometry{top=1in, bottom=1in, left=1in, right=1in}

\title{Lecture Notes: Real Analysis | Uncountability of Real Numbers (Course: MIT 18.100A)}
\author{Thobias K. Høivik}
\date{\today}

\begin{document}

\maketitle
\section*{Proving Uncountability}
\begin{theorem}[Triangle Inequality]
    \(\forall x,y \in \mathbb R : |x+y| \leq |x| + |y|\) 
\end{theorem}
\begin{proof}
    By the definition of absolute value, for any \( x, y \in \mathbb{R} \), we have:
    \[
        -|x| \leq x \leq |x|
    \]
    \[
        -|y| \leq y \leq |y|
    \]
    
    Adding these inequalities together:
    \[
        -(|x| + |y|) \leq x + y \leq |x| + |y|
    \]
    
    By the definition of absolute value, this implies:
    \[
        |x+y| \leq |x| + |y|
    \]
    which proves the theorem.
\end{proof}
\begin{definition}
    Let \(x \in (0,1]\) and let \(d_{-j}\in \{0,1,\dots,9\}\) for \(j \in \mathbb N\).
    We say \(x\) is represented by the digits \(\{d_{-j} : j \in \mathbb N\}\), 
    \(x = 0.d_{-1}d_{-2}\dots\) if \(x = \sup\{d_{-1}10^{-1}+\dots 
    + d_{-n}10^{-n} | n \in \mathbb N\}\).
\end{definition}
\noindent
\textbf{Example:}
\[0.25000 = \sup\{\frac{2}{10},\frac{2}{10}+\frac{5}{100},
\frac{2}{10}+\frac{5}{100}+\frac{0}{1000},\dots\}\]
\[= \sup\{\frac{2}{10},\frac{25}{100}\} = \frac{1}{4}\]
\begin{theorem}
    \noindent
    \begin{enumerate}
        \item For every set of digits \(\{d_{-j} : j\in\mathbb N\}\) 
            with \(d_j \in \{0,1,2,\dots,9\}\), there exists a unique 
            \(x \in [0,1]\) such that \(x = 0.d_{-1}d_{-2}d_{-3}\dots\)

        \item \(\forall x \in (0,1], \exists! \{d_{-j} : j\in\mathbb N\} \)  
            such that \(x = 0.d_{-1}d_{-2}\dots\) and 
            \(0.d_{-1}\dots d_{-n} < x \leq 0.d_{-1}\dots d_{-n} + 10^{-n}\).
    \end{enumerate}
\end{theorem}
\begin{theorem}[Cantor]
    \((0,1]\) is uncountable. 
\end{theorem}
\begin{proof}
    Assume, by contradiction, that \(|(0,1]| = |\mathbb N|\). 
    Then \(\exists f: \mathbb N \to (0,1]\) which is bijective.
    \(\forall n\), write \(f(n) = 0.d_{-1}^nd_{-2}^n\dots\) satisfying 
    \(f(n) \leq 0.d_{-1}^nd_{-2}^n\dots + 10^{-n}\).

    \noindent
    Let \[e_ {-j} = 
    \begin{cases}
        1 \text{ if } d_{-j}^j \neq 1 \\
        2 \text{ if } d_{-j}^j = 1
    \end{cases}
    \]
    By 1) of the previous theorem \(\exists!y \in (0,1] \) such that 
    \(y = 0.e_{-1}e_{-2}\dots\). Since all \(e_{-j}\) are either \(1\) or \(2\), 
    they are non-zero.
    \(\forall n\in \mathbb N : 0.e_{-1}e_{-2}\dots e_{-n} < y \leq 0.e_{-1}
    \dots e_{-n} + 10^{-n}\). 
    \(y\) is then the unique decimal representation of this number from 2) in the 
    previous theorem. Since \(f\) is surjective, \(\exists m \in \mathbb N\)
    such that \(y = f(m)\). 

    \noindent
    Then 
    \[d_{-m}^m = e_{-m} = 
    \begin{cases}
        1 \text{ if } d_{-m}^m \neq 1 \\ 
        2 \text{ if } d_{-m}^m = 1
    \end{cases}
    \neq d_{-m}^m
    \] 
    Thus we have arrived at a contradiction so our assumption that \((0,1]\) is 
    countable is false.
\end{proof}

\section*{Sequences and Series}
\begin{definition}
    A sequence of real numbers is a function \(f: \mathbb N \to \mathbb R\). 
    We denote \(f(n)\) with \(X_n\) and the sequence by \(\{X_n\}_{n=1}^\infty\)
    or \(\{X_n\}\) or otherwise \(X_1,X_2,X_3,\dots\)

    \noindent 
    We may also write \(a_n\).
\end{definition}

\noindent 
\textbf{Example:} \(\{\frac{1}{n}\} = 1, \frac{1}{2}, \frac{1}{4}, \dots\)
\begin{definition}
    A sequence \(X_n\) is bounded if \(\exists b \geq 0\) such that 
    \(\forall n \in \mathbb N:\) \(|X_n| \leq B\).
\end{definition}

\noindent 
\textbf{Example:} \(\{\frac{1}{n}\}\) is bounded by 1 because it is never larger than 1 
for any natural number \(n\).

\noindent 
\textbf{Non-Example: } \(\{n\}\) is not bounded as it will just grow and grow 
for increasing n.
\begin{proof}
    Let \(B \geq 0\). By Archimedes Property \(\exists n \in \mathbb N : n > B\). 
    Therefore \(\{n\}\) is unbounded.
\end{proof}
\begin{definition}
    A sequence \(\{X_n\}\) converges to \(X \in \mathbb R\) if 
    \(\forall \mathcal E > 0, \exists M \in \mathbb N : 
    \forall n \geq M, \\ |X_n - X| < \mathcal E \).
    If a sequence converges we say it's convergent, otherwise it is divergent.
\end{definition}
\begin{theorem}
    If \(x,y \in \mathbb R\) and \(\forall \mathcal E > 0, |x-y| < \mathcal E \Rightarrow
    x = y\).
\end{theorem}
\begin{proof}
    Suppose \(x,y \in \mathbb R\) and \(\forall \mathcal E > 0, |x-y| < \mathcal E\), 
    \(x \neq y\). Then \(|x-y| > 0\). Then by \(|x-y| < \frac{|x-y|}{2} \Rightarrow 
    \frac{1}{2}|x-y| < 0 \Rightarrow |x-y| < 0\) which is not possible.
\end{proof}

\begin{theorem}
    If \(\{X_n\}\) converges to \(x\) and \(y\), then \(x = y\).
\end{theorem}
\begin{proof}
    Let \((a_n)\) be a sequence such that:
    \[
        a_n \to x \quad \text{and} \quad a_n \to y
    \]
    By the definition of convergence, for any \(\epsilon > 0\), there exists a positive integer \(N_1\) such that for all \(n \geq N_1\):

    \[
        |a_n - x| < \epsilon
    \]

    \noindent
    Similarly, since \(a_n \to y\), for the same \(\epsilon > 0\), there exists a positive integer \(N_2\) such that for all \(n \geq N_2\):

    \[
        |a_n - y| < \epsilon
    \]
\noindent 
    Let \(N = \max(N_1, N_2)\). Then, for all \(n \geq N\), we have both:

    \[
        |a_n - x| < \epsilon \quad \text{and} \quad |a_n - y| < \epsilon
    \]
    Now, using the triangle inequality, we obtain:
    \[
        |x - y| = |(x - a_n) + (a_n - y)| \leq |x - a_n| + |a_n - y|
    \]
    For \(n \geq N\), we know:
    \[
        |x - y| \leq |x - a_n| + |a_n - y| < \epsilon + \epsilon = 2\epsilon
    \]
    Since \(\epsilon\) is arbitrary, we can make \(\epsilon\) as small as we want. Thus, for any \(\epsilon > 0\), we have:
    \[
        |x - y| \leq 2\epsilon
    \]
    Taking the limit as \(\epsilon \to 0\), we get:
    \[
        |x - y| = 0
    \]
    Therefore, \(x = y\).


\end{proof}



\end{document}
